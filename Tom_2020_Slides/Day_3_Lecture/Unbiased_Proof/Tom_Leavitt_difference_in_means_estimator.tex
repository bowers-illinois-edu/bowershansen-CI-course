\RequirePackage[l2tabu, orthodox]{nag} % warn about outdated packages
\documentclass[12pt,leqno]{article}
\usepackage[]{graphicx}\usepackage[]{color}
%% maxwidth is the original width if it is less than linewidth
%% otherwise use linewidth (to make sure the graphics do not exceed the margin)
\makeatletter
\def\maxwidth{ %
  \ifdim\Gin@nat@width>\linewidth
    \linewidth
  \else
    \Gin@nat@width
  \fi
}
\makeatother

\newcommand\given[1][]{\:#1\vert\:}
\definecolor{fgcolor}{rgb}{0.345, 0.345, 0.345}
\newcommand{\hlnum}[1]{\textcolor[rgb]{0.686,0.059,0.569}{#1}}%
\newcommand{\hlstr}[1]{\textcolor[rgb]{0.192,0.494,0.8}{#1}}%
\newcommand{\hlcom}[1]{\textcolor[rgb]{0.678,0.584,0.686}{\textit{#1}}}%
\newcommand{\hlopt}[1]{\textcolor[rgb]{0,0,0}{#1}}%
\newcommand{\hlstd}[1]{\textcolor[rgb]{0.345,0.345,0.345}{#1}}%
\newcommand{\hlkwa}[1]{\textcolor[rgb]{0.161,0.373,0.58}{\textbf{#1}}}%
\newcommand{\hlkwb}[1]{\textcolor[rgb]{0.69,0.353,0.396}{#1}}%
\newcommand{\hlkwc}[1]{\textcolor[rgb]{0.333,0.667,0.333}{#1}}%
\newcommand{\hlkwd}[1]{\textcolor[rgb]{0.737,0.353,0.396}{\textbf{#1}}}%
\let\hlipl\hlkwb

\usepackage{framed}
\makeatletter
\newenvironment{kframe}{%
 \def\at@end@of@kframe{}%
 \ifinner\ifhmode%
  \def\at@end@of@kframe{\end{minipage}}%
  \begin{minipage}{\columnwidth}%
 \fi\fi%
 \def\FrameCommand##1{\hskip\@totalleftmargin \hskip-\fboxsep
 \colorbox{shadecolor}{##1}\hskip-\fboxsep
     % There is no \\@totalrightmargin, so:
     \hskip-\linewidth \hskip-\@totalleftmargin \hskip\columnwidth}%
 \MakeFramed {\advance\hsize-\width
   \@totalleftmargin\z@ \linewidth\hsize
   \@setminipage}}%
 {\par\unskip\endMakeFramed%
 \at@end@of@kframe}
\makeatother

\definecolor{shadecolor}{rgb}{.97, .97, .97}
\definecolor{messagecolor}{rgb}{0, 0, 0}
\definecolor{warningcolor}{rgb}{1, 0, 1}
\definecolor{errorcolor}{rgb}{1, 0, 0}
\newenvironment{knitrout}{}{} % an empty environment to be redefined in TeX

\usepackage{alltt}
\usepackage{microtype} %
\usepackage{setspace}
\onehalfspacing
\usepackage{xcolor, color, ucs}     % http://ctan.org/pkg/xcolor
\usepackage{natbib}
\usepackage{booktabs}          % package for thick lines in tables
\usepackage{amsfonts,amsthm,amsmath,amssymb}          % AMS Stuff
\usepackage{empheq}            % To use left brace on {align} environment
\usepackage{graphicx}          % Insert .pdf, .eps or .png
\usepackage{enumitem}          % http://ctan.org/pkg/enumitem
\usepackage[mathscr]{euscript}          % Font for right expectation sign
\usepackage{tabularx}          % Get scale boxes for tables
\usepackage{float}             % Force floats around
\usepackage{afterpage}% http://ctan.org/pkg/afterpage
\usepackage[T1]{fontenc}
\usepackage{rotating}          % Rotate long tables horizontally
\usepackage{bbm}                % for bold betas
\usepackage{csquotes}           % \enquote{} and \textquote[][]{} environments
\usepackage{subfig}
\usepackage{lscape}
\usepackage{titling}            % modify maketitle in latex
% \usepackage{mathtools}          % multlined environment with size option
\usepackage{verbatim}
\usepackage{geometry}
\usepackage{bigfoot}
\usepackage[format=hang,
            font={small},
            labelfont=bf,
            textfont=rm]{caption}
\usepackage{tikz}
\usetikzlibrary{positioning}

\geometry{verbose,margin=2cm,nomarginpar}
\setcounter{secnumdepth}{2}
\setcounter{tocdepth}{2}

\usepackage{url}
\usepackage{relsize}            % \mathlarger{} environment
\usepackage[unicode=true,
            pdfusetitle,
            bookmarks=true,
            bookmarksnumbered=true,
            bookmarksopen=true,
            bookmarksopenlevel=2,
            breaklinks=false,
            pdfborder={0 0 1},
            backref=page,
            colorlinks=true,
            hyperfootnotes=true,
            hypertexnames=false,
            pdfstartview={XYZ null null 1},
            citecolor=blue!70!black,
            linkcolor=red!70!black,
            urlcolor=green!70!black]{hyperref}
\usepackage{hypernat}

\usepackage{multirow}
\usepackage{titlesec}

\titleformat*{\section}{\large\bfseries}
\titleformat*{\subsection}{\normalsize\bfseries}
\usepackage[noabbrev]{cleveref} % Should be loaded after \usepackage{hyperref}

\parskip=12pt
\parindent=0pt
\delimitershortfall=-1pt
\interfootnotelinepenalty=100000

\makeatletter
\def\thm@space@setup{\thm@preskip=0pt
\thm@postskip=0pt}
\makeatother

\makeatletter
% align all math after the command
\newcommand{\mathleft}{\@fleqntrue\@mathmargin\parindent}
\newcommand{\mathcenter}{\@fleqnfalse}
% tilde with text over it
\newcommand{\distas}[1]{\mathbin{\overset{#1}{\kern\z@\sim}}}%
\newsavebox{\mybox}\newsavebox{\mysim}
\newcommand{\distras}[1]{%
  \savebox{\mybox}{\hbox{\kern3pt$\scriptstyle#1$\kern3pt}}%
  \savebox{\mysim}{\hbox{$\sim$}}%
  \mathbin{\overset{#1}{\kern\z@\resizebox{\wd\mybox}{\ht\mysim}{$\sim$}}}%
}
\makeatother

\newtheoremstyle{newstyle}
{12pt} %Aboveskip
{12pt} %Below skip
{\itshape} %Body font e.g.\mdseries,\bfseries,\scshape,\itshape
{} %Indent
{\bfseries} %Head font e.g.\bfseries,\scshape,\itshape
{.} %Punctuation afer theorem header
{ } %Space after theorem header
{} %Heading

\theoremstyle{newstyle}
\newtheorem{thm}{Theorem}
\newtheorem{prop}[thm]{Proposition}
\newtheorem{lem}{Lemma}
\newtheorem{cor}{Corollary}
\newcommand*\diff{\mathop{}\!\mathrm{d}}
\newcommand*\Diff[1]{\mathop{}\!\mathrm{d^#1}}
\newcommand*{\QEDA}{\hfill\ensuremath{\blacksquare}}%
\newcommand*{\QEDB}{\hfill\ensuremath{\square}}%
\DeclareMathOperator{\E}{\mathbb{E}}
\DeclareMathOperator{\R}{\mathbb{R}}
\DeclareMathOperator{\N}{\mathbb{N}}
\DeclareMathOperator{\Z}{\mathbb{Z}}
\DeclareMathOperator{\Q}{\mathbb{Q}}
\DeclareMathOperator{\Var}{\rm{Var}}
\DeclareMathOperator{\Cov}{\rm{Cov}}
\DeclareMathOperator{\e}{\rm{e}}


%\DeclareMathOperator{\Pr}{\rm{Pr}}

% COLORS FOR GRAPHICS (3-class Set1)
\definecolor{Blue}{RGB}{55,126,184}
\definecolor{Red}{RGB}{228,26,28}
\definecolor{Green}{RGB}{77,175,74}

% COLORS FOR EQUATIONS (3-class Dark2)
\definecolor{eqgreen}{RGB}{27,158,119}
\definecolor{eqblue}{RGB}{117,112,179}
\definecolor{eqred}{RGB}{217,95,2}

\begin{document}

\begin{titlepage}
\title{Unbiasedness of Difference-in-Means Estimator under Complete Random Assignment}
\author{Thomas Leavitt}
\date{\today}
\maketitle
\end{titlepage}

\section{Estimation}

No one can observe both potential outcomes for any given unit in a given study population. One can, however, generate a guess about some function of the study population's individual causal effects (e.g., the mean causal effect) using observed outcomes. We call this unobservable, causal quantity the \textit{estimand}. The \textit{estimator}, by contrast, refers to the procedure that generates a guess about the estimand. An \textit{estimate} is the actual output of the estimator once it is applied to a given data set.

One estimand is the mean causal effect, $\bar{\tau} = \left(\frac{1}{n}\right)\sum \limits_{i = 1}^n \tau_i$, where $\bar{\tau}_i = y_{ti} - y_{ci}$. A procedure for generating a guess about $\bar{\tau}$ is the Difference-in-Means estimator, which we can define in terms of observable quantities as follows:
\begin{equation}
\begin{split}
\label{eq: diff-in-means est}
\hat{\bar{\tau}}\left(\mathbf{Z}, \mathbf{Y}\right) & = \frac{\mathbf{Z}^{\prime}\mathbf{Y}}{\mathbf{Z}^{\prime}\mathbf{1}} - \frac{(\mathbf{1} - \mathbf{Z})^{\prime} \mathbf{Y}}{(\mathbf{1} - \mathbf{Z})^{\prime}(\mathbf{1}} \\ 
& = \left(\frac{1}{\sum \limits_{i = 1}^N Z_i}\right) \sum \limits_{i = 1}^N Z_i Y_i - \left(\frac{1}{\sum \limits_{i = 1}^N \left(1 - Z_i\right)}\right) \sum \limits_{i = 1}^N \left(1 - Z_i\right) Y_i.
\end{split}
\end{equation}

\subsection{Estimation under Complete Random Assignment}

\begin{lem} \label{lem: exp val Z}
Under complete, uniform random assignment in which $n_1$ out of $n$ total units are assigned to treatment, $\E_{\Omega}\left[Z_i\right] = \frac{n_1}{n}$ for all $i \in \left\{1, \dots , n\right\}$ units.\footnote{$\E_{\Omega}\left[\cdot\right]$ indicates that the expected value is taken over the set $\Omega$, i.e., the set of possible assignments.}
\end{lem}
\begin{proof}

We will complete this proof in two steps: We will show that (1) the proportion of assignments in which unit $i$ is in the treatment condition is $\frac{n_1}{n}$ and (2) under uniform assignment, the probability that $Z_i = 1$ is equal to this proportion $\frac{n_1}{n}$.

\begin{enumerate}

\item First note that the number of ways to choose a subset of $n_1$ treated units from a fixed population of $n$ units is as follows:

\begin{equation} \label{eq: total assignments}
\binom{n}{n_1} = \frac{n!}{\left(n - n_1\right)!n_1!} = \frac{n!}{n_0!n_1!},
\end{equation}
where $n_0 = n - n_1$ is the number of units assigned to the control condition.

Given that an arbitrary unit $i$ is in the treatment condition and only $n_1$ total units can be in the treatment condition, there are $\binom{n - 1}{n_1 - 1}$ ways in which $n_1 - 1$ other units could be in the treatment condition. Hence, the number of assignments in which unit $i$ is treated and $n_1 - 1$ other units are treated is:

\begin{equation} \label{eq: assignments unit i treated}
\binom{n - 1}{n_1 - 1} = \frac{\left(n - 1\right)!}{\left(\left(n - 1\right) - \left(n_1 - 1\right)\right)!\left(n_1 - 1\right)!}
\end{equation}

To get the proportion of assignments in which unit $i$ is treated, we need to divide \eqref{eq: assignments unit i treated} by \eqref{eq: total assignments}:

\begin{equation} \label{eq: prop assignments unit i treated I}
\frac{\binom{n - 1}{n_1 - 1}}{\binom{n}{n_1}} = \frac{\left(\frac{\left(n - 1\right)!}{\left(\left(n - 1\right) - \left(n_1 - 1\right)\right)!\left(n_1 - 1\right)!}\right)}{\left(\frac{n!}{n_0!n_1!}\right)}
\end{equation}

Now notice that:
\begin{align*}
\left(n - 1\right) - \left(n_1 - 1\right) & = n - 1 - n_1 + 1 \\
& = n - n_1 \\
& = n_0
\end{align*}
We can therefore substitute $n_0$ for $\left(n - 1\right) - \left(n_1 - 1\right)$ in \eqref{eq: prop assignments unit i treated I}, which gives us:

\begin{equation}\label{eq: prop assignments unit i treated II}
\frac{\left(\frac{\left(n - 1\right)!}{n_0!\left(n_1 - 1\right)!}\right)}{\left(\frac{n!}{n_0!n_1!}\right)} \\
\end{equation}

Now we can simply manipulate \eqref{eq: prop assignments unit i treated II} and cancel terms until we are left with $\frac{n_1}{n}$:
\begin{align*}
& = \left(\frac{\left(n - 1\right)!}{n_0!\left(n_1 - 1\right)!}\right)\left(\frac{n_0!n_1!}{n!}\right) \\
& = \left(\frac{\left(n - 1\right)\left(n - 2\right) \dots 1}{n_0\left(n_0 - 1\right) \dots 1 \left(n_1 - 1\right) \dots 1}\right) \left(\frac{n_0\left(n_0 - 1\right) \dots 1 n_1 \left(n_1 - 1\right) \dots 1}{n\left(n - 1\right) \dots 1}\right) \\
& = \frac{\textcolor{blue}{\left(n - 1\right)\left(n - 2\right) \dots 2} \textcolor{red}{n_0\left(n_0 - 1\right) \dots 2} n_1 \textcolor{green}{\left(n_1 - 1\right) \dots 2}}{\textcolor{red}{n_0\left(n_0 - 1\right) \dots 2} \textcolor{green}{\left(n_1 - 1\right) \dots 2} n\textcolor{blue}{\left(n - 1\right) \dots 2}} \\
\end{align*}
All of the matching colors in the numerator and denominator cancel, which leaves us with $\frac{n_1}{n}$. Therefore, exactly $\frac{n_1}{n}$ out of all assignment assignments will be those in which unit $i$ is in the treatment condition. 

\item The total probability of all assignments in which $i$ is treated is simply the sum of the probabilities of those assignments in which unit $i$ is in the treatment condition. Under uniform random assignment, the probability of each assignment is $\frac{1}{\left\lvert\Omega\right\rvert}$, where $\left\lvert\Omega\right\rvert$ is the ``cardinality'' (i.e., number of elements in) the set of assignments, $\Omega$. Thus, the probability that unit $i$ is treated is as follows:
\begin{align*}
\left(\frac{1}{\left\lvert\Omega\right\rvert}\right) \left(\left(\frac{n_1}{n}\right)\left(\left\lvert\Omega\right\rvert\right)\right) \\ 
& = \left(\frac{1}{\left\lvert\Omega\right\rvert}\right) \left(\frac{n_1\left\lvert\Omega\right\rvert}{n}\right) \\
& = \left(\frac{n_1\left\lvert\Omega\right\rvert}{\left\lvert\Omega\right\rvert n}\right) \\ 
& = \frac{n_1}{n}  
 \end{align*}
\end{enumerate}

Since $\Pr\left(Z_i = 1\right) = \frac{n_1}{n}$ for all $i \in \left\{1, \dots , n\right\}$ units, it follows that the expected value of $Z_i \in \left\{0, 1\right\}$ is $\E_{\Omega}\left[Z_i\right] = 1\left(\frac{n_1}{n}\right) + 0\left(1 - \frac{n_1}{n}\right) = \frac{n_1}{n}$.
\end{proof}

\begin{prop} \label{prop: complete ran assign}
Under complete, uniform random assignment, $\mathbb{E}_{\Omega} \left[\frac{\mathbf{Z}^{\prime}\mathbf{Y}}{\mathbf{Z}^{\prime}\mathbf{Z}} - \frac{(\mathbf{1} - \mathbf{Z})^{\prime} \mathbf{Y}}{(\mathbf{1} - \mathbf{Z})^{\prime}(\mathbf{1} - \mathbf{Z})}\right] = \overline{y_{t}} - \overline{y_{c}} = \overline{\tau}$.
\end{prop}
\begin{proof}
\begin{align*}
\mathbb{E}_{\Omega} \left[\frac{\mathbf{Z}^{\prime}\mathbf{Y}}{\mathbf{Z}^{\prime}\mathbf{Z}} - \frac{(\mathbf{1} - \mathbf{Z})^{\prime} \mathbf{Y}}{(\mathbf{1} - \mathbf{Z})^{\prime}(\mathbf{1} - \mathbf{Z})}\right] \\
= \mathbb{E}_{\Omega} \left[\frac{\mathbf{Z}^{\prime}\mathbf{Y}}{\mathbf{Z}^{\prime}\mathbf{Z}}\right] - \mathbb{E}_{\Omega}\left[\frac{(\mathbf{1} - \mathbf{Z})^{\prime} \mathbf{Y}}{(\mathbf{1} - \mathbf{Z})^{\prime}(\mathbf{1} - \mathbf{Z})}\right] \\
\end{align*}
Notice that $\mathbf{Z}^{\prime}\mathbf{Y}$ is simply the sum of observed outcomes among treated units (i.e., the sum of observed outcomes among units for which $Z_i = 1$). The quantity $\mathbf{Z}^{\prime}\mathbf{y_t}$ is the sum of treatment potential outcomes among treated units. Since the observed outcomes for treated units is equal to those units' treatment potential outcomes, we can substitute $\mathbf{Z}^{\prime}\mathbf{y_t}$ for $\mathbf{Z}^{\prime}\mathbf{Y}$. Analogously, we can substitute $\left(\mathbf{1} - \mathbf{Z}\right)^{\prime}\mathbf{y_c}$ for $\left(\mathbf{1} - \mathbf{Z}\right)^{\prime}\mathbf{Y}$. After substituting $\mathbf{Z}^{\prime}\mathbf{y_t}$ for $\mathbf{Z}^{\prime}\mathbf{Y}$ and $\left(\mathbf{1} - \mathbf{Z}\right)^{\prime}\mathbf{y_c}$ for $\left(\mathbf{1} - \mathbf{Z}\right)^{\prime}\mathbf{Y}$, we are left with:
\begin{align*}
= \mathbb{E}_{\Omega} \left[\frac{\mathbf{Z}^{\prime} \mathbf{y_t}}{\mathbf{Z}^{\prime}\mathbf{Z}}\right] - \mathbb{E}_{\Omega}\left[\frac{(\mathbf{1} - \mathbf{Z})^{\prime} \mathbf{y_c}}{(\mathbf{1} - \mathbf{Z})^{\prime}(\mathbf{1} - \mathbf{Z})}\right].
\end{align*}
Notice that under complete random assignment, the number of treatment units, $\mathbf{Z}^{\prime}\mathbf{Z}$, and the number of control units, $\left(\mathbf{1} - \mathbf{Z}\right)^{\prime}\left(\mathbf{1} - \mathbf{Z}\right)$ are fixed constants, which we denote by $n_1$ and $n_0$, respectively.

\begin{align*}
= \mathbb{E}_{\Omega} \left[\frac{\mathbf{Z}^{\prime} \mathbf{y_t}}{n_1}\right] - \mathbb{E}_{\Omega}\left[\frac{(\mathbf{1} - \mathbf{Z})^{\prime} \mathbf{y_c}}{n_0}\right] \\
= \left(\frac{1}{n_1} \right) \mathbb{E}_{\Omega} \left[\mathbf{Z}^{\prime} \mathbf{y_t}\right] - \left(\frac{1}{n_0}\right) \mathbb{E}_{\Omega}\left[(\mathbf{1} - \mathbf{Z})^{\prime} \mathbf{y_c}\right] \\
= \left(\frac{1}{n_1} \right) \mathbb{E}_{\Omega} \left[ \sum \limits_{i = 1}^n Z_i y_{ti}\right] - \left(\frac{1}{n_0}\right) \mathbb{E}_{\Omega}\left[\sum \limits_{i = 1}^n \left(1 - Z_i\right) y_{ci} \right] \\
= \left(\frac{1}{n_1} \right) \mathbb{E}_{\Omega} \left[ Z_1 y_{t1} + \dots + Z_n y_{tn}\right] - \left(\frac{1}{n_0}\right) \mathbb{E}_{\Omega}\left[\left(1 - Z_1\right) y_{c1} + \dots + \left(1 - Z_n\right) y_{cn} \right] \\
= \left(\frac{1}{n_1} \right) \mathbb{E}_{\Omega} \left[ Z_1 y_{t1}\right] + \dots + \mathbb{E}_{\Omega} \left[Z_n y_{tn}\right] - \left(\frac{1}{n_0}\right) \mathbb{E}_{\Omega}\left[\left(1 - Z_1\right) y_{c1}\right] + \dots + \mathbb{E}_{\Omega} \left[\left(1 - Z_n\right) y_{cn} \right] \\
= \left(\frac{1}{n_1} \right) \left(y_{t1}\mathbb{E}_{\Omega} \left[ Z_1 \right] + \dots + y_{tn}\mathbb{E}_{\Omega} \left[Z_n\right]\right) - \left(\frac{1}{n_0}\right) \left(y_{c1}\mathbb{E}_{\Omega}\left[\left(1 - Z_1\right) \right] + \dots + y_{cn}\mathbb{E}_{\Omega} \left[\left(1 - Z_n\right) \right]\right)
\end{align*}


By Lemma \ref{lem: exp val Z}, $\mathbb{E}_{\Omega}\left[Z_i \right] = \left(\frac{n_1}{n}\right)$ for all $i \in \left\{1, \dots , n\right\}$, which implies that $\mathbb{E}_{\Omega}\left[\left(1 - Z_i\right) \right] = 1 - \left(\frac{n_1}{n}\right) = \left(\frac{n_0}{n}\right)$ for all $i \in \left\{1, \dots, n\right\}$. Hence, for all $i \in \left\{1, \dots , n\right\}$, we can substitute $\left(\frac{n_1}{n}\right)$ for $\mathbb{E}_{\Omega}\left[Z_i \right]$ and $\left(\frac{n_0}{n}\right)$ for $\mathbb{E}_{\Omega}\left[1 - Z_i \right]$.

\begin{align*}
= \left(\frac{1}{n_1} \right) \left(y_{t1}\mathbb{E}_{\Omega} \left[ Z_1 \right] + \dots + y_{tn}\mathbb{E}_{\Omega} \left[Z_n\right]\right) - \left(\frac{1}{n_0}\right) \left(y_{c1}\mathbb{E}_{\Omega}\left[\left(1 - Z_1\right) \right] + \dots + y_{cn}\mathbb{E}_{\Omega} \left[\left(1 - Z_n\right) \right]\right) \\ 
= \left(\frac{1}{n_1} \right) \left(y_{t1}\left(\frac{n_1}{n}\right) + \dots + y_{tn}\left(\frac{n_1}{n}\right)\right) - \left(\frac{1}{n_0}\right) \left(y_{c1}\left(\frac{n_0}{n}\right) + \dots + y_{cn}\left(\frac{n_0}{n}\right)\right) \\ 
= \left(\frac{1}{n_1} \right) \left(\frac{n_1}{n}\right) \left(y_{t1} + \dots + y_{tn} \right) - \left(\frac{1}{n_0}\right) \left(\frac{n_0}{n}\right) \left(y_{c1} + \dots + y_{cn} \right) \\
= \left(\frac{1}{n} \right) \left(y_{t1} + \dots + y_{tn} \right) - \left(\frac{1}{n}\right) \left(y_{c1} + \dots + y_{cn} \right) \\
= \frac{\left(y_{t1} + \dots + y_{tn} \right)}{n} - \frac{\left(y_{c1} + \dots + y_{cn} \right)}{n} \\
= \overline{y_{t}} - \overline{y_{c}} \\
= \overline{\tau}.
\end{align*}
\end{proof}



\begin{singlespace}
\bibliographystyle{chicago}
\bibliography{master_bibliography}   % name your BibTeX data base
\end{singlespace}

\end{document}
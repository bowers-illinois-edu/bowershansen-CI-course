\newcommand{\igrphx}[2][width=\linewidth]{\includegraphics[#1]{images/#2}}

\renewcommand{\strut}{\rule{0pt}{3ex}}

% Frame note commands, with time budget updating
\newcounter{timeTotal}
\newcommand{\printUpdateTimeTotal}[1]{ \addtocounter{timeTotal}{#1}
  \textrm{(#1 min budgeted)} \hfill \textbf{Finish by
    \arabic{timeTotal}\ min from start}\\[1ex]}
\newcommand{\tnote}[3][10]{#2 \note{#2: #3 \\[.5ex]} \printUpdateTimeTotal{#1}}
\newcommand{\itnote}[2][10]{\note{ \begin{itemize} #2 \end{itemize}
\vfill \noindent  \printUpdateTimeTotal{#1} }}
\newcommand{\ennote}[2][10]{\note{ \begin{enumerate} #2 \end{enumerate}
\vfill \noindent  \printUpdateTimeTotal{#1} }}
\newcommand{\Note}[2][10]{\note{ #2 \mbox{ }\\ \vfill \noindent  \printUpdateTimeTotal{#1} }}

\newenvironment{Column}[1][.5\linewidth]{\begin{column}{#1}}{\end{column}}

\mode<handout>{\beamertemplatesolidbackgroundcolor{black!5}} 
\mode<article>{\usepackage{fullpage}}
\mode<presentation>
{
  \usetheme{Boadilla}
  % or ...

  \setbeamercovered{transparent}
  % or whatever (possibly just delete it)
}
\usepackage{url}
\usepackage{ulem}

\usepackage[english]{babel}
% or whatever

\usepackage[latin1]{inputenc}
% or whatever

\usepackage{times}
\usepackage[T1]{fontenc}
% Or whatever. Note that the encoding and the font should match. If T1
% does not look nice, try deleting the line with the fontenc.

\AtBeginSubsection[]
{
  \begin{frame}<beamer>
    \frametitle{Outline}
    \tableofcontents[subsectionstyle=show/shaded/hide]
  \end{frame}
}

\AtBeginSection[]
{
  \begin{frame}<beamer>
    \frametitle{Outline}
    \tableofcontents[hideothersubsections,sectionstyle=show/shaded]
  \end{frame}
}


% If you wish to uncover everything in a step-wise fashion, uncomment
% the following command: 

%\beamerdefaultoverlayspecification{<+->}

%USEFUL CODE TEMPLATES:
%\begin{itemize}[<+-| alert@+>]


% \begin{frame}[fragile]


% \begin{columns}
% \column{.4\textwidth}  
%   \begin{itemize}
%   \item<1-| alert@1> Why sample twice?
%   \item<2-| alert@2> Illustrative example.
%   \item<3-| alert@3> PSU's and SSU's.
%   \item<4-| alert@4> Stratification in combination with two-stage cluster sampling.
%   \item<5-| alert@5> Detailed example.
%   \end{itemize}
% \column{.6\textwidth}
% \only<2| handout:0>{
% \includegraphics[width=\textwidth]{nursingHomes}}
% \only<5| handout:1>{
% \includegraphics[width=\textwidth]{hosp_cover}}
% \end{columns}

% FOR INCLUDING R CODE. 
%\usepackage{listings}
%\lstset{language=R}
% \begin{frame}[fragile]
% \frametitle{Some code}
% \begin{lstlisting}
% > plot(myobj)
% > rm(myobj)
% \end{lstlisting}  
% \end{frame}

%\addtocounter{framenumber}{-1}


% Options for packages loaded elsewhere
\PassOptionsToPackage{unicode}{hyperref}
\PassOptionsToPackage{hyphens}{url}
%
\documentclass[
  12pt,
  leqno]{article}
\usepackage{lmodern}
\usepackage{amssymb,amsmath}
\usepackage{ifxetex,ifluatex}
\ifnum 0\ifxetex 1\fi\ifluatex 1\fi=0 % if pdftex
  \usepackage[T1]{fontenc}
  \usepackage[utf8]{inputenc}
  \usepackage{textcomp} % provide euro and other symbols
\else % if luatex or xetex
  \usepackage{unicode-math}
  \defaultfontfeatures{Scale=MatchLowercase}
  \defaultfontfeatures[\rmfamily]{Ligatures=TeX,Scale=1}
\fi
% Use upquote if available, for straight quotes in verbatim environments
\IfFileExists{upquote.sty}{\usepackage{upquote}}{}
\IfFileExists{microtype.sty}{% use microtype if available
  \usepackage[]{microtype}
  \UseMicrotypeSet[protrusion]{basicmath} % disable protrusion for tt fonts
}{}
\makeatletter
\@ifundefined{KOMAClassName}{% if non-KOMA class
  \IfFileExists{parskip.sty}{%
    \usepackage{parskip}
  }{% else
    \setlength{\parindent}{0pt}
    \setlength{\parskip}{6pt plus 2pt minus 1pt}}
}{% if KOMA class
  \KOMAoptions{parskip=half}}
\makeatother
\usepackage{xcolor}
\IfFileExists{xurl.sty}{\usepackage{xurl}}{} % add URL line breaks if available
\IfFileExists{bookmark.sty}{\usepackage{bookmark}}{\usepackage{hyperref}}
\hypersetup{
  pdftitle={Assignment 2 Answer Key},
  hidelinks,
  pdfcreator={LaTeX via pandoc}}
\urlstyle{same} % disable monospaced font for URLs
\usepackage[margin = 1.5cm]{geometry}
\usepackage{color}
\usepackage{fancyvrb}
\newcommand{\VerbBar}{|}
\newcommand{\VERB}{\Verb[commandchars=\\\{\}]}
\DefineVerbatimEnvironment{Highlighting}{Verbatim}{commandchars=\\\{\}}
% Add ',fontsize=\small' for more characters per line
\usepackage{framed}
\definecolor{shadecolor}{RGB}{248,248,248}
\newenvironment{Shaded}{\begin{snugshade}}{\end{snugshade}}
\newcommand{\AlertTok}[1]{\textcolor[rgb]{0.94,0.16,0.16}{#1}}
\newcommand{\AnnotationTok}[1]{\textcolor[rgb]{0.56,0.35,0.01}{\textbf{\textit{#1}}}}
\newcommand{\AttributeTok}[1]{\textcolor[rgb]{0.77,0.63,0.00}{#1}}
\newcommand{\BaseNTok}[1]{\textcolor[rgb]{0.00,0.00,0.81}{#1}}
\newcommand{\BuiltInTok}[1]{#1}
\newcommand{\CharTok}[1]{\textcolor[rgb]{0.31,0.60,0.02}{#1}}
\newcommand{\CommentTok}[1]{\textcolor[rgb]{0.56,0.35,0.01}{\textit{#1}}}
\newcommand{\CommentVarTok}[1]{\textcolor[rgb]{0.56,0.35,0.01}{\textbf{\textit{#1}}}}
\newcommand{\ConstantTok}[1]{\textcolor[rgb]{0.00,0.00,0.00}{#1}}
\newcommand{\ControlFlowTok}[1]{\textcolor[rgb]{0.13,0.29,0.53}{\textbf{#1}}}
\newcommand{\DataTypeTok}[1]{\textcolor[rgb]{0.13,0.29,0.53}{#1}}
\newcommand{\DecValTok}[1]{\textcolor[rgb]{0.00,0.00,0.81}{#1}}
\newcommand{\DocumentationTok}[1]{\textcolor[rgb]{0.56,0.35,0.01}{\textbf{\textit{#1}}}}
\newcommand{\ErrorTok}[1]{\textcolor[rgb]{0.64,0.00,0.00}{\textbf{#1}}}
\newcommand{\ExtensionTok}[1]{#1}
\newcommand{\FloatTok}[1]{\textcolor[rgb]{0.00,0.00,0.81}{#1}}
\newcommand{\FunctionTok}[1]{\textcolor[rgb]{0.00,0.00,0.00}{#1}}
\newcommand{\ImportTok}[1]{#1}
\newcommand{\InformationTok}[1]{\textcolor[rgb]{0.56,0.35,0.01}{\textbf{\textit{#1}}}}
\newcommand{\KeywordTok}[1]{\textcolor[rgb]{0.13,0.29,0.53}{\textbf{#1}}}
\newcommand{\NormalTok}[1]{#1}
\newcommand{\OperatorTok}[1]{\textcolor[rgb]{0.81,0.36,0.00}{\textbf{#1}}}
\newcommand{\OtherTok}[1]{\textcolor[rgb]{0.56,0.35,0.01}{#1}}
\newcommand{\PreprocessorTok}[1]{\textcolor[rgb]{0.56,0.35,0.01}{\textit{#1}}}
\newcommand{\RegionMarkerTok}[1]{#1}
\newcommand{\SpecialCharTok}[1]{\textcolor[rgb]{0.00,0.00,0.00}{#1}}
\newcommand{\SpecialStringTok}[1]{\textcolor[rgb]{0.31,0.60,0.02}{#1}}
\newcommand{\StringTok}[1]{\textcolor[rgb]{0.31,0.60,0.02}{#1}}
\newcommand{\VariableTok}[1]{\textcolor[rgb]{0.00,0.00,0.00}{#1}}
\newcommand{\VerbatimStringTok}[1]{\textcolor[rgb]{0.31,0.60,0.02}{#1}}
\newcommand{\WarningTok}[1]{\textcolor[rgb]{0.56,0.35,0.01}{\textbf{\textit{#1}}}}
\usepackage{graphicx}
\makeatletter
\def\maxwidth{\ifdim\Gin@nat@width>\linewidth\linewidth\else\Gin@nat@width\fi}
\def\maxheight{\ifdim\Gin@nat@height>\textheight\textheight\else\Gin@nat@height\fi}
\makeatother
% Scale images if necessary, so that they will not overflow the page
% margins by default, and it is still possible to overwrite the defaults
% using explicit options in \includegraphics[width, height, ...]{}
\setkeys{Gin}{width=\maxwidth,height=\maxheight,keepaspectratio}
% Set default figure placement to htbp
\makeatletter
\def\fps@figure{htbp}
\makeatother
\setlength{\emergencystretch}{3em} % prevent overfull lines
\providecommand{\tightlist}{%
  \setlength{\itemsep}{0pt}\setlength{\parskip}{0pt}}
\setcounter{secnumdepth}{-\maxdimen} % remove section numbering
% \usepackage{microtype} %
% \usepackage{setspace}
% \onehalfspacing
\usepackage{xcolor, color, ucs}     % http://ctan.org/pkg/xcolor
\usepackage{natbib}
\usepackage{booktabs}          % package for thick lines in tables
\usepackage{amsfonts,amsthm,amsmath,amssymb,bm}          % AMS Fonts
% \usepackage{empheq}            % To use left brace on {align} environment
\usepackage{graphicx}          % Insert .pdf, .eps or .png
\usepackage{enumitem}          % http://ctan.org/pkg/enumitem
% \usepackage[mathscr]{euscript}          % Font for right expectation sign
\usepackage{tabularx}          % Get scale boxes for tables
\usepackage{float}             % Force floats around
\usepackage{rotating}          % Rotate long tables horizontally
\usepackage{bbm}                % for bold betas
\usepackage{csquotes}           % \enquote{} and \textquote[][]{} environments
\usepackage{subfigure}
\usepackage{array}
% \usepackage{cancel}
\usepackage{longtable}
% % \usepackage{lmodern}
% % \usepackage{libertine} \usepackage[libertine]{newtxmath}
% \usepackage{stix}
% % \usepackage[osf,sc]{mathpazo}     % alternative math
% \usepackage[T1]{fontenc}
% \usepackage{fontspec}
% \setmainfont{Times New Roman}
% \usepackage{mathtools}          % multlined environment with size option
% \usepackage{verbatim}
% \usepackage{geometry}
% \usepackage{bigfoot}
% \geometry{verbose,margin=.8in,nomarginpar}
% \setcounter{secnumdepth}{2}
% \setcounter{tocdepth}{2}
% \usepackage{lscape}

\setlist{nosep}


% \usepackage{url}
% \usepackage[nobreak=true]{mdframed} % put box around section with \begin{mdframed}\end{mdframed}

% \usepackage{relsize}            % \mathlarger{} environment
% \usepackage[unicode=true,
%             pdfusetitle,
%             bookmarks=true,
%             bookmarksnumbered=true,
%             bookmarksopen=true,
%             bookmarksopenlevel=2,
%             breaklinks=false,
%             pdfborder={0 0 1},
%             backref=false,
%             colorlinks=true,
%             hypertexnames=false]{hyperref}
% \hypersetup{pdfstartview={XYZ null null 1},
%             citecolor=blue!50,
%             linkcolor=red,
%             urlcolor=green!70!black}

\usepackage{multirow}
% \usepackage{tikz}
% \usetikzlibrary{trees, positioning, arrows, automata}

% \tikzset{
%   treenode/.style = {align=center, inner sep=0pt, text centered,
%     font=\sffamily},
%   arn_n/.style = {treenode, rectangle, black, fill=white, text width=6em},
%   arn_r/.style = {treenode, circle, red, draw=red, text width=1.5em, thick}
% }

\usepackage[noabbrev]{cleveref} % Should be loaded after \usepackage{hyperref}
\usepackage[small,bf]{caption}  % Captions

% \usepackage[obeyFinal,textwidth=0.8in, colorinlistoftodos,prependcaption,textsize=tiny]{todonotes} % \fxnote*[options]{note}{text} to make sticky notes
% \usepackage{xargs}
% \newcommandx{\unsure}[2][1=]{\todo[linecolor=red,backgroundcolor=red!25,bordercolor=red,#1]{#2}}
% \newcommandx{\change}[2][1=]{\todo[linecolor=blue,backgroundcolor=blue!25,bordercolor=blue,#1]{#2}}
% \newcommandx{\info}[2][1=]{\todo[linecolor=OliveGreen,backgroundcolor=OliveGreen!25,bordercolor=OliveGreen,#1]{#2}}
% \newcommandx{\improvement}[2][1=]{\todo[linecolor=Plum,backgroundcolor=Plum!25,bordercolor=Plum,#1]{#2}}

\parskip=8pt
\parindent=0pt
\delimitershortfall=-1pt
\interfootnotelinepenalty=100000

\newcommand{\qedknitr}{\hfill\rule{1.2ex}{1.2ex}}

% \makeatletter
% \def\thm@space@setup{\thm@preskip=0pt
% \thm@postskip=0pt}
% \makeatother

\def\tightlist{}

\makeatletter
% align all math after the command
\newcommand{\mathleft}{\@fleqntrue\@mathmargin\parindent}
\newcommand{\mathcenter}{\@fleqnfalse}
% tilde with text over it
\newcommand{\distas}[1]{\mathbin{\overset{#1}{\kern\z@\sim}}}%
\newsavebox{\mybox}\newsavebox{\mysim}
\newcommand{\distras}[1]{%
  \savebox{\mybox}{\hbox{\kern3pt$\scriptstyle#1$\kern3pt}}%
  \savebox{\mysim}{\hbox{$\sim$}}%
  \mathbin{\overset{#1}{\kern\z@\resizebox{\wd\mybox}{\ht\mysim}{$\sim$}}}%
}
\makeatother

% \newtheoremstyle{newstyle}
% {} %Aboveskip
% {} %Below skip
% {\mdseries} %Body font e.g.\mdseries,\bfseries,\scshape,\itshape
% {} %Indent
% {\bfseries} %Head font e.g.\bfseries,\scshape,\itshape
% {.} %Punctuation afer theorem header
% { } %Space after theorem header
% {} %Heading

\newtheorem{thm}{Theorem}
\newtheorem{prop}[thm]{Proposition}
\newtheorem{lem}{Lemma}
\newtheorem{cor}{Corollary}
\newtheorem{definition}{Definition}
\newcommand*\diff{\mathop{}\!\mathrm{d}}
\newcommand*\Diff[1]{\mathop{}\!\mathrm{d^#1}}
\DeclareMathOperator{\E}{\mathrm{E}}
\DeclareMathOperator{\Var}{\mathrm{Var}}
\DeclareMathOperator{\R}{\mathbb{R}}
\DeclareMathOperator{\1}{\mathbbm{1}}
\newcolumntype{L}[1]{>{\raggedright\let\newline\\\arraybackslash\hspace{0pt}}m{#1}}
\newcolumntype{C}[1]{>{\centering\let\newline\\\arraybackslash\hspace{0pt}}m{#1}}
\newcolumntype{R}[1]{>{\raggedleft\let\newline\\\arraybackslash\hspace{0pt}}m{#1}}

% suppress table numbering
\captionsetup[table]{labelformat=empty}

\ifluatex
  \usepackage{selnolig}  % disable illegal ligatures
\fi
\usepackage[]{natbib}
\bibliographystyle{apsr}

\title{Assignment 2 Answer Key}
\author{\href{mailto:tl2624@columbia.edu}{Thomas Leavitt}}
\date{08 August 2020}

\begin{document}
\maketitle

\section*{Question 1}

Assuming SUTVA (i.e., no interference and no hidden levels of treatment
assignment) and the random assignment of the instrument (encouragement),
the average causal effect of encouragement, \(Z\), on dose, \(D\), is
\begin{align*}
\bar{\tau}_D & = \bar{d}_T - \bar{d}_C \\
& = \left(\frac{1}{n}\right) \sum \limits_{i = 1}^n d_{Ti} - \left(\frac{1}{n}\right) \sum \limits_{i = 1}^nd_{Ci} \\ 
& = \left(\frac{1}{n}\right) \sum \limits_{i = 1}^n \left(d_{Ti} - d_{Ci}\right) \\ 
& = \left(\frac{1}{n}\right) \sum \limits_{i = 1}^n \tau_{Di} \\
& = \left(\frac{1}{n}\right) \sum \limits_{i = 1}^n \tau_{Di}
\end{align*} We can estimate this effect via the \(ITT_D\) estimator,
which is \begin{align*}
\widehat{ITT_D} & = \frac{\mathbf{Z}^{t}\mathbf{D}}{\mathbf{Z}^{t}\mathbf{1}} - \frac{\left(\mathbf{1} - \mathbf{Z}\right)^{t}\mathbf{D}}{\left(\mathbf{1} - \mathbf{Z}\right)^{t}\mathbf{1}} \\ 
& = \left(\frac{1}{\sum_{i = 1}^n Z_i}\right) \sum_{i = 1}^n Z_i D_i - \left(\frac{1}{\sum_{i = 1}^n \left(1 - Z_i\right)}\right) \sum_{i = 1}^n \left(1 - Z_i\right) D_i,
\end{align*} where the superscript \(t\) denotes matrix transposition
and \(D_i = Z_i d_{Ti} + \left(1 - Z_i\right) d_{Ci}\).

For the first study, the estimate of the effect of encouragement on
program viewing is \(\frac{244}{510} - \frac{74}{579} \approx 0.3506\),
and for the second study, the estimate is \$ \frac{117}{259} -
\frac{11}{248} \approx 0.4074\$.

\section*{Question 2}

A hypothetical outcome variable one could collect is ``views on
addiction policy,'\,' measured on a scale from 0 -- 10 in which 0
represents no support for public treatment programs and 10 is full
support. Let's imagine that the sum of this outcome among the 510
treated units is 3,723 and the sum of this outcome among the 579 control
units is 2,895, yielding an average outcome of \(7.3\) and \(5\) in
treated and control groups, respectively. \begin{align*}
\frac{3723}{510} - \frac{2895}{579} & = 7.3 - 5 \\
& = 2.3.
\end{align*}

The CACE (complier average causal effect) estimator is
\(\frac{\widehat{\text{ITT}}_Y}{\widehat{\text{ITT}}_D}\), where
\(\widehat{\text{ITT}}_D\) is as defined above and
\(\widehat{\text{ITT}}_Y\) is \begin{align*}
\widehat{ITT}_Y & = \cfrac{\mathbf{Z}^{t}\mathbf{Y}}{\mathbf{Z}^{t}\mathbf{1}} - \cfrac{\left(\mathbf{1} - \mathbf{Z}\right)^{t}\mathbf{Y}}{\left(\mathbf{1} - \mathbf{Z}\right)^{t}\mathbf{1}} \\ 
& = \left(\cfrac{1}{\sum_{i = 1}^n Z_i}\right) \sum_{i = 1}^n Z_i Y_i - \left(\cfrac{1}{\sum_{i = 1}^n \left(1 - Z_i\right)}\right) \sum_{i = 1}^n \left(1 - Z_i\right) Y_i
\end{align*}

Using the hypothetical results described earlier, out CACE estimate is
simply the ratio of our estimate of \(\text{ITT}_Y\) (the average causal
effect of \(Z\) on \(Y\)) and \(\text{ITT}_D\) (the average causal
effect of \(Z\) on \(D\)), which is \begin{align*}
\frac{2.3}{0.3506} & \approx 6.5602
\end{align*}

\section*{Question 3}

\citet[276--277]{albertsonlawrence2009} write that ``{[}b{]}y using a
random sample, we avoid the external validity problems associated with
samples of convenience that are generally used in laboratory
experiments. In addition, by assigning respondents at random to viewing
and nonviewing conditions, these studies benefit from experimental
control. Yet unlike laboratory experiments, this design allows
respondents to view programs in their own homes, thus more closely
approximating regular viewing conditions.''

\section*{Question 4}

\section*{(a)}

\citet{albertsonlawrence2009} state that the study they analyzed used
random sampling from five metropolitan areas to recruit individuals into
the study. Among individuals who were successfully recruited, the
researchers assigned units to treatment and control groups. Among the
individuals in this study population, the researchers were able to
contact only 80\% of them to measure outcomes in Round 2
\citep[284]{albertsonlawrence2009}.

Since we know that there were \(1089\) individuals who responded in
Round 2 and that these individuals make up 80\% of the round 1 study
popuation, we can reason backwards to infer that there were
\(\frac{1089}{0.8} \approx 1361\) individuals in the study, only
\(1089\) of which responded in Round 2.

Let \(r_{Ti}\) be an indicator for whether subject \(i\) would respond
in Round 2 if assigned to treatment and let \(r_{Ci}\) be an indicator
for whether subject \(i\) would respond in Round 2 if assigned to
control. Potential outcomes are translated into observed outcomes
according to the following equation: \begin{equation}
\label{eq:obs_outcomes}
Y_i = \begin{cases} y_{Ci} + [y_{Ti} - y_{Ci}] Z_i & \text{if } R_i = 1 \\
\text{NA} & \text{if } R_i = 0, \end{cases}
\end{equation} where \(R_i = Z_i r_{Ti} + \left(1 - Z_i\right) r_{Ci}\).

From Equation \ref{eq:obs_outcomes} above, we can see that if
\(R_i = 1\), then the researcher will observe \(y_{Ci}\) for unit \(i\)
if \(Z_i = 0\) and \(y_{Ti}\) for unit \(i\) if \(Z_i = 1\). By
contrast, if \(R_i = 0\), then \(Y_i\) will be unobserved --- i.e., NA.

Now we can define four distinct types of subjects with regard to
attrition:

\begin{table}[h]
\centering
    \begin{tabular}{lll}
    \toprule
    $z_i = 0$ & $z_i = 1$ & Type of Subject       \\
    \midrule
    $r_{Ci} = 1$ & $r_{Ti} = 1$ & \textit{Always-Reporter} \\
    $r_{Ci} = 0$ & $r_{Ti} = 1$ & \textit{If-Treated-Reporter} \\
    $r_{Ci} = 1$ & $r_{Ti} = 0$ & \textit{If-Untreated-Reporter} \\
    $r_{Ci} = 0$ & $r_{Ti} = 0$ & \textit{Never-Reporter} \\
    \bottomrule
    \end{tabular}
\end{table}

\citet{albertsonlawrence2009} measure \(D_i\) and \(Y_i\) among only
80\% of the individuals who were assigned to treatment and control. A
treated individual whose outcome the researchers are able to measure
could be an Always-Reporters or an If-Treated-Reporters. By contrast, a
control individual whose outcome the researchers are able to measure
could be an Always-Reporter or an If-Untreated-Reporter.
\citet{albertsonlawrence2009} restrict their analysis to only the
individuals whose \(D_i\) and \(Y_i\) outcomes they did observe.

To see how attrition can lead to bias in the Difference-in-Means
estimator under complete random assignment, let's first rewrite the
Difference-in-Means estimator as \begin{align*}
\hat{\bar{\tau}}\left(\mathbf{Z}, \mathbf{Y}, \mathbf{R}\right) & = \left(\frac{1}{\sum \limits_{i = 1}^n Z_i R_i}\right) \sum \limits_{i = 1}^n Z_i R_i Y_i - \left(\frac{1}{\sum \limits_{i = 1}^n \left(1 - Z_i\right)R_i}\right) \sum \limits_{i = 1}^n \left(1 - Z_i\right) R_i Y_i.
\end{align*}

If \(r_{TI} = 1\) and \(r_{Ci} = 1\) for all
\(i \in \left\{1, \dots , n\right\}\) units, then the
Difference-in-Means estimator is equivalent to how it is usually written
as \begin{align*}
\hat{\bar{\tau}}\left(\mathbf{Z}, \mathbf{Y}\right) & = \left(\frac{1}{\sum \limits_{i = 1}^n Z_i}\right) \sum \limits_{i = 1}^n Z_i Y_i - \left(\frac{1}{\sum \limits_{i = 1}^n \left(1 - Z_i\right)}\right) \sum \limits_{i = 1}^n \left(1 - Z_i\right) Y_i.
\end{align*}

However, when attrition does exist, it is useful to decompose the
average causal effect as
\(\delta_{AR}\pi_{AR} + \delta_{ITR} \pi_{ITR} + \delta_{IUR}\pi_{IUR} + \delta_{NR} \pi_{NR}\).

Note that
\(\cfrac{1}{\E\left[\sum\limits_{i = 1}^n Z_i R_i\right]} = \cfrac{n}{n_1 R^T}\)
and
\(\cfrac{1}{\E\left[\sum\limits_{i = 1}^n \left(1 - Z_i\right) R_i\right]} = \cfrac{n}{n_0 R^C}\),
where \(R^T = \sum \limits_{i = 1}^n r_{Ti}\) and
\(R^C = \sum \limits_{i = 1}^n r_{Ci}\).

We can also note that

\section*{Question 5}

To answer this question, first note that
\(\bar{\tau}_D = \bar{d}_T - \bar{d}_C\) is equivalent to the proportion
of Compliers, \(\pi^C\), minus the proportion of Defiers, \(\pi^D\). To
see this point, note that \(d_c \in \left\{0, 1\right\}\) and
\(d_t \in \left\{0, 1\right\}\). The individual causal effect,
\(\tau_{Di}\), is \(1\) for Compliers, \(-1\) for Defiers and \(0\) for
both Always-Takers and Never-Takers. Hence,
\(\sum \limits_{i = 1}^n \tau_{Di}\) is the number of Compliers, \(n^C\)
minus the number of Defiers, \(n^D\). The average causal effect of \(Z\)
on \(D\) is \begin{align*}
\bar{\tau}_D & = \left(\frac{1}{n}\right)\sum \limits_{i = 1}^n \left(d_{Ti} - d_{Ci}\right) \\
& = \left(\frac{1}{n}\right) \sum \limits_{i = 1}^n \tau_{Di} \\
& = \left(\frac{1}{n}\right)\left(n^C - n^D\right) \\ 
& = \frac{n^C}{n} - \frac{n^D}{n}\\ 
& = \pi^C - \pi^D
\end{align*}

The strength of an instrument is defined in multiple ways. One
definition states that an instrument (or encouragement) is weak when it
has a small impact on whether or not units in a study actually receive
treatment. In order for the ratio \(\cfrac{\bar{\tau}_Y}{\bar{\tau}_D}\)
to be defined it must be the case that \(\bar{\tau}_D > 0\), although we
would ideally like \(\bar{\tau}_D\) to be much larger than \(0\).
Without assuming no Defiers, i.e., that \$pi\^{}D = \$, the required
condition that \(\bar{\tau}_D > 0\) requires only that the proportion of
Compliers is greater than the proportion of Defiers, \(\pi^C > \pi^D\).

In this class, we have defined the strength of an instrument as
equivalent to the proportion of Compliers, \(\pi^C\). The quantity of
interest is \(\cfrac{\bar{\tau}_Y}{\pi^C}\) and, hence, it must be the
case that \(\pi^C > 0\) in order for this quantity of interest to be
defined. However, defining the strength of instrument as the size of
\(\bar{\tau}_D\) or as the size of \(\pi^C\) is equivalent when there
are no Defiers, i.e., \(\pi^D = 0\).

We cannot actually know the values of \(\bar{\tau}_D\) and \(\pi^C\).
However, the estimator \(\widehat{\text{ITT}}_D\) is unbiased and
consistent for \(\bar{\tau}_D\), where \(\bar{\tau}_D = \pi^C\) when
\(\pi^D = 0\). Written mathematically, unbiasedness and consistency in
this context mean that
\(\E\left[\widehat{\text{ITT}}_D\right] = \bar{\tau}_D\) and
\(\widehat{\text{ITT}}_D \overset{p}{\to} \bar{\tau}_D\).

The estimate of \(\bar{\tau}_D\) in Study 1 (PBS,
\textit{Moyers on Addiction: Close to Home}) is
\(\frac{244}{510} - \frac{74}{579} \approx 0.3506\) and in Study 2
(\textit{Fox News}, ``Channel 11 Special on Proposition 209'') is
\(\frac{117}{259} - \frac{11}{248} \approx 0.4074\).

In the \citet{arceneaux2005} study, we can calculate the proportion of
people in each precinct who were successfully contacted by multiplying
the size of each precinct by the proportion of people contacted.

\scriptsize

\begin{Shaded}
\begin{Highlighting}[]
\NormalTok{acorn\_data \textless{}{-}}\StringTok{ }\KeywordTok{read.csv}\NormalTok{(}\StringTok{"acorn03.csv"}\NormalTok{)}

\NormalTok{n\_contact \textless{}{-}}\StringTok{ }\NormalTok{acorn\_data}\OperatorTok{$}\NormalTok{size }\OperatorTok{*}\StringTok{ }\NormalTok{acorn\_data}\OperatorTok{$}\NormalTok{contact}
\end{Highlighting}
\end{Shaded}

\normalsize

Then we can calculate the proportion of people successfully contacted
among all people in treated precincts and subtract from this value the
proportion of people successfully contacted among all people in control
precincts.

\scriptsize

\begin{Shaded}
\begin{Highlighting}[]
\NormalTok{n\_c \textless{}{-}}\StringTok{ }\KeywordTok{t}\NormalTok{(}\DecValTok{1} \OperatorTok{{-}}\StringTok{ }\NormalTok{acorn\_data}\OperatorTok{$}\NormalTok{z) }\OperatorTok{\%*\%}\StringTok{ }\NormalTok{acorn\_data}\OperatorTok{$}\NormalTok{size}
\NormalTok{n\_t \textless{}{-}}\StringTok{ }\KeywordTok{t}\NormalTok{(acorn\_data}\OperatorTok{$}\NormalTok{z) }\OperatorTok{\%*\%}\StringTok{ }\NormalTok{acorn\_data}\OperatorTok{$}\NormalTok{size}

\KeywordTok{t}\NormalTok{(acorn\_data}\OperatorTok{$}\NormalTok{z) }\OperatorTok{\%*\%}\StringTok{ }\NormalTok{(n\_contact)}\OperatorTok{/}\NormalTok{n\_t }\OperatorTok{{-}}\StringTok{ }\KeywordTok{t}\NormalTok{(}\DecValTok{1} \OperatorTok{{-}}\StringTok{ }\NormalTok{acorn\_data}\OperatorTok{$}\NormalTok{z) }\OperatorTok{\%*\%}\StringTok{ }\NormalTok{(n\_contact)}\OperatorTok{/}\NormalTok{n\_c}
\end{Highlighting}
\end{Shaded}

\begin{verbatim}
          [,1]
[1,] 0.6274073
\end{verbatim}

\normalsize

The estimate of the strength of the instrument in the Acorn study
\citep{arceneaux2005} is
\(\frac{3095}{4933} - \frac{0}{4779} \approx 0.6274\), which is larger
than the estimates of instrument strength in the other two studies.

Therefore, our best guess of which study has the weakest instrument is
Study 1 (PBS, \textit{Moyers on Addiction: Close to Home}) in which we
estimate \(\bar{\tau}_D\) to be roughly \(0.3506\).

\newpage

\renewcommand\refname{References}
  \bibliography{Bibliography.bib}

\end{document}

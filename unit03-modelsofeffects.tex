%  For slides only
%\documentclass{beamer} 
\newcommand{\tcw}{\textcolor{black}}
\newcommand{\mynoteonly}{}
\newcommand{\nottheirhandout}{handout:0}

% % For handout
%% Usage:
%% $ cat handout.tex unitXX-foo.tex | pdflatex
%% $ pdfjam -o unitXX-handout{a/b/...}.pdf texput.pdf '{1stpage}-{lastpage}'
\documentclass[handout]{beamer}
\usepackage{pgfpages}
\pgfpagesuselayout{4 on 1}[letterpaper,border shrink=5mm]%\nofiles

\mode<handout>{\setbeamercolor{background canvas}{bg=black!5}}
\newcommand{\tcw}{\textcolor{structure.bg}}
\newcommand{\mynoteonly}{| handout:0}
\newcommand{\nottheirhandout}{handout:0}

%For handout + mynotes
%%% Usage:
%% $ cat handout+mynotes.tex unitXX-foo.tex | pdflatex
%% $ pdfjam --nup 1x4 -o unitXX-foo-wmn{a/b/...}.pdf texput.pdf '1,{1stpage}-{lastpage}'
\documentclass[handout]{beamer}
\usepackage{pgfpages}
%%\pgfpagesuselayout{2 on 1}[letterpaper,border shrink=5mm] 
\setbeameroption{show notes on second screen=left}
\newcommand{\tcw}{\textcolor{black}}
\newcommand{\mynoteonly}{}
\newcommand{\nottheirhandout}{}

\newcommand{\igrphx}[2][width=\linewidth]{\includegraphics[#1]{images/#2}}

\renewcommand{\strut}{\rule{0pt}{3ex}}

% Frame note commands, with time budget updating
\newcounter{timeTotal}
\newcommand{\printUpdateTimeTotal}[1]{ \addtocounter{timeTotal}{#1}
  \textrm{(#1 min budgeted)} \hfill \textbf{Finish by
    \arabic{timeTotal}\ min from start}\\[1ex]}
\newcommand{\tnote}[3][10]{#2 \note{#2: #3 \\[.5ex]} \printUpdateTimeTotal{#1}}
\newcommand{\itnote}[2][10]{\note{ \begin{itemize} #2 \end{itemize}
\vfill \noindent  \printUpdateTimeTotal{#1} }}
\newcommand{\ennote}[2][10]{\note{ \begin{enumerate} #2 \end{enumerate}
\vfill \noindent  \printUpdateTimeTotal{#1} }}
\newcommand{\Note}[2][10]{\note{ #2 \mbox{ }\\ \vfill \noindent  \printUpdateTimeTotal{#1} }}

\newenvironment{Column}[1][.5\linewidth]{\begin{column}{#1}}{\end{column}}

\mode<handout>{\beamertemplatesolidbackgroundcolor{black!5}} 
\mode<article>{\usepackage{fullpage}}
\mode<presentation>
{
  \usetheme{Boadilla}
  % or ...

  \setbeamercovered{transparent}
  % or whatever (possibly just delete it)
}
\usepackage{url}
\usepackage{ulem}

\usepackage[english]{babel}
% or whatever

\usepackage[latin1]{inputenc}
% or whatever

\usepackage{times}
\usepackage[T1]{fontenc}
% Or whatever. Note that the encoding and the font should match. If T1
% does not look nice, try deleting the line with the fontenc.

\AtBeginSubsection[]
{
  \begin{frame}<beamer>
    \frametitle{Outline}
    \tableofcontents[subsectionstyle=show/shaded/hide]
  \end{frame}
}

\AtBeginSection[]
{
  \begin{frame}<beamer>
    \frametitle{Outline}
    \tableofcontents[hideothersubsections,sectionstyle=show/shaded]
  \end{frame}
}


% If you wish to uncover everything in a step-wise fashion, uncomment
% the following command: 

%\beamerdefaultoverlayspecification{<+->}

%USEFUL CODE TEMPLATES:
%\begin{itemize}[<+-| alert@+>]


% \begin{frame}[fragile]


% \begin{columns}
% \column{.4\textwidth}  
%   \begin{itemize}
%   \item<1-| alert@1> Why sample twice?
%   \item<2-| alert@2> Illustrative example.
%   \item<3-| alert@3> PSU's and SSU's.
%   \item<4-| alert@4> Stratification in combination with two-stage cluster sampling.
%   \item<5-| alert@5> Detailed example.
%   \end{itemize}
% \column{.6\textwidth}
% \only<2| handout:0>{
% \includegraphics[width=\textwidth]{nursingHomes}}
% \only<5| handout:1>{
% \includegraphics[width=\textwidth]{hosp_cover}}
% \end{columns}

% FOR INCLUDING R CODE. 
%\usepackage{listings}
%\lstset{language=R}
% \begin{frame}[fragile]
% \frametitle{Some code}
% \begin{lstlisting}
% > plot(myobj)
% > rm(myobj)
% \end{lstlisting}  
% \end{frame}

%\addtocounter{framenumber}{-1}

\author{Stats 617}
\date{Winter 2016}

\usepackage{amsmath,amsthm}
\usepackage{wasysym,pifont}
\usepackage{Sweave}
\usepackage{ulem}
\usepackage{textcomp}
\usepackage{versions}
%% amsthm-type theorem environment specifications -- 
%% see amsthdoc.pdf in amscls documentation
\theoremstyle{plain}
\newtheorem{prop}{Proposition}[section]
\newtheorem{lem}[prop]{Lemma}

\newtheorem*{thm}{Proposition}
\newtheorem*{cor}{Corollary}

\theoremstyle{definition}
\newtheorem{defn}{Definition}[section]

\newcommand{\Pdistsym}{P}
\newcommand{\Pdistsymn}{P_n}
\newcommand{\Qdistsym}{Q}
\newcommand{\Qdistsymn}{Q_n}
\newcommand{\Qdistsymni}{Q_{n_i}}
\newcommand{\Qdistsymt}{Q[t]}
\newcommand{\dQdP}{\ensuremath{\frac{dQ}{dP}}}
\newcommand{\dQdPn}{\ensuremath{\frac{dQ_{n}}{dP_{n}}}}
\newcommand{\EE}{\ensuremath{\mathbf{E}}}
\newcommand{\EEp}{\ensuremath{\mathbf{E}_{P}}}
\newcommand{\EEpn}{\ensuremath{\mathbf{E}_{P_{n}}}}
\newcommand{\EEq}{\ensuremath{\mathbf{E}_{Q}}}
\newcommand{\EEqn}{\ensuremath{\mathbf{E}_{Q_{n}}}}
\newcommand{\EEqni}{\ensuremath{\mathbf{E}_{Q_{n[i]}}}}
\newcommand{\EEqt}{\ensuremath{\mathbf{E}_{Q[t]}}}
\newcommand{\PP}{\ensuremath{\mathbf{Pr}}}
\newcommand{\PPp}{\ensuremath{\mathbf{Pr}_{P}}}
\newcommand{\PPpn}{\ensuremath{\mathbf{Pr}_{P_{n}}}}
\newcommand{\PPq}{\ensuremath{\mathbf{Pr}_{Q}}}
\newcommand{\PPqn}{\ensuremath{\mathbf{Pr}_{Q_{n}}}}
\newcommand{\PPqt}{\ensuremath{\mathbf{Pr}_{Q[t]}}}
\newcommand{\var}{\ensuremath{\mathbf{V}}}
\newcommand{\varp}{\ensuremath{\mathbf{V}_{P}}}
\newcommand{\varpn}{\ensuremath{\mathbf{V}_{P_{n}}}}
\newcommand{\varq}{\ensuremath{\mathbf{V}_{Q}}}
\newcommand{\cov}{\ensuremath{\mathbf{Cov}}}
\newcommand{\covp}{\ensuremath{\mathbf{Cov}_{P}}}
\newcommand{\covpn}{\ensuremath{\mathbf{Cov}_{P_{n}}}}
\newcommand{\covq}{\ensuremath{\mathbf{Cov}_{Q}}}

\newcommand{\hatvar}{\ensuremath{\widehat{\mathrm{Var}}}}
\newcommand{\hatcov}{\ensuremath{\widehat{\mathrm{Cov}}}}

\newcommand{\sehat}{\ensuremath{\widehat{\mathrm{se}}}}

\newcommand{\combdiff}[1]{\ensuremath{\Delta_{{z}}[#1]}}
\newcommand{\Combdiff}[1]{\ensuremath{\Delta_{{Z}}[#1]}}

\newcommand{\psvec}{\ensuremath{\varphi}}
\newcommand{\psvecgc}{\ensuremath{\tilde{\varphi}}}


\newcommand{\atob}[2]{\ensuremath{#1\!\! :\!\! #2}}
\newcommand{\stratA}{\ensuremath{\mathbf{S}}}
\newcommand{\stratAnumstrat}{\ensuremath{S}}
\newcommand{\sAsi}{\ensuremath{s}}

\newcommand{\permsd}{\ensuremath{\sigma_{\Pdistsym}}}
% \newcommand{\dz}[1]{\ensuremath{d_{z}[{#1}]}}
\newcommand{\dZ}[1]{\ensuremath{d_{Z}[{#1}]}}
\newcommand{\tz}[1]{\ensuremath{t_{{z}}[#1]}}
\newcommand{\tZ}[1]{\ensuremath{t_{{Z}}[#1]}}


\newlength{\tabcolsepadj}
\setlength{\tabcolsepadj}{1.3mm}

%%% NEWBLOCK UNDEFINED BUG
\def\newblock{\hskip .11em plus .33em minus .07em}

%%% tightlist undefined control sequence bug 
%%% http://tex.stackexchange.com/questions/257418/error-tightlist-converting-md-file-into-pdf-using-pandoc
\providecommand{\tightlist}{%
  \setlength{\itemsep}{0pt}\setlength{\parskip}{0pt}}
\newcommand{\igrphx}[2][width=\linewidth]{\includegraphics[#1]{images/#2}}

\renewcommand{\strut}{\rule{0pt}{3ex}}

\newenvironment{Column}[1][.5\linewidth]{\begin{column}{#1}}{\end{column}}

\usetikzlibrary{arrows} % also see \tikzstyle spec below after \begin{doc}

\newcommand{\mlpnode}[1]{\tikz[baseline=-.5ex] \coordinate (#1) {};}
%\newcommand{\mlpnode}[1]{\raisebox{.5ex}{\pnode{#1}}}


\usepackage{xspace}

\includeversion{pedantic} % \excludeversion

\tikzstyle{every picture}+=[remember picture]



\title{Unit 3: Fisherian randomization inference and models of effects}
% \author, date moved to beamer-preamble-*-all.tex

\begin{document}


  \begin{frame}
    \frametitle{Outline \& Readings}

\tableofcontents[subsectionstyle=show/hide/hide]

  \alert{Readings for this Unit:}\\
  \begin{itemize}
  \item Bowers, Fredrickson and Panagopoulos (2013), \S1-4. 
  \item Rosenbaum's (1996) comment on Angrist, Imbens and Rubin's (1996) ``Identification of causal effects using instrumental variables.'' (\texttt{Resources/Papers/rosenbaum1997\{AIRcomment\}.pdf})
  \end{itemize}

\input{announcement-of-the-day}  % announcement-of-the-day.tex not
                                % part of repo

\end{frame}

\section{Tests for normal, not Normal, data distributions}

\begin{frame}[<+->]{Review: permutation tests vs parametric tests}

When comparing 2 or more groups, one can often choose between permutation tests and parametric tests.
\begin{itemize}
\item both begin by computing a test statistic
\item they differ in terms of what they compare the statistic to in order to get a p-value
\item parametric tests lean on stronger assumptions
\item permutation tests can use a model to ``motivate'' a test statistic, then compare that test statistic to a permutation-based reference distribution in order committing to the model's assumptions.
\item likewise, the permutational approach circumvents sample-size requirements for p-values' validity.
\end{itemize}

Software notes:
\begin{itemize}
\item In unit 2, $z$'s were permuted using \texttt{sample(z)}.
\item \texttt{sample()} is part of base R, but ``\textrm{shuffle}'' would be more intuitive.
\item The ``mosaic'' add-on package provides a \texttt{shuffle()} command. 
\end{itemize}

\end{frame}

\subsection{Unstratified case}
\begin{frame}{Test statistics for comparing means with Normal \& non-Normal data}

Stepping back momentarily from issues specific to causal inference, consider the one-way model
$$
Y = \mu + Z\tau + \epsilon,\, \mathbf{E}(\epsilon) = 0.
$$

Suppose the goal is to test the hypothesis that $\tau = 0$. 
  
\begin{itemize}[<+->]
\item If $\epsilon \sim \mathcal{N}(0, \sigma^{2})$, some $\sigma >0$, regardless of whether $Z=1$ or 0, the difference of means is optimal, in various senses, as an estimator of $\tau$. By extension, using it as a test statistic gives favorable power.
\item True both for parametric tests (t-test) and permutation tests. 
\item If $\epsilon$ follows some other distribution, or isn't the same for the two groups, the situation changes.
\item Parametric $t$-tests may have sizes less than their levels. Permutation $t$-tests will maintain size.
\item On the other hand, \textit{power} suffers.
\item A simple explanation: the difference of means reacts so strongly to outliers as to tend to obscure differences due to a treatment effect.
\end{itemize}
\end{frame}

\begin{frame}{M-estimation using Huber's loss/psi function} \framesubtitle{as a remedy for the difference in means' hypersensitivity}
  \begin{columns}
    \begin{Column}
          \begin{center}
         {\usebeamercolor[fg]{titlelike} Huber vs squared-error loss} 
    \end{center}
      \igrphx{huber-loss-fct}
    \end{Column}
    \begin{Column}
          \begin{center}a
         {\usebeamercolor[fg]{titlelike} Huber's psi function} 
    \end{center}

\igrphx{Hubers-Psi-Graph}
    \end{Column}
  \end{columns}
\pause
In practice, Huber's M-estimation solves a regression problem in which
the residuals have been top- and bottom-coded, $e \mapsto \max(-t,
\min(t,e))$ at a data-dependent threshold $t>0$.
\end{frame}
\Note{

Tech note: specificially, just as OLS produces residuals that sum to 0
and are uncorrelated with each column of $\mathbf{x}$, the Huber
M-estimate gives residuals which sum to 0 and are uncorrelated with
the xes, \textit{after} the top and bottom coding.
}
\begin{frame}{Rank-based tests } \framesubtitle{as a remedy for the difference in means' hypersensitivity}
  
\end{frame}
\subsection{Adaptations for stratified data}

\begin{frame}{For permutation tests of block-randomized experiments}
  
  \begin{enumerate}
  \item If you're using a regression model to generate test statistics, consider adding fixed effects for blocks. (Not necessary, but consider it.)
  \item  You \textit{have to} permute $Z$ separately within strata -- i.e.,\\ 
\texttt{> shuffle(z, groups=myblocks)}\\
in order to have a  valid permutation test. True whether or not $\mathrm{Pr}(Z=1|S=s)$ depends on $s$, although more variation in that probability makes things worse.
  \end{enumerate}




\end{frame}

\begin{frame}{Test statistics for comparing means with Normal \& non-Normal data}

If treatment was assigned separately within strata, a two-way anova model is often used:
$$
Y = \sum_{s}\{S=s\}\alpha_{s} + Z\tau + \epsilon,\, \mathbf{E}(\epsilon) = 0.
$$  
I.e., the model adds per-stratum ``fixed effects''.  Similar considerations as above apply. OLS estimates of $\tau$:
\begin{itemize}[<+->]
  \item Are optimal, both as estimates and as statistics for testing hypotheses about $\tau$, if $\epsilon \sim \mathcal{N}(0, \sigma^{2})$;
  \item Need not have the right size, and typically will have suboptimal power, otherwise.
  \item Permutation-based versions of these tests have correct size\ldots
  \item But still suffer in terms of power.
\end{itemize}

\end{frame}

\begin{frame}{Adapting Huber's M-estimation to tests w/ stratified data}


      Two approaches:
      \begin{enumerate}
      \item using Huber's loss, fit a two-way model  --- 
$$
Y = \sum_{s}\{S=s\}\alpha_{s} + Z\tau + \epsilon,\, \mathbf{E}(\epsilon) = 0.
$$  
The test statistic is the $\hat{\tau}$ that results.
\item using Huber's loss, fit a one-way model \textit{with no $Z$-contribution},
$$
Y = \sum_{s}\{S=s\}\alpha_{s} + \epsilon,\, \mathbf{E}(\epsilon) = 0.
$$  
Write residuals from this model as $e$, $\tilde{e} = \psi(e)$.  The test statistic is the difference-of-means in $\psi(e)$ (or, equivalently, $Z'\tilde{e}$).
      \end{enumerate}



  
\end{frame}
\begin{frame}{Example: violence and public infrastructure in Medellin, Colombia}
  
\frametitle{Example: Violence and public infrastructure in Medell{\'i}n, Colombi
a}
\begin{columns}
  \begin{column}{.5\linewidth}
    \begin{itemize}
    \item 2 million residents; 16 districts
    \item Pre-intervention: 60\% poverty rate, 20\% unemployment, homicide 185 p
er 100K
    \item High residential segregation
    \item<2-> 2004-2006: infrastructure intervention for certain poor neighborho
ods.
    \end{itemize}
  \end{column}
  \begin{column}{.5\linewidth}
    \only<1>{\igrphx{medellin-conc-pov}}
\only<2-\mynoteonly>{\igrphx{medellin-gondola}}
  \end{column}
\end{columns}

\end{frame}
\Note{(strata will enter the story because the end analysis involved matching)}



\section{Models of effects}
\begin{frame}{A simple model of effects for the Acorn GOTV experiment}
  
One simple model is that the GOTV campaign increases voter turnout by $p$ percentage points per precinct.

\begin{center}
  \begin{tabular}{r|rr|rr}
  \hline
 & GOTV? & vote03(\%)& $y_c$ & $y_t$ \\ 
  \hline
1 & 0 & 38 & 38 & $38+p$\\
$\vdots$& & & & \\
13 & 0 & 19 & 19& $19+p$\\ 
14 & 0 & 34 & 34& $34+p$\\ 
15 & 1 & 49 & $49-p$& 49\\ 
16 & 1 & 38 & $38-p$& 38\\ 
$\vdots$& & & & \\
28 & 1 & 29 & $29-p$& 29\\
   \hline
\end{tabular}

\end{center}

This family of hypotheses, as $p$ varies between $-100$ and $100$, can be called a \textit{model of effects}.
\end{frame}

\begin{frame}{A more nuanced model for the Acorn GOTV experiment}

Recall that a \textit{response schedule} is a complete specification of unit-level responses to the experiment under every possible random assigment. 

It's often more natural to hypothesize only about how treated units would have responded to control, not also how controls would have responded to treatment. For example, here are 3 competing models of the effects of the Acorn GOTV campaign: 
\begin{itemize}
\item[No effect] says there was no effect (\texttt{RS0})
\item[one per 10] says the GOTV campaign generated 1 vote for every 10 contacts (\texttt{RS1})
\item[one per 5]says the GOTV campaign generated 1 vote for every 5 contacts (\texttt{RS2})
\end{itemize}

\pause
Note that these models don't specify $y_{t}$ for precincts we only got to observe under control, $z=0$ --- they leave ``blanks'' in the potential schedule.  That's OK.

\end{frame}



\begin{frame}[fragile]{A model of effects-type analysis for Medellin homicide rate}

To summarize neighborhood-level homicide rates' departures from 
 trend, over a 4 year period, for each model of effects I fit a random
 effects model of this form:

\begin{verbatim}
 lmer(Count~ yearsPreFrom02 + yearspost+ + moe.offset
             + (yearsPreFrom02 + yearspost+1|nh),
     family=poisson, data=homd)
\end{verbatim}

The \texttt{moe.offset} term was created from a treatment effect on
the log rate scale, \texttt{t.e.}, as follows:
\begin{verbatim}
data <- transform(data, oset=(t.e./4)*(Group=="T")*yearspost)
\end{verbatim}
  
\end{frame}

\begin{frame}<\nottheirhandout>{Outcome analysis for the Medellin study}
  
  \begin{center}
    \igrphx[width=\linewidth]{RIoutcomeAnalysisBarchart}
  \end{center}

\end{frame}


\section{Inference w/ general models of effects}
\begin{frame}{Incomplete response schedules and statistical testing}
   An incomplete response schedule can still give us enough to do statistical testing. For example, if the schedule specifies $y_c$ for all subjects, then we can compare the difference in means in $y_c$ under the realized assigment and all possible other assigments.  

Using this test statistic for a two-sided test gives approximately

\begin{center}
\begin{tabular}{l|rrr} \hline
& \multicolumn{3}{c}{votes/ 10 contacts}\\ 
  Hypothesis & 0 & 1 & 2 \\ \hline
$p$ (2-sided) & .15 & .38 & .003 \\ \hline 
\end{tabular}
\end{center}

(see \texttt{unit3-MOEs.pdf}).
\end{frame}

\begin{frame}{Confidence intervals and models of effects}

  \begin{itemize}[<+->]
  \item The 3 models just considered fall under the broader model that the GOTV  
generated $v$ votes per contact, some $v \in [-1, 1]$.
\item Assuming that model, the 95\% CI for $v$ is the collection of $v$s corresponding to incomplete response schedules that would not be rejected at level .05 - a confidence interval by inversion of a family of hypothesis tests.
\item If you want an estimate to go with such a confidence interval, the convention is to report a \textit{Hodges-Lehmann} estimate --- essentially, limit of 100*$(1-\alpha)$\% CIs, as $\alpha \downarrow 0$.
\item There's no precise analogue of the ``standard error.''
\item However, for some of the spirit of a ``standard error'', report a sometimes 2/3 CI alongside a 95\% CI, as suggested by Mosteller \& Tukey (1977, \textit{Data analysis and regression}).
\item We just backed out estimates of the ``Complier average treatment effect,'' and associated confidence intervals, w/o violating the intention-to-treat principle! (Rosenbaum, 1997).
  \end{itemize}


\end{frame}
\end{document}

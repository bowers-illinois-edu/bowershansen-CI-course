\documentclass{article}
\usepackage{natbib}

\title{Causal Inference Assignment 1}
\date{ICPSR Session 2 (July 19, 2021)}

\author{Due Wednesday 7/28}
\usepackage{../icpsr-classwork}


\begin{document}
\maketitle

\begin{enumerate}
%\item Can the simulation method approximate exact, permutation-based
%  calculation to as much accuracy as is practically meaningful?
%  State a limit on the absolute error of a p-value,
%  $|p_{\text{approx}} - p_{\text{exact}}|$, that seems reasonable to
%  you to tolerate; then use the formula presented in
%  \texttt{unit01-Rex.pdf} to determine how many simulation replicates
%  are need to obtain simulation $p$-values deviating from their exact
%  counterparts by no more than this amount.
%\item For the coffee experiment, does the chi-square procedure report
%  a similar p-value as the fisher test?  (In R,
%  it's \texttt{chisq.test()}. For documentation, enter \texttt{?chisq.test}.)
%\item For this data set, are the chi-square test and the fisher test
%  both admissible? (Hint: what are the sample size requirements of the
%  chi-square procedure? As review, read R's warnings and documentation.)
\item Refer to the the \texttt{acorn} data set that accompanies this assignment.  Using the mean of turnout proportions in
treatment group precincts, $n_{1}^{-1}\mathbf{Z}'\mathbf{y}$, as
test statistic, simulate its rerandomization distribution under the
null hypothesis of strictly no effect, reporting: \label{q:simmoments}
\begin{enumerate}
  \item your simulation $p$-value; 
  \item your simulation approximation of $\mathrm{E}[
    n_{1}^{-1}\mathbf{Z}'\mathbf{y}]$, the null expected value of the test
    statistic; 
  \item your simulation approximation of
    $\mathrm{Var}[n_{1}^{-1}\mathbf{Z}'\mathbf{y}]$, this test
    statistic's variance under the null.
  \end{enumerate}
 Note that $n_1$ is the number of treated units, $n$ is the total number of units, $\mathbf{Z}$ is the random assignment variable and $\mathbf{y}$ is the observed outcome variable
\item Calculate $\mathrm{E} [ n_{1}^{-1}\mathbf{Z}'\mathbf{y} ]$, the expected value of the
  sample mean under the strict null hypothesis of no effect, from first principles --- i.e., without simulations ---  using data in $\mathbf{y}$.  \label{q:exactEV}
\item In this setup, $\mathrm{Var}\, [\bar{y}_{1}]  = \frac{1}{n_{1}}
  \frac{n}{n} \frac{\sum_{i=1}^{n} (y_{i} - \bar y)^{2}}{n-1} $,
  where $\bar{y}_{1} = n_{1}^{-1}\mathbf{Z}'\mathbf{y}$ is the mean of
  $y$s among the group assigned to treatment ($\{i: Z_{i} =1\}$) while
  $\bar{y} = n^{-1}\mathbf{1}'\mathbf{y}$ is the mean of $y$ over
  the full study population.  If you've seen formulas for the sampling
  variance of the mean in earlier stats
  courses, you probably saw
  $$
  \widehat{\mathrm{Var}} \left[\bar{y}_{1}\right] = \frac{s^{2}_{y}}{n_{1}} = \frac{1}{n_{1}}
  \frac{\sum_{i=1}^{n_{1}} (y_{i} - \bar{y}_{1})^{2}}{n_{1}-1}, 
\, \mathrm{and/or}\,
  \mathrm{Var} \left[\bar{y}_{1}\right] = \frac{\sigma_{y}^{2}}{n_{1}} = \frac{1}{n_{1}}
  \frac{\sum_{i=1}^{n} (y_{i} - \bar{y})^{2}}{n}.
  $$
Which gives a larger value, our new $\mathrm{Var}\,
[\bar{y}_{1}]$-formula or the  $\mathrm{Var} \left[\bar{y}_{1}\right]$
formula immediately above?  Explain why it makes
sense that the formula we've given would differ in the direction it does
from this other formula. %, in terms of the difference between this
%sampling situation and the sampling model that more commonly
%accompanies the Central Limit Theorem.
\item Calculate $\mathrm{Var}\, [n_{1}^{-1}\mathbf{Z}'\mathbf{y}]$ using the appropriate
  formula.   Determine the error of the simulation-based
  approximation to this quantity that you reported in question~\ref{q:simmoments},
  expressing it as a percentage of $\mathrm{Var}\,[ n_{1}^{-1}\mathbf{Z}'\mathbf{y}]$. \label{q:exactvar}
\item Determine the Normal theory approximation to
  $\mathrm{Pr}(n_{1}^{-1}\mathbf{Z}'\mathbf{y} \geq
  n_{1}^{-1}\mathbf{z}'\mathbf{y} ) $.  (Hints: Use the variance and
  expected values calculated in~\ref{q:exactvar} and \ref{q:exactEV} to transform your
  observed treatment group mean into a corresponding ``$z$-score.'' 
  To to determine Normal quantiles corresponding to z-scores in R, use
  \texttt{pnorm()}; type \texttt{?pnorm} for help. And $\mathbf{z}'$ refers to
  the observed Acorn data treatment assignment.)
\item %
A researcher plans to ask six subjects to donate time to an adult
literacy program. Each subject will be asked to donate either 30
($Z=0$) or 60 ($Z=1$)
minutes. The researcher is considering three methods for randomizing
the treatment. Method I is to make independent decisions for each
subject, tossing a coin each time. Method C is to
write ``30'' and ``60'' on three playing cards each, and then shuffle
the six cards. Method P tosses one coin for each of the 3 pairs
$(1,2)$, $(3,4)$, $(5,6)$, asking for 30 (60) minutes from exactly one
member of each pair. 
  
\begin{itemize}
\item[a] Discuss strengths \& weaknesses of each method.
\item[b] How would your answers to (a) change if $n: 6 \mapsto 600$?
\item[c] Determine $\mathrm{E}\left[  Z_{1} \right]$  under each method.
\item[d] Determine $\mathrm{E} \left[  Z_{1} + Z_{2} + \cdots + Z_{6} \right]$ under each method.
\end{itemize}

A researcher plans to ask six subjects to donate time to an adult
literacy program. Each subject will be asked to donate either 30
($Z=0$) or 60 ($Z=1$)
minutes. The researcher is considering three methods for randomizing
the treatment. Method I is to make independent decisions for each
subject, tossing a coin each time. Method C is to
write ``30'' and ``60'' on three playing cards each, and then shuffle
the six cards. Method P tosses one coin for each of the 3 pairs
$(1,2)$, $(3,4)$, $(5,6)$, asking for 30 (60) minutes from exactly one
member of each pair. 
  \begin{minipage}{.45\linewidth}
\begin{itemize}
\item[a] Discuss strengths \& weaknesses of each method.
\item[b] How would your answers to (a) change if $n: 6 \mapsto 600$?
\item[c] Determine $\EE[  Z_{1} ]$  under each method.
\item[d] Determine $\EE\big[  Z_{1} + Z_{2} + \cdots + Z_{6} \big]$ under each method.
\item[e] Calculate $\EE[\mathbf{Z}'\mathbf{1}]$ under each of the three methods.
\end{itemize}
\end{minipage}
% \input{gg15ii}
\begin{minipage}{.45\linewidth}
  \begin{itemize}
 \item[f] For which of the methods does $\EE
  \big[\big(\mathbf{Z}'\mathbf{1} -\EE [\mathbf{Z}'\mathbf{1}]\big)^{2}\big] =
  0$?\footnote{I.e., for which does
    $\mathrm{Var}[\mathbf{Z}'\mathbf{1}] = 0$?  (In general,
    $\mathrm{Var}[V] = \EE \big[V - \EE[V] \big]^{2} $.)}
\item[h] For two of the three methods, algebraic principles we've seen
  let you reduce
  $\EE [\frac{\mathbf{Z}'\mathbf{x}}{\mathbf{Z}'\mathbf{1}}]$ to a
  familiar function of $(x_{1}, x_{2}, \ldots, x_{6}) $.  Which 2 are
  these, and why doesn't the same thing work for the third?
\end{itemize}
\end{minipage}
\end{enumerate}
\end{document}

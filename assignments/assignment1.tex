\documentclass{article}
\usepackage{natbib}

\title{Causal Inference Assignment 1}
\date{ICPSR Session 2 (July 18, 2022)}

\author{Due Tuesday 8/1, 3pm}
\usepackage{../icpsr-classwork}


\begin{document}
\maketitle

\begin{enumerate}
\item Confirm to your own satisfaction that for analysis of the coffee experiment, the simulation method can approximate the exact, permutation-based calculation to as much accuracy as is practically meaningful.
\item Does the chi-square procedure report a similar p-value?  (In R,
  it's \texttt{chisq.test()}. For documentation, enter \texttt{?chisq.test}.)
\item For this data set, are the chi-square test and the fisher test
  both admissible? (Hint: read R's warnings and documentation.  Can
  you remember the conditions for the chi-square test for two-way
  tables, from a previous stats class?)
\item Use your simulation to approximate $\mathrm{E} \left[n_{1}^{-1}\mathbf{Z}'\mathbf{y}\right] $ and
  $\mathrm{Var}\left[n_{1}^{-1}\mathbf{Z}'\mathbf{y}\right] $. 
\item Calculate $\mathrm{E}  \left[n_{1}^{-1}\mathbf{Z}'\mathbf{y}\right] $, the exact expected value of the
  sample mean, from first principles.  
 %% Should one state, ``calculate ... , the expected value of the sample mean under the strict null hypothesis of no effect, from first principles''? Or do you want students to reason that, since the y vector is lowercase (and hence fixed), we are implictly invoking Fisher's null?
\item In this setup, $\mathrm{Var}\, \left[n_{1}^{-1}\mathbf{Z}'\mathbf{y}\right]  = n_{1}^{-1}
  \frac{n_{0}}{n} \frac{\sum_{i=1}^{n} (y_{i} - \bar y)^{2}}{n-1} $.
  If you've seen formulas for the sample variance in earlier stats
  courses, you probably saw
  $$
  \mathrm{Var} \left[\bar{y}\right] = \frac{\sigma_{y}^{2}}{n_{1}} = \frac{1}{n_{1}}
  \frac{\sum_{i=1}^{n} (y_{i} - \bar y)^{2}}{n} .
  $$
Which of the two formulas gives a larger value?  Explain why it makes
sense that the formula we've given would differ the direction it does
from this other formula, in terms of the difference between this
sampling situation and the sampling model that more commonly
accompanies the Central Limit Theorem.
\item Calculate $ \mathrm{Var}\, \left[n_{1}^{-1}\mathbf{Z}'\mathbf{y}\right] $ using the appropriate
  formula. 
\item Determine the normal theory approximation to
  $\mathrm{Pr}_{0}(n_{1}^{-1}\mathbf{Z}'\mathbf{y} =1 ) $.  (To do this in R, use
  \texttt{pnorm()}; type \texttt{?pnorm} for help. It's a good idea to
  check your answer against some 
  more user-friendly Normal table; you'll find one of these quickly
  using any search engine.)
\item

A researcher plans to ask six subjects to donate time to an adult
literacy program. Each subject will be asked to donate either 30
($Z=0$) or 60 ($Z=1$)
minutes. The researcher is considering three methods for randomizing
the treatment. Method I is to make independent decisions for each
subject, tossing a coin each time. Method C is to
write ``30'' and ``60'' on three playing cards each, and then shuffle
the six cards. Method P tosses one coin for each of the 3 pairs
$(1,2)$, $(3,4)$, $(5,6)$, asking for 30 (60) minutes from exactly one
member of each pair. 
  
\begin{itemize}
\item[a] Discuss strengths \& weaknesses of each method.
\item[b] How would your answers to (a) change if $n: 6 \mapsto 600$?
\item[c] Determine $\mathrm{E}\left[  Z_{1} \right]$  under each method.
\item[d] Determine $\mathrm{E} \left[  Z_{1} + Z_{2} + \cdots + Z_{6} \right]$ under each method.
\end{itemize}

\item
A researcher plans to ask six subjects to donate time to an adult
literacy program. Each subject will be asked to donate either 30
($Z=0$) or 60 ($Z=1$)
minutes. The researcher is considering three methods for randomizing
the treatment. Method I is to make independent decisions for each
subject, tossing a coin each time. Method C is to
write ``30'' and ``60'' on three playing cards each, and then shuffle
the six cards. Method P tosses one coin for each of the 3 pairs
$(1,2)$, $(3,4)$, $(5,6)$, asking for 30 (60) minutes from exactly one
member of each pair. 

\begin{itemize}
\item[a] Calculate $\EE(\mathbf{Z}'\mathbf{Z})$ under each of the three methods.
\item[b] For which of the methods does $\EE
  \big[\mathbf{Z}'\mathbf{Z} -\EE (\mathbf{Z}'\mathbf{Z})\big]^{2} =
  0$?\footnote{I.e., for which does
    $\mathrm{Var}(\mathbf{Z}'\mathbf{Z}) = 0$?  (In general,
    $\mathrm{Var}(V) = \EE \big[V - \EE(V) \big]^{2} $.)}
\item[c] ``$\EE \frac{\mathbf{Z}'\mathbf{x}}{\mathbf{Z}'\mathbf{Z}}$''
  is another way of writing ``the treatment group
  \underline{\hspace{3em}} of $x$.''  Fill in the blank.
\item[d] For two of the three methods, algebraic principles we've seen
  let you can reduce
  $\EE \frac{\mathbf{Z}'\mathbf{x}}{\mathbf{Z}'\mathbf{Z}}$ to a
  familiar function of $(x_{1}, x_{2}, \ldots, x_{6}) $.  Which 2 are
  these, and why doesn't the same thing work for the third?
\end{itemize}

\end{enumerate}

\end{document}

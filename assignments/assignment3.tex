\documentclass{article}
\usepackage{natbib}

\title{Causal Inference Assignment 3}
\date{ICPSR Session 2 (July 19, 2021)}

\author{Due Friday July 7}
\usepackage{../icpsr-classwork}


\begin{document}
\maketitle

We challenge you to find an existing study (your own or one from someone else,
published or not) and to replicate one of the key analyses in that study using
tools learned in this class. For example, you might find a study using a linear
model for adjustment (i.e. "controlling for") and you might redo that analysis
using matching. And/or you might find a study where you could add or apply
permutation-based or randomization-justified testing (perhaps you are concerned
about the false positive error rate of a study with a weak instrument, a small
number of clusters, a skewed outcome, or a small sample size). If you would like
to do something other than matching and/or permutation-based testing, let us
know.

Data and/or study ideas: \begin{itemize}
 \item You could use your own data if you are replicating one of your own
      previous studies.
 \item You could search the \href{http://dataverse.org/}{Dataverse} or the
      \href{http://www.icpsr.umich.edu/icpsrweb/deposit/pra/index.jsp}{ICPSR
      Replication Archive} or the
      \href{http://www.3ieimpact.org/evaluation/impact-evaluation-replication-programme/}{3IE
      Replication program}
 \item Many journals now maintain or point to replication archives.
 \item Many scholars now maintain replication archives.
 \item You could work on the Medellin data.
\end{itemize}
\end{document}

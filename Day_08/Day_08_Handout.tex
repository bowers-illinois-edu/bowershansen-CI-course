\documentclass[12pt,leqno]{article}
\usepackage[left=1in,top=1in,right=1in,bottom=1in]{geometry}
\newcommand*{\authorfont}{\fontfamily{phv}\selectfont}
\usepackage{lmodern}


  \usepackage[T1]{fontenc}




\usepackage{abstract}
\renewcommand{\abstractname}{}    % clear the title
\renewcommand{\absnamepos}{empty} % originally center

\renewenvironment{abstract}
 {{%
    \setlength{\leftmargin}{0mm}
    \setlength{\rightmargin}{\leftmargin}%
  }%
  \relax}
 {\endlist}

\makeatletter
\def\@maketitle{%
  \newpage
%  \null
%  \vskip 2em%
%  \begin{center}%
  \let \footnote \thanks
    {\fontsize{18}{20}\selectfont\raggedright  \setlength{\parindent}{0pt} \@title \par}%
}
%\fi
\makeatother




\setcounter{secnumdepth}{0}

\usepackage{color}
\usepackage{fancyvrb}
\newcommand{\VerbBar}{|}
\newcommand{\VERB}{\Verb[commandchars=\\\{\}]}
\DefineVerbatimEnvironment{Highlighting}{Verbatim}{commandchars=\\\{\}}
% Add ',fontsize=\small' for more characters per line
\usepackage{framed}
\definecolor{shadecolor}{RGB}{248,248,248}
\newenvironment{Shaded}{\begin{snugshade}}{\end{snugshade}}
\newcommand{\AlertTok}[1]{\textcolor[rgb]{0.94,0.16,0.16}{#1}}
\newcommand{\AnnotationTok}[1]{\textcolor[rgb]{0.56,0.35,0.01}{\textbf{\textit{#1}}}}
\newcommand{\AttributeTok}[1]{\textcolor[rgb]{0.77,0.63,0.00}{#1}}
\newcommand{\BaseNTok}[1]{\textcolor[rgb]{0.00,0.00,0.81}{#1}}
\newcommand{\BuiltInTok}[1]{#1}
\newcommand{\CharTok}[1]{\textcolor[rgb]{0.31,0.60,0.02}{#1}}
\newcommand{\CommentTok}[1]{\textcolor[rgb]{0.56,0.35,0.01}{\textit{#1}}}
\newcommand{\CommentVarTok}[1]{\textcolor[rgb]{0.56,0.35,0.01}{\textbf{\textit{#1}}}}
\newcommand{\ConstantTok}[1]{\textcolor[rgb]{0.00,0.00,0.00}{#1}}
\newcommand{\ControlFlowTok}[1]{\textcolor[rgb]{0.13,0.29,0.53}{\textbf{#1}}}
\newcommand{\DataTypeTok}[1]{\textcolor[rgb]{0.13,0.29,0.53}{#1}}
\newcommand{\DecValTok}[1]{\textcolor[rgb]{0.00,0.00,0.81}{#1}}
\newcommand{\DocumentationTok}[1]{\textcolor[rgb]{0.56,0.35,0.01}{\textbf{\textit{#1}}}}
\newcommand{\ErrorTok}[1]{\textcolor[rgb]{0.64,0.00,0.00}{\textbf{#1}}}
\newcommand{\ExtensionTok}[1]{#1}
\newcommand{\FloatTok}[1]{\textcolor[rgb]{0.00,0.00,0.81}{#1}}
\newcommand{\FunctionTok}[1]{\textcolor[rgb]{0.00,0.00,0.00}{#1}}
\newcommand{\ImportTok}[1]{#1}
\newcommand{\InformationTok}[1]{\textcolor[rgb]{0.56,0.35,0.01}{\textbf{\textit{#1}}}}
\newcommand{\KeywordTok}[1]{\textcolor[rgb]{0.13,0.29,0.53}{\textbf{#1}}}
\newcommand{\NormalTok}[1]{#1}
\newcommand{\OperatorTok}[1]{\textcolor[rgb]{0.81,0.36,0.00}{\textbf{#1}}}
\newcommand{\OtherTok}[1]{\textcolor[rgb]{0.56,0.35,0.01}{#1}}
\newcommand{\PreprocessorTok}[1]{\textcolor[rgb]{0.56,0.35,0.01}{\textit{#1}}}
\newcommand{\RegionMarkerTok}[1]{#1}
\newcommand{\SpecialCharTok}[1]{\textcolor[rgb]{0.00,0.00,0.00}{#1}}
\newcommand{\SpecialStringTok}[1]{\textcolor[rgb]{0.31,0.60,0.02}{#1}}
\newcommand{\StringTok}[1]{\textcolor[rgb]{0.31,0.60,0.02}{#1}}
\newcommand{\VariableTok}[1]{\textcolor[rgb]{0.00,0.00,0.00}{#1}}
\newcommand{\VerbatimStringTok}[1]{\textcolor[rgb]{0.31,0.60,0.02}{#1}}
\newcommand{\WarningTok}[1]{\textcolor[rgb]{0.56,0.35,0.01}{\textbf{\textit{#1}}}}
\usepackage{longtable,booktabs}

\usepackage{graphicx}
% We will generate all images so they have a width \maxwidth. This means
% that they will get their normal width if they fit onto the page, but
% are scaled down if they would overflow the margins.
\makeatletter
\def\maxwidth{\ifdim\Gin@nat@width>\linewidth\linewidth
\else\Gin@nat@width\fi}
\makeatother
\let\Oldincludegraphics\includegraphics
\renewcommand{\includegraphics}[1]{\Oldincludegraphics[width=\maxwidth]{#1}}

\title{Day 08: Testing Causal Hypotheses with Instrumental Variables  }
 



\author{\Large \href{mailto:t.leavitt718@gmail.com}{Thomas Leavitt}\vspace{0.05in} \newline\normalsize\emph{}  }


\date{}

\usepackage{titlesec}

\titleformat*{\section}{\normalsize\bfseries}
\titleformat*{\subsection}{\normalsize\itshape}
\titleformat*{\subsubsection}{\normalsize\itshape}
\titleformat*{\paragraph}{\normalsize\itshape}
\titleformat*{\subparagraph}{\normalsize\itshape}


\usepackage{natbib}
\bibliographystyle{apsr}



\newtheorem{hypothesis}{Hypothesis}
\usepackage{setspace}

\makeatletter
\@ifpackageloaded{hyperref}{}{%
\ifxetex
  \usepackage[setpagesize=false, % page size defined by xetex
              unicode=false, % unicode breaks when used with xetex
              xetex]{hyperref}
\else
  \usepackage[unicode=true]{hyperref}
\fi
}
\@ifpackageloaded{xcolor}{
    \PassOptionsToPackage{usenames,dvipsnames}{xcolor}
}{%
    \usepackage[usenames,dvipsnames]{xcolor}
}
\makeatother
\hypersetup{breaklinks=true,
            bookmarks=true,
            pdfauthor={\href{mailto:t.leavitt718@gmail.com}{Thomas Leavitt} ()},
             pdfkeywords = {},  
            pdftitle={Day 08: Testing Causal Hypotheses with
Instrumental Variables},
            colorlinks=true,
            citecolor=blue,
            urlcolor=blue,
            linkcolor=magenta,
            pdfborder={0 0 0}}
\urlstyle{same}  % don't use monospace font for urls

% \usepackage{microtype} %
% \usepackage{setspace}
% \onehalfspacing
\usepackage{xcolor, color, ucs}     % http://ctan.org/pkg/xcolor
\usepackage{natbib}
\usepackage{booktabs}          % package for thick lines in tables
\usepackage{amsfonts,amsthm,amsmath,amssymb,bm}          % AMS Fonts
% \usepackage{empheq}            % To use left brace on {align} environment
\usepackage{graphicx}          % Insert .pdf, .eps or .png
\usepackage{enumitem}          % http://ctan.org/pkg/enumitem
% \usepackage[mathscr]{euscript}          % Font for right expectation sign
\usepackage{tabularx}          % Get scale boxes for tables
\usepackage{float}             % Force floats around
\usepackage{rotating}          % Rotate long tables horizontally
\usepackage{bbm}                % for bold betas
\usepackage{csquotes}           % \enquote{} and \textquote[][]{} environments
\usepackage{subfigure}
\usepackage{array}
% \usepackage{cancel}
\usepackage{longtable}
% % \usepackage{lmodern}
% % \usepackage{libertine} \usepackage[libertine]{newtxmath}
% \usepackage{stix}
% % \usepackage[osf,sc]{mathpazo}     % alternative math
% \usepackage[T1]{fontenc}
% \usepackage{fontspec}
% \setmainfont{Times New Roman}
% \usepackage{mathtools}          % multlined environment with size option
% \usepackage{verbatim}
% \usepackage{geometry}
% \usepackage{bigfoot}
% \geometry{verbose,margin=.8in,nomarginpar}
% \setcounter{secnumdepth}{2}
% \setcounter{tocdepth}{2}
% \usepackage{lscape}

\setlist{nosep}


% \usepackage{url}
% \usepackage[nobreak=true]{mdframed} % put box around section with \begin{mdframed}\end{mdframed}

% \usepackage{relsize}            % \mathlarger{} environment
% \usepackage[unicode=true,
%             pdfusetitle,
%             bookmarks=true,
%             bookmarksnumbered=true,
%             bookmarksopen=true,
%             bookmarksopenlevel=2,
%             breaklinks=false,
%             pdfborder={0 0 1},
%             backref=false,
%             colorlinks=true,
%             hypertexnames=false]{hyperref}
% \hypersetup{pdfstartview={XYZ null null 1},
%             citecolor=blue!50,
%             linkcolor=red,
%             urlcolor=green!70!black}

% \usepackage{multirow}
% \usepackage{tikz}
% \usetikzlibrary{trees, positioning, arrows, automata}

% \tikzset{
%   treenode/.style = {align=center, inner sep=0pt, text centered,
%     font=\sffamily},
%   arn_n/.style = {treenode, rectangle, black, fill=white, text width=6em},
%   arn_r/.style = {treenode, circle, red, draw=red, text width=1.5em, thick}
% }

\usepackage[noabbrev]{cleveref} % Should be loaded after \usepackage{hyperref}
\usepackage[small,bf]{caption}  % Captions

% \usepackage[obeyFinal,textwidth=0.8in, colorinlistoftodos,prependcaption,textsize=tiny]{todonotes} % \fxnote*[options]{note}{text} to make sticky notes
% \usepackage{xargs}
% \newcommandx{\unsure}[2][1=]{\todo[linecolor=red,backgroundcolor=red!25,bordercolor=red,#1]{#2}}
% \newcommandx{\change}[2][1=]{\todo[linecolor=blue,backgroundcolor=blue!25,bordercolor=blue,#1]{#2}}
% \newcommandx{\info}[2][1=]{\todo[linecolor=OliveGreen,backgroundcolor=OliveGreen!25,bordercolor=OliveGreen,#1]{#2}}
% \newcommandx{\improvement}[2][1=]{\todo[linecolor=Plum,backgroundcolor=Plum!25,bordercolor=Plum,#1]{#2}}

\parskip=8pt
\parindent=0pt
\delimitershortfall=-1pt
\interfootnotelinepenalty=100000

\newcommand{\qedknitr}{\hfill\rule{1.2ex}{1.2ex}}

% \makeatletter
% \def\thm@space@setup{\thm@preskip=0pt
% \thm@postskip=0pt}
% \makeatother

\def\tightlist{}

\makeatletter
% align all math after the command
\newcommand{\mathleft}{\@fleqntrue\@mathmargin\parindent}
\newcommand{\mathcenter}{\@fleqnfalse}
% tilde with text over it
\newcommand{\distas}[1]{\mathbin{\overset{#1}{\kern\z@\sim}}}%
\newsavebox{\mybox}\newsavebox{\mysim}
\newcommand{\distras}[1]{%
  \savebox{\mybox}{\hbox{\kern3pt$\scriptstyle#1$\kern3pt}}%
  \savebox{\mysim}{\hbox{$\sim$}}%
  \mathbin{\overset{#1}{\kern\z@\resizebox{\wd\mybox}{\ht\mysim}{$\sim$}}}%
}
\makeatother

% \newtheoremstyle{newstyle}
% {} %Aboveskip
% {} %Below skip
% {\mdseries} %Body font e.g.\mdseries,\bfseries,\scshape,\itshape
% {} %Indent
% {\bfseries} %Head font e.g.\bfseries,\scshape,\itshape
% {.} %Punctuation afer theorem header
% { } %Space after theorem header
% {} %Heading

\theoremstyle{newstyle}
\newtheorem{thm}{Theorem}
\newtheorem{prop}[thm]{Proposition}
\newtheorem{lem}{Lemma}
\newtheorem{cor}{Corollary}
\newtheorem{definition}{Definition}
\newcommand*\diff{\mathop{}\!\mathrm{d}}
\newcommand*\Diff[1]{\mathop{}\!\mathrm{d^#1}}
\DeclareMathOperator{\E}{\mathbb{E}}
\DeclareMathOperator{\R}{\mathbb{R}}
\newcolumntype{L}[1]{>{\raggedright\let\newline\\\arraybackslash\hspace{0pt}}m{#1}}
\newcolumntype{C}[1]{>{\centering\let\newline\\\arraybackslash\hspace{0pt}}m{#1}}
\newcolumntype{R}[1]{>{\raggedleft\let\newline\\\arraybackslash\hspace{0pt}}m{#1}}

% suppress table numbering
\captionsetup[table]{labelformat=empty}



\begin{document}
	
% \pagenumbering{arabic}% resets `page` counter to 1 
%    

% \maketitle

{% \usefont{T1}{pnc}{m}{n}
\setlength{\parindent}{0pt}
\thispagestyle{plain}
{\fontsize{18}{20}\selectfont\raggedright 
\maketitle  % title \par  

}

{
   \vskip 13.5pt\relax \normalsize\fontsize{11}{12} 
\textbf{\authorfont \href{mailto:t.leavitt718@gmail.com}{Thomas Leavitt}} \hskip 15pt \emph{\small }   

}

}





\vskip 6.5pt

\noindent  \hypertarget{review-estimation}{%
\section{Review: Estimation}\label{review-estimation}}

Yesterday we showed that, in addition to the standard SUTVA assumption,
if three assumptions are true --- (1) exlcusion restriction, (2) no
Defiers and (3) at least one Complier --- then the mean causal effect of
the instrument, \(\bar{\tau}\), divided by the proportion of Compliers,
\(\pi_C\), is equal to the mean causal effect among Compliers,
\(\delta_C\). That is, \begin{equation}
\frac{\bar{\tau}}{\pi_C} = \delta_C
\end{equation}

When the instrument, \(\mathbf{Z}\), is randomly assigned, the
Difference-in-Means estimator,
\(\hat{\bar{\tau}}\left(\mathbf{Z}, \mathbf{Y}\right)\), unbiasedly
estimates \(\bar{\tau}\) and
\(\hat{\bar{\tau}}\left(\mathbf{Z}, \mathbf{D}\right)\) unbiasedly
estimates \(\pi_C\).

The ratio of these two estimators,
\(\cfrac{\hat{\bar{\tau}}\left(\mathbf{Z}, \mathbf{Y}\right)}{\hat{\bar{\tau}}\left(\mathbf{Z}, \mathbf{D}\right)}\),
is a \textit{consistent}, but \textit{not} necessarily \textit{unbiased}
estimator of \(\frac{\bar{\tau}}{\pi_C} = \delta_C\).

What if we wanted not to estimate the mean causal effect among
compliers, but to test hypotheses about causal effects among Compliers?

\hypertarget{hypothesis-testing}{%
\section{Hypothesis Testing}\label{hypothesis-testing}}

\hypertarget{a-simple-example}{%
\subsection{A simple example}\label{a-simple-example}}

\begin{Shaded}
\begin{Highlighting}[]
\KeywordTok{rm}\NormalTok{(}\DataTypeTok{list =} \KeywordTok{ls}\NormalTok{())}
\NormalTok{n \textless{}{-}}\StringTok{ }\DecValTok{8}
\NormalTok{n\_}\DecValTok{1}\NormalTok{ \textless{}{-}}\StringTok{ }\DecValTok{4}

\KeywordTok{set.seed}\NormalTok{(}\DecValTok{1}\OperatorTok{:}\DecValTok{5}\NormalTok{)}
\NormalTok{d\_c \textless{}{-}}\StringTok{ }\KeywordTok{rbinom}\NormalTok{(}\DataTypeTok{n =}\NormalTok{ n, }\DataTypeTok{size =} \DecValTok{1}\NormalTok{, }\DataTypeTok{prob =} \FloatTok{0.3}\NormalTok{)}
\NormalTok{d\_t \textless{}{-}}\StringTok{ }\KeywordTok{rep}\NormalTok{(}\DataTypeTok{x =} \OtherTok{NA}\NormalTok{, }\DataTypeTok{times =} \KeywordTok{length}\NormalTok{(d\_c))}
\CommentTok{\#\# HERE WE SATISFY THE AT LEAST ONE COMPLIER (NON{-}WEAK INSTRUMENT) ASSUMPTION}
\NormalTok{d\_t[}\KeywordTok{which}\NormalTok{(d\_c }\OperatorTok{!=}\StringTok{ }\DecValTok{1}\NormalTok{)] \textless{}{-}}\StringTok{ }\KeywordTok{rbinom}\NormalTok{(}\DataTypeTok{n =} \KeywordTok{length}\NormalTok{(}\KeywordTok{which}\NormalTok{(d\_c }\OperatorTok{!=}\StringTok{ }\DecValTok{1}\NormalTok{)), }\DataTypeTok{size =} \DecValTok{1}\NormalTok{, }\DataTypeTok{prob =} \FloatTok{0.6}\NormalTok{)}
\CommentTok{\#\# HERE WE SATISFY THE NO DEFIERS (MONOTONICITY) ASSUMPTION}
\NormalTok{d\_t[}\KeywordTok{which}\NormalTok{(d\_c }\OperatorTok{==}\StringTok{ }\DecValTok{1}\NormalTok{)] \textless{}{-}}\StringTok{ }\KeywordTok{rep}\NormalTok{(}\DataTypeTok{x =} \DecValTok{1}\NormalTok{, }\DataTypeTok{times =} \KeywordTok{length}\NormalTok{(}\KeywordTok{which}\NormalTok{(d\_c }\OperatorTok{==}\StringTok{ }\DecValTok{1}\NormalTok{)))}
\KeywordTok{cbind}\NormalTok{(d\_c, d\_t)}
\end{Highlighting}
\end{Shaded}

\begin{verbatim}
##      d_c d_t
## [1,]   0   0
## [2,]   0   1
## [3,]   0   1
## [4,]   1   1
## [5,]   0   1
## [6,]   1   1
## [7,]   1   1
## [8,]   0   0
\end{verbatim}

\begin{Shaded}
\begin{Highlighting}[]
\NormalTok{prop\_comp \textless{}{-}}\StringTok{ }\KeywordTok{length}\NormalTok{(}\KeywordTok{which}\NormalTok{(d\_c }\OperatorTok{==}\StringTok{ }\DecValTok{0} \OperatorTok{\&}\StringTok{ }\NormalTok{d\_t }\OperatorTok{==}\StringTok{ }\DecValTok{1}\NormalTok{)) }\OperatorTok{/}\StringTok{ }\NormalTok{n}
\NormalTok{prop\_def \textless{}{-}}\StringTok{ }\KeywordTok{length}\NormalTok{(}\KeywordTok{which}\NormalTok{(d\_c }\OperatorTok{==}\StringTok{ }\DecValTok{1} \OperatorTok{\&}\StringTok{ }\NormalTok{d\_t }\OperatorTok{==}\StringTok{ }\DecValTok{0}\NormalTok{)) }\OperatorTok{/}\StringTok{ }\NormalTok{n}
\NormalTok{prop\_at \textless{}{-}}\StringTok{ }\KeywordTok{length}\NormalTok{(}\KeywordTok{which}\NormalTok{(d\_c }\OperatorTok{==}\StringTok{ }\DecValTok{1} \OperatorTok{\&}\StringTok{ }\NormalTok{d\_t }\OperatorTok{==}\StringTok{ }\DecValTok{1}\NormalTok{)) }\OperatorTok{/}\StringTok{ }\NormalTok{n}
\NormalTok{prop\_nt \textless{}{-}}\StringTok{ }\KeywordTok{length}\NormalTok{(}\KeywordTok{which}\NormalTok{(d\_c }\OperatorTok{==}\StringTok{ }\DecValTok{0} \OperatorTok{\&}\StringTok{ }\NormalTok{d\_t }\OperatorTok{==}\StringTok{ }\DecValTok{0}\NormalTok{)) }\OperatorTok{/}\StringTok{ }\NormalTok{n}

\CommentTok{\#\# HERE WE SATISFY THE EXCLUSION RESTRICTION ASSUMPTION BY LETTING}
\CommentTok{\#\# y\_c = y\_t FOR ALL ALWAYS{-}TAKERS AND NEVER{-}TAKERS AND}
\CommentTok{\#\# WE ALSO SATISFY THE SUTVA ASSUMPTION BY LETTING ALL UNITS HAVE}
\CommentTok{\#\# ONLY TWO POT OUTS}
\KeywordTok{set.seed}\NormalTok{(}\DecValTok{1}\OperatorTok{:}\DecValTok{5}\NormalTok{)}
\NormalTok{y\_c \textless{}{-}}\StringTok{ }\KeywordTok{round}\NormalTok{(}\DataTypeTok{x =} \KeywordTok{rnorm}\NormalTok{(}\DataTypeTok{n =} \DecValTok{8}\NormalTok{, }\DataTypeTok{mean =} \DecValTok{20}\NormalTok{, }\DataTypeTok{sd =} \DecValTok{10}\NormalTok{), }\DataTypeTok{digits =} \DecValTok{0}\NormalTok{)}
\NormalTok{y\_t\_null\_false \textless{}{-}}\StringTok{ }\KeywordTok{rep}\NormalTok{(}\DataTypeTok{x =} \OtherTok{NA}\NormalTok{, }\DataTypeTok{times =}\NormalTok{ n)}
\NormalTok{y\_t\_null\_false[}\KeywordTok{which}\NormalTok{(d\_c }\OperatorTok{==}\StringTok{ }\DecValTok{0} \OperatorTok{\&}\StringTok{ }\NormalTok{d\_t }\OperatorTok{==}\StringTok{ }\DecValTok{1}\NormalTok{)] \textless{}{-}}\StringTok{ }\NormalTok{y\_c[}\KeywordTok{which}\NormalTok{(d\_c }\OperatorTok{==}\StringTok{ }\DecValTok{0} \OperatorTok{\&}\StringTok{ }\NormalTok{d\_t }\OperatorTok{==}\StringTok{ }\DecValTok{1}\NormalTok{)] }\OperatorTok{+}
\StringTok{  }\KeywordTok{round}\NormalTok{(}\DataTypeTok{x =} \KeywordTok{rnorm}\NormalTok{(}\DataTypeTok{n =} \KeywordTok{length}\NormalTok{(}\KeywordTok{which}\NormalTok{(d\_c }\OperatorTok{==}\StringTok{ }\DecValTok{0} \OperatorTok{\&}\StringTok{ }\NormalTok{d\_t }\OperatorTok{==}\StringTok{ }\DecValTok{1}\NormalTok{)),}
                  \DataTypeTok{mean =} \DecValTok{10}\NormalTok{,}
                  \DataTypeTok{sd =} \DecValTok{4}\NormalTok{),}
        \DataTypeTok{digits =} \DecValTok{0}\NormalTok{)}
\NormalTok{y\_t\_null\_false[}\OperatorTok{!}\NormalTok{(d\_c }\OperatorTok{==}\StringTok{ }\DecValTok{0} \OperatorTok{\&}\StringTok{ }\NormalTok{d\_t }\OperatorTok{==}\StringTok{ }\DecValTok{1}\NormalTok{)] \textless{}{-}}\StringTok{ }\NormalTok{y\_c[}\OperatorTok{!}\NormalTok{(d\_c }\OperatorTok{==}\StringTok{ }\DecValTok{0} \OperatorTok{\&}\StringTok{ }\NormalTok{d\_t }\OperatorTok{==}\StringTok{ }\DecValTok{1}\NormalTok{)]}
\KeywordTok{cbind}\NormalTok{(y\_c, y\_t\_null\_false)}
\end{Highlighting}
\end{Shaded}

\begin{verbatim}
##      y_c y_t_null_false
## [1,]  14             14
## [2,]  22             34
## [3,]  12             21
## [4,]  36             36
## [5,]  23             39
## [6,]  12             12
## [7,]  25             25
## [8,]  27             27
\end{verbatim}

\begin{Shaded}
\begin{Highlighting}[]
\NormalTok{true\_data \textless{}{-}}\StringTok{ }\KeywordTok{data.frame}\NormalTok{(}\DataTypeTok{y\_t =}\NormalTok{ y\_t\_null\_false,}
                        \DataTypeTok{y\_c =}\NormalTok{ y\_c,}
                        \DataTypeTok{d\_t =}\NormalTok{ d\_t,}
                        \DataTypeTok{d\_c =}\NormalTok{ d\_c,}
                        \DataTypeTok{tau =}\NormalTok{ y\_t\_null\_false }\OperatorTok{{-}}\StringTok{ }\NormalTok{y\_c)}

\NormalTok{true\_data }\OperatorTok{\%\textless{}\textgreater{}\%}\StringTok{ }\KeywordTok{mutate}\NormalTok{(}\DataTypeTok{type =} \OtherTok{NA}\NormalTok{,}
                      \DataTypeTok{type =} \KeywordTok{ifelse}\NormalTok{(}\DataTypeTok{test =}\NormalTok{ d\_c }\OperatorTok{==}\StringTok{ }\DecValTok{0} \OperatorTok{\&}\StringTok{ }\NormalTok{d\_t }\OperatorTok{==}\StringTok{ }\DecValTok{0}\NormalTok{,}
                                    \DataTypeTok{yes =} \StringTok{"Never Taker"}\NormalTok{,}
                                    \DataTypeTok{no =}\NormalTok{ type),}
                      \DataTypeTok{type =} \KeywordTok{ifelse}\NormalTok{(}\DataTypeTok{test =}\NormalTok{ d\_c }\OperatorTok{==}\StringTok{ }\DecValTok{0} \OperatorTok{\&}\StringTok{ }\NormalTok{d\_t }\OperatorTok{==}\StringTok{ }\DecValTok{1}\NormalTok{,}
                                    \DataTypeTok{yes =} \StringTok{"Complier"}\NormalTok{,}
                                    \DataTypeTok{no =}\NormalTok{ type),}
                      \DataTypeTok{type =} \KeywordTok{ifelse}\NormalTok{(}\DataTypeTok{test =}\NormalTok{ d\_c }\OperatorTok{==}\StringTok{ }\DecValTok{1} \OperatorTok{\&}\StringTok{ }\NormalTok{d\_t }\OperatorTok{==}\StringTok{ }\DecValTok{0}\NormalTok{,}
                                    \DataTypeTok{yes =} \StringTok{"Defier"}\NormalTok{,}
                                    \DataTypeTok{no =}\NormalTok{ type),}
                      \DataTypeTok{type =} \KeywordTok{ifelse}\NormalTok{(}\DataTypeTok{test =}\NormalTok{ d\_c }\OperatorTok{==}\StringTok{ }\DecValTok{1} \OperatorTok{\&}\StringTok{ }\NormalTok{d\_t }\OperatorTok{==}\StringTok{ }\DecValTok{1}\NormalTok{,}
                                    \DataTypeTok{yes =} \StringTok{"Always Taker"}\NormalTok{,}
                                    \DataTypeTok{no =}\NormalTok{ type))}
\end{Highlighting}
\end{Shaded}

\begin{Shaded}
\begin{Highlighting}[]
\KeywordTok{kable}\NormalTok{(true\_data)}
\end{Highlighting}
\end{Shaded}

\begin{longtable}[]{@{}rrrrrl@{}}
\toprule
y\_t & y\_c & d\_t & d\_c & tau & type\tabularnewline
\midrule
\endhead
14 & 14 & 0 & 0 & 0 & Never Taker\tabularnewline
34 & 22 & 1 & 0 & 12 & Complier\tabularnewline
21 & 12 & 1 & 0 & 9 & Complier\tabularnewline
36 & 36 & 1 & 1 & 0 & Always Taker\tabularnewline
39 & 23 & 1 & 0 & 16 & Complier\tabularnewline
12 & 12 & 1 & 1 & 0 & Always Taker\tabularnewline
25 & 25 & 1 & 1 & 0 & Always Taker\tabularnewline
27 & 27 & 0 & 0 & 0 & Never Taker\tabularnewline
\bottomrule
\end{longtable}

\begin{Shaded}
\begin{Highlighting}[]
\NormalTok{Omega \textless{}{-}}\StringTok{ }\KeywordTok{apply}\NormalTok{(}\DataTypeTok{X =} \KeywordTok{combn}\NormalTok{(}\DataTypeTok{x =} \DecValTok{1}\OperatorTok{:}\NormalTok{n,}
                         \DataTypeTok{m =}\NormalTok{ n\_}\DecValTok{1}\NormalTok{),}
               \DataTypeTok{MARGIN =} \DecValTok{2}\NormalTok{,}
               \DataTypeTok{FUN =} \ControlFlowTok{function}\NormalTok{(x) \{ }\KeywordTok{as.integer}\NormalTok{(}\DecValTok{1}\OperatorTok{:}\NormalTok{n }\OperatorTok{\%in\%}\StringTok{ }\NormalTok{x) \})}

\NormalTok{assign\_vec\_probs \textless{}{-}}\StringTok{ }\KeywordTok{rep}\NormalTok{(}\DataTypeTok{x =}\NormalTok{ (}\DecValTok{1}\OperatorTok{/}\DecValTok{70}\NormalTok{), }\DataTypeTok{times =} \KeywordTok{ncol}\NormalTok{(Omega))}
\CommentTok{\#\# Omega contains all possible assignments, so the probs should add up to 1}
\KeywordTok{stopifnot}\NormalTok{(}\KeywordTok{sum}\NormalTok{(assign\_vec\_probs)}\OperatorTok{==}\DecValTok{1}\NormalTok{) }
\KeywordTok{stopifnot}\NormalTok{(}\KeywordTok{length}\NormalTok{(assign\_vec\_probs)}\OperatorTok{==}\KeywordTok{ncol}\NormalTok{(Omega) )}

\KeywordTok{set.seed}\NormalTok{(}\DecValTok{1}\OperatorTok{:}\DecValTok{5}\NormalTok{)}
\NormalTok{obs\_z \textless{}{-}}\StringTok{ }\NormalTok{Omega[,}\KeywordTok{sample}\NormalTok{(}\DataTypeTok{x =} \DecValTok{1}\OperatorTok{:}\KeywordTok{ncol}\NormalTok{(Omega), }\DataTypeTok{size =} \DecValTok{1}\NormalTok{)]}

\CommentTok{\#obs\_ys \textless{}{-} apply(X = Omega,}
\CommentTok{\#\# MARGIN = 2,}
\CommentTok{\#\# FUN = function(x) \{ x * y\_t\_null\_false + (1 {-} x) * y\_c \})}

\CommentTok{\#obs\_ds \textless{}{-} apply(X = Omega,}
\CommentTok{\#\#MARGIN = 2,}
\CommentTok{\#\#FUN = function(x) \{ x * d\_t + (1 {-} x) * d\_c \}) }

\NormalTok{obs\_y \textless{}{-}}\StringTok{ }\NormalTok{obs\_z }\OperatorTok{*}\StringTok{ }\NormalTok{true\_data}\OperatorTok{$}\NormalTok{y\_t }\OperatorTok{+}\StringTok{ }\NormalTok{(}\DecValTok{1} \OperatorTok{{-}}\StringTok{ }\NormalTok{obs\_z) }\OperatorTok{*}\StringTok{ }\NormalTok{true\_data}\OperatorTok{$}\NormalTok{y\_c}
\NormalTok{obs\_d \textless{}{-}}\StringTok{ }\NormalTok{obs\_z }\OperatorTok{*}\StringTok{ }\NormalTok{true\_data}\OperatorTok{$}\NormalTok{d\_t }\OperatorTok{+}\StringTok{ }\NormalTok{(}\DecValTok{1} \OperatorTok{{-}}\StringTok{ }\NormalTok{obs\_z) }\OperatorTok{*}\StringTok{ }\NormalTok{true\_data}\OperatorTok{$}\NormalTok{d\_c}

\NormalTok{obs\_diff\_means \textless{}{-}}\StringTok{ }\KeywordTok{as.numeric}\NormalTok{((}\KeywordTok{t}\NormalTok{(obs\_z) }\OperatorTok{\%*\%}\StringTok{ }\NormalTok{obs\_y) }\OperatorTok{/}\StringTok{ }\NormalTok{(}\KeywordTok{t}\NormalTok{(obs\_z) }\OperatorTok{\%*\%}\StringTok{ }\NormalTok{obs\_z) }\OperatorTok{{-}}
\StringTok{                               }\NormalTok{(}\KeywordTok{t}\NormalTok{(}\DecValTok{1} \OperatorTok{{-}}\StringTok{ }\NormalTok{obs\_z) }\OperatorTok{\%*\%}\StringTok{ }\NormalTok{obs\_y) }\OperatorTok{/}\StringTok{ }\NormalTok{(}\KeywordTok{t}\NormalTok{(}\DecValTok{1} \OperatorTok{{-}}\StringTok{ }\NormalTok{obs\_z) }\OperatorTok{\%*\%}\StringTok{ }\NormalTok{(}\DecValTok{1} \OperatorTok{{-}}\StringTok{ }\NormalTok{obs\_z)))}

\KeywordTok{coef}\NormalTok{(}\KeywordTok{lm}\NormalTok{(}\DataTypeTok{formula =}\NormalTok{ obs\_y }\OperatorTok{\textasciitilde{}}\StringTok{ }\NormalTok{obs\_z))[[}\StringTok{"obs\_z"}\NormalTok{]]}
\end{Highlighting}
\end{Shaded}

\begin{verbatim}
## [1] 16.75
\end{verbatim}

Our observed data is as follows:

\begin{table}[H]
\centering
    \begin{tabular}{l|l|l|l|l|l|l}
    $\mathbf{z}$ & $\mathbf{y}$ & $\mathbf{y_c}$ & $\mathbf{y_t}$ & $\mathbf{d}$ & $\mathbf{d_c}$ & $\mathbf{d_t}$ \\ \hline
    1 & 14 & ? & 14 & 0 & ? & 0 \\
    0 & 22 & 22 & ? & 0 & 0 & ? \\
    1 & 21 & ? & 21 & 1 & ? & 1 \\
    1 & 36 & ? & 36 & 1 & ? & 1 \\
    0 & 23 & 23 & ? & 0 & 0 & ? \\
    0 & 12 & 12 & ? & 1 & 1 & ? \\
    0 & 25 & 25 & ? & 1 & 1 & ? \\
    1 & 27 & ?  & 27 & 0 & ? & 0\\
    \end{tabular}
    \caption{Observed Experimental Data}
\end{table}

The observed Difference-in-Means test statistic is \(16.75\).

The null hypothesis of no complier causal effect states that the
individual causal effect of \(\mathbf{Z}\) on \(\mathbf{Y}\) is \(0\)
among units who are Compliers.

Along with the exclusion restriction (i.e., that the individual causal
effect is \(0\) for Always Takers and Never Takers) and the assumption
of no Defiers, we can ``fill in'' missing potential outcomes according
to the null hypothesis of no complier causal effect as follows:
\begin{align*}
Y_{c,0,i} & = 
\begin{cases}
Y_i - D_i \tau_i & \text{if } D_i = 1 \\
Y_i + \left(1 - D_i\right) \tau_i & \text{if } D_i = 0
\end{cases} \\
Y_{t,0,i} & = 
\begin{cases}
Y_i - D_i \tau_i & \text{if } D_i = 1 \\
Y_i + \left(1 - D_i\right) \tau_i & \text{if } D_i = 0,
\end{cases}
\end{align*} where \(\tau_i = 0\) for all \(i\).

\begin{Shaded}
\begin{Highlighting}[]
\CommentTok{\#\# tau = 0 under exclusion restriction and no complier causal effect}
\NormalTok{tau \textless{}{-}}\StringTok{ }\DecValTok{0}

\NormalTok{null\_y\_c \textless{}{-}}\StringTok{ }\NormalTok{obs\_y }\OperatorTok{{-}}\StringTok{ }\NormalTok{obs\_d }\OperatorTok{*}\StringTok{ }\NormalTok{tau}
\NormalTok{null\_y\_t \textless{}{-}}\StringTok{ }\NormalTok{obs\_y }\OperatorTok{+}\StringTok{ }\NormalTok{(}\DecValTok{1} \OperatorTok{{-}}\StringTok{ }\NormalTok{obs\_d) }\OperatorTok{*}\StringTok{ }\NormalTok{tau}
\end{Highlighting}
\end{Shaded}

According to the null hypothesis of no effect among compliers and the
exclusion restriction, the full schedule of potential outcomes under the
null hypothesis is as follows:

\begin{table}[H]
\centering
    \begin{tabular}{l|l|l|l|l|l|l}
    $\mathbf{z}$ & $\mathbf{y}$ & $\mathbf{y_c}$ & $\mathbf{y_t}$ & $\mathbf{d}$ & $\mathbf{d_c}$ & $\mathbf{d_t}$ \\ \hline
    1 & 14 & 14 & 14 & 0 & ? & 0 \\
    0 & 22 & 22 & 22 & 0 & 0 & ? \\
    1 & 21 & 21 & 21 & 1 & ? & 1 \\
    1 & 36 & 36 & 36 & 1 & ? & 1 \\
    0 & 23 & 23 & 23 & 0 & 0 & ? \\
    0 & 12 & 12 & 12 & 1 & 1 & ? \\
    0 & 25 & 25 & 25 & 1 & 1 & ? \\
    1 & 27 & 27  & 27 & 0 & ? & 0\\
    \end{tabular}
    \caption{Potential outcomes under the exclusion restriction and null hypothesis of no complier causal effect}
    \label{tab: pot outs under null}
\end{table}

The null potential outcomes are a function of \(\mathbf{D}\). But once
we have constructed these potential outcomes according to the null
hypothesis, we summarize the data under the null via a test statistic
that is a function of \(\mathbf{Z}\) and \(\mathbf{Y}\),
\(t\left(\mathbf{Z}, \mathbf{Y}\right)\). For this example, we will
stick with the Difference-in-Means test statistic.

We can now exactly enumerate all possible realizations of data if Table
\ref{tab: pot outs under null} were the true state of the world.

\begin{Shaded}
\begin{Highlighting}[]
\NormalTok{obs\_null\_pot\_outs \textless{}{-}}\StringTok{ }\KeywordTok{sapply}\NormalTok{(}\DataTypeTok{X =} \DecValTok{1}\OperatorTok{:}\KeywordTok{ncol}\NormalTok{(Omega),}
                            \DataTypeTok{FUN =} \ControlFlowTok{function}\NormalTok{(x) \{ Omega[,x] }\OperatorTok{*}\StringTok{ }\NormalTok{null\_y\_t }\OperatorTok{+}\StringTok{ }\NormalTok{(}\DecValTok{1} \OperatorTok{{-}}\StringTok{ }\NormalTok{Omega[,x]) }\OperatorTok{*}\StringTok{ }\NormalTok{null\_y\_c \})}

\NormalTok{null\_test\_stat\_dist \textless{}{-}}\StringTok{ }\KeywordTok{sapply}\NormalTok{(}\DataTypeTok{X =} \DecValTok{1}\OperatorTok{:}\KeywordTok{ncol}\NormalTok{(Omega),}
                              \DataTypeTok{FUN =} \ControlFlowTok{function}\NormalTok{(x) \{ }\KeywordTok{mean}\NormalTok{(obs\_null\_pot\_outs[,x][}\KeywordTok{which}\NormalTok{(Omega[,x] }\OperatorTok{==}\StringTok{ }\DecValTok{1}\NormalTok{)]) }\OperatorTok{{-}}
\StringTok{                                  }\KeywordTok{mean}\NormalTok{(obs\_null\_pot\_outs[,x][}\KeywordTok{which}\NormalTok{(Omega[,x] }\OperatorTok{==}\StringTok{ }\DecValTok{0}\NormalTok{)])\})}

\NormalTok{null\_test\_stats\_data \textless{}{-}}\StringTok{ }\KeywordTok{data.frame}\NormalTok{(}\DataTypeTok{null\_test\_stat =}\NormalTok{ null\_test\_stat\_dist,}
                                   \DataTypeTok{prob =}\NormalTok{ assign\_vec\_probs)}
\end{Highlighting}
\end{Shaded}

\begin{Shaded}
\begin{Highlighting}[]
\KeywordTok{ggplot}\NormalTok{(}\DataTypeTok{data =}\NormalTok{ null\_test\_stats\_data,}
                         \DataTypeTok{mapping =} \KeywordTok{aes}\NormalTok{(}\DataTypeTok{x =}\NormalTok{ null\_test\_stat,}
                                       \DataTypeTok{y =}\NormalTok{ prob)) }\OperatorTok{+}
\StringTok{  }\KeywordTok{geom\_bar}\NormalTok{(}\DataTypeTok{stat =} \StringTok{"identity"}\NormalTok{) }\OperatorTok{+}
\StringTok{  }\KeywordTok{geom\_vline}\NormalTok{(}\DataTypeTok{xintercept =}\NormalTok{ obs\_diff\_means,}
             \DataTypeTok{color =} \StringTok{"red"}\NormalTok{,}
             \DataTypeTok{linetype =} \StringTok{"dashed"}\NormalTok{) }\OperatorTok{+}
\StringTok{  }\KeywordTok{xlab}\NormalTok{(}\DataTypeTok{label =} \StringTok{"Null Test Statistics"}\NormalTok{) }\OperatorTok{+}
\StringTok{  }\KeywordTok{ylab}\NormalTok{(}\DataTypeTok{label =} \StringTok{"Probability"}\NormalTok{)}
\end{Highlighting}
\end{Shaded}

\begin{figure}
\centering
\includegraphics{Day_08_Handout_files/figure-latex/null_dist_plot-1.pdf}
\caption{Null Distribution of Test Statistic}
\end{figure}

\newpage

How would we calculate a p-value in \texttt{[R]}? Recall the expression
for an upper p-value from Day 3:

\begin{equation}
\Pr\left(t\left(\mathbf{z}, \mathbf{y}_0 \right) \geq T \right) = \sum \limits_{\mathbf{z} \in \Omega} \mathbbm{1}\left[t\left(\mathbf{z}, \mathbf{y}_0 \right) \geq T\right] \Pr\left(\mathbf{Z} = \mathbf{z}\right),
\end{equation} where \(\mathbbm{1}\left[\cdot\right]\) is an indicator
function that is \(1\) if the argument to the function is true and \(0\)
if it is false, \(t\left(\mathbf{z}, \mathbf{y}_0 \right)\) is the null
test statistic and \(T\) is the observed test statistic.

\begin{Shaded}
\begin{Highlighting}[]
\NormalTok{upper\_p\_value \textless{}{-}}\StringTok{ }\KeywordTok{sum}\NormalTok{((null\_test\_stat\_dist }\OperatorTok{\textgreater{}=}\StringTok{ }\NormalTok{obs\_diff\_means) }\OperatorTok{*}\StringTok{ }\NormalTok{assign\_vec\_probs)}
\end{Highlighting}
\end{Shaded}

\newcommand{\V}{\mathbb{V}}

We could also use a Normal approximation p-value, where the upper
p-value, \(p_u\), is: \begin{equation}
p_u =  1 - \Phi\left(\frac{t\left(\mathbf{z}, \mathbf{y}_0 \right) - \E\left[t\left(\mathbf{z}, \mathbf{y}_0 \right)\right]}{\sqrt{\mathbb{V}\left[t\left(\mathbf{z}, \mathbf{y}_0 \right)\right]}}\right).
\end{equation}

\begin{Shaded}
\begin{Highlighting}[]
\KeywordTok{source}\NormalTok{(}\StringTok{"true\_diff\_means\_var\_fun.R"}\NormalTok{)}

\NormalTok{var\_null\_test\_stat \textless{}{-}}\StringTok{ }\KeywordTok{true\_diff\_means\_var}\NormalTok{(}\DataTypeTok{.n =}\NormalTok{ n,}
                                          \DataTypeTok{.n\_1 =}\NormalTok{ n\_}\DecValTok{1}\NormalTok{,}
                                          \DataTypeTok{.y\_c =}\NormalTok{ null\_y\_c,}
                                          \DataTypeTok{.y\_t =}\NormalTok{ null\_y\_t)}\OperatorTok{$}\NormalTok{var}

\NormalTok{z\_score \textless{}{-}}\StringTok{ }\NormalTok{(obs\_diff\_means }\OperatorTok{{-}}\StringTok{ }\DecValTok{0}\NormalTok{)}\OperatorTok{/}\KeywordTok{sqrt}\NormalTok{(var\_null\_test\_stat)}

\DecValTok{1} \OperatorTok{{-}}\StringTok{ }\KeywordTok{pnorm}\NormalTok{(}\DataTypeTok{q =}\NormalTok{ z\_score)}
\end{Highlighting}
\end{Shaded}

\begin{verbatim}
## [1] 0.0118582
\end{verbatim}

What if we were to construct potential outcomes under a null with values
of \(\tau_i\) that were not all equal to 0? Consider these two possible
states of the world:

\begin{table}[H]
\centering
    \begin{tabular}{l|l|l|l|l|l|l}
    $\mathbf{z}$ & $\mathbf{y}$ & $\mathbf{y_{c0}}$ & $\mathbf{y_{t0}}$ & $\mathbf{d}$ & $\mathbf{d_{c0}}$ & $\mathbf{d_{t0}}$ \\ \hline
    1 & 14 & 14 & 14 & 0 & 0 & 0 \\
    0 & 22 & 22 & 27 & 0 & 0 & 0 \\
    1 & 21 & 16 & 21 & 1 & 1 & 1 \\
    1 & 36 & 31 & 36 & 1 & 1 & 1 \\
    0 & 23 & 23 & 28 & 0 & 0 & 0 \\
    0 & 12 & 12 & 17 & 1 & 1 & 1 \\
    0 & 25 & 25 & 25 & 1 & 1 & 1 \\
    1 & 27 & 27  & 27 & 0 & 0 & 0\\
    \end{tabular}
    \hfill 
    \begin{tabular}{l|l|l|l|l|l|l}
    $\mathbf{z}$ & $\mathbf{y}$ & $\mathbf{y_{c0}}$ & $\mathbf{y_{t0}}$ & $\mathbf{d}$ & $\mathbf{d_{c0}}$ & $\mathbf{d_{t0}}$ \\ \hline
    1 & 14 & 14 & 14 & 0 & 0 & 0 \\
    0 & 22 & 22 & 27 & 0 & 0 & 1 \\
    1 & 21 & 16 & 21 & 1 & 0 & 1 \\
    1 & 36 & 31 & 36 & 1 & 0 & 1 \\
    0 & 23 & 23 & 28 & 0 & 0 & 1 \\
    0 & 12 & 12 & 17 & 1 & 1 & 1 \\
    0 & 25 & 25 & 25 & 1 & 1 & 1 \\
    1 & 27 & 27  & 27 & 0 & 0 & 0\\
    \end{tabular}
\end{table}

Both hypothetical states of the world yield observationally equivalent
null distributions of the test statistic,
\(t\left(\mathbf{Z}, \mathbf{Y}_0\right)\).

\begin{Shaded}
\begin{Highlighting}[]
\NormalTok{df\_a \textless{}{-}}\StringTok{ }\KeywordTok{data.frame}\NormalTok{(}\DataTypeTok{y\_c0 =} \KeywordTok{c}\NormalTok{(}\DecValTok{14}\NormalTok{, }\DecValTok{27}\NormalTok{, }\DecValTok{21}\NormalTok{, }\DecValTok{36}\NormalTok{, }\DecValTok{28}\NormalTok{, }\DecValTok{17}\NormalTok{, }\DecValTok{25}\NormalTok{, }\DecValTok{27}\NormalTok{),}
                   \DataTypeTok{y\_t0 =} \KeywordTok{c}\NormalTok{(}\DecValTok{14}\NormalTok{, }\DecValTok{27}\NormalTok{, }\DecValTok{21}\NormalTok{, }\DecValTok{36}\NormalTok{, }\DecValTok{28}\NormalTok{, }\DecValTok{17}\NormalTok{, }\DecValTok{25}\NormalTok{, }\DecValTok{27}\NormalTok{),}
                   \DataTypeTok{d\_c0 =} \KeywordTok{c}\NormalTok{(}\DecValTok{0}\NormalTok{, }\DecValTok{0}\NormalTok{, }\DecValTok{1}\NormalTok{, }\DecValTok{1}\NormalTok{, }\DecValTok{0}\NormalTok{, }\DecValTok{1}\NormalTok{, }\DecValTok{1}\NormalTok{, }\DecValTok{0}\NormalTok{),}
                   \DataTypeTok{d\_t0 =} \KeywordTok{c}\NormalTok{(}\DecValTok{0}\NormalTok{, }\DecValTok{0}\NormalTok{, }\DecValTok{1}\NormalTok{, }\DecValTok{1}\NormalTok{, }\DecValTok{0}\NormalTok{, }\DecValTok{1}\NormalTok{, }\DecValTok{1}\NormalTok{, }\DecValTok{0}\NormalTok{))}

\NormalTok{df\_b \textless{}{-}}\StringTok{ }\KeywordTok{data.frame}\NormalTok{(}\DataTypeTok{y\_c0 =} \KeywordTok{c}\NormalTok{(}\DecValTok{14}\NormalTok{, }\DecValTok{27}\NormalTok{, }\DecValTok{21}\NormalTok{, }\DecValTok{36}\NormalTok{, }\DecValTok{28}\NormalTok{, }\DecValTok{17}\NormalTok{, }\DecValTok{25}\NormalTok{, }\DecValTok{27}\NormalTok{),}
                   \DataTypeTok{y\_t0 =} \KeywordTok{c}\NormalTok{(}\DecValTok{14}\NormalTok{, }\DecValTok{27}\NormalTok{, }\DecValTok{21}\NormalTok{, }\DecValTok{36}\NormalTok{, }\DecValTok{28}\NormalTok{, }\DecValTok{17}\NormalTok{, }\DecValTok{25}\NormalTok{, }\DecValTok{27}\NormalTok{),}
                   \DataTypeTok{d\_c0 =} \KeywordTok{c}\NormalTok{(}\DecValTok{0}\NormalTok{, }\DecValTok{0}\NormalTok{, }\DecValTok{0}\NormalTok{, }\DecValTok{0}\NormalTok{, }\DecValTok{0}\NormalTok{, }\DecValTok{1}\NormalTok{, }\DecValTok{1}\NormalTok{, }\DecValTok{1}\NormalTok{),}
                   \DataTypeTok{d\_t0 =} \KeywordTok{c}\NormalTok{(}\DecValTok{0}\NormalTok{, }\DecValTok{1}\NormalTok{, }\DecValTok{1}\NormalTok{, }\DecValTok{1}\NormalTok{, }\DecValTok{1}\NormalTok{, }\DecValTok{1}\NormalTok{, }\DecValTok{1}\NormalTok{, }\DecValTok{0}\NormalTok{))}

\NormalTok{obs\_null\_pot\_outs\_a \textless{}{-}}\StringTok{ }\KeywordTok{sapply}\NormalTok{(}\DataTypeTok{X =} \DecValTok{1}\OperatorTok{:}\KeywordTok{ncol}\NormalTok{(Omega),}
                              \DataTypeTok{FUN =} \ControlFlowTok{function}\NormalTok{(x) \{ Omega[,x] }\OperatorTok{*}\StringTok{ }\NormalTok{df\_a}\OperatorTok{$}\NormalTok{y\_t0 }\OperatorTok{+}\StringTok{ }\NormalTok{(}\DecValTok{1} \OperatorTok{{-}}\StringTok{ }\NormalTok{Omega[,x]) }\OperatorTok{*}\StringTok{ }\NormalTok{df\_a}\OperatorTok{$}\NormalTok{y\_c0 \})}

\NormalTok{null\_test\_stat\_dist\_a \textless{}{-}}\StringTok{ }\KeywordTok{sapply}\NormalTok{(}\DataTypeTok{X =} \DecValTok{1}\OperatorTok{:}\KeywordTok{ncol}\NormalTok{(Omega),}
                                \DataTypeTok{FUN =} \ControlFlowTok{function}\NormalTok{(x) \{ }\KeywordTok{mean}\NormalTok{(obs\_null\_pot\_outs\_a[,x][}\KeywordTok{which}\NormalTok{(Omega[,x] }\OperatorTok{==}\StringTok{ }\DecValTok{1}\NormalTok{)]) }\OperatorTok{{-}}
\StringTok{                                    }\KeywordTok{mean}\NormalTok{(obs\_null\_pot\_outs\_a[,x][}\KeywordTok{which}\NormalTok{(Omega[,x] }\OperatorTok{==}\StringTok{ }\DecValTok{0}\NormalTok{)])\})}

\NormalTok{obs\_null\_pot\_outs\_b \textless{}{-}}\StringTok{ }\KeywordTok{sapply}\NormalTok{(}\DataTypeTok{X =} \DecValTok{1}\OperatorTok{:}\KeywordTok{ncol}\NormalTok{(Omega),}
                              \DataTypeTok{FUN =} \ControlFlowTok{function}\NormalTok{(x) \{ Omega[,x] }\OperatorTok{*}\StringTok{ }\NormalTok{df\_b}\OperatorTok{$}\NormalTok{y\_t0 }\OperatorTok{+}\StringTok{ }\NormalTok{(}\DecValTok{1} \OperatorTok{{-}}\StringTok{ }\NormalTok{Omega[,x]) }\OperatorTok{*}\StringTok{ }\NormalTok{df\_b}\OperatorTok{$}\NormalTok{y\_c0 \})}

\NormalTok{null\_test\_stat\_dist\_b \textless{}{-}}\StringTok{ }\KeywordTok{sapply}\NormalTok{(}\DataTypeTok{X =} \DecValTok{1}\OperatorTok{:}\KeywordTok{ncol}\NormalTok{(Omega),}
                                \DataTypeTok{FUN =} \ControlFlowTok{function}\NormalTok{(x) \{ }\KeywordTok{mean}\NormalTok{(obs\_null\_pot\_outs\_b[,x][}\KeywordTok{which}\NormalTok{(Omega[,x] }\OperatorTok{==}\StringTok{ }\DecValTok{1}\NormalTok{)]) }\OperatorTok{{-}}
\StringTok{                                    }\KeywordTok{mean}\NormalTok{(obs\_null\_pot\_outs\_b[,x][}\KeywordTok{which}\NormalTok{(Omega[,x] }\OperatorTok{==}\StringTok{ }\DecValTok{0}\NormalTok{)])\})}

\KeywordTok{cbind}\NormalTok{(null\_test\_stat\_dist\_a, null\_test\_stat\_dist\_b)}
\end{Highlighting}
\end{Shaded}

\begin{verbatim}
##       null_test_stat_dist_a null_test_stat_dist_b
##  [1,]                  0.25                  0.25
##  [2,]                 -3.75                 -3.75
##  [3,]                 -9.25                 -9.25
##  [4,]                 -5.25                 -5.25
##  [5,]                 -4.25                 -4.25
##  [6,]                  3.75                  3.75
##  [7,]                 -1.75                 -1.75
##  [8,]                  2.25                  2.25
##  [9,]                  3.25                  3.25
## [10,]                 -5.75                 -5.75
## [11,]                 -1.75                 -1.75
## [12,]                 -0.75                 -0.75
## [13,]                 -7.25                 -7.25
## [14,]                 -6.25                 -6.25
## [15,]                 -2.25                 -2.25
## [16,]                  0.75                  0.75
## [17,]                 -4.75                 -4.75
## [18,]                 -0.75                 -0.75
## [19,]                  0.25                  0.25
## [20,]                 -8.75                 -8.75
## [21,]                 -4.75                 -4.75
## [22,]                 -3.75                 -3.75
## [23,]                -10.25                -10.25
## [24,]                 -9.25                 -9.25
## [25,]                 -5.25                 -5.25
## [26,]                 -1.25                 -1.25
## [27,]                  2.75                  2.75
## [28,]                  3.75                  3.75
## [29,]                 -2.75                 -2.75
## [30,]                 -1.75                 -1.75
## [31,]                  2.25                  2.25
## [32,]                 -6.75                 -6.75
## [33,]                 -5.75                 -5.75
## [34,]                 -1.75                 -1.75
## [35,]                 -7.25                 -7.25
## [36,]                  7.25                  7.25
## [37,]                  1.75                  1.75
## [38,]                  5.75                  5.75
## [39,]                  6.75                  6.75
## [40,]                 -2.25                 -2.25
## [41,]                  1.75                  1.75
## [42,]                  2.75                  2.75
## [43,]                 -3.75                 -3.75
## [44,]                 -2.75                 -2.75
## [45,]                  1.25                  1.25
## [46,]                  5.25                  5.25
## [47,]                  9.25                  9.25
## [48,]                 10.25                 10.25
## [49,]                  3.75                  3.75
## [50,]                  4.75                  4.75
## [51,]                  8.75                  8.75
## [52,]                 -0.25                 -0.25
## [53,]                  0.75                  0.75
## [54,]                  4.75                  4.75
## [55,]                 -0.75                 -0.75
## [56,]                  2.25                  2.25
## [57,]                  6.25                  6.25
## [58,]                  7.25                  7.25
## [59,]                  0.75                  0.75
## [60,]                  1.75                  1.75
## [61,]                  5.75                  5.75
## [62,]                 -3.25                 -3.25
## [63,]                 -2.25                 -2.25
## [64,]                  1.75                  1.75
## [65,]                 -3.75                 -3.75
## [66,]                  4.25                  4.25
## [67,]                  5.25                  5.25
## [68,]                  9.25                  9.25
## [69,]                  3.75                  3.75
## [70,]                 -0.25                 -0.25
\end{verbatim}

\hypertarget{applied-example}{%
\section{Applied Example}\label{applied-example}}

Let's estimate the average causal effect among Compliers and also test
the null hypothesis that the causal effect is \(0\) among all compliers.

\begin{Shaded}
\begin{Highlighting}[]
\NormalTok{adam\_smith\_data \textless{}{-}}\StringTok{ }\KeywordTok{data.frame}\NormalTok{(}\DataTypeTok{call =} \KeywordTok{c}\NormalTok{(}\KeywordTok{rep}\NormalTok{(}\DataTypeTok{x =} \DecValTok{0}\NormalTok{, }\DataTypeTok{times =} \DecValTok{1325}\NormalTok{), }\KeywordTok{rep}\NormalTok{(}\DataTypeTok{x =} \DecValTok{1}\NormalTok{, }\DataTypeTok{times =} \DecValTok{1325}\NormalTok{)),}
                              \DataTypeTok{contact =} \KeywordTok{c}\NormalTok{(}\KeywordTok{rep}\NormalTok{(}\DataTypeTok{x =} \DecValTok{0}\NormalTok{, }\DataTypeTok{times =} \DecValTok{1325} \OperatorTok{+}\StringTok{ }\DecValTok{375}\NormalTok{), }\KeywordTok{rep}\NormalTok{(}\DataTypeTok{x =} \DecValTok{1}\NormalTok{, }\DataTypeTok{times =} \DecValTok{950}\NormalTok{)),}
                              \DataTypeTok{vote =} \KeywordTok{c}\NormalTok{(}\KeywordTok{rep}\NormalTok{(}\DataTypeTok{x =} \DecValTok{0}\NormalTok{, }\DataTypeTok{times =} \DecValTok{1010}\NormalTok{), }\KeywordTok{rep}\NormalTok{(}\DataTypeTok{x =} \DecValTok{1}\NormalTok{, }\DataTypeTok{times =} \DecValTok{315}\NormalTok{),}
                                       \KeywordTok{rep}\NormalTok{(}\DataTypeTok{x =} \DecValTok{0}\NormalTok{, }\DataTypeTok{times =} \DecValTok{293}\NormalTok{), }\KeywordTok{rep}\NormalTok{(}\DataTypeTok{x =} \DecValTok{1}\NormalTok{, }\DataTypeTok{times =} \DecValTok{82}\NormalTok{),}
                                       \KeywordTok{rep}\NormalTok{(}\DataTypeTok{x =} \DecValTok{1}\NormalTok{, }\DataTypeTok{times =} \DecValTok{310}\NormalTok{), }\KeywordTok{rep}\NormalTok{(}\DataTypeTok{x =} \DecValTok{0}\NormalTok{, }\DataTypeTok{times =} \DecValTok{640}\NormalTok{)))}
\end{Highlighting}
\end{Shaded}


\newpage
\singlespacing 
\bibliography{Bibliography.bib}

\end{document}

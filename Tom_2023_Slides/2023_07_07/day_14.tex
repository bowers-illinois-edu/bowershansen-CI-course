%%%%%%%%%%%%%%%%%%%%%%%%%%%%%%%%%%%%%%%%%
% Beamer Presentation
% LaTeX Template
% Version 1.0 (10/11/12)
%
% This template has been downloaded from:
% http://www.LaTeXTemplates.com
%
% License:
% CC BY-NC-SA 3.0 (http://creativecommons.org/licenses/by-nc-sa/3.0/)
%
%%%%%%%%%%%%%%%%%%%%%%%%%%%%%%%%%%%%%%%%%

%------------------------------------------------------------------------------------------------
%   PACKAGES AND THEMES
%------------------------------------------------------------------------------------------------

\documentclass[table, xcolor = {dvipsnames}, 9pt]{beamer}
\usepackage{tikz}
\usetikzlibrary{backgrounds}
\usetikzlibrary{arrows,shapes}
\usetikzlibrary{tikzmark}
\usetikzlibrary{calc}
\usetikzlibrary{colorbrewer}
\mode<presentation> {

% The Beamer class comes with a number of default slide themes
% which change the colors and layouts of slides. Below this is a list
% of all the themes, uncomment each in turn to see what they look like.

\usetheme{default}
%\usetheme{AnnArbor}
%\usetheme{Antibes}
%\usetheme{Bergen}
%\usetheme{Berkeley}
%\usetheme{Berlin}
%\usetheme{Boadilla}
%\usetheme{CambridgeUS}
%\usetheme{Copenhagen}
%\usetheme{Darmstadt}
%\usetheme{Dresden}
%\usetheme{Frankfurt}
%\usetheme{Goettingen}
%\usetheme{Hannover}
%\usetheme{Ilmenau}
%\usetheme{JuanLesPins}
%\usetheme{Luebeck}
%\usetheme{Madrid}
\usetheme{metropolis}
%\usetheme{Malmoe}
%\usetheme{Marburg}
%\usetheme{Montpellier}
%\usetheme{PaloAlto}
%\usetheme{Pittsburgh}
%\usetheme{Rochester}
%\usetheme{Singapore}
%\usetheme{Szeged}
%\usetheme{Warsaw}

% As well as themes, the Beamer class has a number of color themes
% for any slide theme. Uncomment each of these in turn to see how it
% changes the colors of your current slide theme.

%\usecolortheme{albatross}
%\usecolortheme{beaver}
%\usecolortheme{beetle}
%\usecolortheme{crane}
%\usecolortheme{dolphin}
%\usecolortheme{dove}
%\usecolortheme{fly}
%\usecolortheme{lily}
%\usecolortheme{orchid}
%\usecolortheme{rose}
\usecolortheme{seagull}
%\usecolortheme{seahorse}
%\usecolortheme{whale}
%\usecolortheme{wolverine}
\usefonttheme{professionalfonts}
%\setbeamertemplate{footline} % To remove the footer line in all slides uncomment this line
%\setbeamertemplate{footline}[page number] % To replace the footer line in all slides with a simple slide count uncomment this line

%\setbeamertemplate{navigation symbols}{} % To remove the navigation symbols from the bottom of all slides uncomment this line
}

\usepackage{graphicx} % Allows including images
\usepackage{booktabs} % Allows the use of \toprule, \midrule and \bottomrule in tables
\usepackage{multirow}
\usepackage{xspace}
\usepackage{natbib}
\usepackage{hyperref}
\usepackage{diagbox}
\usepackage{makecell}
\usepackage{xparse}
\usepackage{subfig}
\usepackage{amsfonts,amsthm,amsmath,amssymb}
\usepackage{mathtools, nccmath}
\usepackage[ruled, vlined, linesnumbered]{algorithm2e}
\usepackage{wrapfig}
\usepackage{comment}  
\usepackage{bbm}
\usepackage{bm}
\usepackage{empheq}
\usepackage{pgfplots}
\usepgfplotslibrary{colorbrewer}
\usepackage{animate}
\usepackage{array}
\usepackage{ragged2e}
\newcolumntype{P}[1]{>{\RaggedRight\hspace{0pt}}p{#1}}
\newcolumntype{X}[1]{>{\RaggedRight\hspace*{0pt}}p{#1}}

% color box
\usepackage{tcolorbox}
\usepackage{tikz}
\usetikzlibrary{backgrounds}
\usetikzlibrary{arrows,shapes}
\usetikzlibrary{tikzmark}
\usetikzlibrary{calc}
% Commands for Highlighting text -- non tikz method
\newcommand{\highlight}[2]{\colorbox{#1!17}{$\displaystyle #2$}}
%\newcommand{\highlight}[2]{\colorbox{#1!17}{$#2$}}
\newcommand{\highlightdark}[2]{\colorbox{#1!47}{$\displaystyle #2$}}

% my custom colors for shading
\colorlet{mhpurple}{Plum!80}


% Commands for Highlighting text -- non tikz method
\renewcommand{\highlight}[2]{\colorbox{#1!17}{#2}}
\renewcommand{\highlightdark}[2]{\colorbox{#1!47}{#2}}

\usepgfplotslibrary{colorbrewer}

\newcommand\mybox[2][]{\tikz[overlay]\node[fill=lightgray,inner sep=2pt, anchor=text, rectangle, rounded corners=1mm,#1] {#2};\phantom{#2}}
\hypersetup{unicode=true,
            bookmarksnumbered=true,
            bookmarksopen=true,
            bookmarksopenlevel=2,
            breaklinks=false,
            pdfborder={0 0 1},
            hypertexnames=false,
            pdfstartview={XYZ null null 1}}
\usepackage{xcolor}
\hypersetup{
    colorlinks,
    linkcolor={red!50!black},
    citecolor={blue!50!black},
    urlcolor={blue!80!black}
}
\newcommand\myheading[1]{%
  \par\bigskip
  {\Large\bfseries#1}\par\smallskip}
\newcommand\given[1][]{\:#1\vert\:}
\theoremstyle{plain}
\newtheorem{thm}{Theorem}
\newtheorem{prop}{Proposition\thisthmnumber}
\newtheorem{lem}{Lemma\thisthmnumber}
\newtheorem{cor}{Corollary}
\newtheorem{defin}{Definition}
\newtheorem{algo}{Algorithm}
\newcommand*\diff{\mathop{}\!\mathrm{d}}
\newcommand*\Diff[1]{\mathop{}\!\mathrm{d^#1}}
\newcommand{\thisthmnumber}{}
\newcommand*{\QEDA}{\hfill\ensuremath{\blacksquare}}%
\newcommand*{\QEDB}{\hfill\ensuremath{\square}}%
\newcommand{\norm}[1]{\left\lVert#1\right\rVert}
\DeclareMathOperator{\N}{\mathbb{N}}
\DeclareMathOperator{\E}{\rm{E}}
\DeclareMathOperator{\R}{\mathbb{R}}
\DeclareMathOperator{\Var}{\rm{Var}}
\DeclareMathOperator{\Cov}{\rm{Cov}}
\DeclareMathOperator{\e}{\rm{e}}
\DeclareMathOperator{\logit}{\rm{logit}}
\DeclareMathOperator{\indep}{{\perp\!\!\!\perp}}
\DeclareMathOperator{\rank}{rank}
\DeclareMathOperator*{\argmin}{arg\,min}
\DeclareMathOperator*{\argmax}{arg\,max}
%\DeclareMathOperator{\Pr}{\rm{Pr}}
%------------------------------------------------------------------------
% TITLE PAGE
%-----------------------------------------------------------------------
\pagestyle{empty}
\title[]{Difference-in-Differences Designs} % The short title appears at the bottom of every slide, the full title is only on the title page

\author{Thomas Leavitt and Jake Bowers} % Your name
\institute[] % Your institution as it will appear on the bottom of every slide, may be shorthand to save space
{
%Columbia University \\ % Your institution for the title page
\medskip
%\textit{tl2624@columbia.edu} % Your email address
}
\date{July 7, 2023} % Date, can be changed to a custom date

\NewDocumentEnvironment{statement}{mo}
 {%
  \IfValueT{#2}{\renewcommand{\thisthmnumber}{ #2}}\begin{#1}%
 }
 {\end{#1}}

\begin{document}

\begin{frame}
\titlepage % Print the title page as the first slide
\end{frame}

%\begin{frame}
%\frametitle{Overview} % Table of contents slide, comment this block out to remove it
%\tableofcontents % Throughout your presentation, if you choose to use \section{} and \subsection{} commands, these will automatically be printed on this slide as an overview of your presentation
%\end{frame}

%------------------------------------------------------------------------
% PRESENTATION SLIDES
%----------------------------------------------------------------
\section{Introduction}
\begin{frame}[t]
\frametitle{Causal inference and DID}
\vfill
\begin{itemize}
\item Scholars typically want to infer causal effects \vfill
\begin{itemize}
\item Neyman--Rubin causal model \citep{neyman1923,rubin1974}  \vfill
\end{itemize} 
\item Randomized experiments are ``gold standard'' for inferring causal effects \vfill
\item But experiments often \textit{not} feasible or ethical \vfill
\item Usually observational designs aim to justify \textit{as-if-randomized} assumption \vfill
\item Other observational designs justify assumptions about how treated units' outcomes would have evolved in absence of treatment \vfill
\begin{itemize}
\item Difference-in-Differences (DID) is archetypal example of such designs \vfill
\end{itemize} 
\end{itemize}
\vfill
\end{frame}
%----------------------------------------------------------------
\section{Identification of ATT}
\begin{frame}[t]
\frametitle{Identification of ATT}
\vfill
\begin{itemize}
\item \textcolor{Dark2-A}{\textbf{Setting}}: \vfill
\begin{itemize}
\item Two populations, $G = 0$ and $G = 1$, over periods $t = 0, 1, \ldots , T$ \vfill
\item Population $G = 1$ is treated between $T - 1$ and $T$ \vfill
\item Approx. \textrm{i.i.d.} sampling from $G = 0$ and from $G = 1$ \vfill
\end{itemize}
\item \textcolor{Dark2-A}{\textbf{Causal target}}: Treated population's average effect is \vfill
\begin{equation*}
\textrm{ATT} \coloneqq \E\left[Y_{i,T}(1) - Y_{i,T}(0) \given G_i = 1\right]
\end{equation*} \vfill
\end{itemize}
\vfill
\end{frame}
%---------------------------------------------------------------
\begin{frame}[t]
\frametitle{Identification of ATT}
\vfill
\begin{figure}
\includegraphics[width = \linewidth]{Identification/Part 1/Identification_Part_1.pdf}
\end{figure}
\vfill
\end{frame}
%---------------------------------------------------------------
\begin{frame}[t]
\frametitle{Identification of ATT}
\vfill
\begin{figure}
\includegraphics[width = \linewidth]{Identification/Part 2/Identification_Part_2.pdf}
\end{figure}
\vfill
\end{frame}
%---------------------------------------------------------------
\begin{frame}[t]
\frametitle{Identification of ATT}
\vfill
\begin{figure}
\includegraphics[width = \linewidth]{Identification/Part 3/Identification_Part_3.pdf}
\end{figure}
\vfill
\end{frame}
%---------------------------------------------------------------
\begin{frame}[t]
\frametitle{Identification of ATT}
\vfill
\begin{figure}
\includegraphics[width = \linewidth]{Identification/Part 4/Identification_Part_4.pdf}
\end{figure}
\vfill
\end{frame}
%---------------------------------------------------------------
\begin{frame}[t]
\frametitle{Identification of ATT}
\vfill
\begin{figure}
\includegraphics[width = \linewidth]{Identification/Part 5/Identification_Part_5.pdf}
\end{figure}
\vfill
\end{frame}
%---------------------------------------------------------------
\section{Canonical DID design}
\begin{frame}[t]
\frametitle{Canonical DID design}
\vfill
\begin{itemize}
\item Canonical setup: \vfill
\begin{itemize}
\item Two populations measured over time \vfill
\item One population treated between two time periods and the other is not \vfill
\end{itemize} \vfill
\vfill
\end{itemize} \vfill
\begin{figure}
\includegraphics[width = 0.9\linewidth]{DID_plot.pdf}
\vskip1ex
\end{figure}
\vfill
\end{frame}
%----------------------------------------------------------------
\begin{frame}[t]
\frametitle{Canonical DID design}
\vfill
DID estimator: 
\begin{align*}
\text{treated sample's after-minus-before mean } - \\ \text{control sample's after-minus-before mean} 
\end{align*} \vfill
\vfill
\begin{figure}[htb]
    \vspace{2\baselineskip}
    \begin{equation*}
        \widehat{\text{DID}}_t \coloneqq \highlight{Dark2-D}{$\dfrac{\tikzmarknode{sumtreatpost}{\sum}_{i: G_{i} = 1} \tikzmarknode{treatpotout}{Y}_{i,t}}{\tikzmarknode{ntreatpost}{n}_{1,t}}$} - \highlight{Dark2-D}{$\dfrac{\tikzmarknode{sumtreatpre}{\sum}_{i: G_{i} = 1} \tikzmarknode{treatpreout}{Y}_{i,t-1}}{\tikzmarknode{ntreatpre}{n}_{1,t}}$} - \left(\highlight{Dark2-D}{$\dfrac{\tikzmarknode{sumcontpost}{\sum}_{i: G_{i} = 0} \tikzmarknode{contpostout}{Y}_{i,t}}{\tikzmarknode{ncontpost}{n}_{0,t}}$} - \highlight{Dark2-D}{$\dfrac{\tikzmarknode{sumcontpre}{\sum}_{i: G_{i} = 0} \tikzmarknode{contpreout}{Y}_{i,t-1}}{\tikzmarknode{ncontpre}{n}_{0,t}}$}\right)
    \end{equation*}
    \begin{tikzpicture}[overlay,remember picture,>=stealth,nodes={align=left,inner ysep=1pt},<-]

        \path (treatpotout.north) ++ (-3.25em,2.3em) node[anchor=south west, color=Dark2-D!67] (scalep){\scriptsize \textsf{Observable outcome}};
        \draw[<->,color=Dark2-D!57] (treatpotout.north) -- ++(0,0.67)  -| node[] {} (treatpotout.north);

        \path (treatpreout.north) ++ (-2em,2.3em) node[anchor = south west, color=Dark2-D!67] (scalep){\scriptsize \textsf{Lagged outcome}};
        \draw[<->, color = Dark2-D!57] (treatpreout.north) -- ++(0,0.67)  -| node[] {} (treatpreout.north);

        % sumtreatpost to sumtreatpre
        \path (sumtreatpost.north) ++ (0em,6.7em) node[anchor=south west, color=Dark2-D!67] (scalep){\scriptsize \textsf{Sum over treated units}};
        \draw[<->,color=Dark2-D!57] (sumtreatpost.north) -- ++(0,2)  -| node[] {} (sumtreatpre.north);

        % ntreatpost to ntreatpre
        \path (ntreatpost.south) ++ (0em,-3.3em) node[anchor=south west,color=Dark2-D!67] (scalep){\scriptsize \textsf{Number of treated units}};
        \draw[<->,color=Dark2-D!57] (ntreatpost.south) -- ++(0,-0.67)  -| node[] {} (ntreatpre.south);

        \path (contpostout.north) ++ (-3.25em,2.3em) node[anchor=south west, color=Dark2-D!67] (scalep){\scriptsize \textsf{Observable outcome}};
        \draw[<->,color=Dark2-D!57] (contpostout.north) -- ++(0,0.67)  -| node[] {} (contpostout.north);

        \path (contpreout.north) ++ (-2em,2.3em) node[anchor=south west, color=Dark2-D!67] (scalep){\scriptsize \textsf{Lagged outcome}};
        \draw[<->,color=Dark2-D!57] (contpreout.north) -- ++(0,0.67)  -| node[] {} (contpreout.north);

        % sumcontpost to sumcontpre
        \path (sumcontpost.north) ++ (-0.5em,6.7em) node[anchor=south west, color=Dark2-D!67] (scalep){\scriptsize \textsf{Sum over untreated units}};
        \draw[<->,color=Dark2-D!57] (sumcontpost.north) -- ++(0,2)  -| node[] {} (sumcontpre.north);

        % ncontpost to ncontpre
        \path (ncontpost.south) ++ (-0.5em,-3.3em) node[anchor=south west,color=Dark2-D!67] (scalep){\scriptsize \textsf{Number of untreated units}};
        \draw[<->,color=Dark2-D!57] (ncontpost.south) -- ++(0,-0.67)  -| node[] {} (ncontpre.south);

    \end{tikzpicture}
    \vspace{\baselineskip}
\end{figure} \vfill
\vfill
\end{frame}
%----------------------------------------------------------------
\begin{frame}[t]
\frametitle{DID Estimation}
\vfill
\begin{figure}
\includegraphics[width = 0.9\linewidth]{DID_pt_plot.pdf}
\vskip1ex
\end{figure} \vfill
\vspace{-2em}
\begin{itemize} \vfill
\item Parallel trends: \vfill
\begin{align*}
\text{Treated population's counterfactual after-minus-before mean} = \\ \text{control population's after-minus-before mean} 
\end{align*} \vfill
\end{itemize}  
\vfill
\end{frame}
%----------------------------------------------------------------
\begin{frame}[t]
\frametitle{DID Estimation}
Parallel trends $\implies$ \\ Descriptive difference between treated and control populations $=$ ATT 
\begin{itemize}
\item Descriptive difference
\begin{equation*}
\begin{split}
\underbrace{\E\left[Y_{iT}(1) - Y_{iT-1} \given G_i = 1\right] - \E\left[Y_{iT} - Y_{iT-1} \given G_i = 0\right]}_{\text{Descriptive difference between treated and control populations}}
\end{split} 
\end{equation*}
\item Parallel trends: 
\vskip3ex 
\small
\large
\begin{equation*}
\begin{split}
\E\left[Y_{iT}(0) \tikzmark{treatedafterbeforemeannotreat}- Y_{iT-1} \given G_i = 1\right] & = \E\left[Y_{iT} - Y_{iT-1} \given G_i = 0\right]
\end{split} 
\end{equation*}
\normalsize
\vspace{-1em}
\item Rearranging parallel trends assumption yields
\small
\vspace{0.75em}
\begin{align*}
\underbrace{\E\left[Y_{iT}(0) \given G_i = 1\right]}_{\text{Counterfactual}} - \E\left[Y_{T-1} \given G_i = 1 \right]+ \E\left[Y_{iT-1} \given G_i = 0\right] & = \E\left[Y_{iT} \given G_i = 0\right]
\end{align*}
\normalsize
\vspace{0.75em}
\item Substitute for $\E\left[Y_{iT} \given G_i = 0\right]$ 
\small\begin{align*}
\underbrace{\E\left[Y_{iT}(1) - Y_{iT-1} \given G_i = 1\right] - \E\left[Y_{iT} - Y_{iT-1} \given G_i = 0\right]}_{\text{Descriptive difference between treated and control populations}} \\ = \underbrace{\E\left[Y_{iT}(1) - Y_{iT}(0) \given G_i = 1\right]}_{\text{ATT}}
\end{align*} 
\normalsize
\end{itemize}
\end{frame}
%----------------------------------------------------------------
\section{Problems with canonical DID}
\begin{frame}{Problems with canonical DID}
\vfill
\begin{enumerate} \vfill
\item Identification: Scale dependence \vfill
\item Estimation: Parallel trends assumption is restrictive \vfill
\item Inference: Accounts for only sampling uncertainty \vfill
\end{enumerate} \vfill
\end{frame}
%----------------------------------------------------------------
\begin{frame}{Identification}
\begin{itemize}
\item[] 
\begin{figure}[H]
\includegraphics[width = \linewidth]{standard_scale_DID_plot.pdf}
\end{figure}
\item $\text{ATT} \approx 12.08$ and $\E\left[\widehat{\text{DID}}_{T}\right] \approx 21.01$ 
\end{itemize}
\end{frame}
%----------------------------------------------------------------
\begin{frame}{Identification}
\begin{itemize}
\item[] 
\begin{figure}[H]
\includegraphics[width = \linewidth]{log_scale_DID_plot.pdf}
\end{figure}
\item $\text{ATT} = 0.25$ and $\E\left[\widehat{\text{DID}}_{T}\right] = 0.25$  
\end{itemize}
\end{frame}
%----------------------------------------------------------------
\begin{frame}{Identification}
\begin{center}
``It makes no sense, scientifically or mathematically, to say that the null hypothesis of no treatment effect is false for a response, $R_i$, but true for its logarithm, $\log\left(R_i\right)$'' \citep[][p.~164]{rosenbaum2017}
\end{center}
\end{frame}
%----------------------------------------------------------------
\begin{frame}[t]
\frametitle{Estimation}
\vfill
\begin{itemize} \vfill
\item Nonparametric identification is possible under parallel trends \vfill
\item Common belief holds that: \\ If parallel trends is false, then nonparametric identification is \textit{not} possible \vfill
\item But why should we abandon DID if another trend assumption is plausible? \vfill
\end{itemize}
\vfill
\end{frame}
%----------------------------------------------------------------
\begin{frame}[t]
\frametitle{Inference}
\vfill 
\begin{itemize}
\item Two primary types of uncertainty in causal inference \citep{abadieetal2020}: \vfill
\begin{enumerate}
\item \textbf{Sampling}: Inability to observe all units in a target population \vfill
\item \textbf{Causal}: Inability to observe counterfactual potential outcomes \textit{among whichever units one samples}  \vfill
\end{enumerate}
\item E.g., randomized experiment on random sample from population: \vfill
\begin{itemize}
\item Uncertainty in estimator of population average causal effect reflects variation in \vfill
\begin{enumerate}
\item average causal effect across possible \textit{samples} \vfill
\item causal estimates across possible \textit{assignments} conditional on each possible sample \vfill
\end{enumerate} 
\end{itemize}
\item Quantification of uncertainty (standard errors, $p$-values, confidence intervals) in DID reflects only sampling uncertainty \vfill
\item[] $\implies$ We are too likely to detect an effect when none exists \vfill
\item Quantification of causal uncertainty is difficult because DID based on parallel trends, \textit{not} as-if-randomized assumption \\ 
\citep{manskipepper2018} \vfill
\end{itemize}
\vfill
\end{frame}
%----------------------------------------------------------------
\section{Generalization of DID design}
\begin{frame}[t]
\frametitle{General identification of ATT}
\vfill
\begin{itemize}
\item \textcolor{Dark2-A}{\textbf{General identification condition}}: \vfill
\small
\begin{equation*}
\begin{split}
\textcolor{Dark2-D}{\textbf{Equal-expected-prediction-errors}} \coloneqq \\
\underbrace{\E\left[Y_{i,T}(0) \given G_i = 1\right] - \E\left[\widehat{Y}_{i,T} \given G_i = 1\right]}_{\text{Counterfactual prediction error}} = \underbrace{\E\left[Y_{i,T}(0) \given G_i = 0\right] - \E\left[\widehat{Y}_{i,T} \given G_i = 0\right]}_{\text{Observable prediction error}}
\end{split}
\end{equation*}
\normalsize
\vfill 
\item Equal-expected-prediction-errors $\implies$ \vfill
\small
\begin{equation*} \label{eq: ATT est desc diff}
\begin{split}
\textrm{ATT} = \\ 
\E\left[Y_{i,T}(1) \given G_i = 1\right] - \E\left[\widehat{Y}_{i,T} \given G_i = 1\right] - \left(\E\left[Y_{i,T}(0) \given G_i = 0\right] - \E\left[\widehat{Y}_{i,T} \given G_i = 0\right]\right)
\end{split}
\end{equation*}
\normalsize \vfill
\item So $\textrm{ATT}$ is identified \vfill
\end{itemize}
\end{frame}
%---------------------------------------------------------------
\begin{frame}[t, label = general ID part IV]
\frametitle{General identification of ATT}
\vfill
\begin{itemize} \vfill
\item General ID assumption is agnostic about predictive function \vfill
\item Familiar designs are special cases of this general framework \vfill
\begin{itemize} \vfill
\item DID, Two-way fixed effects (TWFE), lagged dependent variable (LDV), etc.
\item E.g., parallel-trends assumption of canonical DID design when \vfill
\small
\begin{equation*}
\widehat{Y}_{i,t} =
\begin{cases}
\dfrac{\sum_{i: G_{i} = 1} Y_{i,t-1}}{n_{1,t-1}} & \text{if } G_i = 1 \\[10pt]
\dfrac{\sum_{i: G_{i} = 0} Y_{i,t-1}}{n_{0,t-1}} & \text{if } G_i = 0
\end{cases}
\end{equation*} \vfill
\normalsize
for $t = 1, \dots , T$ \vfill
\end{itemize} \vfill
\end{itemize}
\vfill
\end{frame}
%---------------------------------------------------------------
\begin{frame}[t]
\frametitle{General estimation of ATT}
\vfill
\begin{itemize}
\item \textcolor{Dark2-A}{\textbf{General estimator}}: \vfill
\end{itemize}
\vspace{2em}
\begin{figure}[htb]
    \vspace{2\baselineskip}
    \begin{equation*}
        \widehat{\Delta}_t \coloneqq \highlight{Dark2-D}{$\dfrac{\tikzmarknode{sumtreatpost}{\sum}_{i: G_{i} = 1} \tikzmarknode{treatpotout}{Y}_{i,t}}{\tikzmarknode{ntreatpost}{n}_{1,t}}$} - \highlight{Dark2-D}{$\dfrac{\tikzmarknode{sumtreatpre}{\sum}_{i: G_{i} = 1} \tikzmarknode{treatpreout}{\widehat{Y}}_{i,t}}{\tikzmarknode{ntreatpre}{n}_{1,t}}$} - \left(\highlight{Dark2-D}{$\dfrac{\tikzmarknode{sumcontpost}{\sum}_{i: G_{i} = 0} \tikzmarknode{contpostout}{Y}_{i,t}}{\tikzmarknode{ncontpost}{n}_{0,t}}$} - \highlight{Dark2-D}{$\dfrac{\tikzmarknode{sumcontpre}{\sum}_{i: G_{i} = 0} \tikzmarknode{contpreout}{\widehat{Y}}_{i,t}}{\tikzmarknode{ncontpre}{n}_{0,t}}$}\right)
    \end{equation*}
    \begin{tikzpicture}[overlay,remember picture,>=stealth,nodes={align=left,inner ysep=1pt},<-]

        \path (treatpotout.north) ++ (-3.25em,2.3em) node[anchor=south west, color=Dark2-D!67] (scalep){\scriptsize \textsf{Observable outcome}};
        \draw[<->,color=Dark2-D!57] (treatpotout.north) -- ++(0,0.67)  -| node[] {} (treatpotout.north);

        \path (treatpreout.north) ++ (-2em,2.3em) node[anchor = south west, color=Dark2-D!67] (scalep){\scriptsize \textsf{Prediction}};
        \draw[<->, color = Dark2-D!57] (treatpreout.north) -- ++(0,0.67)  -| node[] {} (treatpreout.north);

        % sumtreatpost to sumtreatpre
        \path (sumtreatpost.north) ++ (0em,6.7em) node[anchor=south west, color=Dark2-D!67] (scalep){\scriptsize \textsf{Sum over treated units}};
        \draw[<->,color=Dark2-D!57] (sumtreatpost.north) -- ++(0,2)  -| node[] {} (sumtreatpre.north);

        % ntreatpost to ntreatpre
        \path (ntreatpost.south) ++ (0em,-3.3em) node[anchor=south west,color=Dark2-D!67] (scalep){\scriptsize \textsf{Number of treated units}};
        \draw[<->,color=Dark2-D!57] (ntreatpost.south) -- ++(0,-0.67)  -| node[] {} (ntreatpre.south);

        \path (contpostout.north) ++ (-3.25em,2.3em) node[anchor=south west, color=Dark2-D!67] (scalep){\scriptsize \textsf{Observable outcome}};
        \draw[<->,color=Dark2-D!57] (contpostout.north) -- ++(0,0.67)  -| node[] {} (contpostout.north);

        \path (contpreout.north) ++ (-2em,2.3em) node[anchor=south west, color=Dark2-D!67] (scalep){\scriptsize \textsf{Prediction}};
        \draw[<->,color=Dark2-D!57] (contpreout.north) -- ++(0,0.67)  -| node[] {} (contpreout.north);

        % sumcontpost to sumcontpre
        \path (sumcontpost.north) ++ (-0.5em,6.7em) node[anchor=south west, color=Dark2-D!67] (scalep){\scriptsize \textsf{Sum over untreated units}};
        \draw[<->,color=Dark2-D!57] (sumcontpost.north) -- ++(0,2)  -| node[] {} (sumcontpre.north);

        % ncontpost to ncontpre
        \path (ncontpost.south) ++ (-0.5em,-3.3em) node[anchor=south west,color=Dark2-D!67] (scalep){\scriptsize \textsf{Number of untreated units}};
        \draw[<->,color=Dark2-D!57] (ncontpost.south) -- ++(0,-0.67)  -| node[] {} (ncontpre.south);

    \end{tikzpicture}
    \vspace{\baselineskip}
\end{figure} \vfill
where $\widehat{Y}_{i,t} = \hat{y}_0\left(\bm{X}_{i,t}\right)$ \vfill 
\begin{itemize} \vfill
\item $\hat{y}_0\left(\cdot\right)$ is arbitrary function that predicts untreated POs \vfill
\item $\bm{X}_{i,t}$ is vector of predictors measured prior to period $t$ \vfill 
\end{itemize}
\end{frame}
%---------------------------------------------------------------
\begin{frame}[t]
\frametitle{General inference of ATT}
\vfill
\begin{itemize}
\item Conformal inference to account for predictive uncertainty \vfill
\item[] \citep{vovketal2005,shafervovk2008,chernozhukovetal2021} \vfill
\item Usual sampling-based inference to account for sampling uncertainty \vfill
\end{itemize}
\vfill
\end{frame}
%----------------------------------------------------------------
\begin{frame}[allowframebreaks]
\frametitle{References} 
\scriptsize
\bibliographystyle{chicago}
\bibliography{DID_bibliography}   % name your BibTeX data base
\end{frame}
%----------------------------------------------------------------
\end{document}
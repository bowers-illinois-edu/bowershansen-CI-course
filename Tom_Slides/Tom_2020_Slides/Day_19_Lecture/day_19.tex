%%%%%%%%%%%%%%%%%%%%%%%%%%%%%%%%%%%%%%%%%
% Beamer Presentation
% LaTeX Template
% Version 1.0 (10/11/12)
%
% This template has been downloaded from:
% http://www.LaTeXTemplates.com
%
% License:
% CC BY-NC-SA 3.0 (http://creativecommons.org/licenses/by-nc-sa/3.0/)
%
%%%%%%%%%%%%%%%%%%%%%%%%%%%%%%%%%%%%%%%%%

%----------------------------------------------------------------------------------------
% PACKAGES AND THEMES
%----------------------------------------------------------------------------------------

\documentclass[table, xcolor={dvipsnames}, 9pt]{beamer}
\usepackage{tikz}
\usetikzlibrary{positioning}
\mode<presentation> {

% The Beamer class comes with a number of default slide themes
% which change the colors and layouts of slides. Below this is a list
% of all the themes, uncomment each in turn to see what they look like.

%\usetheme{default}
%\usetheme{AnnArbor}
%\usetheme{Antibes}
%\usetheme{Bergen}
%\usetheme{Berkeley}
%\usetheme{Berlin}
%\usetheme{Boadilla}
%\usetheme{CambridgeUS}
%\usetheme{Copenhagen}
%\usetheme{Darmstadt}
%\usetheme{Dresden}
%\usetheme{Frankfurt}
%\usetheme{Goettingen}
%\usetheme{Hannover}
%\usetheme{Ilmenau}
%\usetheme{JuanLesPins}
%\usetheme{Luebeck}
% \usetheme{Madrid}
\usetheme{metropolis}
%\usetheme{Malmoe}
%\usetheme{Marburg}
%\usetheme{Montpellier}
%\usetheme{PaloAlto}
%\usetheme{Pittsburgh}
%\usetheme{Rochester}
%\usetheme{Singapore}
%\usetheme{Szeged}
%\usetheme{Warsaw}

% As well as themes, the Beamer class has a number of color themes
% for any slide theme. Uncomment each of these in turn to see how it
% changes the colors of your current slide theme.

%\usecolortheme{albatross}
%\usecolortheme{beaver}
%\usecolortheme{beetle}
%\usecolortheme{crane}
%\usecolortheme{dolphin}
%\usecolortheme{dove}
%\usecolortheme{fly}
%\usecolortheme{lily}
%\usecolortheme{orchid}
%\usecolortheme{rose}
%\usecolortheme{seagull}
%\usecolortheme{seahorse}
%\usecolortheme{whale}
%\usecolortheme{wolverine}

%\setbeamertemplate{footline} % To remove the footer line in all slides uncomment this line
%\setbeamertemplate{footline}[page number] % To replace the footer line in all slides with a simple slide count uncomment this line

%\setbeamertemplate{navigation symbols}{} % To remove the navigation symbols from the bottom of all slides uncomment this line
}
\setbeamertemplate{footline}{}
\addtobeamertemplate{footnote}{}{\vspace{24pt}}
\usepackage{graphicx} % Allows including images
\usepackage{booktabs} % Allows the use of \toprule, \midrule and \bottomrule in tables
\usepackage{multirow}
\usepackage{natbib}
\usepackage[]{hyperref}
\usepackage{diagbox}
\usepackage{makecell}
\usepackage{subfig}
\usepackage{amsmath}
\usepackage{amsfonts,amsthm,amsmath,amssymb}    
\usepackage{bbm}
\usepackage{bm}
\usepackage{empheq}
\makeatletter
\let\save@measuring@true\measuring@true
\def\measuring@true{%
  \save@measuring@true
  \def\beamer@sortzero##1{\beamer@ifnextcharospec{\beamer@sortzeroread{##1}}{}}%
  \def\beamer@sortzeroread##1<##2>{}%
  \def\beamer@finalnospec{}%
}
\makeatother
\hypersetup{unicode=true,
            pdfusetitle,
            bookmarks=true,
            bookmarksnumbered=true,
            bookmarksopen=true,
            bookmarksopenlevel=2,
            breaklinks=false,
            pdfborder={0 0 1},
            backref=true,
            hypertexnames=false,
            pdfstartview={XYZ null null 1}}
\usepackage{xcolor}
\newcommand\myheading[1]{%
  \par\bigskip
  {\Large\bfseries#1}\par\smallskip}
\newcommand\given[1][]{\:#1\vert\:}
\theoremstyle{newstyle}
\newtheorem{thm}{Theorem}
\newtheorem{prop}[thm]{Proposition}
\newtheorem{lem}{Lemma}
\newtheorem{cor}{Corollary}
\newtheorem{defin}{Definition}
\newcommand*\diff{\mathop{}\!\mathrm{d}}
\newcommand*\Diff[1]{\mathop{}\!\mathrm{d^#1}}
\newcommand*{\QEDA}{\hfill\ensuremath{\blacksquare}}%
\newcommand*{\QEDB}{\hfill\ensuremath{\square}}%
\DeclareMathOperator{\E}{\mathrm{E}}
\DeclareMathOperator{\R}{\mathbb{R}}
\DeclareMathOperator{\Var}{\rm{Var}}
\DeclareMathOperator{\Cov}{\rm{Cov}}
\DeclareMathOperator{\e}{\rm{e}}
\DeclareMathOperator{\logit}{\rm{logit}}
\DeclareMathOperator{\indep}{{\perp\!\!\!\perp}}
%\DeclareMathOperator{\Pr}{\rm{Pr}}
\newenvironment{Column}[1][.5\linewidth]{\begin{column}{#1}}{\end{column}}
%----------------------------------------------------------------------------------------
% TITLE PAGE
%----------------------------------------------------------------------------------------

\title[]{Difference-in-Differences} % The short title appears at the bottom of every slide, the full title is only on the title page

\author{Thomas Leavitt} % Your name
\institute[] % Your institution as it will appear on the bottom of every slide, may be shorthand to save space
{
% Your institution for the title page
\medskip
\textit{} % Your email address
}
\date{\today} % Date, can be changed to a custom date

\begin{document}

\begin{frame}
\titlepage % Print the title page as the first slide
\end{frame}

%\begin{frame}
%\frametitle{Overview} % Table of contents slide, comment this block out to remove it
%\tableofcontents % Throughout your presentation, if you choose to use \section{} and \subsection{} commands, these will automatically be printed on this slide as an overview of your presentation
%\end{frame}

%------------------------------------------------------------------------
% PRESENTATION SLIDES
%------------------------------------------------------------------------
\section{Introduction}
\begin{frame}{Difference-in-Differences}
\begin{itemize}
\item Difference-in-Difference (DID) is one of most popuar \textit{design-based} methods
\item \pause Inference based on ability to accurately predict treated units' unobserved counterfactual outcomes
\item \pause Uses after-minus-before difference in control group to predict how treated group would have changed had treatment not occurred  
\item \pause Inference is \textit{not} based on the assumption that control units could have been treated and treated units could have been untreated
\item \pause DID is a design \textit{without controls} --- i.e., controls were ``were never eligible for treatment'' \citep[][155]{rosenbaum2017}
\end{itemize}
\end{frame}
%------------------------------------------------------------------------
\begin{frame}{Difference-in-Differences}
\begin{figure}[H]
\includegraphics[width = \linewidth]{DID_plot.pdf}
\end{figure}
\end{frame}
%------------------------------------------------------------------------
\begin{frame}{Formal framework}
\begin{itemize}
\item Consider a population, $\mathcal{P}$, of $N$ units that belong to one of two groups, $G = 0$ and $G = 1$, which are of sizes $N_0$ and $N_1$
\item \pause All units bear measurements over $T$ time periods, $t \in \left\{1, \dots , T\right\}$
\item \pause All units in $\mathcal{P}$ are untreated in time periods $t \in \left\{1, \dots , T-1\right\}$
\item \pause In time period $T$, group $G = 1$ is treated and group $G = 0$ is not such that $G = Z_T$ 
\item \pause The ATT in the population, $\text{ATT}_{\mathcal{P}}$, is $\left(\mu_{1T} \given Z_T = 1\right) - \left(\mu_{0T} \given Z_T = 1\right)$
\item \pause We have an unbiased estimator for $\left(\mu_{1T} \given Z_T = 1\right)$, but not for $\left(\mu_{0T} \given Z_T = 1\right)$
\end{itemize}	
\end{frame}
%------------------------------------------------------------------------
\begin{frame}{Causal identification}
\begin{itemize}
\item Let $\mathcal{S}$ be an i.i.d. sample of size $n$ from the population $\mathcal{P}$ that is stratified by $G$, where $n_0$ and $n_1$ are the fixed numbers of sampled units from $G = 0$ and $G = 1$
\item \pause Define DID estimator as difference between
\begin{enumerate}
\item \pause after-minus-before sample mean among treated units
\item \pause after-minus-before sample mean among control units
\end{enumerate} \pause 
\begin{align*}
\widehat{\text{DID}}_{\mathcal{P}} = \left(\frac{1}{n_1}\right)\sum_{i = n_0 + 1}^{n} \left[Y_{iT}(1) - Y_{iT-1}\right] - \left[\left(\frac{1}{n_0}\right)\sum_{i = 1}^{n_0} \left[Y_{iT} - Y_{iT-1}\right]\right]
\end{align*}
\item \pause DID estimator is unbiased for \textit{descriptive difference} between treated and control populations:
\begin{align*} \pause 
\E\left[\widehat{\text{DID}}_{\mathcal{P}}\right] & = \left(\mu_{1T} - \mu_{T-1} \given Z_T = 1\right) - \left(\mu_{T} - \mu_{T-1} \given Z_T = 0\right)
\end{align*}
\item \pause Under what conditions does this descriptive difference equal the ATT?
\end{itemize}	
\end{frame}
%------------------------------------------------------------------------
\begin{frame}{Parallel trends assumption}
\begin{itemize}
\item Parallel trends assumption: \pause
\begin{align*}  
\underbrace{\left(\mu_{0T}\given Z_T = 1\right)}_{\text{Counterfactual}} - \left(\mu_{T-1} \given Z_T = 1\right) & = \left(\mu_{T} \given Z_T = 0\right) - \left(\mu_{T-1} \given Z_T = 0\right)
\end{align*}
\item \pause Rearranging yields
\begin{align*}
\underbrace{\left(\mu_{0T}\given Z_T = 1\right)}_{\text{Counterfactual}} - \left(\mu_{T-1} \given Z_T = 1\right) + \left(\mu_{T-1} \given Z_T = 0\right) & = \left(\mu_{T} \given Z_T = 0\right)
\end{align*}
\item \pause Substitute for $\left(\mu_{T} \given Z_T = 0\right)$
\item \pause Then simple algebra yields \pause 
\begin{align*}
\underbrace{\left(\mu_{1T} - \mu_{T-1} \given Z_T = 1\right) - \left(\mu_{T} - \mu_{T-1} \given Z_T = 0\right)}_{ = \E\left[\widehat{\text{DID}}_{\mathcal{P}}\right]} & = \underbrace{\left(\mu_{1T} - \mu_{0T} \given Z_T = 1\right)}_{ = \text{ATT}_{\mathcal{P}}}
\end{align*}
\item \pause Hence, $\widehat{\text{DID}}_{\mathcal{P}}$ unbiased for $\text{ATT}_{\mathcal{P}}$
\end{itemize}
\end{frame}
%------------------------------------------------------------------------
\begin{frame}{Parallel trends assumption}
\begin{figure}[H]
\includegraphics[width = \linewidth]{DID_no_pt_plot.pdf}
\end{figure}
\end{frame}
%------------------------------------------------------------------------
\section{Problems with standard DID}
\begin{frame}{Problems with standard DID}
\begin{enumerate}
\item Scale dependence problem
\item \pause Statistical uncertainty
\end{enumerate}
\end{frame}
%------------------------------------------------------------------------
\begin{frame}{Scale dependence}
\begin{itemize}
\item[]	
\begin{figure}[H]
\includegraphics[width = \linewidth]{standard_scale_DID_plot.pdf}
\end{figure}
\item $\text{ATT}_{\mathcal{P}} \approx 12.08$
\item $\E\left[\widehat{\text{DID}}_{\mathcal{P}}\right] \approx 21.01$	
\end{itemize}
\end{frame}
%------------------------------------------------------------------------
\begin{frame}{Scale dependence problem}
\begin{itemize}
\item[]	
\begin{figure}[H]
\includegraphics[width = \linewidth]{log_scale_DID_plot.pdf}
\end{figure}
\item $\text{ATT}_{\mathcal{P}} = 0.25$
\item $\E\left[\widehat{\text{DID}}_{\mathcal{P}}\right] = 0.25$	
\end{itemize}
\end{frame}
%------------------------------------------------------------------------
\begin{frame}{Scale dependence problem}
\begin{center}
``It makes no sense, scientifically or mathematically, to say that the null hypothesis of no treatment effect is false for a response, $R_i$, but true for its logarithm, $\log\left(R_i\right)$'' \citep{rosenbaum2017}
\end{center}
\end{frame}
%------------------------------------------------------------------------
\section{Statistical uncertainty}
\begin{frame}{}
\begin{itemize}
\item Two types of design-based inference: 
\begin{enumerate}
\item[(1)] \pause  Descriptive inference:
\begin{itemize}
\item[] \pause $\text{Sample} \to \text{Population}$ based on sampling mechanism  
\end{itemize}
\item[(2)] \pause Causal inference: 
\begin{itemize}
\item[] \pause  $\text{Observed outcomes} \to \text{counterfactual outcomes}$ based on assignment mechanism 
\end{itemize}
\end{enumerate}	
\item \pause  Causal uncertainty is difficult to represent: 
\begin{itemize}
\item \pause DID design's validity depends on parallel trends, \textit{not} assignment mechanism  
\item No analogy with randomized experiment 
\end{itemize}	
\item Inference reflects only sampling uncertainty
\item To see this point, ask what is our uncertainty when our target is the ATT in our sample?
\end{itemize}
\end{frame}
%------------------------------------------------------------------------
\section{Decomposition of DID uncertainty}
\begin{frame}
\frametitle{Decomposition of DID uncertainty}
\begin{itemize}
\item Define $\Delta_{\mathcal{P}}$ as difference in counterfactual trends: \pause 
\begin{align*}
\Delta_{\mathcal{P}} = \underbrace{\E\left[Y_{iT}(0) \given z_{iT} = 1\right]}_{\text{Inestimable}} - \underbrace{\E\left[Y_{iT-1} \given z_{iT} = 1\right]}_{\text{Estimable}} - \\ \left[\underbrace{\E\left[Y_{iT} \given z_{iT} = 0\right]}_{\text{Estimable}} - \underbrace{\E\left[Y_{iT-1} \given z_{iT} = 0\right]}_{\text{Estimable}}\right]
\end{align*} 
\item \pause $\E\left[\widehat{\text{DID}}_{\mathcal{P}}\right] - \Delta_{\mathcal{P}} = \text{ATT}_{\mathcal{P}}$
\item \pause Parallel trends is special case in which \normalsize $\Delta_{\mathcal{P}} = 0$ \large
\item \pause Decompose $\text{ATT}_{\mathcal{P}}$  as $\text{ATT}_{\mathcal{P}} = \underbrace{\E\left[\widehat{\text{DID}}_{\mathcal{P}}\right]}_{\substack{\text{Sampling} \\ \text{uncertainty}}} - \underbrace{\Delta_{\mathcal{P}}}_{\substack{\text{Counterfactual} \\ \text{uncertainty}}}$ 
\item \pause How do we account for counterfactual uncertainty?
\end{itemize}
\end{frame}
%------------------------------------------------------------------------
\section{Counterfactual uncertainty}
\begin{frame}
\frametitle{Counterfactual uncertainty}
\begin{itemize}
\item Propose new finite sample estimator: $\widehat{\text{DID}}_{\mathcal{S}} - \Delta^{\text{imp}}_{\mathcal{S}}$
\begin{itemize}
\item \pause Assume target is ATT in finite sample such that \pause 
\begin{align*}
\text{ATT}_{\mathcal{S}} = \left(\frac{1}{n_1}\right) \sum \limits_{i = n_0 + 1}^n \left(y_{iT}(1) - y_{iT}(0)\right)
\end{align*}
and \pause
\begin{align*}
\Delta_{\mathcal{S}} = \underbrace{\left(\frac{1}{n_1}\right)\sum_{i = n_0 + 1}^{n} y_{iT}(0)}_{\text{unobserved}} - \underbrace{\left(\frac{1}{n_1}\right)\sum_{i = n_0 + 1}^{n} y_{iT-1}}_{\text{observed}} - \\ \left[\underbrace{\left(\frac{1}{n_0}\right)\sum_{i = 1}^{n_0} y_{iT}}_{\text{observed}} - \underbrace{\left(\frac{1}{n_0}\right)\sum_{i = 1}^{n_0} y_{iT-1}}_{\text{observed}}\right]
\end{align*} 
\item \pause Distribution on predictions of $\left(\frac{1}{n_1}\right)\sum \limits_{i = n_0 + 1}^{n} y_{iT}(0) \implies$ distribution on $\Delta_{\mathcal{S}}$
\item \pause Fit predictive models to pre-treatment data in treated and control groups and then impute for $\left(\frac{1}{n_1}\right)\sum \limits_{i = n_0 + 1}^{n} y_{iT}(0)$
\end{itemize}
\end{itemize}
\end{frame}
%------------------------------------------------------------------------
\begin{frame}
\frametitle{Counterfactual uncertainty}
\begin{itemize}
\item Uncertainty represented via quasi Bayesian approach \citep[][chapter 7]{gelmanhill2006}
\begin{itemize}
\item \pause Intuitively, random distribution of plausible projected counterfactuals based on pre-treatment trends
\item \pause Formally, data-driven prior distribution on nuisance parameter, $\Delta_{\mathcal{S}}$
\end{itemize}	
\item \pause Uncertainty in counterfactual predictions decreasing in 
\begin{enumerate}
\item \pause Stability of pre-treatment trends 
\item \pause Amount of pre-treatment data
\end{enumerate} 
\item \pause Derive conditions for unbiasedness of Bayes' estimator under conditions more general than parallel trends
\item \pause Show robustness to violations of conditions
\end{itemize}
\end{frame}
%------------------------------------------------------------------------
\begin{frame}[allowframebreaks]
\frametitle{References} 
\scriptsize
\bibliographystyle{chicago}
\bibliography{master_bibliography}   % name your BibTeX data base
\end{frame}
%------------------------------------------------------------------------
\end{document}
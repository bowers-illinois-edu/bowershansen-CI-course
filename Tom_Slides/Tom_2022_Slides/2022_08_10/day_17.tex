%%%%%%%%%%%%%%%%%%%%%%%%%%%%%%%%%%%%%%%%%
% Beamer Presentation
% LaTeX Template
% Version 1.0 (10/11/12)
%
% This template has been downloaded from:
% http://www.LaTeXTemplates.com
%
% License:
% CC BY-NC-SA 3.0 (http://creativecommons.org/licenses/by-nc-sa/3.0/)
%
%%%%%%%%%%%%%%%%%%%%%%%%%%%%%%%%%%%%%%%%%

%----------------------------------------------------------------------------------------
% PACKAGES AND THEMES
%----------------------------------------------------------------------------------------

\documentclass[table, xcolor={dvipsnames}, 9pt]{beamer}
\usepackage{tikz}
\usetikzlibrary{positioning}
\mode<presentation> {

% The Beamer class comes with a number of default slide themes
% which change the colors and layouts of slides. Below this is a list
% of all the themes, uncomment each in turn to see what they look like.

%\usetheme{default}
%\usetheme{AnnArbor}
%\usetheme{Antibes}
%\usetheme{Bergen}
%\usetheme{Berkeley}
%\usetheme{Berlin}
%\usetheme{Boadilla}
%\usetheme{CambridgeUS}
%\usetheme{Copenhagen}
%\usetheme{Darmstadt}
%\usetheme{Dresden}
%\usetheme{Frankfurt}
%\usetheme{Goettingen}
%\usetheme{Hannover}
%\usetheme{Ilmenau}
%\usetheme{JuanLesPins}
%\usetheme{Luebeck}
% \usetheme{Madrid}
\usetheme{metropolis}
%\usetheme{Malmoe}
%\usetheme{Marburg}
%\usetheme{Montpellier}
%\usetheme{PaloAlto}
%\usetheme{Pittsburgh}
%\usetheme{Rochester}
%\usetheme{Singapore}
%\usetheme{Szeged}
%\usetheme{Warsaw}

% As well as themes, the Beamer class has a number of color themes
% for any slide theme. Uncomment each of these in turn to see how it
% changes the colors of your current slide theme.

%\usecolortheme{albatross}
%\usecolortheme{beaver}
%\usecolortheme{beetle}
%\usecolortheme{crane}
%\usecolortheme{dolphin}
%\usecolortheme{dove}
%\usecolortheme{fly}
%\usecolortheme{lily}
%\usecolortheme{orchid}
%\usecolortheme{rose}
%\usecolortheme{seagull}
%\usecolortheme{seahorse}
%\usecolortheme{whale}
%\usecolortheme{wolverine}

%\setbeamertemplate{footline} % To remove the footer line in all slides uncomment this line
%\setbeamertemplate{footline}[page number] % To replace the footer line in all slides with a simple slide count uncomment this line

%\setbeamertemplate{navigation symbols}{} % To remove the navigation symbols from the bottom of all slides uncomment this line
}
\setbeamertemplate{footline}{}
\addtobeamertemplate{footnote}{}{\vspace{24pt}}
\usepackage{graphicx} % Allows including images
\usepackage{booktabs} % Allows the use of \toprule, \midrule and \bottomrule in tables
\usepackage{multirow}
\usepackage{natbib}
\usepackage[]{hyperref}
\usepackage{diagbox}
\usepackage{makecell}
\usepackage{subfig}
\usepackage{amsmath}
\usepackage{amsfonts,amsthm,amsmath,amssymb}    
\usepackage{bbm}
\usepackage{bm}
\usepackage{empheq}
\makeatletter
\let\save@measuring@true\measuring@true
\def\measuring@true{%
  \save@measuring@true
  \def\beamer@sortzero##1{\beamer@ifnextcharospec{\beamer@sortzeroread{##1}}{}}%
  \def\beamer@sortzeroread##1<##2>{}%
  \def\beamer@finalnospec{}%
}
\makeatother
\hypersetup{unicode=true,
            pdfusetitle,
            bookmarks=true,
            bookmarksnumbered=true,
            bookmarksopen=true,
            bookmarksopenlevel=2,
            breaklinks=false,
            pdfborder={0 0 1},
            backref=true,
            hypertexnames=false,
            pdfstartview={XYZ null null 1}}
\usepackage{xcolor}
\newcommand\myheading[1]{%
  \par\bigskip
  {\Large\bfseries#1}\par\smallskip}
\newcommand\given[1][]{\:#1\vert\:}
\theoremstyle{newstyle}
\newtheorem{thm}{Theorem}
\newtheorem{prop}[thm]{Proposition}
\newtheorem{lem}{Lemma}
\newtheorem{cor}{Corollary}
\newtheorem{defin}{Definition}
\newcommand*\diff{\mathop{}\!\mathrm{d}}
\newcommand*\Diff[1]{\mathop{}\!\mathrm{d^#1}}
\newcommand*{\QEDA}{\hfill\ensuremath{\blacksquare}}%
\newcommand*{\QEDB}{\hfill\ensuremath{\square}}%
\DeclareMathOperator{\E}{\mathrm{E}}
\DeclareMathOperator{\R}{\mathbb{R}}
\DeclareMathOperator{\Var}{\rm{Var}}
\DeclareMathOperator{\Cov}{\rm{Cov}}
\DeclareMathOperator{\e}{\rm{e}}
\DeclareMathOperator{\logit}{\rm{logit}}
\DeclareMathOperator{\indep}{{\perp\!\!\!\perp}}
%\DeclareMathOperator{\Pr}{\rm{Pr}}
\newenvironment{Column}[1][.5\linewidth]{\begin{column}{#1}}{\end{column}}
%----------------------------------------------------------------------------------------
% TITLE PAGE
%----------------------------------------------------------------------------------------

\title[]{Sensitivity analysis} % The short title appears at the bottom of every slide, the full title is only on the title page

\author{Thomas Leavitt} % Your name
\institute[] % Your institution as it will appear on the bottom of every slide, may be shorthand to save space
{
% Your institution for the title page
\medskip
\textit{} % Your email address
}
\date{August 10, 2022} % Date, can be changed to a custom date

\begin{document}

\begin{frame}
\titlepage % Print the title page as the first slide
\end{frame}

%\begin{frame}
%\frametitle{Overview} % Table of contents slide, comment this block out to remove it
%\tableofcontents % Throughout your presentation, if you choose to use \section{} and \subsection{} commands, these will automatically be printed on this slide as an overview of your presentation
%\end{frame}

%------------------------------------------------------------------------
% PRESENTATION SLIDES
%------------------------------------------------------------------------
\section{Conservative sensitivity analysis}
\begin{frame}{Conservative sensitivity analysis in stratified studies}
\vfill
\begin{itemize} \vfill
\item Let the index $b \in \left\{1, \ldots, B\right\}$ runs over the $B$ strata \vfill
\item  Then order $n$ units lexically by stratum and then in decreasing order within strata by value of outcome $y_{bi}$.
\begin{itemize} \vfill
\item  E.g., in \texttt{R}: \texttt{data <- dplyr::arrange(.data = data, strata, desc(y))} \vfill
\end{itemize}	 \vfill
\item  The worst-case $\mathbf{u} \in \mathcal{U}^+$ cannot be found by finding $\mathbf{u}_b \in \mathcal{U}_b^+$ in each stratum one at a time \vfill
\item  All $n$ coordinates of $\mathbf{u} \in \mathcal{U}^+$ must be found simultaneously \vfill
\item  Hence, there are $\prod \limits_{b = 1}^B \left(n_b - 1\right)$ candidate values to consider \vfill
\begin{itemize} \vfill
\item  Usually computationally intractable, even in relatively small studies  \vfill
\end{itemize} \vfill
\item So, in practice, we can use asymptotic approximation \\ \citep{gastwirthetal2000,rosenbaum2018} \vfill
\end{itemize} \vfill
\end{frame}
%----------------------------------------------------------------
\begin{frame}{Asymptotic separability algorithm}
\vfill
\begin{itemize} \vfill
\item An asymptotically valid upper bound to $p$-value is attained by finding $\bm{u}_b$ in each stratum one at a time \citep{gastwirthetal2000} \vfill
\begin{enumerate} \vfill
\item In each stratum, find $\bm{u}_b$ that maximizes expectation of test-stat under null \vfill
\item If multiple $\bm{u}_b$ yield same expectation, choose $\bm{u}_b$ that yields largest variance \vfill
\item[] \textbf{Intuition}: Put as much mass as possible in upper tail of null distribution \vfill
\end{enumerate} \vfill
\item[$\star$] ``Separable'' refers to finding overall ``worst-case'' $\bm{u} = \begin{bmatrix} \bm{u}_1 & \ldots & \bm{u}_B \end{bmatrix}^{\top}$ by finding ``worst case'' $\bm{u}_b$ in each stratum \textbf{separately} \vfill
\end{itemize} \vfill
\end{frame}
%----------------------------------------------------------------
\begin{frame}{Asymptotic separability algorithm}
\vfill
\begin{itemize} \vfill
\item For any matched design, the $\bm{u}$ this procedure picks may not yield the largest $p$-value, but ... \vfill
\item As $B \to \infty$ with bounded $\max n_b$, there is \textbf{negligible difference} between \vfill
\begin{enumerate}
\item $p$-value from true ``worst case'' $\bm{u}$ and \vfill
\item $p$-value from $\bm{u}$ picked by separable algorithm \vfill
\end{enumerate} \vfill
\item Separable algorithm yields asymptotic rejection probability $\leq \alpha$ for all configurations of $\bm{u} \in \mathcal{U}$ when sensitivity models holds at $\Gamma \geq 1$ and sharp null is true \vfill
\begin{itemize} \vfill
\item I.e., controls Type I error probability for any given random assignment violation characterized by $\Gamma \geq 1$ \vfill
\end{itemize} \vfill
\item Algorithm is used in \texttt{sensitivitymw} and \texttt{sensitivitymv} \texttt{R} packages \citep{rosenbaum2015} \vfill
\end{itemize} \vfill
\end{frame}
%----------------------------------------------------------------
\begin{frame}{Asymptotic separability algorithm}
\vfill
\begin{itemize} \vfill
\item Why is asymptotic separability helpful? \vfill
\begin{itemize} \vfill
\item With data from \citet{cerdaetal2012}, we made following matched design: \vfill
\vspace{2em}
\begin{table}[H]
\begin{tabular}{l|l|l|l|l}
\textbf{Structure} & $8:1$ & $7:1$ & $1:1$ & $1:16$ \\ \hline 
\textbf{No.~of strata} & $1$   & $1$   & $4$   & $1$ \\
\end{tabular}
\caption{Structure of matched sets}
\end{table} \vfill
\item \textbf{Without} separability, total number of candidate $\bm{u}$s is \vfill
\begin{align*}
(9 - 1) \times (8 - 1) \times (2 - 1) \times (2 - 1) \times (2 - 1) \times (2 - 1) \times (17 - 1) & = 896
\end{align*} \vfill
\item \textbf{With} separability, total number of candidate $\bm{u}$s is \vfill
\begin{align*}
(9 - 1) + (8 - 1) + (2 - 1) + (2 - 1) + (2 - 1) + (2 - 1) + (17 - 1) & = 35
\end{align*} \vfill
\item With many strata, \vfill
\item[] separability $\rightarrow$ finding (approx.) ``worst case'' $\bm{u}$ is computationally tractable! \vfill
\end{itemize} \vfill
\end{itemize} \vfill
\end{frame}
%----------------------------------------------------------------
\begin{frame}{Difficulties of asymptotic separability algorithm}
\vfill
\begin{itemize} \vfill
\item What do we do when asymptotic approximation is poor and exact solution computationally intractable? \vfill
\begin{itemize} \vfill
\item E.g., $B = 10$ and $n_b = 101$ for each $b$ \vfill
\item[] There are $100^{10}$ candidate values of $\bm{u} \in \mathcal{U}^+$ \vfill
\item[] Separable algorithm assesses only $100 \times 10 = 1000$ candidate vectors \vfill
\item[] But separable algorithm works best with many small strata \vfill
\end{itemize} \vfill
\item \citet{rosenbaum2018} to the rescue!
\end{itemize} \vfill
\end{frame}
%----------------------------------------------------------------
\begin{frame}{An alternative approximation}
\vfill
\begin{itemize} \vfill
\item What does \citet{rosenbaum2018} do? \vfill
\begin{enumerate} \vfill
\item In any matched design, separable algorithm is \textbf{always} (slightly) liberal \vfill
\begin{itemize} \vfill
\item I.e., finds $\bm{u}$ with $p$-value (slightly) less than $p$-value from true ``worst case'' $\bm{u}$ \vfill
\item Algorithm's $p$-value \textbf{never} greater than true ``worst case'' $p$-value \vfill
\end{itemize} \vfill
\item Alternative Taylor approximation algorithm is \textbf{always} (slightly) conservative \vfill
\begin{itemize} \vfill
\item I.e., finds $\bm{u}$ with $p$-value (slightly) greater than $p$-value from true ``worst case'' $\bm{u}$ \vfill
\item Algorithm's $p$-value \textbf{never} less than than true ``worst case'' $p$-value \vfill
\end{itemize} \vfill
\end{enumerate} \vfill
\item So we use conservative algorithm for sensitivity analysis and compare it to separable algorithm to check that former is not \textbf{too} conservative \vfill
\begin{itemize} \vfill
\item In seemingly all examples, $p$-values from the two algorithms differ negligibly \vfill
\item Easy to implement via \texttt{senstrat} package in \texttt{R} \vfill
\end{itemize} \vfill
\end{itemize} \vfill
\end{frame}
%----------------------------------------------------------------
\begin{frame}[allowframebreaks]
\frametitle{References} 
\scriptsize
\bibliographystyle{chicago}
\bibliography{sensitivity_analysis}   % name your BibTeX data base
\end{frame}
%----------------------------------------------------------------
\end{document}

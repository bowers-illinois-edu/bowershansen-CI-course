% Options for packages loaded elsewhere
\PassOptionsToPackage{unicode}{hyperref}
\PassOptionsToPackage{hyphens}{url}
%
\documentclass[
  12pt,
  leqno]{article}
\usepackage{amsmath,amssymb}
\usepackage{iftex}
\ifPDFTeX
  \usepackage[T1]{fontenc}
  \usepackage[utf8]{inputenc}
  \usepackage{textcomp} % provide euro and other symbols
\else % if luatex or xetex
  \usepackage{unicode-math} % this also loads fontspec
  \defaultfontfeatures{Scale=MatchLowercase}
  \defaultfontfeatures[\rmfamily]{Ligatures=TeX,Scale=1}
\fi
\usepackage{lmodern}
\ifPDFTeX\else
  % xetex/luatex font selection
\fi
% Use upquote if available, for straight quotes in verbatim environments
\IfFileExists{upquote.sty}{\usepackage{upquote}}{}
\IfFileExists{microtype.sty}{% use microtype if available
  \usepackage[]{microtype}
  \UseMicrotypeSet[protrusion]{basicmath} % disable protrusion for tt fonts
}{}
\makeatletter
\@ifundefined{KOMAClassName}{% if non-KOMA class
  \IfFileExists{parskip.sty}{%
    \usepackage{parskip}
  }{% else
    \setlength{\parindent}{0pt}
    \setlength{\parskip}{6pt plus 2pt minus 1pt}}
}{% if KOMA class
  \KOMAoptions{parskip=half}}
\makeatother
\usepackage{xcolor}
\usepackage[margin = 1.5cm]{geometry}
\usepackage{color}
\usepackage{fancyvrb}
\newcommand{\VerbBar}{|}
\newcommand{\VERB}{\Verb[commandchars=\\\{\}]}
\DefineVerbatimEnvironment{Highlighting}{Verbatim}{commandchars=\\\{\}}
% Add ',fontsize=\small' for more characters per line
\usepackage{framed}
\definecolor{shadecolor}{RGB}{248,248,248}
\newenvironment{Shaded}{\begin{snugshade}}{\end{snugshade}}
\newcommand{\AlertTok}[1]{\textcolor[rgb]{0.94,0.16,0.16}{#1}}
\newcommand{\AnnotationTok}[1]{\textcolor[rgb]{0.56,0.35,0.01}{\textbf{\textit{#1}}}}
\newcommand{\AttributeTok}[1]{\textcolor[rgb]{0.13,0.29,0.53}{#1}}
\newcommand{\BaseNTok}[1]{\textcolor[rgb]{0.00,0.00,0.81}{#1}}
\newcommand{\BuiltInTok}[1]{#1}
\newcommand{\CharTok}[1]{\textcolor[rgb]{0.31,0.60,0.02}{#1}}
\newcommand{\CommentTok}[1]{\textcolor[rgb]{0.56,0.35,0.01}{\textit{#1}}}
\newcommand{\CommentVarTok}[1]{\textcolor[rgb]{0.56,0.35,0.01}{\textbf{\textit{#1}}}}
\newcommand{\ConstantTok}[1]{\textcolor[rgb]{0.56,0.35,0.01}{#1}}
\newcommand{\ControlFlowTok}[1]{\textcolor[rgb]{0.13,0.29,0.53}{\textbf{#1}}}
\newcommand{\DataTypeTok}[1]{\textcolor[rgb]{0.13,0.29,0.53}{#1}}
\newcommand{\DecValTok}[1]{\textcolor[rgb]{0.00,0.00,0.81}{#1}}
\newcommand{\DocumentationTok}[1]{\textcolor[rgb]{0.56,0.35,0.01}{\textbf{\textit{#1}}}}
\newcommand{\ErrorTok}[1]{\textcolor[rgb]{0.64,0.00,0.00}{\textbf{#1}}}
\newcommand{\ExtensionTok}[1]{#1}
\newcommand{\FloatTok}[1]{\textcolor[rgb]{0.00,0.00,0.81}{#1}}
\newcommand{\FunctionTok}[1]{\textcolor[rgb]{0.13,0.29,0.53}{\textbf{#1}}}
\newcommand{\ImportTok}[1]{#1}
\newcommand{\InformationTok}[1]{\textcolor[rgb]{0.56,0.35,0.01}{\textbf{\textit{#1}}}}
\newcommand{\KeywordTok}[1]{\textcolor[rgb]{0.13,0.29,0.53}{\textbf{#1}}}
\newcommand{\NormalTok}[1]{#1}
\newcommand{\OperatorTok}[1]{\textcolor[rgb]{0.81,0.36,0.00}{\textbf{#1}}}
\newcommand{\OtherTok}[1]{\textcolor[rgb]{0.56,0.35,0.01}{#1}}
\newcommand{\PreprocessorTok}[1]{\textcolor[rgb]{0.56,0.35,0.01}{\textit{#1}}}
\newcommand{\RegionMarkerTok}[1]{#1}
\newcommand{\SpecialCharTok}[1]{\textcolor[rgb]{0.81,0.36,0.00}{\textbf{#1}}}
\newcommand{\SpecialStringTok}[1]{\textcolor[rgb]{0.31,0.60,0.02}{#1}}
\newcommand{\StringTok}[1]{\textcolor[rgb]{0.31,0.60,0.02}{#1}}
\newcommand{\VariableTok}[1]{\textcolor[rgb]{0.00,0.00,0.00}{#1}}
\newcommand{\VerbatimStringTok}[1]{\textcolor[rgb]{0.31,0.60,0.02}{#1}}
\newcommand{\WarningTok}[1]{\textcolor[rgb]{0.56,0.35,0.01}{\textbf{\textit{#1}}}}
\usepackage{longtable,booktabs,array}
\usepackage{calc} % for calculating minipage widths
% Correct order of tables after \paragraph or \subparagraph
\usepackage{etoolbox}
\makeatletter
\patchcmd\longtable{\par}{\if@noskipsec\mbox{}\fi\par}{}{}
\makeatother
% Allow footnotes in longtable head/foot
\IfFileExists{footnotehyper.sty}{\usepackage{footnotehyper}}{\usepackage{footnote}}
\makesavenoteenv{longtable}
\usepackage{graphicx}
\makeatletter
\def\maxwidth{\ifdim\Gin@nat@width>\linewidth\linewidth\else\Gin@nat@width\fi}
\def\maxheight{\ifdim\Gin@nat@height>\textheight\textheight\else\Gin@nat@height\fi}
\makeatother
% Scale images if necessary, so that they will not overflow the page
% margins by default, and it is still possible to overwrite the defaults
% using explicit options in \includegraphics[width, height, ...]{}
\setkeys{Gin}{width=\maxwidth,height=\maxheight,keepaspectratio}
% Set default figure placement to htbp
\makeatletter
\def\fps@figure{htbp}
\makeatother
\setlength{\emergencystretch}{3em} % prevent overfull lines
\providecommand{\tightlist}{%
  \setlength{\itemsep}{0pt}\setlength{\parskip}{0pt}}
\setcounter{secnumdepth}{-\maxdimen} % remove section numbering
\usepackage{xcolor, color, ucs}     % http://ctan.org/pkg/xcolor
\usepackage{natbib}
\usepackage{booktabs}          % package for thick lines in tables
\usepackage{amsfonts,amsthm,amsmath}          % AMS Fonts
\usepackage{graphicx}          % Insert .pdf, .eps or .png
\usepackage{enumitem}          % http://ctan.org/pkg/enumitem
\usepackage{tabularx}          % Get scale boxes for tables
\usepackage{float}             % Force floats around
\usepackage{rotating}          % Rotate long tables horizontally
\usepackage{bbm}                % for bold betas
\usepackage{csquotes}           % \enquote{} and \textquote[][]{} environments
\usepackage{subfigure}
\usepackage{array}

\usepackage{longtable}


\setlist{nosep}

\usepackage{setspace}
\doublespacing



\usepackage{multirow}

\usepackage{hyperref}
\usepackage[noabbrev]{cleveref} % Should be loaded after \usepackage{hyperref}
\usepackage[small,bf]{caption}  % Captions


\parskip=8pt
\parindent=0pt
\delimitershortfall=-1pt
\interfootnotelinepenalty=100000

\newcommand{\qedknitr}{\hfill\rule{1.2ex}{1.2ex}}

%

\def\tightlist{}

\makeatletter
% align all math after the command
\newcommand{\mathleft}{\@fleqntrue\@mathmargin\parindent}
\newcommand{\mathcenter}{\@fleqnfalse}
% tilde with text over it
\newcommand{\distas}[1]{\mathbin{\overset{#1}{\kern\z@\sim}}}%
\newsavebox{\mybox}\newsavebox{\mysim}
\newcommand{\distras}[1]{%
  \savebox{\mybox}{\hbox{\kern3pt$\scriptstyle#1$\kern3pt}}%
  \savebox{\mysim}{\hbox{$\sim$}}%
  \mathbin{\overset{#1}{\kern\z@\resizebox{\wd\mybox}{\ht\mysim}{$\sim$}}}%
}
\makeatother

% \newtheoremstyle{newstyle}
% {} %Aboveskip
% {} %Below skip
% {\mdseries} %Body font e.g.\mdseries,\bfseries,\scshape,\itshape
% {} %Indent
% {\bfseries} %Head font e.g.\bfseries,\scshape,\itshape
% {.} %Punctuation afer theorem header
% { } %Space after theorem header
% {} %Heading

\newtheorem{thm}{Theorem}
\newtheorem{prop}[thm]{Proposition}
\newtheorem{lem}{Lemma}
\newtheorem{cor}{Corollary}
\newtheorem{definition}{Definition}
\newcommand*\diff{\mathop{}\!\mathrm{d}}
\newcommand*\Diff[1]{\mathop{}\!\mathrm{d^#1}}
\DeclareMathOperator{\E}{\mathrm{E}}
\DeclareMathOperator{\Var}{\mathrm{Var}}
\DeclareMathOperator{\R}{\mathbb{R}}
\newcolumntype{L}[1]{>{\raggedright\let\newline\\\arraybackslash\hspace{0pt}}m{#1}}
\newcolumntype{C}[1]{>{\centering\let\newline\\\arraybackslash\hspace{0pt}}m{#1}}
\newcolumntype{R}[1]{>{\raggedleft\let\newline\\\arraybackslash\hspace{0pt}}m{#1}}

% suppress table numbering
\captionsetup[table]{labelformat=empty}

\ifLuaTeX
  \usepackage{selnolig}  % disable illegal ligatures
\fi
\usepackage[]{natbib}
\bibliographystyle{apsr}
\usepackage{bookmark}
\IfFileExists{xurl.sty}{\usepackage{xurl}}{} % add URL line breaks if available
\urlstyle{same}
\hypersetup{
  pdftitle={Regression Discontinuity Design},
  hidelinks,
  pdfcreator={LaTeX via pandoc}}

\title{Regression Discontinuity Design}
\author{\href{mailto:thomas.leavitt@baruch.cuny.edu}{Thomas Leavitt}}
\date{July 2, 2024}

\begin{document}
\maketitle

\section{Application to close elections}

Today we are going to use the data on close US House of Representatives
races 1942--2008 used in \citet{caugheysekhon2011}.\footnote{The full
  replication data is available for download
  \href{http://sekhon.berkeley.edu/rep/RDReplication.zip}{here}. But we
  read it directly below.} \citet{caugheysekhon2011} engage in a debate
whose participants seek to identify the causal effect of the so-called
``incumbency advantage.'' That is, what effect does a candidate's status
as an incumbent have on whether or not that candidate wins an election?
Obviously, whether or not a candidate is an incumbent is \emph{not}
randomly assigned.

Let's first load the data: \scriptsize

\begin{Shaded}
\begin{Highlighting}[]
\NormalTok{rdd\_data }\OtherTok{\textless{}{-}} \FunctionTok{read\_dta}\NormalTok{(}\StringTok{"http://jakebowers.org/Matching/RDReplication.dta"}\NormalTok{) }\SpecialCharTok{\%\textgreater{}\%}
    \FunctionTok{filter}\NormalTok{(Use }\SpecialCharTok{==} \DecValTok{1}\NormalTok{)  }\DocumentationTok{\#\# Use is indicator for whether unit is included in RD incumbency advantage sample}
\end{Highlighting}
\end{Shaded}

\normalsize

The ``running variable'' is called \texttt{DifDPct}, which is defined as
the Democratic margin of victory or defeat in the election; in other
words, DifDPct is the difference between the percentage of all votes
that were cast for the leading Democrat in the race and the percentage
cast for the leading non-Democrat. Races in which no Democrat ran or in
which the top two vote-getters were both Democrats are coded as missing.

\scriptsize

\begin{Shaded}
\begin{Highlighting}[]
\NormalTok{running\_var }\OtherTok{\textless{}{-}} \FunctionTok{matrix}\NormalTok{(}\FunctionTok{c}\NormalTok{(}\StringTok{"DifDPct"}\NormalTok{, }\StringTok{"Democrat Margin of Victory"}\NormalTok{), }\AttributeTok{ncol =} \DecValTok{2}\NormalTok{, }\AttributeTok{byrow =} \ConstantTok{TRUE}\NormalTok{)}

\FunctionTok{dimnames}\NormalTok{(running\_var) }\OtherTok{\textless{}{-}} \FunctionTok{list}\NormalTok{(}\DecValTok{1}\NormalTok{, }\FunctionTok{c}\NormalTok{(}\StringTok{"Running Variable"}\NormalTok{, }\StringTok{"Description"}\NormalTok{))}

\FunctionTok{kable}\NormalTok{(running\_var)}
\end{Highlighting}
\end{Shaded}

\begin{longtable}[]{@{}ll@{}}
\toprule\noalign{}
Running Variable & Description \\
\midrule\noalign{}
\endhead
\bottomrule\noalign{}
\endlastfoot
DifDPct & Democrat Margin of Victory \\
\end{longtable}

\normalsize

The treatment variable is whether or not the Democratic candidate wins
the election or not. If the candidate wins the election, then that
candidate is assigned to ``treatment.'' If the candidate loses the
election, then he or she is assigned to ``control.'\,'

\scriptsize

\begin{Shaded}
\begin{Highlighting}[]
\NormalTok{treatment }\OtherTok{\textless{}{-}} \FunctionTok{matrix}\NormalTok{(}\FunctionTok{c}\NormalTok{(}\StringTok{"DemWin"}\NormalTok{, }\StringTok{"Democrat Wins Election"}\NormalTok{), }\AttributeTok{ncol =} \DecValTok{2}\NormalTok{, }\AttributeTok{byrow =} \ConstantTok{TRUE}\NormalTok{)}
\FunctionTok{dimnames}\NormalTok{(treatment) }\OtherTok{\textless{}{-}} \FunctionTok{list}\NormalTok{(}\DecValTok{1}\NormalTok{, }\FunctionTok{c}\NormalTok{(}\StringTok{"Treatment"}\NormalTok{, }\StringTok{"Description"}\NormalTok{))}
\FunctionTok{kable}\NormalTok{(treatment)}
\end{Highlighting}
\end{Shaded}

\begin{longtable}[]{@{}ll@{}}
\toprule\noalign{}
Treatment & Description \\
\midrule\noalign{}
\endhead
\bottomrule\noalign{}
\endlastfoot
DemWin & Democrat Wins Election \\
\end{longtable}

\normalsize

Now let's quickly look at the empirical distribution of the treatment
variable:

\scriptsize

\begin{Shaded}
\begin{Highlighting}[]
\FunctionTok{table}\NormalTok{(rdd\_data}\SpecialCharTok{$}\NormalTok{DemWin)}
\end{Highlighting}
\end{Shaded}

\begin{verbatim}

   0    1 
4507 5677 
\end{verbatim}

\normalsize

In \citet{caugheysekhon2011}, the primary outcome variables of interest
are as follows: whether a democrat wins the next election, the
proportion voting for a democrat in the next election, and the
democratic vote margin in the next election.

\scriptsize

\begin{Shaded}
\begin{Highlighting}[]
\NormalTok{dvs }\OtherTok{\textless{}{-}} \FunctionTok{matrix}\NormalTok{(}\FunctionTok{c}\NormalTok{(}\StringTok{"DWinNxt"}\NormalTok{, }\StringTok{"Dem Win t + 1"}\NormalTok{, }\StringTok{"DPctNxt"}\NormalTok{, }\StringTok{"Dem t + 1"}\NormalTok{, }\StringTok{"DifDPNxt"}\NormalTok{, }\StringTok{"Dem Margin t + 1"}\NormalTok{),}
    \AttributeTok{ncol =} \DecValTok{2}\NormalTok{, }\AttributeTok{byrow =} \ConstantTok{TRUE}\NormalTok{)}

\FunctionTok{dimnames}\NormalTok{(dvs) }\OtherTok{\textless{}{-}} \FunctionTok{list}\NormalTok{(}\FunctionTok{seq}\NormalTok{(}\AttributeTok{from =} \DecValTok{1}\NormalTok{, }\AttributeTok{to =} \DecValTok{3}\NormalTok{, }\AttributeTok{by =} \DecValTok{1}\NormalTok{), }\FunctionTok{c}\NormalTok{(}\StringTok{"Outcome"}\NormalTok{, }\StringTok{"Description"}\NormalTok{))}

\FunctionTok{kable}\NormalTok{(dvs)}
\end{Highlighting}
\end{Shaded}

\begin{longtable}[]{@{}ll@{}}
\toprule\noalign{}
Outcome & Description \\
\midrule\noalign{}
\endhead
\bottomrule\noalign{}
\endlastfoot
DWinNxt & Dem Win t + 1 \\
DPctNxt & Dem t + 1 \\
DifDPNxt & Dem Margin t + 1 \\
\end{longtable}

\normalsize

The relevant baseline covariates (all measured prior to the realization
of the running variable) are:

\scriptsize

\begin{Shaded}
\begin{Highlighting}[]
\NormalTok{covs }\OtherTok{\textless{}{-}} \FunctionTok{matrix}\NormalTok{(}\FunctionTok{c}\NormalTok{(}\StringTok{"DWinPrv"}\NormalTok{, }\StringTok{"Dem Win t {-} 1"}\NormalTok{,}
                 \StringTok{"DPctPrv"}\NormalTok{, }\StringTok{"Dem \% t {-} 1"}\NormalTok{,}
                 \StringTok{"DifDPPrv"}\NormalTok{, }\StringTok{"Dem \% Margin t {-} 1"}\NormalTok{,}
                 \StringTok{"IncDWNOM1"}\NormalTok{, }\StringTok{"Inc\textquotesingle{}s D1 NOMINATE"}\NormalTok{,}
                 \StringTok{"DemInc"}\NormalTok{, }\StringTok{"Dem Inc in Race"}\NormalTok{,}
                 \StringTok{"NonDInc"}\NormalTok{, }\StringTok{"Rep Inc in Race"}\NormalTok{,}
                 \StringTok{"PrvTrmsD"}\NormalTok{, }\StringTok{"Dem\textquotesingle{}s \# Prev Terms"}\NormalTok{,}
                 \StringTok{"PrvTrmsO"}\NormalTok{, }\StringTok{"Rep\textquotesingle{}s \# Prev Terms"}\NormalTok{,}
                 \StringTok{"RExpAdv"}\NormalTok{, }\StringTok{"Rep Experience Adv"}\NormalTok{,}
                 \StringTok{"DExpAdv"}\NormalTok{, }\StringTok{"Dem Experience Adv"}\NormalTok{,}
                 \StringTok{"ElcSwing"}\NormalTok{, }\StringTok{"Partisan Swing"}\NormalTok{,}
                 \StringTok{"CQRating3"}\NormalTok{, }\StringTok{"CQ Rating \{{-}1, 0, 1\}"}\NormalTok{,}
                 \StringTok{"DSpndPct"}\NormalTok{, }\StringTok{"Dem Spending \%"}\NormalTok{,}
                 \StringTok{"DDonaPct"}\NormalTok{, }\StringTok{"Dem Donation \%"}\NormalTok{,}
                 \StringTok{"SoSDem"}\NormalTok{, }\StringTok{"Dem Sec of State"}\NormalTok{,}
                 \StringTok{"GovDem"}\NormalTok{, }\StringTok{"Dem Governor"}\NormalTok{,}
                 \StringTok{"DifPVDec"}\NormalTok{, }\StringTok{"Dem Pres \% Margin"}\NormalTok{, }\DocumentationTok{\#\# average over decade}
                 \StringTok{"DemOpen"}\NormalTok{, }\StringTok{"Dem{-}held Open Seat"}\NormalTok{,}
                 \StringTok{"NonDOpen"}\NormalTok{, }\StringTok{"Rep{-}held Open Seat"}\NormalTok{,}
                 \StringTok{"OpenSeat"}\NormalTok{, }\StringTok{"Open Seat"}\NormalTok{,}
                 \StringTok{"VtTotPct"}\NormalTok{, }\StringTok{"Voter Turnout \%"}\NormalTok{,}
                 \StringTok{"GovWkPct"}\NormalTok{, }\StringTok{"Pct Gov\textquotesingle{}t Worker"}\NormalTok{,}
                 \StringTok{"UrbanPct"}\NormalTok{, }\StringTok{"Pct Urban"}\NormalTok{,}
                 \StringTok{"BlackPct"}\NormalTok{, }\StringTok{"Pct Black"}\NormalTok{,}
                 \StringTok{"ForgnPct"}\NormalTok{, }\StringTok{"Pct Foreign Born"}\NormalTok{),}
               \AttributeTok{ncol =} \DecValTok{2}\NormalTok{,}
               \AttributeTok{byrow =} \ConstantTok{TRUE}\NormalTok{)}

\FunctionTok{dimnames}\NormalTok{(covs) }\OtherTok{\textless{}{-}} \FunctionTok{list}\NormalTok{(}\FunctionTok{seq}\NormalTok{(}\AttributeTok{from =} \DecValTok{1}\NormalTok{,}
                           \AttributeTok{to =} \DecValTok{25}\NormalTok{,}
                           \AttributeTok{by =} \DecValTok{1}\NormalTok{),}
                       \FunctionTok{c}\NormalTok{(}\StringTok{"Covariate"}\NormalTok{, }\StringTok{"Description"}\NormalTok{))}

\FunctionTok{kable}\NormalTok{(covs)}
\end{Highlighting}
\end{Shaded}

\begin{longtable}[]{@{}ll@{}}
\toprule\noalign{}
Covariate & Description \\
\midrule\noalign{}
\endhead
\bottomrule\noalign{}
\endlastfoot
DWinPrv & Dem Win t - 1 \\
DPctPrv & Dem \% t - 1 \\
DifDPPrv & Dem \% Margin t - 1 \\
IncDWNOM1 & Inc's D1 NOMINATE \\
DemInc & Dem Inc in Race \\
NonDInc & Rep Inc in Race \\
PrvTrmsD & Dem's \# Prev Terms \\
PrvTrmsO & Rep's \# Prev Terms \\
RExpAdv & Rep Experience Adv \\
DExpAdv & Dem Experience Adv \\
ElcSwing & Partisan Swing \\
CQRating3 & CQ Rating \{-1, 0, 1\} \\
DSpndPct & Dem Spending \% \\
DDonaPct & Dem Donation \% \\
SoSDem & Dem Sec of State \\
GovDem & Dem Governor \\
DifPVDec & Dem Pres \% Margin \\
DemOpen & Dem-held Open Seat \\
NonDOpen & Rep-held Open Seat \\
OpenSeat & Open Seat \\
VtTotPct & Voter Turnout \% \\
GovWkPct & Pct Gov't Worker \\
UrbanPct & Pct Urban \\
BlackPct & Pct Black \\
ForgnPct & Pct Foreign Born \\
\end{longtable}

\normalsize

\section{Local as-if randomization framework}

\subsection{Optimal bandwidth selection}

Let's specify a set of candidate bandwidths and then sequentially test
covariate balance. Before actually testing, though, we want to specify a
balance criterion and then maximize effective sample size subject to
that criterion.

\scriptsize

\begin{Shaded}
\begin{Highlighting}[]
\NormalTok{bal\_fmla }\OtherTok{\textless{}{-}} \FunctionTok{reformulate}\NormalTok{(}\AttributeTok{termlabels =}\NormalTok{ covs[}\DecValTok{1}\SpecialCharTok{:}\DecValTok{25}\NormalTok{], }\AttributeTok{response =} \StringTok{"DemWin"}\NormalTok{)}

\NormalTok{candidate\_bands }\OtherTok{\textless{}{-}} \FunctionTok{seq}\NormalTok{(}\AttributeTok{from =} \SpecialCharTok{{-}}\DecValTok{5}\NormalTok{, }\AttributeTok{to =} \DecValTok{5}\NormalTok{, }\AttributeTok{by =} \FloatTok{0.1}\NormalTok{)}
\end{Highlighting}
\end{Shaded}

\normalsize

Now let's first filter our dataset and check for balance in the largest
candidate bandwidth spanning from \(-5\) to \(5\).

\scriptsize

\begin{Shaded}
\begin{Highlighting}[]
\NormalTok{lower\_bound }\OtherTok{\textless{}{-}} \FunctionTok{seq}\NormalTok{(}\AttributeTok{from =} \SpecialCharTok{{-}}\DecValTok{5}\NormalTok{, }\AttributeTok{to =} \SpecialCharTok{{-}}\FloatTok{0.1}\NormalTok{, }\AttributeTok{by =} \FloatTok{0.1}\NormalTok{)}

\NormalTok{upper\_bound }\OtherTok{\textless{}{-}} \FunctionTok{seq}\NormalTok{(}\AttributeTok{from =} \FloatTok{0.1}\NormalTok{, }\AttributeTok{to =} \DecValTok{5}\NormalTok{, }\AttributeTok{by =} \FloatTok{0.1}\NormalTok{) }\SpecialCharTok{\%\textgreater{}\%}
    \FunctionTok{sort}\NormalTok{(}\AttributeTok{decreasing =} \ConstantTok{TRUE}\NormalTok{)}

\NormalTok{rdd\_dataA }\OtherTok{\textless{}{-}}\NormalTok{ rdd\_data}
\NormalTok{rdd\_dataA }\OtherTok{\textless{}{-}}\NormalTok{ dplyr}\SpecialCharTok{::}\FunctionTok{filter}\NormalTok{(}\AttributeTok{.data =}\NormalTok{ rdd\_dataA, DifDPct }\SpecialCharTok{\textgreater{}}\NormalTok{ lower\_bound[}\DecValTok{1}\NormalTok{] }\SpecialCharTok{\&}\NormalTok{ DifDPct }\SpecialCharTok{\textless{}}
\NormalTok{    upper\_bound[}\DecValTok{1}\NormalTok{])}

\NormalTok{rdd\_dataA }\SpecialCharTok{\%$\%}
    \FunctionTok{summary}\NormalTok{(DifDPct)}

\FunctionTok{xBalance}\NormalTok{(}\AttributeTok{fmla =}\NormalTok{ bal\_fmla, }\AttributeTok{data =}\NormalTok{ rdd\_dataA, }\AttributeTok{report =} \StringTok{"chisquare.test"}\NormalTok{)}

\NormalTok{xb1 }\OtherTok{\textless{}{-}} \FunctionTok{xBalance}\NormalTok{(}\AttributeTok{fmla =}\NormalTok{ bal\_fmla, }\AttributeTok{data =}\NormalTok{ rdd\_dataA, }\AttributeTok{report =} \StringTok{"all"}\NormalTok{)}
\NormalTok{xb1}\SpecialCharTok{$}\NormalTok{results}
\end{Highlighting}
\end{Shaded}

\normalsize

Now let's write a function to perform this same procedure over all
candidate bandwidth sizes beginning with the largest candidate bandwidth
and subsequently testing smaller and smaller bandwidths in order.

\scriptsize

\begin{Shaded}
\begin{Highlighting}[]
\NormalTok{chi\_squared\_balance }\OtherTok{\textless{}{-}} \ControlFlowTok{function}\NormalTok{(lb, ub, running\_var, bal\_fmla, data) \{}

\NormalTok{    data }\OtherTok{\textless{}{-}}\NormalTok{ dplyr}\SpecialCharTok{::}\FunctionTok{filter}\NormalTok{(}\AttributeTok{.data =}\NormalTok{ data, running\_var }\SpecialCharTok{\textgreater{}}\NormalTok{ lb }\SpecialCharTok{\&}\NormalTok{ running\_var }\SpecialCharTok{\textless{}}\NormalTok{ ub)}

\NormalTok{    ess }\OtherTok{\textless{}{-}} \FunctionTok{nrow}\NormalTok{(data)}

\NormalTok{    p\_value }\OtherTok{\textless{}{-}} \FunctionTok{xBalance}\NormalTok{(}\AttributeTok{fmla =}\NormalTok{ bal\_fmla, }\AttributeTok{data =}\NormalTok{ data, }\AttributeTok{report =} \StringTok{"chisquare.test"}\NormalTok{)}\SpecialCharTok{$}\NormalTok{overall[[}\DecValTok{3}\NormalTok{]]}

\NormalTok{    bands }\OtherTok{\textless{}{-}} \FunctionTok{cbind}\NormalTok{(ess, p\_value, lb, ub)}

    \FunctionTok{return}\NormalTok{(bands)}
\NormalTok{\}}
\end{Highlighting}
\end{Shaded}

\normalsize

Now use the function:

\scriptsize

\begin{Shaded}
\begin{Highlighting}[]
\NormalTok{lbs }\OtherTok{\textless{}{-}} \FunctionTok{seq}\NormalTok{(}\AttributeTok{from =} \SpecialCharTok{{-}}\FloatTok{0.5}\NormalTok{, }\AttributeTok{to =} \SpecialCharTok{{-}}\FloatTok{0.1}\NormalTok{, }\AttributeTok{by =} \FloatTok{0.1}\NormalTok{)}
\NormalTok{ubs }\OtherTok{\textless{}{-}} \FunctionTok{seq}\NormalTok{(}\AttributeTok{from =} \FloatTok{0.5}\NormalTok{, }\AttributeTok{to =} \FloatTok{0.1}\NormalTok{, }\AttributeTok{by =} \SpecialCharTok{{-}}\FloatTok{0.01}\NormalTok{)}

\NormalTok{band\_df }\OtherTok{\textless{}{-}} \FunctionTok{data.frame}\NormalTok{(}\FunctionTok{t}\NormalTok{(}\FunctionTok{sapply}\NormalTok{(}\AttributeTok{X =} \DecValTok{1}\SpecialCharTok{:}\FunctionTok{length}\NormalTok{(lbs), }\AttributeTok{FUN =} \ControlFlowTok{function}\NormalTok{(x) \{}
    \FunctionTok{chi\_squared\_balance}\NormalTok{(}\AttributeTok{lb =}\NormalTok{ lbs[x], }\AttributeTok{ub =}\NormalTok{ ubs[x], }\AttributeTok{running\_var =}\NormalTok{ rdd\_data}\SpecialCharTok{$}\NormalTok{DifDPct,}
        \AttributeTok{bal\_fmla =}\NormalTok{ bal\_fmla, }\AttributeTok{data =}\NormalTok{ rdd\_data)}
\NormalTok{\}))) }\SpecialCharTok{\%\textgreater{}\%}
    \FunctionTok{rename}\NormalTok{(}\AttributeTok{ess =}\NormalTok{ X1, }\AttributeTok{p\_value =}\NormalTok{ X2)}

\FunctionTok{kable}\NormalTok{(band\_df)}
\end{Highlighting}
\end{Shaded}

\begin{longtable}[]{@{}rrrr@{}}
\toprule\noalign{}
ess & p\_value & X3 & X4 \\
\midrule\noalign{}
\endhead
\bottomrule\noalign{}
\endlastfoot
85 & 0.3693273 & -0.5 & 0.50 \\
76 & 0.5882135 & -0.4 & 0.49 \\
65 & 0.3450826 & -0.3 & 0.48 \\
55 & 0.4195984 & -0.2 & 0.47 \\
48 & 0.5330565 & -0.1 & 0.46 \\
\end{longtable}

\normalsize

\scriptsize

\begin{Shaded}
\begin{Highlighting}[]
\NormalTok{g }\OtherTok{\textless{}{-}} \FunctionTok{ggplot}\NormalTok{(}\AttributeTok{data =}\NormalTok{ rdd\_data, }\AttributeTok{mapping =} \FunctionTok{aes}\NormalTok{(}\AttributeTok{x =}\NormalTok{ DifDPct, }\AttributeTok{y =}\NormalTok{ DPctNxt, }\AttributeTok{color =}\NormalTok{ DemWin)) }\SpecialCharTok{+}
    \FunctionTok{geom\_point}\NormalTok{()}
\NormalTok{g }\SpecialCharTok{+} \FunctionTok{xlim}\NormalTok{(}\FunctionTok{c}\NormalTok{(}\SpecialCharTok{{-}}\FloatTok{0.4}\NormalTok{, }\FloatTok{0.4}\NormalTok{))}
\end{Highlighting}
\end{Shaded}

\includegraphics{rdd_worksheet_files/figure-latex/unnamed-chunk-11-1.pdf}
\normalsize

\subsection{Outcome analysis}

We could also perform permutation inference within the window around the
cutpoint under the assumption that local randomization holds (useful
especially if the window contains relatively few observations, skewed
outcomes, etc..):

\scriptsize

\begin{Shaded}
\begin{Highlighting}[]
\NormalTok{rdd\_data47 }\OtherTok{\textless{}{-}}\NormalTok{ rdd\_data }\SpecialCharTok{\%\textgreater{}\%}
    \FunctionTok{filter}\NormalTok{(DifDPct }\SpecialCharTok{\textgreater{}}\NormalTok{ lower\_bound[}\DecValTok{47}\NormalTok{] }\SpecialCharTok{\&}\NormalTok{ DifDPct }\SpecialCharTok{\textless{}}\NormalTok{ upper\_bound[}\DecValTok{47}\NormalTok{])}

\NormalTok{rdd\_data47 }\SpecialCharTok{\%\textgreater{}\%}
    \FunctionTok{nrow}\NormalTok{()}
\end{Highlighting}
\end{Shaded}

\begin{verbatim}
[1] 70
\end{verbatim}

\begin{Shaded}
\begin{Highlighting}[]
\FunctionTok{set.seed}\NormalTok{(}\DecValTok{1}\SpecialCharTok{:}\DecValTok{5}\NormalTok{)}
\NormalTok{sharp\_null\_dist }\OtherTok{\textless{}{-}} \FunctionTok{replicate}\NormalTok{(}\AttributeTok{n =} \DecValTok{10}\SpecialCharTok{\^{}}\DecValTok{3}\NormalTok{, }\AttributeTok{expr =} \FunctionTok{coef}\NormalTok{(}\FunctionTok{lm}\NormalTok{(}\AttributeTok{formula =}\NormalTok{ DPctNxt }\SpecialCharTok{\textasciitilde{}} \FunctionTok{sample}\NormalTok{(DemWin),}
    \AttributeTok{data =}\NormalTok{ rdd\_data47))[[}\DecValTok{2}\NormalTok{]])}

\NormalTok{obsstat }\OtherTok{\textless{}{-}} \FunctionTok{coef}\NormalTok{(}\FunctionTok{lm}\NormalTok{(}\AttributeTok{formula =}\NormalTok{ DPctNxt }\SpecialCharTok{\textasciitilde{}}\NormalTok{ DemWin, }\AttributeTok{data =}\NormalTok{ rdd\_data47))[[}\StringTok{"DemWin"}\NormalTok{]]}

\NormalTok{upper\_p\_val }\OtherTok{\textless{}{-}} \FunctionTok{mean}\NormalTok{(sharp\_null\_dist }\SpecialCharTok{\textgreater{}=}\NormalTok{ obsstat)}
\NormalTok{lower\_p\_val }\OtherTok{\textless{}{-}} \FunctionTok{mean}\NormalTok{(sharp\_null\_dist }\SpecialCharTok{\textless{}=}\NormalTok{ obsstat)}
\NormalTok{two\_sided\_p\_val }\OtherTok{\textless{}{-}} \FunctionTok{min}\NormalTok{(}\DecValTok{1}\NormalTok{, }\DecValTok{2} \SpecialCharTok{*} \FunctionTok{min}\NormalTok{(upper\_p\_val, lower\_p\_val))}

\NormalTok{two\_sided\_p\_val}
\end{Highlighting}
\end{Shaded}

\begin{verbatim}
[1] 0
\end{verbatim}

\begin{Shaded}
\begin{Highlighting}[]
\NormalTok{rdd\_data47}\SpecialCharTok{$}\NormalTok{DemWinF }\OtherTok{\textless{}{-}} \FunctionTok{factor}\NormalTok{(rdd\_data47}\SpecialCharTok{$}\NormalTok{DemWin)}
\end{Highlighting}
\end{Shaded}

\normalsize

We can now perform outcome analysis on the detrended outcome variable:

\scriptsize

\begin{Shaded}
\begin{Highlighting}[]
\NormalTok{tmp\_lm }\OtherTok{\textless{}{-}} \FunctionTok{lm}\NormalTok{(DPctNxt }\SpecialCharTok{\textasciitilde{}}\NormalTok{ DifDPct, }\AttributeTok{data =}\NormalTok{ rdd\_data47)}

\NormalTok{rdd\_data47 }\SpecialCharTok{\%\textless{}\textgreater{}\%}
    \FunctionTok{mutate}\NormalTok{(}\AttributeTok{resid\_DPctNxt =} \FunctionTok{resid}\NormalTok{(tmp\_lm))}

\FunctionTok{lm\_robust}\NormalTok{(}\AttributeTok{formula =}\NormalTok{ resid\_DPctNxt }\SpecialCharTok{\textasciitilde{}}\NormalTok{ DemWin, }\AttributeTok{data =}\NormalTok{ rdd\_data47)}
\end{Highlighting}
\end{Shaded}

\begin{verbatim}
             Estimate Std. Error    t value  Pr(>|t|)  CI Lower  CI Upper DF
(Intercept) -2.285391   2.569980 -0.8892638 0.3769964 -7.413706  2.842925 68
DemWin       4.570781   3.613191  1.2650262 0.2101800 -2.639229 11.780791 68
\end{verbatim}

\normalsize

\citet{saleshansen2020}, however, propose robust regression, which is
less sensitive to violations of the regression model's assumptions ---
still removing linear trend here.

\scriptsize

\begin{Shaded}
\begin{Highlighting}[]
\NormalTok{tmp\_rlm }\OtherTok{\textless{}{-}} \FunctionTok{rlm}\NormalTok{(DPctNxt }\SpecialCharTok{\textasciitilde{}}\NormalTok{ DifDPct, }\AttributeTok{data =}\NormalTok{ rdd\_data47)}
\NormalTok{rdd\_data47 }\SpecialCharTok{\%\textless{}\textgreater{}\%}
    \FunctionTok{mutate}\NormalTok{(}\AttributeTok{rlm\_resid\_DPctNxt =} \FunctionTok{resid}\NormalTok{(tmp\_rlm))}
\FunctionTok{lm\_robust}\NormalTok{(}\AttributeTok{formula =}\NormalTok{ rlm\_resid\_DPctNxt }\SpecialCharTok{\textasciitilde{}}\NormalTok{ DemWin, }\AttributeTok{data =}\NormalTok{ rdd\_data47)}
\end{Highlighting}
\end{Shaded}

\begin{verbatim}
             Estimate Std. Error   t value   Pr(>|t|)  CI Lower   CI Upper DF
(Intercept) -4.477620   2.544731 -1.759565 0.08297946 -9.555552  0.6003131 68
DemWin       5.865897   3.590151  1.633886 0.10690469 -1.298138 13.0299312 68
\end{verbatim}

\normalsize

We don't actually need to know the true form of
\(\E[\left[Y | R =r\right]\). Instead, we can use methods, such as local
regression, to approximate the function of \(\E[\left[Y | R = r\right]\)
at values of \(R\) below and above the cutpoint.

\scriptsize

\begin{Shaded}
\begin{Highlighting}[]
\NormalTok{rdd\_data }\SpecialCharTok{\%\textless{}\textgreater{}\%}
    \FunctionTok{arrange}\NormalTok{(DifDPct)}

\NormalTok{rdd\_data\_cp }\OtherTok{\textless{}{-}}\NormalTok{ dplyr}\SpecialCharTok{::}\FunctionTok{select}\NormalTok{(}\AttributeTok{.data =}\NormalTok{ rdd\_data, DifDPct, DPctNxt, DemWin) }\SpecialCharTok{\%\textgreater{}\%}
    \FunctionTok{mutate}\NormalTok{(}\AttributeTok{cp =} \FunctionTok{ifelse}\NormalTok{(}\AttributeTok{test =}\NormalTok{ DifDPct }\SpecialCharTok{\textless{}} \DecValTok{0}\NormalTok{, }\AttributeTok{yes =} \StringTok{"Below Cutpoint"}\NormalTok{, }\AttributeTok{no =} \StringTok{"Above Cutpoint"}\NormalTok{))}
\end{Highlighting}
\end{Shaded}

\normalsize

\scriptsize

\includegraphics{rdd_worksheet_files/figure-latex/unnamed-chunk-16-1.pdf}
\normalsize

\section{Continuity in Potential Outcomes}

Let's first use a simple linear model to estimate the relationship on
both sides of the cutpoint. Here we are not using a window.

\scriptsize

\begin{Shaded}
\begin{Highlighting}[]
\NormalTok{lm\_predict\_below }\OtherTok{\textless{}{-}} \FunctionTok{predict}\NormalTok{(}\AttributeTok{object =} \FunctionTok{lm}\NormalTok{(}\AttributeTok{formula =}\NormalTok{ DPctNxt }\SpecialCharTok{\textasciitilde{}}\NormalTok{ DifDPct, }\AttributeTok{data =}\NormalTok{ rdd\_data,}
    \AttributeTok{subset =}\NormalTok{ DifDPct }\SpecialCharTok{\textless{}} \DecValTok{0}\NormalTok{), }\AttributeTok{newdata =} \FunctionTok{data.frame}\NormalTok{(}\AttributeTok{DifDPct =} \DecValTok{0}\NormalTok{))}

\NormalTok{lm\_predict\_above }\OtherTok{\textless{}{-}} \FunctionTok{predict}\NormalTok{(}\AttributeTok{object =} \FunctionTok{lm}\NormalTok{(}\AttributeTok{formula =}\NormalTok{ DPctNxt }\SpecialCharTok{\textasciitilde{}}\NormalTok{ DifDPct, }\AttributeTok{data =}\NormalTok{ rdd\_data,}
    \AttributeTok{subset =}\NormalTok{ DifDPct }\SpecialCharTok{\textgreater{}} \DecValTok{0}\NormalTok{), }\AttributeTok{newdata =} \FunctionTok{data.frame}\NormalTok{(}\AttributeTok{DifDPct =} \DecValTok{0}\NormalTok{))}

\NormalTok{lm\_predict\_above }\SpecialCharTok{{-}}\NormalTok{ lm\_predict\_below}
\end{Highlighting}
\end{Shaded}

\begin{verbatim}
       1 
11.74521 
\end{verbatim}

\normalsize

We could also do the same thing with LOESS regression:

\scriptsize

\begin{Shaded}
\begin{Highlighting}[]
\NormalTok{loess\_predict\_below }\OtherTok{\textless{}{-}} \FunctionTok{predict}\NormalTok{(}\AttributeTok{object =} \FunctionTok{loess}\NormalTok{(}\AttributeTok{formula =}\NormalTok{ DPctNxt }\SpecialCharTok{\textasciitilde{}}\NormalTok{ DifDPct, }\AttributeTok{data =}\NormalTok{ rdd\_data,}
    \AttributeTok{subset =}\NormalTok{ DifDPct }\SpecialCharTok{\textless{}} \DecValTok{0}\NormalTok{, }\AttributeTok{surface =} \StringTok{"direct"}\NormalTok{), }\AttributeTok{newdata =} \FunctionTok{data.frame}\NormalTok{(}\AttributeTok{DifDPct =} \DecValTok{0}\NormalTok{))}

\NormalTok{loess\_predict\_above }\OtherTok{\textless{}{-}} \FunctionTok{predict}\NormalTok{(}\AttributeTok{object =} \FunctionTok{loess}\NormalTok{(}\AttributeTok{formula =}\NormalTok{ DPctNxt }\SpecialCharTok{\textasciitilde{}}\NormalTok{ DifDPct, }\AttributeTok{data =}\NormalTok{ rdd\_data,}
    \AttributeTok{subset =}\NormalTok{ DifDPct }\SpecialCharTok{\textgreater{}} \DecValTok{0}\NormalTok{, }\AttributeTok{surface =} \StringTok{"direct"}\NormalTok{), }\AttributeTok{newdata =} \FunctionTok{data.frame}\NormalTok{(}\AttributeTok{DifDPct =} \DecValTok{0}\NormalTok{))}

\NormalTok{loess\_predict\_above }\SpecialCharTok{{-}}\NormalTok{ loess\_predict\_below}
\end{Highlighting}
\end{Shaded}

\begin{verbatim}
       1 
9.721584 
\end{verbatim}

\normalsize

We could also use a \(p\)th order polynomial. For example:

\scriptsize

\begin{Shaded}
\begin{Highlighting}[]
\NormalTok{lm\_poly\_predict\_below }\OtherTok{\textless{}{-}} \FunctionTok{predict}\NormalTok{(}\AttributeTok{object =} \FunctionTok{lm}\NormalTok{(}\AttributeTok{formula =}\NormalTok{ DPctNxt }\SpecialCharTok{\textasciitilde{}} \FunctionTok{I}\NormalTok{(DifDPct}\SpecialCharTok{\^{}}\DecValTok{3}\NormalTok{) }\SpecialCharTok{+} \FunctionTok{I}\NormalTok{(DifDPct}\SpecialCharTok{\^{}}\DecValTok{2}\NormalTok{) }\SpecialCharTok{+}
\NormalTok{    DifDPct, }\AttributeTok{data =}\NormalTok{ rdd\_data, }\AttributeTok{subset =}\NormalTok{ DifDPct }\SpecialCharTok{\textless{}} \DecValTok{0}\NormalTok{), }\AttributeTok{newdata =} \FunctionTok{data.frame}\NormalTok{(}\AttributeTok{DifDPct =} \DecValTok{0}\NormalTok{))}

\NormalTok{lm\_poly\_predict\_above }\OtherTok{\textless{}{-}} \FunctionTok{predict}\NormalTok{(}\AttributeTok{object =} \FunctionTok{lm}\NormalTok{(}\AttributeTok{formula =}\NormalTok{ DPctNxt }\SpecialCharTok{\textasciitilde{}} \FunctionTok{I}\NormalTok{(DifDPct}\SpecialCharTok{\^{}}\DecValTok{3}\NormalTok{) }\SpecialCharTok{+} \FunctionTok{I}\NormalTok{(DifDPct}\SpecialCharTok{\^{}}\DecValTok{2}\NormalTok{) }\SpecialCharTok{+}
\NormalTok{    DifDPct, }\AttributeTok{data =}\NormalTok{ rdd\_data, }\AttributeTok{subset =}\NormalTok{ DifDPct }\SpecialCharTok{\textgreater{}} \DecValTok{0}\NormalTok{), }\AttributeTok{newdata =} \FunctionTok{data.frame}\NormalTok{(}\AttributeTok{DifDPct =} \DecValTok{0}\NormalTok{))}

\NormalTok{lm\_poly\_predict\_above }\SpecialCharTok{{-}}\NormalTok{ lm\_poly\_predict\_below}
\end{Highlighting}
\end{Shaded}

\begin{verbatim}
       1 
10.63556 
\end{verbatim}

\normalsize

This next calculates a bandwidth and then does what we just did above
(all in one step):

\scriptsize

\begin{Shaded}
\begin{Highlighting}[]
\NormalTok{rdest1 }\OtherTok{\textless{}{-}} \FunctionTok{RDestimate}\NormalTok{(}\AttributeTok{formula =}\NormalTok{ DPctNxt }\SpecialCharTok{\textasciitilde{}}\NormalTok{ DifDPct, }\AttributeTok{data =}\NormalTok{ rdd\_data, }\AttributeTok{se.type =} \StringTok{"HC3"}\NormalTok{)}
\FunctionTok{summary}\NormalTok{(rdest1)}
\end{Highlighting}
\end{Shaded}

\begin{verbatim}

Call:
RDestimate(formula = DPctNxt ~ DifDPct, data = rdd_data, se.type = "HC3")

Type:
sharp 

Estimates:
           Bandwidth  Observations  Estimate  Std. Error  z value  Pr(>|z|) 
LATE        6.624     1048           9.531    1.586       6.008    1.877e-09
Half-BW     3.312      523          11.023    2.470       4.463    8.100e-06
Double-BW  13.249     2013           9.249    1.064       8.692    3.556e-18
              
LATE       ***
Half-BW    ***
Double-BW  ***
---
Signif. codes:  0 '***' 0.001 '**' 0.01 '*' 0.05 '.' 0.1 ' ' 1

F-statistics:
           F       Num. DoF  Denom. DoF  p
LATE        86.05  3         1044        0
Half-BW     38.33  3          519        0
Double-BW  218.56  3         2009        0
\end{verbatim}

\begin{Shaded}
\begin{Highlighting}[]
\FunctionTok{plot}\NormalTok{(rdest1)}
\end{Highlighting}
\end{Shaded}

\includegraphics{rdd_worksheet_files/figure-latex/unnamed-chunk-20-1.pdf}
\normalsize

\scriptsize

\begin{Shaded}
\begin{Highlighting}[]
\NormalTok{rdrest1 }\OtherTok{\textless{}{-}} \FunctionTok{rdrobust}\NormalTok{(}\AttributeTok{y =}\NormalTok{ rdd\_data}\SpecialCharTok{$}\NormalTok{DPctNxt, }\AttributeTok{x =}\NormalTok{ rdd\_data}\SpecialCharTok{$}\NormalTok{DifDPct)}
\FunctionTok{summary}\NormalTok{(rdrest1)}
\end{Highlighting}
\end{Shaded}

\begin{verbatim}
Sharp RD estimates using local polynomial regression.

Number of Obs.                 8594
BW type                       mserd
Kernel                   Triangular
VCE method                       NN

Number of Obs.                 3708         4886
Eff. Number of Obs.            1697         1435
Order est. (p)                    1            1
Order bias  (q)                   2            2
BW est. (h)                  21.156       21.156
BW bias (b)                  35.032       35.032
rho (h/b)                     0.604        0.604
Unique Obs.                    3576         4098

=============================================================================
        Method     Coef. Std. Err.         z     P>|z|      [ 95% C.I. ]       
=============================================================================
  Conventional     9.381     0.847    11.080     0.000     [7.722 , 11.041]    
        Robust         -         -     9.710     0.000     [7.676 , 11.559]    
=============================================================================
\end{verbatim}

\begin{Shaded}
\begin{Highlighting}[]
\FunctionTok{rdplot}\NormalTok{(}\AttributeTok{y =}\NormalTok{ rdd\_data}\SpecialCharTok{$}\NormalTok{DPctNxt, }\AttributeTok{x =}\NormalTok{ rdd\_data}\SpecialCharTok{$}\NormalTok{DifDPct)}
\end{Highlighting}
\end{Shaded}

\includegraphics{rdd_worksheet_files/figure-latex/unnamed-chunk-21-1.pdf}
\normalsize

\newpage

\renewcommand\refname{References}
  \bibliography{rddbibliography.bib}

\end{document}

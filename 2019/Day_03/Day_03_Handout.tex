\documentclass[12pt,leqno]{article}
\usepackage[left=1in,top=1in,right=1in,bottom=1in]{geometry}
\newcommand*{\authorfont}{\fontfamily{phv}\selectfont}
\usepackage{lmodern}


  \usepackage[T1]{fontenc}




\usepackage{abstract}
\renewcommand{\abstractname}{}    % clear the title
\renewcommand{\absnamepos}{empty} % originally center

\renewenvironment{abstract}
 {{%
    \setlength{\leftmargin}{0mm}
    \setlength{\rightmargin}{\leftmargin}%
  }%
  \relax}
 {\endlist}

\makeatletter
\def\@maketitle{%
  \newpage
%  \null
%  \vskip 2em%
%  \begin{center}%
  \let \footnote \thanks
    {\fontsize{18}{20}\selectfont\raggedright  \setlength{\parindent}{0pt} \@title \par}%
}
%\fi
\makeatother




\setcounter{secnumdepth}{0}

\usepackage{color}
\usepackage{fancyvrb}
\newcommand{\VerbBar}{|}
\newcommand{\VERB}{\Verb[commandchars=\\\{\}]}
\DefineVerbatimEnvironment{Highlighting}{Verbatim}{commandchars=\\\{\}}
% Add ',fontsize=\small' for more characters per line
\usepackage{framed}
\definecolor{shadecolor}{RGB}{248,248,248}
\newenvironment{Shaded}{\begin{snugshade}}{\end{snugshade}}
\newcommand{\AlertTok}[1]{\textcolor[rgb]{0.94,0.16,0.16}{#1}}
\newcommand{\AnnotationTok}[1]{\textcolor[rgb]{0.56,0.35,0.01}{\textbf{\textit{#1}}}}
\newcommand{\AttributeTok}[1]{\textcolor[rgb]{0.77,0.63,0.00}{#1}}
\newcommand{\BaseNTok}[1]{\textcolor[rgb]{0.00,0.00,0.81}{#1}}
\newcommand{\BuiltInTok}[1]{#1}
\newcommand{\CharTok}[1]{\textcolor[rgb]{0.31,0.60,0.02}{#1}}
\newcommand{\CommentTok}[1]{\textcolor[rgb]{0.56,0.35,0.01}{\textit{#1}}}
\newcommand{\CommentVarTok}[1]{\textcolor[rgb]{0.56,0.35,0.01}{\textbf{\textit{#1}}}}
\newcommand{\ConstantTok}[1]{\textcolor[rgb]{0.00,0.00,0.00}{#1}}
\newcommand{\ControlFlowTok}[1]{\textcolor[rgb]{0.13,0.29,0.53}{\textbf{#1}}}
\newcommand{\DataTypeTok}[1]{\textcolor[rgb]{0.13,0.29,0.53}{#1}}
\newcommand{\DecValTok}[1]{\textcolor[rgb]{0.00,0.00,0.81}{#1}}
\newcommand{\DocumentationTok}[1]{\textcolor[rgb]{0.56,0.35,0.01}{\textbf{\textit{#1}}}}
\newcommand{\ErrorTok}[1]{\textcolor[rgb]{0.64,0.00,0.00}{\textbf{#1}}}
\newcommand{\ExtensionTok}[1]{#1}
\newcommand{\FloatTok}[1]{\textcolor[rgb]{0.00,0.00,0.81}{#1}}
\newcommand{\FunctionTok}[1]{\textcolor[rgb]{0.00,0.00,0.00}{#1}}
\newcommand{\ImportTok}[1]{#1}
\newcommand{\InformationTok}[1]{\textcolor[rgb]{0.56,0.35,0.01}{\textbf{\textit{#1}}}}
\newcommand{\KeywordTok}[1]{\textcolor[rgb]{0.13,0.29,0.53}{\textbf{#1}}}
\newcommand{\NormalTok}[1]{#1}
\newcommand{\OperatorTok}[1]{\textcolor[rgb]{0.81,0.36,0.00}{\textbf{#1}}}
\newcommand{\OtherTok}[1]{\textcolor[rgb]{0.56,0.35,0.01}{#1}}
\newcommand{\PreprocessorTok}[1]{\textcolor[rgb]{0.56,0.35,0.01}{\textit{#1}}}
\newcommand{\RegionMarkerTok}[1]{#1}
\newcommand{\SpecialCharTok}[1]{\textcolor[rgb]{0.00,0.00,0.00}{#1}}
\newcommand{\SpecialStringTok}[1]{\textcolor[rgb]{0.31,0.60,0.02}{#1}}
\newcommand{\StringTok}[1]{\textcolor[rgb]{0.31,0.60,0.02}{#1}}
\newcommand{\VariableTok}[1]{\textcolor[rgb]{0.00,0.00,0.00}{#1}}
\newcommand{\VerbatimStringTok}[1]{\textcolor[rgb]{0.31,0.60,0.02}{#1}}
\newcommand{\WarningTok}[1]{\textcolor[rgb]{0.56,0.35,0.01}{\textbf{\textit{#1}}}}


\title{Day 03: Research Design and Hypothesis Testing  }



\author{\Large \href{mailto:t.leavitt718@gmail.com}{Thomas Leavitt}\vspace{0.05in} \newline\normalsize\emph{}  }


\date{}

\usepackage{titlesec}

\titleformat*{\section}{\normalsize\bfseries}
\titleformat*{\subsection}{\normalsize\itshape}
\titleformat*{\subsubsection}{\normalsize\itshape}
\titleformat*{\paragraph}{\normalsize\itshape}
\titleformat*{\subparagraph}{\normalsize\itshape}


\usepackage{natbib}
\bibliographystyle{apsr}



\newtheorem{hypothesis}{Hypothesis}
\usepackage{setspace}

\makeatletter
\@ifpackageloaded{hyperref}{}{%
\ifxetex
  \usepackage[setpagesize=false, % page size defined by xetex
              unicode=false, % unicode breaks when used with xetex
              xetex]{hyperref}
\else
  \usepackage[unicode=true]{hyperref}
\fi
}
\@ifpackageloaded{xcolor}{
    \PassOptionsToPackage{usenames,dvipsnames}{xcolor}
}{%
    \usepackage[usenames,dvipsnames]{xcolor}
}
\makeatother
\hypersetup{breaklinks=true,
            bookmarks=true,
            pdfauthor={\href{mailto:t.leavitt718@gmail.com}{Thomas Leavitt} ()},
             pdfkeywords = {},  
            pdftitle={Day 03: Research Design and Hypothesis Testing},
            colorlinks=true,
            citecolor=blue,
            urlcolor=blue,
            linkcolor=magenta,
            pdfborder={0 0 0}}
\urlstyle{same}  % don't use monospace font for urls

% \usepackage{microtype} %
% \usepackage{setspace}
% \onehalfspacing
\usepackage{xcolor, color, ucs}     % http://ctan.org/pkg/xcolor
\usepackage{natbib}
\usepackage{booktabs}          % package for thick lines in tables
\usepackage{amsfonts,amsthm,amsmath,amssymb,bm}          % AMS Fonts
% \usepackage{empheq}            % To use left brace on {align} environment
\usepackage{graphicx}          % Insert .pdf, .eps or .png
\usepackage{enumitem}          % http://ctan.org/pkg/enumitem
% \usepackage[mathscr]{euscript}          % Font for right expectation sign
\usepackage{tabularx}          % Get scale boxes for tables
\usepackage{float}             % Force floats around
\usepackage{rotating}          % Rotate long tables horizontally
\usepackage{bbm}                % for bold betas
\usepackage{csquotes}           % \enquote{} and \textquote[][]{} environments
\usepackage{subfigure}
\usepackage{array}
% \usepackage{cancel}
\usepackage{longtable}
% % \usepackage{lmodern}
% % \usepackage{libertine} \usepackage[libertine]{newtxmath}
% \usepackage{stix}
% % \usepackage[osf,sc]{mathpazo}     % alternative math
% \usepackage[T1]{fontenc}
% \usepackage{fontspec}
% \setmainfont{Times New Roman}
% \usepackage{mathtools}          % multlined environment with size option
% \usepackage{verbatim}
% \usepackage{geometry}
% \usepackage{bigfoot}
% \geometry{verbose,margin=.8in,nomarginpar}
% \setcounter{secnumdepth}{2}
% \setcounter{tocdepth}{2}
% \usepackage{lscape}

\setlist{nosep}


% \usepackage{url}
% \usepackage[nobreak=true]{mdframed} % put box around section with \begin{mdframed}\end{mdframed}

% \usepackage{relsize}            % \mathlarger{} environment
% \usepackage[unicode=true,
%             pdfusetitle,
%             bookmarks=true,
%             bookmarksnumbered=true,
%             bookmarksopen=true,
%             bookmarksopenlevel=2,
%             breaklinks=false,
%             pdfborder={0 0 1},
%             backref=false,
%             colorlinks=true,
%             hypertexnames=false]{hyperref}
% \hypersetup{pdfstartview={XYZ null null 1},
%             citecolor=blue!50,
%             linkcolor=red,
%             urlcolor=green!70!black}

% \usepackage{multirow}
% \usepackage{tikz}
% \usetikzlibrary{trees, positioning, arrows, automata}

% \tikzset{
%   treenode/.style = {align=center, inner sep=0pt, text centered,
%     font=\sffamily},
%   arn_n/.style = {treenode, rectangle, black, fill=white, text width=6em},
%   arn_r/.style = {treenode, circle, red, draw=red, text width=1.5em, thick}
% }

\usepackage[noabbrev]{cleveref} % Should be loaded after \usepackage{hyperref}
\usepackage[small,bf]{caption}  % Captions

% \usepackage[obeyFinal,textwidth=0.8in, colorinlistoftodos,prependcaption,textsize=tiny]{todonotes} % \fxnote*[options]{note}{text} to make sticky notes
% \usepackage{xargs}
% \newcommandx{\unsure}[2][1=]{\todo[linecolor=red,backgroundcolor=red!25,bordercolor=red,#1]{#2}}
% \newcommandx{\change}[2][1=]{\todo[linecolor=blue,backgroundcolor=blue!25,bordercolor=blue,#1]{#2}}
% \newcommandx{\info}[2][1=]{\todo[linecolor=OliveGreen,backgroundcolor=OliveGreen!25,bordercolor=OliveGreen,#1]{#2}}
% \newcommandx{\improvement}[2][1=]{\todo[linecolor=Plum,backgroundcolor=Plum!25,bordercolor=Plum,#1]{#2}}

\parskip=8pt
\parindent=0pt
\delimitershortfall=-1pt
\interfootnotelinepenalty=100000

\newcommand{\qedknitr}{\hfill\rule{1.2ex}{1.2ex}}

% \makeatletter
% \def\thm@space@setup{\thm@preskip=0pt
% \thm@postskip=0pt}
% \makeatother

\def\tightlist{}

\makeatletter
% align all math after the command
\newcommand{\mathleft}{\@fleqntrue\@mathmargin\parindent}
\newcommand{\mathcenter}{\@fleqnfalse}
% tilde with text over it
\newcommand{\distas}[1]{\mathbin{\overset{#1}{\kern\z@\sim}}}%
\newsavebox{\mybox}\newsavebox{\mysim}
\newcommand{\distras}[1]{%
  \savebox{\mybox}{\hbox{\kern3pt$\scriptstyle#1$\kern3pt}}%
  \savebox{\mysim}{\hbox{$\sim$}}%
  \mathbin{\overset{#1}{\kern\z@\resizebox{\wd\mybox}{\ht\mysim}{$\sim$}}}%
}
\makeatother

% \newtheoremstyle{newstyle}
% {} %Aboveskip
% {} %Below skip
% {\mdseries} %Body font e.g.\mdseries,\bfseries,\scshape,\itshape
% {} %Indent
% {\bfseries} %Head font e.g.\bfseries,\scshape,\itshape
% {.} %Punctuation afer theorem header
% { } %Space after theorem header
% {} %Heading

\theoremstyle{newstyle}
\newtheorem{thm}{Theorem}
\newtheorem{prop}[thm]{Proposition}
\newtheorem{lem}{Lemma}
\newtheorem{cor}{Corollary}
\newtheorem{definition}{Definition}
\newcommand*\diff{\mathop{}\!\mathrm{d}}
\newcommand*\Diff[1]{\mathop{}\!\mathrm{d^#1}}
\DeclareMathOperator{\E}{\mathbb{E}}
\DeclareMathOperator{\R}{\mathbb{R}}
\newcolumntype{L}[1]{>{\raggedright\let\newline\\\arraybackslash\hspace{0pt}}m{#1}}
\newcolumntype{C}[1]{>{\centering\let\newline\\\arraybackslash\hspace{0pt}}m{#1}}
\newcolumntype{R}[1]{>{\raggedleft\let\newline\\\arraybackslash\hspace{0pt}}m{#1}}

% suppress table numbering
\captionsetup[table]{labelformat=empty}


\begin{document}
	
% \pagenumbering{arabic}% resets `page` counter to 1 
%
% \maketitle

{% \usefont{T1}{pnc}{m}{n}
\setlength{\parindent}{0pt}
\thispagestyle{plain}
{\fontsize{18}{20}\selectfont\raggedright 
\maketitle  % title \par  

}

{
   \vskip 13.5pt\relax \normalsize\fontsize{11}{12} 
\textbf{\authorfont \href{mailto:t.leavitt718@gmail.com}{Thomas Leavitt}} \hskip 15pt \emph{\small }   

}

}





\vskip 6.5pt

\noindent  \hypertarget{review-of-probability}{%
\section{Review of Probability}\label{review-of-probability}}

\begin{itemize}

\item Let $\Omega$ denote the \textit{sample space}, i.e., the set of possible events that could result from a random process.

\item Let $\mathcal{F}$ denote the set of all measurable subsets of the sample space, $\Omega$.\footnote{More formally, we stipulate that the set $\mathcal{F}$ of subsets of $\Omega$ satisfies the following conditions: (1) $\Omega \in \mathcal{F}$; (2) if an event $x \in \mathcal{F}$, then $x^c \in \mathcal{F}$, where $x^c$ is the elements of $\mathcal{F}$ that are not in $x$ (i.e., $x^c$ is the complement of $x$); (3) if $x_1, x_2, x_3, \dots \in \mathcal{F}$, then $x_1 \cup x_2 \cup x_3 \cup \dots \in \mathcal{F}$. The formal name for a set of a sample space's subsets that satisfies these three conditions is $\sigma$-algebra.}

\item Let $\Pr: \mathcal{F} \mapsto [0, 1]$ denote a \textit{probability measure}, i.e., a function that maps every element in $\mathcal{F}$ to exactly one element on the closed interval $[0, 1]$, which satisfies the following three conditions:
\begin{enumerate}
\item For all $x \in \mathcal{F}$, $\Pr(x) \geq 0$, where $\Pr(x)$ is finite and is a real number, i.e., $\Pr(x) \in \R$.
\item $\Pr(\Omega) = 1$.
\item If $x_1, x_2, x_3, \dots \in \mathcal{F}$ are pairwise disjoint, i.e., for all $i$ and $j \neq i$, $x_i \cap x_j = \emptyset$, then $\Pr\left(x_1 \cup x_2 \cup x_3 \cup \dots\right) = \Pr(x_1) + \Pr(x_2) + \Pr(x_3) + \dots = \sum \limits_{i = 1}^{\infty} \Pr(x_i)$
\end{enumerate}
\end{itemize}

We call these three parts --- \(\Omega\), \(\mathcal{F}\) and \(\Pr\)
--- a \enquote{probability space} or \enquote{probability triple.}

We can now formally define a random variable as follows:

\begin{definition}
Given a probability triple, $\left(\Omega, \mathcal{F}, \Pr\right)$, a random variable is a function $X: \Omega \mapsto \mathbb{R}$ such that, for all $x \in \mathbb{R}$, $\left\{\omega \in \Omega: X\left(\omega\right) \leq x \right\} \in \mathcal{F}$.
\end{definition}

Without getting into technicalities, we can unpack Definition 1 as
follows: There is a subset of events such that \(X\left(\omega\right)\)
takes on a value less than or equal to an arbitrarily chosen real
number, \(x\). If this subset of events belongs to the set of all
measurable subsets, \(\mathcal{F}\), of the sample space, \(\Omega\),
then we can coherently define a function that assigns a probability to
the event that \(X\left(\omega\right)\) takes on a value less than or
equal to \(x\). A random variable pertains to this subset of events upon
which we can coherently define a probability measure.

\hypertarget{discrete-random-variables-and-probability-mass-functions}{%
\subsection{Discrete Random Variables and Probability Mass
Functions}\label{discrete-random-variables-and-probability-mass-functions}}

A random variable that can take on a countable number of real values is
a \textit{discrete random variable}. We can now define a probability
mass function (PMF) on a discrete random variable as follows:

\begin{definition}
For a discrete random variable, $X$, the probability mass functon (PMF) of $X$ is $f\left(x\right) = \Pr\left(X = x\right)$.
\end{definition}

That is, a PMF is a function that assigns a probability to each possible
value that a random variable could take on.

Let's imagine that we have a coin that, once flipped, can land either
heads or tails. We can therefore represent the sample space as
\(\Omega \in \left\{\text{Heads}, \text{Tails} \right\}\) and the random
variable, \(Z\), as \(Z\left(\text{Heads}\right) = 0\) and
\(Z\left(\text{Tails}\right) = 1\). Now let's define a PMF, \(f(z)\):
\begin{equation}
f(z) =
\begin{cases}
p^z\left(1 - p\right)^{(1 - z)} & \text{if } z \in \left\{0, 1\right\} \\
0 & \text{otherwise},
\end{cases}
\end{equation} where \(p \in [0, 1]\) is the probability that a coin
lands tails and, hence, that \(Z = 1\).

\hypertarget{summarizing-random-variables}{%
\subsection{Summarizing Random
Variables}\label{summarizing-random-variables}}

For a random variable variable \(X\) with PMF
\(f(x) = \Pr\left(X = x\right)\), the expected value of \(X\) is:

\begin{definition}
$\E\left[X\right] = \sum_{x} x \Pr\left(X = x\right)$,
where $x$ ranges over the set of possible events in the sample space of $X$.
\end{definition}

Let's say that we have a random variable \(Z\) that can take on the
values \(Z = 1\) or \(Z = 0\). Let the probability, \(p\), that
\(Z = 1\) be equal to \(0.5\) and let the probability, \(1 - p\), that
\(Z = 0\) be equal to \(1 - 0.5 = 0.5\). What is the expected value of
the random variable \(Z\)?

\begin{align*}
\E\left[Z\right] & = \sum_{z} z \Pr\left(Z = z\right) \\
& = (1)\Pr\left(Z = 1\right) + 0\Pr\left(Z = 0\right) \\
& = (1)(0.5) + 0(1 - 0.5) \\
& = 0.5
\end{align*}

We are often also interested in the variance of a random variable, i.e.,
the expected squared distance of an outcome from the expected outcome.
\newcommand{\V}{\mathbb{V}}

\begin{definition}
\begin{align*}
\mathbb{V}\left[X\right] & = \E\left[\left(x - \E\left[X\right]\right)^2\right] \\
& = \sum_x \left(x - \E\left[X\right]\right)^2 \Pr\left(X = x\right).
\end{align*}
\end{definition}

What is the variance of the random variable \(Z\)?

\begin{align*}
\mathbb{V}\left[Z\right] & = \sum_z \left(z - \E\left[Z\right]\right)^2 \Pr\left(Z = z\right) \\
& = \left(1  - 0.5\right)^2 \Pr\left(Z = 1\right) + \left(0  - 0.5\right)^2 \Pr\left(Z = 0\right)  \\
& = \left(1  - 0.5\right)^2 \left(0.5\right) + \left(0  - 0.5\right)^2 \left(1 - 0.5\right) \\ 
& = \left(0.25\right) \left(0.5\right) + \left(0.25\right) \left(1 - 0.5\right) \\ 
& = 0.25
\end{align*}

\newtheorem{student_exercise}{Student Exercise}
\begin{student_exercise}
Let's imagine again that the random variable $Z$ can take on the values $Z = 1$ or $Z = 0$. But now the probability, $p$, that $Z = 1$ is $0.75$ and the probability, $1 - p$, that $Z = 0$ is $1 - 0.75 = 0.25$. What is the expected value of $Z$? What is the variance of $Z$?
\end{student_exercise}

\begin{Shaded}
\begin{Highlighting}[]
\NormalTok{exp_val_Z <-}\StringTok{ }\DecValTok{1} \OperatorTok{*}\StringTok{ }\FloatTok{0.75} \OperatorTok{+}\StringTok{ }\DecValTok{0} \OperatorTok{*}\StringTok{ }\NormalTok{(}\DecValTok{1} \OperatorTok{-}\StringTok{ }\FloatTok{0.75}\NormalTok{)}

\NormalTok{var_Z <-}\StringTok{ }\NormalTok{(}\DecValTok{1} \OperatorTok{-}\StringTok{ }\NormalTok{exp_val_Z)}\OperatorTok{^}\DecValTok{2} \OperatorTok{*}\StringTok{ }\FloatTok{0.75} \OperatorTok{+}\StringTok{ }\NormalTok{(}\DecValTok{0} \OperatorTok{-}\StringTok{ }\NormalTok{exp_val_Z)}\OperatorTok{^}\DecValTok{2} \OperatorTok{*}\StringTok{ }\NormalTok{(}\DecValTok{1} \OperatorTok{-}\StringTok{ }\FloatTok{0.75}\NormalTok{)}
\end{Highlighting}
\end{Shaded}

\hypertarget{research-design}{%
\section{Research Design}\label{research-design}}

At its most basic level, a research design refers to the process by
which units come to be in one study condition instead of another, i.e.,
each random \(Z_i\) comes to take on the value of \(1\) (treatment) or
\(0\) (control). More formally, we denote the collection of the values
of random variables \(Z_i\) for all
\(i \in \left\{1, \ldots , n\right\}\) units by the vector
\(\mathbf{Z}^{\prime} = \begin{bmatrix} Z_1 & \cdots & Z_n \end{bmatrix}\)
and define a research design as:

\begin{enumerate}

\item A set of possible ways (events) in which the whole vector $\mathbf{Z}$ could occur; and 

\item A probability function on this set of possible events.

\end{enumerate}

\hypertarget{simple-individual-assignment}{%
\subsection{Simple, Individual
Assignment}\label{simple-individual-assignment}}

Under completely unconstrained simple, individual assignment, the number
of units in the treatment condition can range from \(0\) to \(n\) and
the number of units in the control condition can likewise range from
\(n - 0\) to \(n - n\). For example, in an experiment with \(8\) units,
simple, individual assignment can allow \(0\), \(1\), \(2\), \(3\),
\(4\), \(5\), \(6\), \(7\) or \(8\) units to be in the treatment and
control conditions, respectively. More formally, we write the set,
\(\Omega\), of possible ways that a researcher can assign all
individuals to study conditions as follows: \begin{equation}
  \Omega = \left\{0, 1\right\}^n =
  \left\{
    \begin{bmatrix} 0 \\ 0 \\ \vdots \\ 0 \\ 0 \end{bmatrix},
    \begin{bmatrix} 1 \\ 0 \\ \vdots \\ 0 \\ 0 \end{bmatrix},
    \begin{bmatrix} 0 \\ 1 \\ \vdots \\ 0 \\ 0 \end{bmatrix},
    \cdots ,
    \begin{bmatrix} 1 \\ 0 \\ \vdots \\ 1 \\ 1 \end{bmatrix},
    \begin{bmatrix} 0 \\ 1 \\ \vdots \\ 1 \\ 1 \end{bmatrix},
    \begin{bmatrix} 1 \\ 1 \\ \vdots \\ 1 \\ 1 \end{bmatrix}
  \right\}.
  \label{eq: omega simple assignment}
\end{equation} We can write the number of possible assignments in the
set \(\Omega\) by \(\left\lvert\Omega\right\rvert\) (the ``cardinality
of Omega''), under simple, individual assignment as follows:
\begin{align*}
  \left\lvert\Omega\right\rvert & = \binom{n}{0} + \binom{n}{1} + \cdots + \binom{n}{n-1} + \binom{n}{n} \\ 
              & = \sum \limits_{n_1 = 0}^{n} \binom{n}{n_1},
\end{align*} where \(n_1 = \sum_{i = 1}^n z_i\) is the number of units
in the treatment condition, which can range from \(0\) to \(n\), and
\(\binom{n}{n_1} = \frac{n!}{\left(n - n_1\right)!n_1!}\) is the number
of ways to choose \(n_1\) units from a total of \(n\) units. Conversely,
\(n_0 = \sum_{i = 1}^n \left(1 - z_i\right)\) is the number of units in
the control condition, which can range from \(n - 0\) to \(n - n\). In
practice, researchers who control the assignment process will typically
forbid assignments in which all units are in either condition, in which
case
\(\left\lvert\Omega\right\rvert = \sum \limits_{n_1 = 1}^{n-1} \binom{n}{n_1}\).

\begin{student_exercise}
Let's imagine that in a study of $n = 8$ units, each of the $i \in \left\{1, \ldots , 8\right\}$ units is assigned to $Z = 1$ or $Z = 0$ by an independent flip of a fair coin. What is the number of possible assignments? What is the probability associated with each one of these assignments, bearing in mind that the probability of $n$ independent events, $\begin{bmatrix} z_1 & \ldots & z_n \end{bmatrix}$, is $\prod \limits_{i = 1}^n \Pr\left(Z_i = z\right)$?
\end{student_exercise}

\begin{Shaded}
\begin{Highlighting}[]
\NormalTok{sra_treated <-}\StringTok{ }\KeywordTok{lapply}\NormalTok{(}\DataTypeTok{X =} \DecValTok{0}\OperatorTok{:}\DecValTok{8}\NormalTok{,}
                      \DataTypeTok{FUN =} \ControlFlowTok{function}\NormalTok{(x) \{ }\KeywordTok{combn}\NormalTok{(}\DataTypeTok{x =} \DecValTok{8}\NormalTok{,}
                                                \DataTypeTok{m =}\NormalTok{ x) \})}

\NormalTok{sra_z_vecs <-}\StringTok{ }\KeywordTok{lapply}\NormalTok{(}\DataTypeTok{X =} \DecValTok{1}\OperatorTok{:}\KeywordTok{length}\NormalTok{(sra_treated),}
                     \DataTypeTok{FUN =} \ControlFlowTok{function}\NormalTok{(t) \{ }\KeywordTok{apply}\NormalTok{(}\DataTypeTok{X =}\NormalTok{ sra_treated[[t]],}
                                               \DataTypeTok{MARGIN =} \DecValTok{2}\NormalTok{,}
                                               \DataTypeTok{FUN =} \ControlFlowTok{function}\NormalTok{(x) \{ }\KeywordTok{as.integer}\NormalTok{(}\DecValTok{1}\OperatorTok{:}\DecValTok{8} \OperatorTok\StringTok{ }\NormalTok{x) \}) \})}

\NormalTok{sra_z_vecs_mat <-}\StringTok{ }\KeywordTok{matrix}\NormalTok{(}\DataTypeTok{data =} \KeywordTok{unlist}\NormalTok{(sra_z_vecs),}
                         \DataTypeTok{nrow =} \DecValTok{8}\NormalTok{,}
                         \DataTypeTok{byrow =} \OtherTok{FALSE}\NormalTok{)}

\NormalTok{indiv_probs <-}\StringTok{ }\KeywordTok{rep}\NormalTok{(}\DataTypeTok{x =}\NormalTok{ (}\DecValTok{1}\OperatorTok{/}\DecValTok{2}\NormalTok{), }\DataTypeTok{times =} \DecValTok{8}\NormalTok{)}

\NormalTok{sra_vec_probs <-}\StringTok{ }\KeywordTok{apply}\NormalTok{(}\DataTypeTok{X =}\NormalTok{ sra_z_vecs_mat,}
                       \DataTypeTok{MARGIN =} \DecValTok{2}\NormalTok{,}
                       \DataTypeTok{FUN =} \ControlFlowTok{function}\NormalTok{(x) \{ }\KeywordTok{prod}\NormalTok{(indiv_probs}\OperatorTok{^}\NormalTok{(x) }\OperatorTok{*}\StringTok{ }\NormalTok{(}\DecValTok{1} \OperatorTok{-}\StringTok{ }\NormalTok{indiv_probs)}\OperatorTok{^}\NormalTok{(}\DecValTok{1} \OperatorTok{-}\StringTok{ }\NormalTok{x)) \})}
\end{Highlighting}
\end{Shaded}

\begin{student_exercise}
Under the random assignment method described above, what is $\E\left[\mathbf{Z}^{\prime}\mathbf{Z}\right]$? What is $\mathbb{V}\left[\mathbf{Z}^{\prime}\mathbf{Z}\right]$?
\end{student_exercise}

\begin{Shaded}
\begin{Highlighting}[]
\NormalTok{N_1s <-}\StringTok{ }\KeywordTok{apply}\NormalTok{(}\DataTypeTok{X =}\NormalTok{ sra_z_vecs_mat,}
              \DataTypeTok{MARGIN =} \DecValTok{2}\NormalTok{,}
              \DataTypeTok{FUN =}\NormalTok{ sum)}

\NormalTok{N_}\DecValTok{1}\NormalTok{_probs <-}\StringTok{ }\KeywordTok{sapply}\NormalTok{(}\DataTypeTok{X =} \DecValTok{0}\OperatorTok{:}\DecValTok{8}\NormalTok{,}
                    \DataTypeTok{FUN =} \ControlFlowTok{function}\NormalTok{(n_}\DecValTok{1}\NormalTok{) \{ }\KeywordTok{sum}\NormalTok{(sra_vec_probs[}\KeywordTok{which}\NormalTok{(N_1s }\OperatorTok{==}\StringTok{ }\NormalTok{n_}\DecValTok{1}\NormalTok{)])  \})}

\NormalTok{N_}\DecValTok{1}\NormalTok{_ran_var <-}\StringTok{ }\KeywordTok{cbind}\NormalTok{(}\DecValTok{0}\OperatorTok{:}\DecValTok{8}\NormalTok{, N_}\DecValTok{1}\NormalTok{_probs)}

\KeywordTok{colnames}\NormalTok{(N_}\DecValTok{1}\NormalTok{_ran_var) <-}\StringTok{ }\KeywordTok{c}\NormalTok{(}\StringTok{"N_1"}\NormalTok{, }\StringTok{"Prob"}\NormalTok{)}
\end{Highlighting}
\end{Shaded}

\begin{Shaded}
\begin{Highlighting}[]
\NormalTok{exp_val_n_}\DecValTok{1}\NormalTok{ <-}\StringTok{ }\KeywordTok{sum}\NormalTok{(N_}\DecValTok{1}\NormalTok{_ran_var[,}\StringTok{"N_1"}\NormalTok{] }\OperatorTok{*}\StringTok{ }\NormalTok{N_}\DecValTok{1}\NormalTok{_ran_var[,}\StringTok{"Prob"}\NormalTok{])}

\NormalTok{var_n_}\DecValTok{1}\NormalTok{ <-}\StringTok{ }\KeywordTok{sum}\NormalTok{((N_}\DecValTok{1}\NormalTok{_ran_var[,}\StringTok{"N_1"}\NormalTok{] }\OperatorTok{-}\StringTok{ }\NormalTok{exp_val_n_}\DecValTok{1}\NormalTok{)}\OperatorTok{^}\DecValTok{2} \OperatorTok{*}\StringTok{ }\NormalTok{N_}\DecValTok{1}\NormalTok{_ran_var[,}\StringTok{"Prob"}\NormalTok{])}
\end{Highlighting}
\end{Shaded}

\hypertarget{complete-individual-assignment}{%
\subsection{Complete, Individual
Assignment}\label{complete-individual-assignment}}

Complete, individual assignment differs from simple, individual
assignment only in that the value of \(n_1\) is fixed across all
possible assignments. In other words, we are conditioning on the event
that \(N_1 = n_1\) by assigning a probability of \(0\) to all
assignments in which \(N_1 \neq n_1\).

For example, the random assignment process in Fisher's lady tasting tea
experiment is such that exactly \(4\) cups are assigned to treatment
(\enquote{milk first}) and \(8 - 4 = 4\) cups are assigned to control
(\enquote{tea first}). Under this assignment process, there are
\(\binom{8}{4} = 70\) possible assignments.

To show that the probability of each assignment is \(\cfrac{1}{70}\), we
can appeal to the definition of conditional probability.

\begin{definition}
Given two events $x$ and $y$, where $\Pr\left(y\right) > 0$, the conditional probability of $x$ given $y$ is
\begin{align*}
\Pr\left(x \vert y \right) & = \cfrac{\Pr\left(x, y\right)}{\Pr\left(y\right)}.
\end{align*}
\end{definition}

By the definition of conditional probability, \begin{align*}
\cfrac{\Pr\left(\mathbf{Z} = \mathbf{z}, \mathbf{Z}^{\prime}\mathbf{Z} = 4\right)}{\Pr\left(\mathbf{Z}^{\prime}\mathbf{Z} = 4\right)}
\end{align*}

By the definition of joint probability, we know that
\(\Pr\left(\mathbf{Z} = \mathbf{z}, \mathbf{Z}^{\prime}\mathbf{Z} = 4\right) \equiv \Pr\left(\mathbf{Z} = \mathbf{z}\right)\Pr\left(\mathbf{Z}^{\prime}\mathbf{Z} = 4 \vert \mathbf{Z} = \mathbf{z}\right)\).
Since
\(\Pr\left(\mathbf{Z}^{\prime}\mathbf{Z} = 4 \vert \mathbf{Z} = \mathbf{z}\right)\)
can equal only 0 or 1 depending on whether the assignment vector
\(\mathbf{z}\) has four treated units or not, we can represent the joint
probability of \(\mathbf{Z} = \mathbf{z}\) and
\(\mathbf{Z}^{\prime}\mathbf{Z}\) as follows:

\begin{align*}
\Pr\left(\mathbf{Z} = \mathbf{z}, \mathbf{Z}^{\prime}\mathbf{Z} = 4\right) & = \begin{cases}
0 & \text{if } \sum \limits_{i = 1}^n z_i \neq 4 \\
\prod_{i = 1}^8 0.5^{z_i}\left(1 - 0.5\right)^{(1 - z_i)} & \text{otherwise} 
\end{cases}
\end{align*}

Now let's consider
\(\Pr\left(\mathbf{Z}^{\prime}\mathbf{Z}\right) = 4\). Formally, let the
index
\(j \in \left\{1, \dots, \#\Omega \vert \mathbf{z}^{\prime}\mathbf{z} = 4\right\}\)
run over the set of all assignment vectors with exactly four treatment
units. Based on the axiom of countable additivity (this is the third
axiom of probability, which you should look up if more interested), the
probability of the event that \(\mathbf{z}^{\prime}\mathbf{z} = 4\) is
simply the sum of probabilities of assignment vectors in which
\(\mathbf{z}^{\prime}\mathbf{z} = 4\). Hence, \begin{align*}
\Pr\left(\mathbf{Z}^{\prime}\mathbf{Z} = 4\right) & = \sum_{j = 1} \Pr\left(\mathbf{z}_j\right) 
\end{align*}

\begin{Shaded}
\begin{Highlighting}[]
\NormalTok{cra_total_prob <-}\StringTok{ }\KeywordTok{sum}\NormalTok{(sra_vec_probs[}\KeywordTok{which}\NormalTok{(N_1s }\OperatorTok{==}\StringTok{ }\DecValTok{4}\NormalTok{)])}

\NormalTok{cra_vec_probs <-}\StringTok{ }\NormalTok{sra_vec_probs[}\KeywordTok{which}\NormalTok{(N_1s }\OperatorTok{==}\StringTok{ }\DecValTok{4}\NormalTok{)]}\OperatorTok{/}\NormalTok{cra_total_prob}

\NormalTok{n <-}\StringTok{ }\DecValTok{8}
\NormalTok{n_}\DecValTok{1}\NormalTok{ <-}\StringTok{ }\DecValTok{4}

\NormalTok{treated <-}\StringTok{ }\KeywordTok{combn}\NormalTok{(}\DataTypeTok{x =} \DecValTok{1}\OperatorTok{:}\NormalTok{n,}
                 \DataTypeTok{m =}\NormalTok{ n_}\DecValTok{1}\NormalTok{) }

\NormalTok{Omega <-}\StringTok{ }\KeywordTok{apply}\NormalTok{(}\DataTypeTok{X =}\NormalTok{ treated,}
               \DataTypeTok{MARGIN =} \DecValTok{2}\NormalTok{,}
               \DataTypeTok{FUN =} \ControlFlowTok{function}\NormalTok{(x) \{ }\KeywordTok{as.integer}\NormalTok{(}\DecValTok{1}\OperatorTok{:}\NormalTok{n }\OperatorTok\StringTok{ }\NormalTok{x) \})}

\NormalTok{cra_vec_probs <-}\StringTok{ }\KeywordTok{rep}\NormalTok{(}\DataTypeTok{x =}\NormalTok{ (}\DecValTok{1}\OperatorTok{/}\KeywordTok{ncol}\NormalTok{(Omega)), }\DataTypeTok{times =} \KeywordTok{ncol}\NormalTok{(Omega))}
\end{Highlighting}
\end{Shaded}

\hypertarget{why-does-research-design-matter-hypothesis-testing}{%
\section{Why Does Research Design Matter? Hypothesis
Testing}\label{why-does-research-design-matter-hypothesis-testing}}

\hypertarget{null-and-alternative-hypotheses}{%
\subsection{Null and Alternative
Hypotheses}\label{null-and-alternative-hypotheses}}

Recall that the observed data of Fisher's lady tasting tea experiment
were as follows:

\begin{table}[H]
\centering
    \begin{tabular}{l|l|l|l|l}
    Unit & $\mathbf{z}$ & $\mathbf{y}$ & $\mathbf{y_c}$ & $\mathbf{y_t}$ \\ \hline
    1    & 1            & 1            & ?                & 1 \\
    2    & 1            & 1            & ?                & 1  \\
    3    & 1            & 1            & ?                & 1  \\
    4    & 1            & 1            & ?                & 1  \\
    5    & 0            & 0            & 0                & ?  \\
    6    & 0            & 0            & 0                & ?  \\
    7    & 0            & 0            & 0                & ?  \\
    8    & 0            & 0            & 0                & ?  \\
    \end{tabular}
    \caption{Observed Data}
\end{table}

\begin{Shaded}
\begin{Highlighting}[]
\NormalTok{tea_data_frame <-}\StringTok{ }\KeywordTok{data.frame}\NormalTok{(}\DataTypeTok{unit =} \DecValTok{1}\OperatorTok{:}\DecValTok{8}\NormalTok{,}
                             \DataTypeTok{z =} \KeywordTok{c}\NormalTok{(}\KeywordTok{rep}\NormalTok{(}\DataTypeTok{x =} \DecValTok{1}\NormalTok{, }\DataTypeTok{times =} \DecValTok{4}\NormalTok{),}
                                   \KeywordTok{rep}\NormalTok{(}\DataTypeTok{x =} \DecValTok{0}\NormalTok{, }\DataTypeTok{times =} \DecValTok{4}\NormalTok{)),}
                             \DataTypeTok{y =} \KeywordTok{c}\NormalTok{(}\KeywordTok{rep}\NormalTok{(}\DataTypeTok{x =} \DecValTok{1}\NormalTok{, }\DataTypeTok{times =} \DecValTok{4}\NormalTok{),}
                                   \KeywordTok{rep}\NormalTok{(}\DataTypeTok{x =} \DecValTok{0}\NormalTok{, }\DataTypeTok{times =} \DecValTok{4}\NormalTok{)),}
                             \DataTypeTok{y_c =} \KeywordTok{c}\NormalTok{(}\KeywordTok{rep}\NormalTok{(}\DataTypeTok{x =} \OtherTok{NA}\NormalTok{, }\DataTypeTok{times =} \DecValTok{4}\NormalTok{),}
                                     \KeywordTok{rep}\NormalTok{(}\DataTypeTok{x =} \DecValTok{0}\NormalTok{, }\DataTypeTok{times =} \DecValTok{4}\NormalTok{)),}
                             \DataTypeTok{y_t =} \KeywordTok{c}\NormalTok{(}\KeywordTok{rep}\NormalTok{(}\DataTypeTok{x =} \DecValTok{1}\NormalTok{, }\DataTypeTok{times =} \DecValTok{4}\NormalTok{),}
                                     \KeywordTok{rep}\NormalTok{(}\DataTypeTok{x =} \OtherTok{NA}\NormalTok{, }\DataTypeTok{times =} \DecValTok{4}\NormalTok{)))}

\NormalTok{tea_data_frame}
\end{Highlighting}
\end{Shaded}

Each cup in the experiment has two potential outcomes, \(y_{c_i}\) and
\(y_{t_i}\), but only one of these two outcomes can be observed. The
potential outcome that is observed is a random variable defined as
\(Y_i = Z_i y_{t_i} + \left(1 - Z_i\right)y_{c_i}\). Notice that \(Y_i\)
inherits its randomness solely from \(Z_i\).

The sharp null hypothesis postulates that each unit's treatment
potential outcome, \(y_{t_i}\), is equal to its control potential
outcome, \(y_{c_i}\). More formally, the sharp null hypothesis states
that \(y_{c,i} = y_{t,i}\) for all
\(i \in \left\{1, \ldots , n\right\}\) units.

In other words, the outcome that was observed for each cup (i.e.,
whether it was marked as \enquote{milk first} or \enquote{tea first})
would have been the same had that cup been assigned to the alternative
experimental condition.

\begin{table}[H]
\centering
    \begin{tabular}{l|l|l|l|l}
    Unit & $\mathbf{z}$ & $\mathbf{y}$ & $\mathbf{y_c}$ & $\mathbf{y_t}$ \\ \hline
    1    & 1            & 1            & 1                & 1 \\
    2    & 1            & 1            & 1                & 1  \\
    3    & 1            & 1            & 1                & 1  \\
    4    & 1            & 1            & 1                & 1  \\
    5    & 0            & 0            & 0                & 0  \\
    6    & 0            & 0            & 0                & 0  \\
    7    & 0            & 0            & 0                & 0  \\
    8    & 0            & 0            & 0                & 0  \\
    \end{tabular}
    \caption{Potential Outcomes under the Sharp Null Hypothesis of No Effect}    
\end{table}

\begin{Shaded}
\begin{Highlighting}[]
\NormalTok{sharp_null_tea_data_frame <-}\StringTok{ }\KeywordTok{data.frame}\NormalTok{(}\DataTypeTok{unit =} \DecValTok{1}\OperatorTok{:}\DecValTok{8}\NormalTok{,}
                                        \DataTypeTok{z =} \KeywordTok{c}\NormalTok{(}\KeywordTok{rep}\NormalTok{(}\DataTypeTok{x =} \DecValTok{1}\NormalTok{, }\DataTypeTok{times =} \DecValTok{4}\NormalTok{),}
                                              \KeywordTok{rep}\NormalTok{(}\DataTypeTok{x =} \DecValTok{0}\NormalTok{, }\DataTypeTok{times =} \DecValTok{4}\NormalTok{)),}
                                        \DataTypeTok{y =} \KeywordTok{c}\NormalTok{(}\KeywordTok{rep}\NormalTok{(}\DataTypeTok{x =} \DecValTok{1}\NormalTok{, }\DataTypeTok{times =} \DecValTok{4}\NormalTok{),}
                                              \KeywordTok{rep}\NormalTok{(}\DataTypeTok{x =} \DecValTok{0}\NormalTok{, }\DataTypeTok{times =} \DecValTok{4}\NormalTok{)),}
                                        \DataTypeTok{y_c =} \KeywordTok{c}\NormalTok{(}\KeywordTok{rep}\NormalTok{(}\DataTypeTok{x =} \DecValTok{1}\NormalTok{, }\DataTypeTok{times =} \DecValTok{4}\NormalTok{),}
                                                \KeywordTok{rep}\NormalTok{(}\DataTypeTok{x =} \DecValTok{0}\NormalTok{, }\DataTypeTok{times =} \DecValTok{4}\NormalTok{)),}
                                        \DataTypeTok{y_t =} \KeywordTok{c}\NormalTok{(}\KeywordTok{rep}\NormalTok{(}\DataTypeTok{x =} \DecValTok{1}\NormalTok{, }\DataTypeTok{times =} \DecValTok{4}\NormalTok{),}
                                                \KeywordTok{rep}\NormalTok{(}\DataTypeTok{x =} \DecValTok{0}\NormalTok{, }\DataTypeTok{times =} \DecValTok{4}\NormalTok{)))}

\NormalTok{sharp_null_tea_data_frame}
\end{Highlighting}
\end{Shaded}

The alternative hypothesis of ``perfect discrimination,'' by contrast,
stipulates that \(y_{c,i} = 0, y_{t,i} = 1\) for all
\(i \in \left\{1, \dots, n \right\}\) units.

\begin{table}[H]
\centering
    \begin{tabular}{l|l|l|l|l}
    Unit & $\mathbf{z}$ & $\mathbf{y}$ & $\mathbf{y_c}$ & $\mathbf{y_t}$ \\ \hline
    1    & 1            & 1            & 0                & 1 \\
    2    & 1            & 1            & 0                & 1  \\
    3    & 1            & 1            & 0                & 1  \\
    4    & 1            & 1            & 0                & 1  \\
    5    & 0            & 0            & 0                & 1  \\
    6    & 0            & 0            & 0                & 1  \\
    7    & 0            & 0            & 0                & 1  \\
    8    & 0            & 0            & 0                & 1  \\
    \end{tabular}
    \caption{Potential Outcomes under the Alternative Hypothesis}    
\end{table}

\begin{Shaded}
\begin{Highlighting}[]
\NormalTok{alt_hyp_tea_data_frame <-}\StringTok{ }\KeywordTok{data.frame}\NormalTok{(}\DataTypeTok{unit =} \DecValTok{1}\OperatorTok{:}\DecValTok{8}\NormalTok{,}
                                     \DataTypeTok{z =} \KeywordTok{c}\NormalTok{(}\KeywordTok{rep}\NormalTok{(}\DataTypeTok{x =} \DecValTok{1}\NormalTok{, }\DataTypeTok{times =} \DecValTok{4}\NormalTok{),}
                                           \KeywordTok{rep}\NormalTok{(}\DataTypeTok{x =} \DecValTok{0}\NormalTok{, }\DataTypeTok{times =} \DecValTok{4}\NormalTok{)),}
                                     \DataTypeTok{y =} \KeywordTok{c}\NormalTok{(}\KeywordTok{rep}\NormalTok{(}\DataTypeTok{x =} \DecValTok{1}\NormalTok{, }\DataTypeTok{times =} \DecValTok{4}\NormalTok{),}
                                           \KeywordTok{rep}\NormalTok{(}\DataTypeTok{x =} \DecValTok{0}\NormalTok{, }\DataTypeTok{times =} \DecValTok{4}\NormalTok{)),}
                                     \DataTypeTok{y_c =} \KeywordTok{c}\NormalTok{(}\KeywordTok{rep}\NormalTok{(}\DataTypeTok{x =} \DecValTok{0}\NormalTok{, }\DataTypeTok{times =} \DecValTok{4}\NormalTok{),}
                                             \KeywordTok{rep}\NormalTok{(}\DataTypeTok{x =} \DecValTok{0}\NormalTok{, }\DataTypeTok{times =} \DecValTok{4}\NormalTok{)),}
                                     \DataTypeTok{y_t =} \KeywordTok{c}\NormalTok{(}\KeywordTok{rep}\NormalTok{(}\DataTypeTok{x =} \DecValTok{1}\NormalTok{, }\DataTypeTok{times =} \DecValTok{4}\NormalTok{),}
                                             \KeywordTok{rep}\NormalTok{(}\DataTypeTok{x =} \DecValTok{1}\NormalTok{, }\DataTypeTok{times =} \DecValTok{4}\NormalTok{)))}

\NormalTok{alt_hyp_tea_data_frame}
\end{Highlighting}
\end{Shaded}

\hypertarget{test-statistics}{%
\subsection{Test Statistics}\label{test-statistics}}

Fisher's test statistic is \(\mathbf{z}'\mathbf{y}\). Other test
statistics are also possible. For example, the proportion of the
focal-group cups that were correctly identified is
\(\mathbf{Z}'\mathbf{y}/n_1\), where \(n_{1}\) is the design constant
\(\sum_{i} Z_{i}\), here 4.

Yet another possibility would be the difference in proportions of focal
and non-focal group cups that were identified as being in the focal
group, \(\mathbf{Z}'\mathbf{y}/n_{1} - \mathbf{Z}'\mathbf{y}/n_{0}\),
where \(n_{0}= \sum_{i} 1- Z_{i} = 4\).

We can calculate these test statistics on the realized experimental data
to get our observed test statistic.

\begin{Shaded}
\begin{Highlighting}[]
\NormalTok{z <-}\StringTok{ }\KeywordTok{c}\NormalTok{(}\KeywordTok{rep}\NormalTok{(}\DataTypeTok{x =} \DecValTok{1}\NormalTok{, }\DataTypeTok{times =} \DecValTok{4}\NormalTok{), }\KeywordTok{rep}\NormalTok{(}\DataTypeTok{x =} \DecValTok{0}\NormalTok{, }\DataTypeTok{times =} \DecValTok{4}\NormalTok{))}
\NormalTok{y <-}\StringTok{ }\KeywordTok{c}\NormalTok{(}\KeywordTok{rep}\NormalTok{(}\DataTypeTok{x =} \DecValTok{1}\NormalTok{, }\DataTypeTok{times =} \DecValTok{4}\NormalTok{), }\KeywordTok{rep}\NormalTok{(}\DataTypeTok{x =} \DecValTok{0}\NormalTok{, }\DataTypeTok{times =} \DecValTok{4}\NormalTok{))}
\NormalTok{n <-}\StringTok{ }\KeywordTok{length}\NormalTok{(z)}
\NormalTok{n_t <-}\StringTok{ }\KeywordTok{sum}\NormalTok{(z)}

\NormalTok{obs_treat_sum <-}\StringTok{ }\KeywordTok{as.integer}\NormalTok{(}\KeywordTok{t}\NormalTok{(z) }\OperatorTok\StringTok{ }\NormalTok{y)}

\NormalTok{obs_treat_mean <-}\StringTok{ }\NormalTok{(}\DecValTok{1}\OperatorTok{/}\KeywordTok{sum}\NormalTok{(z)) }\OperatorTok{*}\StringTok{ }\KeywordTok{as.integer}\NormalTok{(}\KeywordTok{t}\NormalTok{(z) }\OperatorTok\StringTok{ }\NormalTok{y)}

\NormalTok{obs_mean_diff <-}\StringTok{ }\NormalTok{(}\DecValTok{1}\OperatorTok{/}\KeywordTok{sum}\NormalTok{(z)) }\OperatorTok{*}\StringTok{ }\KeywordTok{as.integer}\NormalTok{(}\KeywordTok{t}\NormalTok{(z) }\OperatorTok\StringTok{ }\NormalTok{y) }\OperatorTok{-}
\StringTok{  }\NormalTok{(}\DecValTok{1}\OperatorTok{/}\KeywordTok{sum}\NormalTok{(}\DecValTok{1} \OperatorTok{-}\StringTok{ }\NormalTok{z)) }\OperatorTok{*}\StringTok{ }\KeywordTok{as.integer}\NormalTok{(}\KeywordTok{t}\NormalTok{(}\DecValTok{1} \OperatorTok{-}\StringTok{ }\NormalTok{z) }\OperatorTok\StringTok{ }\NormalTok{y)}
\end{Highlighting}
\end{Shaded}

\hypertarget{exact-p-values}{%
\subsection{Exact P-Values}\label{exact-p-values}}

We first need to define a null distribution of the test statistic, i.e.,
the test statistics that would be observed under all possible random
assignments if the sharp null were true. Let's write functions for each
of these test statistics:

\begin{Shaded}
\begin{Highlighting}[]
\NormalTok{treat_sum <-}\StringTok{ }\ControlFlowTok{function}\NormalTok{(.Z,}
\NormalTok{                      .y_c,}
\NormalTok{                      .y_t) \{}
  
\NormalTok{  y =}\StringTok{ }\NormalTok{.Z }\OperatorTok{*}\StringTok{ }\NormalTok{.y_t }\OperatorTok{+}\StringTok{ }\NormalTok{(}\DecValTok{1} \OperatorTok{-}\StringTok{ }\NormalTok{.Z) }\OperatorTok{*}\StringTok{ }\NormalTok{.y_c}
  
  \KeywordTok{return}\NormalTok{( }\KeywordTok{as.integer}\NormalTok{(}\KeywordTok{t}\NormalTok{(.Z) }\OperatorTok\StringTok{ }\NormalTok{y) )}
  
\NormalTok{\}}

\NormalTok{treat_mean <-}\StringTok{ }\ControlFlowTok{function}\NormalTok{(.Z, .y_c, .y_t) \{}
  
\NormalTok{  y =}\StringTok{ }\NormalTok{.Z }\OperatorTok{*}\StringTok{ }\NormalTok{.y_t }\OperatorTok{+}\StringTok{ }\NormalTok{(}\DecValTok{1} \OperatorTok{-}\StringTok{ }\NormalTok{.Z) }\OperatorTok{*}\StringTok{ }\NormalTok{.y_c}
  
  \KeywordTok{return}\NormalTok{( (}\DecValTok{1}\OperatorTok{/}\KeywordTok{sum}\NormalTok{(.Z)) }\OperatorTok{*}\StringTok{ }\KeywordTok{as.integer}\NormalTok{(}\KeywordTok{t}\NormalTok{(.Z) }\OperatorTok\StringTok{ }\NormalTok{y) ) }
  
\NormalTok{\}}

\NormalTok{mean_diff <-}\StringTok{ }\ControlFlowTok{function}\NormalTok{(.Z, .y_c, .y_t) \{}
  
\NormalTok{  y =}\StringTok{ }\NormalTok{.Z }\OperatorTok{*}\StringTok{ }\NormalTok{.y_t }\OperatorTok{+}\StringTok{ }\NormalTok{(}\DecValTok{1} \OperatorTok{-}\StringTok{ }\NormalTok{.Z) }\OperatorTok{*}\StringTok{ }\NormalTok{.y_c}
  
  \KeywordTok{return}\NormalTok{( (}\DecValTok{1}\OperatorTok{/}\KeywordTok{sum}\NormalTok{(.Z)) }\OperatorTok{*}\StringTok{ }\KeywordTok{as.integer}\NormalTok{(}\KeywordTok{t}\NormalTok{(.Z) }\OperatorTok\StringTok{ }\NormalTok{y) }\OperatorTok{-}\StringTok{ }\NormalTok{(}\DecValTok{1}\OperatorTok{/}\KeywordTok{sum}\NormalTok{(}\DecValTok{1} \OperatorTok{-}\StringTok{ }\NormalTok{.Z)) }\OperatorTok{*}\StringTok{ }\KeywordTok{as.integer}\NormalTok{(}\KeywordTok{t}\NormalTok{(}\DecValTok{1} \OperatorTok{-}\StringTok{ }\NormalTok{.Z) }\OperatorTok\StringTok{ }\NormalTok{y) ) }
  
\NormalTok{  \}}
\end{Highlighting}
\end{Shaded}

Now let's generate the null distribution of Fisher's test statistic:

\begin{Shaded}
\begin{Highlighting}[]
\NormalTok{y_c_null <-}\StringTok{ }\KeywordTok{c}\NormalTok{(}\KeywordTok{rep}\NormalTok{(}\DataTypeTok{x =} \DecValTok{1}\NormalTok{, }\DataTypeTok{times =} \DecValTok{4}\NormalTok{), }\KeywordTok{rep}\NormalTok{(}\DataTypeTok{x =} \DecValTok{0}\NormalTok{, }\DataTypeTok{times =} \DecValTok{4}\NormalTok{))}
\NormalTok{y_t_null <-}\StringTok{ }\KeywordTok{c}\NormalTok{(}\KeywordTok{rep}\NormalTok{(}\DataTypeTok{x =} \DecValTok{1}\NormalTok{, }\DataTypeTok{times =} \DecValTok{4}\NormalTok{), }\KeywordTok{rep}\NormalTok{(}\DataTypeTok{x =} \DecValTok{0}\NormalTok{, }\DataTypeTok{times =} \DecValTok{4}\NormalTok{))}

\NormalTok{null_test_stats <-}\StringTok{ }\KeywordTok{apply}\NormalTok{(}\DataTypeTok{X =}\NormalTok{ Omega,}
                         \DataTypeTok{MARGIN =} \DecValTok{2}\NormalTok{,}
                         \DataTypeTok{FUN =} \ControlFlowTok{function}\NormalTok{(x) \{ }\KeywordTok{treat_sum}\NormalTok{(}\DataTypeTok{.Z =}\NormalTok{ x,}
                                                       \DataTypeTok{.y_c =}\NormalTok{ y_c_null,}
                                                       \DataTypeTok{.y_t =}\NormalTok{ y_t_null) \})}

\NormalTok{null_prob_dist <-}\StringTok{ }\KeywordTok{cbind}\NormalTok{(}\DecValTok{0}\OperatorTok{:}\DecValTok{4}\NormalTok{,}
                        \KeywordTok{sapply}\NormalTok{(}\DataTypeTok{X =} \DecValTok{0}\OperatorTok{:}\DecValTok{4}\NormalTok{,}
                               \DataTypeTok{FUN =} \ControlFlowTok{function}\NormalTok{(x) \{ }\KeywordTok{sum}\NormalTok{((null_test_stats }\OperatorTok{==}\StringTok{ }\NormalTok{x) }\OperatorTok{*}\StringTok{ }\NormalTok{cra_vec_probs) \}))}

\KeywordTok{colnames}\NormalTok{(null_prob_dist) <-}\StringTok{ }\KeywordTok{c}\NormalTok{(}\StringTok{"Test_Stat"}\NormalTok{, }\StringTok{"Prob"}\NormalTok{)}

\NormalTok{null_prob_dist}
\end{Highlighting}
\end{Shaded}

To calculate the one sided p-value, we need to know the probability of
observing a null test statistic greater than or equal to the observed
test statistic. Formally, we can write this p-value as follows:

\begin{equation}
\Pr\left(t\left(\mathbf{z}, \mathbf{y}_0 \right) \geq T \right) = \sum \limits_{\mathbf{z} \in \Omega} \mathbbm{1}\left[t\left(\mathbf{z}, \mathbf{y}_0 \right) \geq T\right] \Pr\left(\mathbf{Z} = \mathbf{z}\right),
\end{equation} where \(\mathbbm{1}\left[\cdot\right]\) is an indicator
function that is \(1\) if the argument to the function is true and \(0\)
if it is false.

Now let's calculate the p-value:

\begin{Shaded}
\begin{Highlighting}[]
\NormalTok{exact_p_value <-}\StringTok{ }\KeywordTok{sum}\NormalTok{((null_prob_dist[,}\StringTok{"Test_Stat"}\NormalTok{] }\OperatorTok{>=}\StringTok{ }\NormalTok{obs_treat_sum) }\OperatorTok{*}\StringTok{ }\NormalTok{null_prob_dist[,}\StringTok{"Prob"}\NormalTok{])}
\end{Highlighting}
\end{Shaded}

We can also get the same exact p-value from the \texttt{fisher.test}
command in \texttt{[R]}:

\begin{Shaded}
\begin{Highlighting}[]
\NormalTok{coffee_experiment <-}\StringTok{ }\KeywordTok{matrix}\NormalTok{(}\DataTypeTok{data =} \KeywordTok{c}\NormalTok{(}\DecValTok{4}\NormalTok{, }\DecValTok{0}\NormalTok{, }\DecValTok{0}\NormalTok{, }\DecValTok{4}\NormalTok{),}
                            \DataTypeTok{nrow =} \DecValTok{2}\NormalTok{,}
                            \DataTypeTok{ncol =} \DecValTok{2}\NormalTok{,}
                            \DataTypeTok{byrow =} \OtherTok{TRUE}\NormalTok{)}

\KeywordTok{colnames}\NormalTok{(coffee_experiment) <-}\StringTok{ }\KeywordTok{c}\NormalTok{(}\StringTok{"0"}\NormalTok{, }\StringTok{"1"}\NormalTok{)}

\KeywordTok{rownames}\NormalTok{(coffee_experiment) <-}\StringTok{ }\KeywordTok{c}\NormalTok{(}\StringTok{"0"}\NormalTok{, }\StringTok{"1"}\NormalTok{)}

\KeywordTok{fisher.test}\NormalTok{(}\DataTypeTok{x =}\NormalTok{ coffee_experiment,}
            \DataTypeTok{alternative =} \StringTok{"greater"}\NormalTok{)}
\end{Highlighting}
\end{Shaded}

The exact p-value from the Fisher test is
\(\frac{1}{70} \approx 0.01429\).

\hypertarget{type-i-and-type-ii-errors}{%
\subsection{Type I and Type II Errors}\label{type-i-and-type-ii-errors}}

Hypothesis tests are subject to at least two types of errors. One could,
first, reject the null hypothesis when it is true (a type I error) or,
second, fail to reject the null hypothesis when it is false (a type II
error). Two features of hypothesis tests related to these two potential
errors are the \(\alpha\) size of the test and the \textit{power} of the
test. We now define the \(\alpha\) size (as distinct from the \(\alpha\)
level) and power of hypothesis tests.

A test's \(\alpha\) \textit{level} is, in the words of
\citet{rosenbaum2010}, that test's \enquote{promise} that the
probability of a Type I error (i.e., the probability of a \(p\)-value
less than \(\alpha\) when the null hypothesis is true) is less than or
equal to the \(\alpha\) level. The test's \(\alpha\) \textit{size}, on
the other hand, is the test's true probability of a Type I error, which,
in general, can be greater than, equal to or less than the \(\alpha\)
level ``promised'' by the test. In contrast to the \(\alpha\) level and
size of a test, a test's power is the probability of a \(p\)-value less
than the \(\alpha\) level when the null hypothesis is false. In other
words, power is \(1\) minus the Type II error probability; hence, as the
power of a test increases, the Type II error probability decreases.

Let's calculate the probability of rejecting the sharp null when it is
true (i.e., the type I error rate).

\begin{Shaded}
\begin{Highlighting}[]
\NormalTok{y_c_null_true <-}\StringTok{ }\KeywordTok{c}\NormalTok{(}\KeywordTok{rep}\NormalTok{(}\DataTypeTok{x =} \DecValTok{1}\NormalTok{, }\DataTypeTok{times =} \DecValTok{4}\NormalTok{), }\KeywordTok{rep}\NormalTok{(}\DataTypeTok{x =} \DecValTok{0}\NormalTok{, }\DataTypeTok{times =} \DecValTok{4}\NormalTok{))}
\NormalTok{y_t_null_true <-}\StringTok{ }\KeywordTok{c}\NormalTok{(}\KeywordTok{rep}\NormalTok{(}\DataTypeTok{x =} \DecValTok{1}\NormalTok{, }\DataTypeTok{times =} \DecValTok{4}\NormalTok{), }\KeywordTok{rep}\NormalTok{(}\DataTypeTok{x =} \DecValTok{0}\NormalTok{, }\DataTypeTok{times =} \DecValTok{4}\NormalTok{))}

\NormalTok{get_p_value <-}\StringTok{ }\ControlFlowTok{function}\NormalTok{(.y_c,}
\NormalTok{                        .y_t,}
\NormalTok{                        .Omega,}
\NormalTok{                        .probs) \{}
  
\NormalTok{  obs_outs =}\StringTok{ }\KeywordTok{sapply}\NormalTok{(}\DataTypeTok{X =} \DecValTok{1}\OperatorTok{:}\KeywordTok{ncol}\NormalTok{(.Omega),}
                    \DataTypeTok{FUN =} \ControlFlowTok{function}\NormalTok{(x) \{ Omega[,x] }\OperatorTok{*}\StringTok{ }\NormalTok{.y_t }\OperatorTok{+}\StringTok{ }\NormalTok{(}\DecValTok{1} \OperatorTok{-}\StringTok{ }\NormalTok{Omega[,x]) }\OperatorTok{*}\StringTok{ }\NormalTok{.y_c \})}
  
\NormalTok{  obs_test_stats =}\StringTok{ }\KeywordTok{sapply}\NormalTok{(}\DataTypeTok{X =} \DecValTok{1}\OperatorTok{:}\KeywordTok{ncol}\NormalTok{(.Omega),}
                          \DataTypeTok{FUN =} \ControlFlowTok{function}\NormalTok{(x)\{ }\KeywordTok{as.integer}\NormalTok{(}\KeywordTok{t}\NormalTok{(Omega[,x]) }\OperatorTok\StringTok{ }\NormalTok{obs_outs[,x]) \})}
  
\NormalTok{  null_dists <-}\StringTok{ }\KeywordTok{list}\NormalTok{()}
  
  \ControlFlowTok{for}\NormalTok{(i }\ControlFlowTok{in} \DecValTok{1}\OperatorTok{:}\KeywordTok{ncol}\NormalTok{(.Omega))\{}
    
\NormalTok{    null_dists[[i]] =}\StringTok{ }\KeywordTok{sapply}\NormalTok{(}\DataTypeTok{X =} \DecValTok{1}\OperatorTok{:}\KeywordTok{ncol}\NormalTok{(.Omega),}
                             \DataTypeTok{FUN =} \ControlFlowTok{function}\NormalTok{(x) \{ }\KeywordTok{as.integer}\NormalTok{(}\KeywordTok{t}\NormalTok{(Omega[,x]) }\OperatorTok\StringTok{ }\NormalTok{obs_outs[,i]) \})}
\NormalTok{  \}}
  
\NormalTok{  p_values =}\StringTok{ }\KeywordTok{sapply}\NormalTok{(}\DataTypeTok{X =} \DecValTok{1}\OperatorTok{:}\KeywordTok{ncol}\NormalTok{(.Omega),}
                    \DataTypeTok{FUN =} \ControlFlowTok{function}\NormalTok{(x) \{ }\KeywordTok{sum}\NormalTok{((null_dists[[x]] }\OperatorTok{>=}\StringTok{ }\NormalTok{obs_test_stats[x]) }\OperatorTok{*}\StringTok{ }\NormalTok{.probs)  \})}
  
  \KeywordTok{return}\NormalTok{(}\KeywordTok{list}\NormalTok{(}\StringTok{"obs_test_stats"}\NormalTok{ =}\StringTok{ }\NormalTok{obs_test_stats,}
              \StringTok{"p_values"}\NormalTok{ =}\StringTok{ }\NormalTok{p_values))}
  
\NormalTok{\}}

\NormalTok{p_values_null_true <-}\StringTok{ }\KeywordTok{get_p_value}\NormalTok{(}\DataTypeTok{.y_c =}\NormalTok{ y_c_null_true,}
                                  \DataTypeTok{.y_t =}\NormalTok{ y_t_null_true,}
                                  \DataTypeTok{.Omega =}\NormalTok{ Omega,}
                                  \DataTypeTok{.probs =}\NormalTok{ cra_vec_probs)}\OperatorTok{$}\NormalTok{p_values}

\CommentTok{## The probability of rejecting the null hypothesis when it is true}
\KeywordTok{sum}\NormalTok{((p_values_null_true }\OperatorTok{<=}\StringTok{ }\FloatTok{0.05}\NormalTok{) }\OperatorTok{*}\StringTok{ }\NormalTok{cra_vec_probs)}
\end{Highlighting}
\end{Shaded}

Let's now calculate the probability of rejecting the sharp null when it
is false and the alternative hypothesis is true (i.e., the power of the
test).

\begin{Shaded}
\begin{Highlighting}[]
\NormalTok{y_c_null_false <-}\StringTok{ }\KeywordTok{rep}\NormalTok{(}\DataTypeTok{x =} \DecValTok{0}\NormalTok{, }\DataTypeTok{times =} \DecValTok{8}\NormalTok{)}
\NormalTok{y_t_null_false <-}\StringTok{ }\KeywordTok{rep}\NormalTok{(}\DataTypeTok{x =} \DecValTok{1}\NormalTok{, }\DataTypeTok{times =} \DecValTok{8}\NormalTok{)}

\NormalTok{get_p_value <-}\StringTok{ }\ControlFlowTok{function}\NormalTok{(.y_c,}
\NormalTok{                        .y_t,}
\NormalTok{                        .Omega,}
\NormalTok{                        .probs) \{}
  
\NormalTok{  obs_outs =}\StringTok{ }\KeywordTok{sapply}\NormalTok{(}\DataTypeTok{X =} \DecValTok{1}\OperatorTok{:}\KeywordTok{ncol}\NormalTok{(.Omega),}
                    \DataTypeTok{FUN =} \ControlFlowTok{function}\NormalTok{(x) \{ Omega[,x] }\OperatorTok{*}\StringTok{ }\NormalTok{.y_t }\OperatorTok{+}\StringTok{ }\NormalTok{(}\DecValTok{1} \OperatorTok{-}\StringTok{ }\NormalTok{Omega[,x]) }\OperatorTok{*}\StringTok{ }\NormalTok{.y_c \})}
  
\NormalTok{  obs_test_stats =}\StringTok{ }\KeywordTok{sapply}\NormalTok{(}\DataTypeTok{X =} \DecValTok{1}\OperatorTok{:}\KeywordTok{ncol}\NormalTok{(.Omega),}
                          \DataTypeTok{FUN =} \ControlFlowTok{function}\NormalTok{(x)\{ }\KeywordTok{as.integer}\NormalTok{(}\KeywordTok{t}\NormalTok{(Omega[,x]) }\OperatorTok\StringTok{ }\NormalTok{obs_outs[,x]) \})}
  
\NormalTok{  null_dists <-}\StringTok{ }\KeywordTok{list}\NormalTok{()}
  
  \ControlFlowTok{for}\NormalTok{(i }\ControlFlowTok{in} \DecValTok{1}\OperatorTok{:}\KeywordTok{ncol}\NormalTok{(.Omega))\{}
    
\NormalTok{    null_dists[[i]] =}\StringTok{ }\KeywordTok{sapply}\NormalTok{(}\DataTypeTok{X =} \DecValTok{1}\OperatorTok{:}\KeywordTok{ncol}\NormalTok{(.Omega),}
                             \DataTypeTok{FUN =} \ControlFlowTok{function}\NormalTok{(x) \{ }\KeywordTok{as.integer}\NormalTok{(}\KeywordTok{t}\NormalTok{(Omega[,x]) }\OperatorTok\StringTok{ }\NormalTok{obs_outs[,i]) \})}
\NormalTok{  \}}
  
\NormalTok{  p_values =}\StringTok{ }\KeywordTok{sapply}\NormalTok{(}\DataTypeTok{X =} \DecValTok{1}\OperatorTok{:}\KeywordTok{ncol}\NormalTok{(.Omega),}
                    \DataTypeTok{FUN =} \ControlFlowTok{function}\NormalTok{(x) \{ }\KeywordTok{sum}\NormalTok{((null_dists[[x]] }\OperatorTok{>=}\StringTok{ }\NormalTok{obs_test_stats[x]) }\OperatorTok{*}\StringTok{ }\NormalTok{.probs)  \})}
  
  \KeywordTok{return}\NormalTok{(}\KeywordTok{list}\NormalTok{(}\StringTok{"obs_test_stats"}\NormalTok{ =}\StringTok{ }\NormalTok{obs_test_stats,}
              \StringTok{"p_values"}\NormalTok{ =}\StringTok{ }\NormalTok{p_values))}
  
\NormalTok{\}}

\NormalTok{p_values_null_false <-}\StringTok{ }\KeywordTok{get_p_value}\NormalTok{(}\DataTypeTok{.y_c =}\NormalTok{ y_c_null_false,}
                                  \DataTypeTok{.y_t =}\NormalTok{ y_t_null_false,}
                                  \DataTypeTok{.Omega =}\NormalTok{ Omega,}
                                  \DataTypeTok{.probs =}\NormalTok{ cra_vec_probs)}\OperatorTok{$}\NormalTok{p_values}

\CommentTok{## The probability of rejecting the null hypothesis when it is true}
\KeywordTok{sum}\NormalTok{((p_values_null_false }\OperatorTok{<=}\StringTok{ }\FloatTok{0.05}\NormalTok{) }\OperatorTok{*}\StringTok{ }\NormalTok{cra_vec_probs)}
\end{Highlighting}
\end{Shaded}

\newpage
\singlespacing 
\bibliography{Bibliography.bib}

\end{document}

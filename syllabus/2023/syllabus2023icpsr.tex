\documentclass[12pt]{article}

%\usepackage{epsf,graphicx}
\usepackage[T1]{fontenc}
\usepackage{mathptmx}
\usepackage[top=1in, bottom=1in, right=1in, left=1in]{geometry}
\usepackage{graphicx}
\usepackage{multirow}
\usepackage{natbib}
\usepackage{bibentry} %% to do bib in syllabus
\newcommand{\bibverse}[1]{\begin{verse} \bibentry{#1}. \end{verse}}
\usepackage{hyperref}
\usepackage{parskip}
\usepackage{bookmark}
\usepackage{rotating}
\usepackage{lscape}

\usepackage{setspace}
\setcounter{secnumdepth}{4}

\title{Causal Inference for the Social Sciences}

\date{ICPSR Session 2 (July 19, 2021)}
 %misnomer, as of now file only carries date

\author{Jake Bowers \thanks{Political Science and Statistics, University of Illinois @ Urbana-Champaign; \mbox{jwbowers@illinois.edu}} \and
    Thomas (Tom) Leavitt \thanks{Marxe School of Public and International Affairs at Baruch College, City University of New York (CUNY); \mbox{thomas.leavitt@baruch.cuny.edu}}
}


\begin{document}

\maketitle
\begin{abstract}
\noindent This course introduces methods and concepts used to infer causal effects from comparisons of intervention and control groups. We'll use the potential outcomes framework of causality to show how a study's research design provides a foundation for estimation and testing. We focus, first, on properties of estimators and tests in randomized experiments, e.g., unbiasedness, consistency, controlled error rates. We then turn to research designs that are either partially controlled (e.g., experiments with noncompliance and/or attrition) or uncontrolled (e.g., observational studies). For observational studies, we focus primarily on matching methods implemented via \texttt{optmatch} and related packages in \texttt{R}. Finally, we turn to sensitivity analysis --- namely, how to assess how inferences would change should certain assumptions about the research design be false. Examples throughout the course are drawn from economics, political science, public health, and sociology.

The course assumes familiarity with linear algebra and strong knowledge of statistical concepts, such as sampling distributions, statistical inference, and hypothesis testing. The course will rely on \texttt{R} for computation.

\end{abstract}

% \section{Meetings}

The course meets from 8:30 AM to 11:45 AM US Eastern Time, Monday through Friday, from June 16
through July 3. Ben will be the primary instructor for the half of the
course. Jake will be the primary instructor for the second half.

%Tom holds office hours 12--3 pm, in Newberry 301. Ben and/or Jake
%have office hours Monday, Wednesday and Friday between 9:30 and 10:30
%a.m., in Newberry 323. Additional office hours may be announced.

%Readings and other course materials are distributed via Box:
%\begin{quote}
%\url{https://umich.box.com/v/ICPSRcausal-Bowers-etal-2019} .
%\end{quote}
%(You won't have to authenticate to use this link, but it expires after
%August 2019.)
% (There is also a
% course Canvas site,  \href{https://umich.instructure.com/courses/187094}{``POLSCI 832 228
% SU 2017''}, to be opened during the first week of classes.)


\clearpage

\nobibliography*


\section*{Overview}

We may all warn our freshmen that association is not causation, but inferring causation has always been a central aim both for statisticians and for their collaborators. Until recently, however, inference of causation from statistical evidence depended on murky, scarcely attainable requirements; in practice, the weight of casual arguments was largely determined by the scientific authority of the people making them.

Requirements for causal inference become more clear when they are framed in terms of \emph{potential outcomes}.  This was first done by Neyman, who in the 1920s used potential outcomes to model agricultural experiments.  Fisher independently proposed a related but distinct, ultimately more influential, analysis of experiments in 1935, and a rich strain of causal analysis developed among his intellectual progeny. It clarified the differing requirements for causal inference with experiments and with observational data, isolating the distinct
contributions required of the statistician and of his disciplinary collaborators; generated more satisfying methods with which to address potential confounding due to measured variables; qualitatively and quantitatively advanced our grasp of unmeasured confounding and its potential ramifications; furnished statistical methods with which to eke more out of the strongest study designs, under fewer assumptions; and articulated principles with which to understand study designs as a spectrum, rather than a dichotomy between ``good" experiments and ``bad" observational studies. Understanding the methods and outlook of the school founded by Fisher's student W. G. Cochran will be the central task of this course.

The course begins by applying the Fisher and Neyman-Rubin approaches to statistical inference for counterfactual causal efffects to randomized experiments, touching on considerations
specific to clustered treatment assignment, ``small'' sample sizes and treatment effect heterogeneity. The next segment addresses conceptual and methodogical challenges of applying the same models to analysis of non-experimental data. This course segment covers ignorability, selection, ``common support,'' covariate balance, paired comparisons,
optimal matching and propensity scores. With these foundations in place, the course then turns to sensitivity analysis --- namely, how to make principled assessments of how inferences would change should crucial assumptions be false. Over the course's three weeks, the course becomes
progressively less conceptual and more applied with increasing
emphasis on computing strategies in R.

\section*{Administrative}

\subsection*{Textbooks}

The main texts for the course are

\begin{verse} \bibentry{rosenbaum2017} \end{verse}

\begin{verse} \bibentry{rosenbaum2010} \end{verse}

\begin{verse} \bibentry{rosenbaum2002a} \end{verse}

These three textbooks are presented in varying difficulty and we will draw from all three. Although we won't follow these books closely, their goals and methods align with the course's, and they will be useful as references and supplements.

Other texts that we draw on include

\begin{verse} \bibentry{gerbergreen2012}  \end{verse}

\begin{verse} \bibentry{imbensrubin2015} \end{verse}

% If you'll be using experiments, propensity scores, regression
% discontinuity, or instrumental variables in your work, then each of
% these is a worthwhile investment.

%Several other graduate-level monographs focus or touch
%on causal inference.  Texts that we've found to contain helpful
%discussion include:

%\begin{verse}

%\bibentry{angristpischke2008}

%\bibentry{aronowmiller2019}

%\bibentry{berk2004}

%\bibentry{gelmanhill2006}

%\bibentry{morganwinship2015}

%\bibentry{murnanewillett2010}

%\bibentry{dunning2012}

%\end{verse}

Other readings will be assigned and distributed electronically.

If you're new to \texttt{R}, we suggest getting a hold of:

\begin{verse} \bibentry{fox2016}  \end{verse}

\begin{verse} \bibentry{wickhamgrolemund2017} \end{verse}

\texttt{R} software will be required for several specific segments of the course. With some independent effort, students not familiar with \texttt{R} in advance should be able to learn enough \texttt{R} during the course to complete these assignments. We also recommend some work with \texttt{R} --- for example, via working through some online \texttt{R} courses --- before the course for students who have never used it before.


\subsection*{Assignments}
%Computer or pencil and paper exercises will be assigned periodically and collected on Mondays and on Thursdays at the beginning of class.
Assignments are due each Friday, at the beginning of
class. Parts of the assignment will be given at the beginning of the week, but other parts will be given during class, over the course of the week.  Late homework will not be accepted without cause (or prior arrangement with the teaching assistant).

Participation is expected. It can take various forms:
\begin{enumerate}
\item Doing in-class exercises and discussing them with your peers;
\item From time to time, making a clarification or raising a clarifying questions;
\item \label{it:part0} Contributing to in-class discussions.
%\item \label{it:part0} Using a github pull request to suggest a clarification or other enhancement to a course slide or worksheet.
\item \label{it:part1} Drop by one of the professor's office hours to share a point that you \textit{and at least one classmate} would like to have clarified or amplified, or to point out a connection to your field;
%\item \label{it:part2} Give a 5-10 minute in-class presentation of a paper in your field that uses methods or designs we're discussing in the course.
\end{enumerate}
If you are taking the course for a grade, make a point of doing at
least one of \ref{it:part0} and \ref{it:part1}. %and \ref{it:part2}.  There'll be an electronic sign-up for \ref{it:part2}.

% Course participants will be expected either to give a presentation or to submit a short paper about assigned readings related to their fields of study.  Presentations will be given alone or in pairs, and last 10-20 minutes; short papers should be 5-8 pages long.  Students must decide no later than the end of the second week whether they’ll present or submit a paper, and on what topic; presentations will be given at times designated by the instructor, whereas papers must be submitted by the beginning of the fourth week.

\newpage
\begin{landscape}
\section*{Course Schedule}\label{sec:schedule}

The course schedule is below. In the Course content section, we provide more extensive readings on each topic in the schedule, as well as additional ``special topics.'' The last two days of the course are reserved for special topics chosen by students.

\begin{table}[!htbp]
\centering
%{\large\renewcommand{\arraystretch}{.8}
\resizebox{9in}{!}{\begin{tabular}{|l|l|l|l|l|}
\hline
\textbf{Date} & \textbf{Instructor} & \textbf{Topic} & \textbf{Required readings} & \textbf{Application} \\ \hline
Tues, June 20 & Tom & Introduction: Randomized experiments and potential outcomes & \begin{tabular}{@{}c@{}} Holland (1986) \\Kinder and Palfrey (1993, Section 1.2) \\ Fisher (1935, Introduction) \\ Rosenbaum (2017, Chapter 2) \end{tabular} & N/A \\ \cline{1-5}
June 21 & Tom & Randomized experiments: Fisherian exact tests & \begin{tabular}{@{}c@{}} Rosenbaum (2017, Chapter 3) \end{tabular} & ~ \\ \cline{1-4}
June 22 & \multirow{2}{*}{Tom} & \multirow{2}{*}{Randomized experiments: Neymanian estimation and inference} & \begin{tabular}{@{}c@{}} Gerber and Green (2012, Chapter 2) \\ Aronow and Middleton (2013) \\ Middleton and Aronow (2015) \\ Gerber and Green (2012, pp. 51 -- 61) \end{tabular} & \multirow{5}{*}{Arceneaux (2005)} \\ \cline{1-4}
June 23 (HW 1 due) & Tom & Randomized experiments: Covariance adjustment \& regression & \begin{tabular}{@{}c@{}} Gerber and Green (2012, Chapter 4) \\ Rosenbaum (2002) \\ Lin (2013) \end{tabular} & ~ \\ \hline
June 26 & Jake & \multirow{2}{*}{Randomized experiments: Noncompliance and attrition} & \begin{tabular}{@{}c@{}} Gerber and Green (2012, Chapters 5 -- 6) \\ Rosenbaum (1996) \\ Rosenbaum (2010, Section 5.3)\end{tabular} & \multirow{2}{*}{Albertson and Lawrence (2009)} \\ \cline{1-5}
June 27 & Jake & Observational studies: Introduction & \begin{tabular}{@{}c@{}} Bind and Rubin (2019) \\ Gelman and Hill (2006, Sections 9.0 -- 9.2) \\ Berk (2010) \end{tabular} &  \\ \cline{1-4}
June 28 & Jake & Introduction to matching & \begin{tabular}{@{}c@{}} Rosenbaum (2017, pp. 65 -- 90) \end{tabular} & ~ \\ \cline{1-4}
June 29  & Jake & Nuts \& bolts of matching & \begin{tabular}{@{}c@{}} Rosenbaum (2010, Chapters 7 -- 8) \\ Rosenbaum (2017, Chapter 11) \end{tabular} & \multirow{6}{*}{Cerd\'{a} et al (2012)} \\ \cline{1-4}
June 30 (HW 2 due) & Jake & Propensity score methods & \begin{tabular}{@{}c@{}} Rosenbaum (2017, pp. 90 -- 96) \\ Hansen (2011) \end{tabular} & ~ \\ \cline{1-4}
July 3 & Jake & Assessments of covariate balance & \multirow{2}{*}{\begin{tabular}{@{}c@{}} Hansen and Bowers (2008) \\ Gerber and Green (2021, pp. 71 -- 79) \end{tabular}} & ~ \\ \cline{1-4}
July 4 & Jake & Outcome analysis after matching & & ~ \\ \cline{1-4}
July 5 & Jake & Fisherian sensitivity analysis & \begin{tabular}{@{}c@{}}
Rosenbaum (2017, Chapter 9) \\ Gastwirth et al (2000) \\ Rosenbaum (2015) \\ Rosenbaum (2018) \end{tabular} & ~ \\ \cline{1-4}
July 6 & Jake & Neymanian sensitivity analysis & \begin{tabular}{@{}c@{}} Fogarty (2023) \end{tabular} & ~ \\ \cline{1-4}
July 7 (HW 3 due) & Jake & Design sensitivity & \begin{tabular}{@{}c@{}} Rosenbaum (2004) \\ Rosenbaum (2017, Chapter 10) \end{tabular} & ~ \\ \hline
\end{tabular}}
%}
\end{table}

\end{landscape}
\newpage
\section*{Course content}

\section{Potential outcomes and random assignment}

%Medical- and social-science data generating processes can be difficult to capture accurately in a single regression equation, for various reasons. The statistical foundations of randomized experiments are much more satisfying, particularly when they are taken on their own terms. Fisher and Neyman did this earlier in the 1920s and 30s, in work that Rubin, Holland and others reinvigorated beginning in the 1970s. The course begins by surveying the circle of ideas to emerge from this.

\paragraph*{Required}

\begin{verse}\bibentry{holland1986}, Sections~1 -- 4. (The article that brought the ``Rubin Causal Model'' to  statisticians' attention.)\end{verse}

\begin{verse} Section~1.2, ``Experimentation defined,''  of
\bibentry{kinderpalfrey1993}.  (Particularly pp. 5 -- 10.) \end{verse}

\begin{verse} \bibentry{rosenbaum2017}, Chapter 2. \end{verse}

\begin{verse} Chapter~1, ``Introduction,'' of \bibentry {fisher1935a} and pages 131 -- 135 of \bibentry{box1978} for historical context. \end{verse}

\paragraph*{Recommended}

\begin{verse} \bibentry{bowersleavitt2020} \end{verse}

\section{Random assignment as a basis for inference}

\subsection*{Application}

\begin{verse} \bibentry{arceneaux2005} \end{verse}

\subsection{Inference for causal effects: the Neyman tradition}

\subsubsection{Estimation of average causal effects}

\paragraph*{Required}

\begin{verse} Chapter~2 and pp. 51 -- 61 of \bibentry{gerbergreen2012} \end{verse}
\begin{verse} \bibentry{aronowmiddleton2013} \end{verse}
\begin{verse} \bibentry{middletonaronow2015} \end{verse}

\subsubsection{Variance estimation and hypothesis testing}

\paragraph*{Required}

\begin{verse}
  Chapter~3 of \bibentry{gerbergreen2012}.
\end{verse}

\begin{verse} Endnote spanning pages A-32 and 33,
  \bibentry{freedmanpisanipurves1998}.  (This can be read as a pr{\'e}cis
  of: \bibentry{neyman1990}.)
\end{verse}

\begin{verse} Chapter 6, pp. 87 -- 98 of \bibentry{imbensrubin2015} \end{verse}

\paragraph*{Recommended}

\begin{verse} Chapter~6 of \bibentry{dunning2012} \end{verse}

\begin{verse} \bibentry{liding2017} \end{verse}

\begin{verse} \bibentry{ding2017a} \end{verse}

\begin{verse} \bibentry{aronowetal2014} \end{verse}

\subsection{Inference for causal effects: the Fisherian tradition}

\paragraph*{Required}

\begin{verse}
  Chapter 3 of \bibentry{rosenbaum2017}.
\end{verse}

\paragraph*{Recommended}

\begin{verse}
  Pages 27 -- 49 of \bibentry{rosenbaum2002a}
\end{verse}

\begin{verse}
  Chapter 2 of \bibentry{rosenbaum2010}.
\end{verse}

\subsection{Covariance adjustment in randomized experiments}

\paragraph*{Required}

\begin{verse}
  Chapter~4 of \bibentry{gerbergreen2012}.
\end{verse}

\begin{verse}
  \bibentry{rosenbaum2002c}.
\end{verse}

\begin{verse} \bibentry{lin2013} \end{verse}

\paragraph*{Recommended}

\begin{verse} \bibentry{miratrixetal2013} \end{verse}

\begin{verse} \bibentry{freedman2008a} \end{verse}

\begin{verse} \bibentry{freedman2008b} \end{verse}

\begin{verse} \bibentry{freedman2008c} \end{verse}

\begin{verse} \bibentry{samiiaronow2012} \end{verse}

\begin{verse} \bibentry{aronowsamii2016} \end{verse}

\begin{verse} \bibentry{abadieetal2020} \end{verse}


\section{Noncompliance and Attrition}

\subsection*{Applications}

\begin{verse} \bibentry{albertsonlawrence2009} \end{verse}

\subsection{Noncompliance and instrumental variables}

\paragraph*{Required}

\begin{verse} Chapters 5 and 6 of \bibentry{gerbergreen2012}. \end{verse}

\begin{verse}
  Section 5.3, ``Instruments,'' of \bibentry{rosenbaum2010} \end{verse}

\begin{verse}
\bibentry{rosenbaum1996} \end{verse}

\paragraph*{Recommended}

\begin{verse} Section 2.3 of \bibentry{rosenbaum2002c} \end{verse}

\begin{verse} \bibentry{angristetal1996} \end{verse}

\begin{verse} \bibentry{imbensrosenbaum2005} \end{verse}

\begin{verse} \bibentry{kangpeckkeele2018} \end{verse}

\begin{verse} \bibentry{hansenbowers2008} \end{verse}

\subsection{Attrition, or missing outcomes}

\paragraph*{Recommended}

\begin{verse} Chapter 7 of \bibentry{gerbergreen2012} \end{verse}

\begin{verse} \bibentry{lee2009} \end{verse}

\begin{verse} \bibentry{aronowetal2019} \end{verse}

\begin{verse} \bibentry{horowitzmanski2000} \end{verse}

\begin{verse} \bibentry{coppocketal2017} \end{verse}

\section{Observational Studies}

\subsection{Introduction:  ``Controlling for'' in observational studies}

\paragraph*{Required}

\begin{verse} \bibentry{bindrubin2019} \end{verse}

\begin{verse} Sections 9.0 -- 9.2 (especially discussion of interpolation and extrapolation) of \bibentry{gelmanhill2006} \end{verse}

\bibverse{berk2010}

\paragraph*{Recommended}

\begin{verse} Chapter 5 of \bibentry{rosenbaum2017} \end{verse}

\begin{verse} \bibentry{cochran1965} \end{verse}

\begin{verse} On the problem of kitchen sink regressions, \bibentry{achen2002} \end{verse}

\begin{verse} Chapters 11 and 19 (on overly influential points) of \bibentry{fox2016} \end{verse}

\subsection{Matching: An introduction}

\subsection*{Application}

\begin{verse} \bibentry{cerdaetal2012} \end{verse}

\paragraph*{Required}

\begin{verse}
  Pages 65 -- 90 of \bibentry{rosenbaum2017}
\end{verse}

\begin{verse}
  \bibentry{rosenbaum2020}
\end{verse}

\paragraph*{Recommended}

\begin{verse}
  Chapter 3 of \bibentry{rosenbaum2002a}, specifically Sections 3.1 -- 3.2 and 3.4 -- 3.5.
\end{verse}

\begin{verse} \bibentry{rosenbaum2001d} \end{verse}

\bibverse{bifulco2012}

\bibverse{arceneauxetal2010}

\subsection{Nuts \& bolts of matching}

\paragraph*{Required}

\begin{verse}
  Chapters 7 -- 8 of \bibentry{rosenbaum2010}, 
\end{verse}

\begin{verse}
  Chapter 11 \bibentry{rosenbaum2017}, 
\end{verse}

\subsection{Propensity scores methods}

\paragraph*{Required}

\begin{verse} Pages 90 -- 96 of \bibentry{rosenbaum2017} \end{verse}

\begin{verse}
  \bibentry{hansen2011}
\end{verse}

\paragraph*{Recommended}

\begin{verse}
  \bibentry{rubin1979}
\end{verse}

\begin{verse}
  \bibentry{robinsetal2000a}
\end{verse}

\bibverse{hoetal2007}

\begin{verse}
  Chapters 13 of \bibentry{rosenbaum2010}
\end{verse}

\begin{verse}
  \bibentry{rosenbaumrubin1985}
\end{verse}

\subsection{Covariate balance and outcome analysis after matching}

\paragraph*{Required}

\begin{verse}
  \bibentry{hansenbowers2008}
\end{verse}

\begin{verse}
 Pages 71 -- 79 of \bibentry{gerbergreen2012}
\end{verse}

\paragraph*{Recommended}

\begin{verse}
  \bibentry{fogarty2018a}
\end{verse}

\begin{verse}
  \bibentry{pashleymiratrix2020}
\end{verse}

\begin{verse}
  \bibentry{imai2008a}
\end{verse}

\begin{verse}
  \bibentry{imaietal2008}
\end{verse}

\section{Sensitivity analysis}

\subsection*{Application}

\begin{verse} \bibentry{cerdaetal2012} \end{verse}

\subsection{Sensitivity analysis for sharp nulls}

\paragraph*{Required}

\begin{verse}
  Chapter 11 \bibentry{rosenbaum2017}, 
\end{verse}

\paragraph*{Recommended}

\begin{verse} Chapter 4 of \bibentry{rosenbaum2002a} \end{verse}

\begin{verse} \bibentry{rosenbaum2018} \end{verse}

\begin{verse} \bibentry{rosenbaumkrieger1990} \end{verse}

\begin{verse} \bibentry{hansenrosenbaumsmall2014} \end{verse}

\begin{verse} \bibentry{hsusmall2013} \end{verse}

\subsection{Sensitivity analysis for weak nulls}

\paragraph*{Required}

\begin{verse} \bibentry{fogarty2023} \end{verse}

\paragraph*{Recommended}

\begin{verse} \bibentry{fogarty2020b} \end{verse}

\begin{verse} \bibentry{fogartyetal2017} \end{verse}

\section{Design sensitivity}

\paragraph*{Required}

\begin{verse} \bibentry{rosenbaum2004} \end{verse}

\begin{verse} Chapter 10 of \bibentry{rosenbaum2017} \end{verse}

\section{Additional topics}

\subsection{Nonbipartite matching}

\begin{verse} Chapter 11 of \bibentry{rosenbaum2010} \end{verse}

\begin{verse} \bibentry{luetal2001} \end{verse}

\begin{verse} Chapter 11 of \bibentry{rosenbaum2010} \end{verse}

\begin{verse} \bibentry{luetal2011} \end{verse}

\begin{verse} \bibentry{lu2005} \end{verse}

\begin{verse} \bibentry{baiocchietal2010} \end{verse}

\begin{verse} \bibentry{zubizarretaetal2013} \end{verse}

\subsection{Regression discontinuity designs and ``natural experiments''}

\begin{verse} \bibentry{caugheysekhon2011} \end{verse}

\begin{verse} \bibentry{cattaneoetal2020} \end{verse}

\begin{verse} \bibentry{saleshansen2020} \end{verse}

\begin{verse} \bibentry{keeletitiunikzubizarreta2015} \end{verse}

\begin{verse} \bibentry{sekhontitiunik2017} \end{verse}

\begin{verse} \bibentry{sekhontitiunik2016} \end{verse}

\begin{verse} \bibentry{hansensales2015} \end{verse}

\begin{verse} \bibentry{hahnetal2001} \end{verse}

\begin{verse} \bibentry{imbenslemieux2008} \end{verse}

\begin{verse} \bibentry{lee2008} \end{verse}

\begin{verse} \bibentry{gelmanimbens2019} \end{verse}

\begin{verse} \bibentry{mccrary2008} \end{verse}

\begin{verse} Chapter 6 of \bibentry{angristpischke2008} \end{verse}

\subsection{Interference}

\bibverse{rosenbaum2007a}

\bibverse{bowersetal2013}

\bibverse{aronowsamii2017}

\bibverse{manski2013b}

\bibverse{atheyetal2018}

\subsection{Factorial and complex experiments}

\begin{verse} \bibentry{dasguptaetal2015} \end{verse}

\begin{verse} \bibentry{hainmuelleretal2014} \end{verse}

\begin{verse} \bibentry{egamiimai2019} \end{verse}

\begin{verse} \bibentry{gerberetal2010} \end{verse}

\subsection{Difference-in-Differences}

\begin{verse} Pages 162 -- 167 of \bibentry{rosenbaum2017} \end{verse}

\begin{verse} Section 4.1 of \bibentry{gerbergreen2012} \end{verse}

\begin{verse} Chapter 5 of \bibentry{angristpischke2008} \end{verse}

\begin{verse} \bibentry{lechner2011} \end{verse}

\begin{verse} \bibentry{manskipepper2018} \end{verse}

\begin{verse} \bibentry{dingli2019} \end{verse}

\begin{verse} \bibentry{imaikim2021} \end{verse}

\subsection{Special topics in matching: Multilevel, risk set, cardinality, generalized full matching and fine balance}

\begin{verse} \bibentry{zubizarretakeele2017} \end{verse}

\begin{verse} \bibentry{yangetal2016} \end{verse}

\begin{verse} \bibentry{pimenteletal2018} \end{verse}

\begin{verse} Chapter 12 of \bibentry{rosenbaum2010} \end{verse}

\begin{verse} \bibentry{lietal2001} \end{verse}

\begin{verse} Chapter 10 of \bibentry{rosenbaum2010} \end{verse}

\begin{verse} \bibentry{kilciogluzubizarreta2016} \end{verse}

\begin{verse} \bibentry{savjehigginssekhon2021} \end{verse}

\subsection{Attributable effects and inference of random causal quantities}

\begin{verse} \bibentry{rosenbaum2001a} \end{verse}

\begin{verse} \bibentry{rosenbaum2003} \end{verse}

\begin{verse} \bibentry{hansenbowers2009} \end{verse}

\begin{verse} \bibentry{sekhonshem-tov2021} \end{verse}

\begin{verse} \bibentry{keeleetal2017} \end{verse}

\begin{verse} \bibentry{rosenbaum2007} \end{verse}

\subsection{Regression-based sensitivity analysis}

\bibverse{hosmanetal2010}

\bibverse{cinellihazlett2020}

\bibverse{imbens2003}

\bibverse{oster2019}

\subsection{External validity}

\begin{verse} \bibentry{kernetal2016} \end{verse}

\begin{verse} \bibentry{miratrixetal2018a} \end{verse}

\begin{verse} \bibentry{bennettetal2020} \end{verse}

\begin{verse} \bibentry{silberetal2014} \end{verse}

\begin{verse} \bibentry{westreichetal2019} \end{verse}

\subsection{Weighting methods}

\begin{verse} \bibentry{chattopadhyayetal2020} \end{verse}

\begin{verse} \bibentry{imairatkovic2014} \end{verse}

\begin{verse} \bibentry{zubizarreta2015} \end{verse}

\begin{verse} \bibentry{wongchan2018} \end{verse}

\begin{verse} \bibentry{tan2020} \end{verse}

\begin{verse} \bibentry{wangzubizarreta2020} \end{verse}

\begin{verse} \bibentry{hainmueller2012} \end{verse}

\subsection{Synthetic control}

\begin{verse} \bibentry{abadieetal2010} \end{verse}

\begin{verse} \bibentry{abadieetal2012} \end{verse}

\begin{verse} \bibentry{ben-michaeletal2021} \end{verse}

\subsection{Bayesian causal inference}

\begin{verse} \bibentry{rubin1978a} \end{verse}

\begin{verse} Chapter 8 of \bibentry{imbensrubin2015} \end{verse}

\begin{verse} \bibentry{imbensrubin1997} \end{verse}

\begin{verse} \bibentry{keelequinn2017} \end{verse}

\begin{verse} \bibentry{dingmiratrix2019} \end{verse}

\clearpage


%\section{Additional Notes}
\clearpage
\bibliographystyle{plain}
\nobibliography{bib_2023}
%\nobibliography{master,abbrev_long,causalinference,biomedicalapplications,misc}
%\nobibliography{../2013/BIB/master,../2013/BIB/abbrev_long,../2013/BIB/causalinference,../2013/BIB/biomedicalapplications,../2013/BIB/misc}
%"master" is Ryan Moore's file (as of summer 2012).
%"causalinference" is Ben's, and lives (as of summer 2013) in
% http://dept.stat.lsa.umich.edu/~bbh/texmf/bibtex/bib/


\end{document}

%%%%%%%%%%%%%%%%%%%%%%%%%%%%%%%%%%%%%%%%%
% Beamer Presentation
% LaTeX Template
% Version 1.0 (10/11/12)
%
% This template has been downloaded from:
% http://www.LaTeXTemplates.com
%
% License:
% CC BY-NC-SA 3.0 (http://creativecommons.org/licenses/by-nc-sa/3.0/)
%
%%%%%%%%%%%%%%%%%%%%%%%%%%%%%%%%%%%%%%%%%

%----------------------------------------------------------------------------------------
% PACKAGES AND THEMES
%----------------------------------------------------------------------------------------

\documentclass[table, xcolor={dvipsnames}, 9pt]{beamer}
\usepackage{tikz}
\usetikzlibrary{positioning}
\mode<presentation> {

% The Beamer class comes with a number of default slide themes
% which change the colors and layouts of slides. Below this is a list
% of all the themes, uncomment each in turn to see what they look like.

%\usetheme{default}
%\usetheme{AnnArbor}
%\usetheme{Antibes}
%\usetheme{Bergen}
%\usetheme{Berkeley}
%\usetheme{Berlin}
%\usetheme{Boadilla}
%\usetheme{CambridgeUS}
%\usetheme{Copenhagen}
%\usetheme{Darmstadt}
%\usetheme{Dresden}
%\usetheme{Frankfurt}
%\usetheme{Goettingen}
%\usetheme{Hannover}
%\usetheme{Ilmenau}
%\usetheme{JuanLesPins}
%\usetheme{Luebeck}
% \usetheme{Madrid}
\usetheme{metropolis}
%\usetheme{Malmoe}
%\usetheme{Marburg}
%\usetheme{Montpellier}
%\usetheme{PaloAlto}
%\usetheme{Pittsburgh}
%\usetheme{Rochester}
%\usetheme{Singapore}
%\usetheme{Szeged}
%\usetheme{Warsaw}

% As well as themes, the Beamer class has a number of color themes
% for any slide theme. Uncomment each of these in turn to see how it
% changes the colors of your current slide theme.

%\usecolortheme{albatross}
%\usecolortheme{beaver}
%\usecolortheme{beetle}
%\usecolortheme{crane}
%\usecolortheme{dolphin}
%\usecolortheme{dove}
%\usecolortheme{fly}
%\usecolortheme{lily}
%\usecolortheme{orchid}
%\usecolortheme{rose}
%\usecolortheme{seagull}
%\usecolortheme{seahorse}
%\usecolortheme{whale}
%\usecolortheme{wolverine}

%\setbeamertemplate{footline} % To remove the footer line in all slides uncomment this line
%\setbeamertemplate{footline}[page number] % To replace the footer line in all slides with a simple slide count uncomment this line

%\setbeamertemplate{navigation symbols}{} % To remove the navigation symbols from the bottom of all slides uncomment this line
}
\setbeamertemplate{footline}{}
\addtobeamertemplate{footnote}{}{\vspace{24pt}}
\usepackage{graphicx} % Allows including images
\usepackage{booktabs} % Allows the use of \toprule, \midrule and \bottomrule in tables
\usepackage{multirow}
\usepackage{natbib}
\usepackage[]{hyperref}
\usepackage{diagbox}
\usepackage{makecell}
\usepackage{subfig}
\usepackage{amsmath}
\usepackage{amsfonts,amsthm,amsmath,amssymb}    
\usepackage{bbm}
\usepackage{bm}
\usepackage{empheq}
\makeatletter
\let\save@measuring@true\measuring@true
\def\measuring@true{%
  \save@measuring@true
  \def\beamer@sortzero##1{\beamer@ifnextcharospec{\beamer@sortzeroread{##1}}{}}%
  \def\beamer@sortzeroread##1<##2>{}%
  \def\beamer@finalnospec{}%
}
\makeatother
\hypersetup{unicode=true,
            pdfusetitle,
            bookmarks=true,
            bookmarksnumbered=true,
            bookmarksopen=true,
            bookmarksopenlevel=2,
            breaklinks=false,
            pdfborder={0 0 1},
            backref=true,
            hypertexnames=false,
            pdfstartview={XYZ null null 1}}
\usepackage{xcolor}
\newcommand\myheading[1]{%
  \par\bigskip
  {\Large\bfseries#1}\par\smallskip}
\newcommand\given[1][]{\:#1\vert\:}
\theoremstyle{newstyle}
\newtheorem{thm}{Theorem}
\newtheorem{prop}[thm]{Proposition}
\newtheorem{lem}{Lemma}
\newtheorem{cor}{Corollary}
\newtheorem{defin}{Definition}
\newcommand*\diff{\mathop{}\!\mathrm{d}}
\newcommand*\Diff[1]{\mathop{}\!\mathrm{d^#1}}
\newcommand*{\QEDA}{\hfill\ensuremath{\blacksquare}}%
\newcommand*{\QEDB}{\hfill\ensuremath{\square}}%
\DeclareMathOperator{\E}{\mathrm{E}}
\DeclareMathOperator{\R}{\mathbb{R}}
\DeclareMathOperator{\Var}{\rm{Var}}
\DeclareMathOperator{\Cov}{\rm{Cov}}
\DeclareMathOperator{\e}{\rm{e}}
\DeclareMathOperator{\logit}{\rm{logit}}
\DeclareMathOperator{\indep}{{\perp\!\!\!\perp}}
%\DeclareMathOperator{\Pr}{\rm{Pr}}
\newenvironment{Column}[1][.5\linewidth]{\begin{column}{#1}}{\end{column}}
%----------------------------------------------------------------------------------------
% TITLE PAGE
%----------------------------------------------------------------------------------------

\title[]{Sensitivity analysis} % The short title appears at the bottom of every slide, the full title is only on the title page

\author{Thomas Leavitt} % Your name
\institute[] % Your institution as it will appear on the bottom of every slide, may be shorthand to save space
{
% Your institution for the title page
\medskip
\textit{} % Your email address
}
\date{August 9, 2022} % Date, can be changed to a custom date

\begin{document}

\begin{frame}
\titlepage % Print the title page as the first slide
\end{frame}

%\begin{frame}
%\frametitle{Overview} % Table of contents slide, comment this block out to remove it
%\tableofcontents % Throughout your presentation, if you choose to use \section{} and \subsection{} commands, these will automatically be printed on this slide as an overview of your presentation
%\end{frame}

%------------------------------------------------------------------------
% PRESENTATION SLIDES
%------------------------------------------------------------------------
\section{Motivation: Sensitivity analysis}
\begin{frame}{Motivation: Sensitivity analysis}
\vfill
\begin{itemize} \vfill
\item In week 1, we conducted Fisherian inference in randomized experiments \vfill
\item[] We focused specifically on tests of sharp null of no effects \vfill
\item In weeks 2 -- 3, we focused on observational designs \vfill
\item[] Mainly how to design them to approximate ideal randomized experiment \vfill
\item[] If observational study is as-if randomized $\rightarrow$ use inference from week 1 \vfill
\item We now focus on how to assess violations to as-if randomized assumption \vfill
\begin{itemize} \vfill
\item \textbf{Today and tomorrow}: Fisherian inference about sharp causal effects \vfill
\item \textbf{Thursday}: Neymanian inference about weak causal effects \vfill
\end{itemize} \vfill
\end{itemize} \vfill
\end{frame}
%----------------------------------------------------------------
\section{Na\"{i}ve matching model}
\begin{frame}{Na\"{i}ve matching model}
\vfill
\begin{itemize} \vfill
\item In observational study, units individually assigned to treatment or control by $n$ \textit{independent}, but not necessarily \textit{identically distributed}, coin tosses: \vfill
\begin{itemize} \vfill
\item $Z_i \sim \pi_i^{z_i} \left(1 - \pi_i\right)^{1 - z_i}, i = 1, \dots n$, where $\pi_i \in (0, 1)$ for all $i$ \vfill
\end{itemize} \vfill
\item  Consider the following model for the distribution of treatment assignments:  \vfill
\begin{align*} 
\Pr\left(\mathbf{Z} = \mathbf{z}\right) & = \prod \limits_{i = 1}^n \lambda\left(\mathbf{x}_i\right)^{z_i} \left(1 - \lambda\left(\mathbf{x}_i\right)\right)^{1 - z_i}.
\end{align*} \vfill
\item  If units $i$ and $j \neq i$ have $\bm{x}_i = \bm{x}_j$, then $\lambda\left(\mathbf{x}_i \right) = \lambda\left(\mathbf{x}_j \right)$ \vfill
\item  So we analyze matched design \textit{as-if} it is block randomized experiment \vfill
\item But what if $i$ and $j$ are imbalanced on unobserved confounder $\bm{u}$? \vfill
\end{itemize} \vfill
\end{frame}
%----------------------------------------------------------------
\section{A model for sensitivity to hidden bias}
\begin{frame}{Sensitivity analysis model}
\vfill
\begin{itemize}
\item WLOG, suppose perfect matching on observed covariates
\item Also WLOG, suppose single unobserved covariate $\bm{u} \in [0, 1]^n$ \\ \citep[][p. 300, footnote 33]{rosenbaum2017a}  \vfill
\item Write model for the distribution of treatment assignments as  \vfill
\begin{align} 
\Pr\left(\mathbf{Z} = \mathbf{z}\right) & = \prod \limits_{i = 1}^n \lambda\left(u_i\right)^{z_i} \left(1 - \lambda\left(u_i\right)\right)^{1 - z_i}, \\
\pi_i = \lambda\left(u_i\right) & = \cfrac{\exp\left(\gamma u_i\right)}{1 + \exp\left(\gamma u_i\right)}, \, \gamma \geq 0 \label{eq: sens probs}
\end{align} \vfill
\item  Equation \eqref{eq: sens probs} implies that \vfill
\begin{equation}
\Pr\left(\bm{Z} = \bm{z} \given n_T\right) = \dfrac{\exp\left(\gamma \bm{z}^{\top} \bm{u}\right)}{\sum \limits_{\bm{z} \in \Omega} (\gamma \bm{z}^{\top} \bm{u})}
\end{equation} \vfill
\end{itemize}
\end{frame}
%----------------------------------------------------------------
\begin{frame}{Sensitivity analysis model}
\vfill
\begin{itemize} \vfill
\item The model in Equation \eqref{eq: sens probs} is equivalent to \vfill
\begin{align*}
\dfrac{1}{\Gamma} \leq \dfrac{\pi_i / (1 - \pi_i)}{\pi_j / (1 - \pi_j)} \leq \Gamma \text{ for any } i \text{ and } j,
\end{align*} \vfill
where $\Gamma = \exp(\gamma)$ \citep[see][]{rosenbaum1995a} \vfill
\item \textbf{Intuition}: $\Gamma$ is maximum treatment odds ratio between any two units \vfill
\end{itemize}
\end{frame}
%----------------------------------------------------------------

\begin{frame}{Sensitivity analysis model: example}
\vfill
\begin{itemize} \vfill
\item Consider matched set with $6$ units and exact balance on $\bm{x}$: \vfill
\begin{table}[H]
\centering{}
    \begin{tabular}{l|l|l}
   $\bm{z}$ & $\bm{x}$ & $\bm{y}$ \\ \hline
   $1$ & $1$ & $22$ \\
  $0$ & $1$ & $8$ \\
  $0$ & $1$ & $11$ \\
  $1$ & $1$ & $15$ \\
  $1$ & $1$ & $18$ \\
  $0$ & $1$ & $1$ \\
    \end{tabular}
\caption{Matched set with exact within-set balance}
\label{tab: matched set}
\end{table} \vfill
\item Condition on observed number of treated units, $n_T = 3$, and assume we have mini-randomized experiment in this matched set \vfill
\end{itemize} \vfill
\end{frame}
%----------------------------------------------------------------
\begin{frame}{Sensitivity analysis model: example}
\vfill
\begin{itemize} \vfill
\item Assuming that $\Pr\left(\bm{Z} = \bm{z}\right)$ is constant for all $\bm{z} \in \Omega$ with $n_T = 3$, the distribution under sharp null is \vfill
\begin{figure}
\includegraphics[width = 0.9\linewidth]{null_dist_z_8_plot.pdf}
\end{figure} \vfill
\item This is what we have assumed thus far in conducting inference
\end{itemize}\vfill
\end{frame}
%----------------------------------------------------------------
\begin{frame}{Sensitivity analysis model: example}
\vfill
\begin{itemize}
\item How would our $p$-value change if there were unobserved $\bm{u}$? \vfill 
\item Depends on $\gamma \geq 0$ and the true, but unknown values of $\bm{u}$. \vfill
\item E.g., for any value of $\gamma \geq 0$, we would not worry about $\bm{u}$ if it were \vfill
\begin{align*}
\bm{u} & = \begin{bmatrix} 1 & 1 & 1 & 1 & 1 & 1 \end{bmatrix} \text{ or } \\
\bm{u} & = \begin{bmatrix} 0 & 0 & 0 & 0 & 0 & 0 \end{bmatrix} \text{ or } \\
\bm{u} & = \begin{bmatrix} 0.25 & 0.25 & 0.25 & 0.25 & 0.25 & 0.25 \end{bmatrix}
\end{align*} \vfill
\item If $\bm{u}$ is same for every unit, our $p$-value does not change for any $\gamma \geq 0$ \vfill
\item Also, for some $\gamma \geq 0$, there may be $\bm{u}$ that ``helps us'' \vfill
\item[] I.e., $p$-value could become even smaller \vfill
\end{itemize} \vfill
\end{frame}
%----------------------------------------------------------------
\begin{frame}{Sensitivity analysis model: example}
\vfill
\begin{itemize} \vfill
\item E.g., if $\bm{u} = \begin{bmatrix} 0 & 1 & 1 & 0 & 0 & 1 \end{bmatrix}$, then \vfill
\begin{figure}
\includegraphics[width = 0.9\linewidth]{anti_con_sens_null_dist_z_8.pdf}
\end{figure} \vfill
\item However, we often want to be maximally conservative \vfill
\item So we find $\bm{u}$ that, for fixed $\gamma \geq 0$, yields largest $p$-value \vfill
\end{itemize} \vfill
\end{frame}
%----------------------------------------------------------------
\section{Conservative sensitivity analysis}
\begin{frame}{Conservative sensitivity analysis}
\vfill
\begin{itemize} \vfill
\item For fixed $\gamma \geq 0$, find $\mathbf{u}$ that maximizes $p$-value  \vfill
\begin{itemize} \vfill
\item  I.e., find $\max \limits_{\mathbf{u} \in \mathcal{U}} \Pr\left(t\left(\mathbf{Z}, \mathbf{y}\right) \geq t^{\text{obs}} \right)$, where $t^{\text{obs}}$ is our observed test-statistic \vfill
\end{itemize}	\vfill
\item  For large class of test statistics, \citet{rosenbaumkrieger1990} show \vfill
\begin{itemize} \vfill
\item First, renumber the subjects such that $y_1 \geq y_2 \geq \ldots \geq y_n$  \vfill
\item[] E.g., in \texttt{R}: \texttt{data <- dplyr::arrange(.data = data, desc(y))} \vfill
\item The $\mathbf{u} \in \mathcal{U}$ that maximizes the $p$-value is some $\mathbf{u} \in \mathcal{U}^+ \subset \mathcal{U} = \left[0, 1\right]^n$, where  \vfill
\begin{align*}
\mathcal{U}^+ & = \left\{
\begin{bmatrix} 1 \\ 0 \\ 0 \\ \vdots \\ 0 \end{bmatrix},
\begin{bmatrix} 1 \\ 1 \\ 0 \\ \vdots \\ 0 \end{bmatrix},
\ldots , 
\begin{bmatrix} 1 \\ 1 \\ \vdots \\ 1 \\ 0 \end{bmatrix}
\right\}
\end{align*} \vfill
\end{itemize} \vfill
\item  All $\mathbf{u} \in \mathcal{U}^+$ characterized by \vfill
\begin{enumerate}
\item At least one $1$ and one $0$ and \vfill
\item The $1$s all appear for units with largest outcomes \vfill
\end{enumerate}
\end{itemize} \vfill
\end{frame}
%----------------------------------------------------------------
\begin{frame}{Conservative sensitivity analysis}
\vfill
\begin{itemize} \vfill
\item With one matched set, we can \vfill
\begin{enumerate} \vfill
\item Enumerate entire set of $\mathcal{U}^+$; \vfill
\item For fixed $\gamma \geq 0$, calculate $p$-value for every $\bm{u} \in \mathcal{U}^+$; \vfill
\item Then choose $\bm{u} \in \mathcal{U}^+$ that maximizes $p$-value for this $\gamma \geq 0$; \vfill
\item Do this for increasing values of $\gamma$ and stop when unable to reject sharp null \vfill
\end{enumerate} \vfill
\end{itemize} \vfill
\end{frame}
%----------------------------------------------------------------
\begin{frame}{Conservative sensitivity analysis}
\vfill
\begin{itemize} \vfill
\item For matched set in Table \ref{tab: matched set} and $\Gamma = \exp(\gamma) = 2$, the worst case $\bm{u}$ is \vfill
\begin{table}[H]
\centering{}
    \begin{tabular}{l|l|l}
  $\bm{z}$ & $\bm{u}$ & $\bm{y}$ \\ \hline
  $1$ & $\textcolor{red}{1}$ & $\textcolor{red}{22}$ \\
  $0$ & $\textcolor{blue}{0}$ & $\textcolor{blue}{8}$ \\
  $0$ & $\textcolor{blue}{0}$ & $\textcolor{blue}{11}$ \\
  $1$ & $\textcolor{red}{1}$ & $\textcolor{red}{15}$ \\
  $1$ & $\textcolor{red}{1}$ & $\textcolor{red}{18}$ \\
  $0$ & $\textcolor{blue}{0}$ & $\textcolor{blue}{1}$ \\
    \end{tabular}
\caption{Worst-case $\bm{u}$ for test of sharp null of no effects with $\Gamma = \exp(\gamma) = 2$}
\end{table} \vfill
\end{itemize}
\end{frame}
%----------------------------------------------------------------
\begin{frame}{Conservative sensitivity analysis}
\vfill
The $p$-value under this ``worst-case'' $\bm{u}$ for $\Gamma = \exp(\gamma) = 2$ is
\vfill
\begin{figure}[H]
\includegraphics[width=0.9\linewidth]{sens_null_dist_z_8.pdf}
\end{figure} \vfill
\end{frame}
%----------------------------------------------------------------
\begin{frame}{Conservative sensitivity analysis}
\vfill
The ``worst-case'' $p$-value over increasing values of $\Gamma$ is
\vfill
\begin{figure}[H]
\includegraphics[width=0.9\linewidth]{sens_plot.pdf}
\end{figure} \vfill
\end{frame}
%----------------------------------------------------------------
\section{Conservative sensitivity analysis with many strata}
\begin{frame}{Conservative sensitivity analysis in stratified studies}
\vfill
\begin{itemize} \vfill
\item Let the index $b \in \left\{1, \ldots, B\right\}$ runs over the $B$ strata \vfill
\item  Then order $n$ units lexically by stratum and then in decreasing order within strata by value of outcome $y_{bi}$.
\begin{itemize} \vfill
\item  E.g., in \texttt{R}: \texttt{data <- dplyr::arrange(.data = data, strata, desc(y))} \vfill
\end{itemize}	 \vfill
\item  The worst-case $\mathbf{u} \in \mathcal{U}^+$ cannot be found by finding $\mathbf{u}_b \in \mathcal{U}_b^+$ in each stratum one at a time \vfill
\item  All $n$ coordinates of $\mathbf{u} \in \mathcal{U}^+$ must be found simultaneously \vfill
\item  Hence, there are $\prod \limits_{b = 1}^B \left(n_b - 1\right)$ candidate values to consider \vfill
\begin{itemize} \vfill
\item  Usually computationally intractable, even in relatively small studies  \vfill
\end{itemize} \vfill
\item So, in practice, we use asymptotic approximation \\ \citep{gastwirthetal2000,rosenbaum2018} \vfill
\end{itemize} \vfill
\end{frame}
%----------------------------------------------------------------
\begin{frame}[allowframebreaks]
\frametitle{References} 
\scriptsize
\bibliographystyle{chicago}
\bibliography{Master_Bibliography}   % name your BibTeX data base
\end{frame}
%----------------------------------------------------------------
\end{document}

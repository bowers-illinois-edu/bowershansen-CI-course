%  For slides only
%\documentclass{beamer} 
\newcommand{\tcw}{\textcolor{black}}
\newcommand{\mynoteonly}{}
\newcommand{\nottheirhandout}{handout:0}

% % For handout
%%% Usage:
%% $ cat handout.tex unitXX-foo.tex | pdflatex
%% $ pdfjam -o unitXX-handout{a/b/...}.pdf texput.pdf '{1stpage}-{lastpage}'
\documentclass[handout]{beamer}
\usepackage{pgfpages}
\pgfpagesuselayout{4 on 1}[letterpaper,border shrink=5mm]%\nofiles

\mode<handout>{\setbeamercolor{background canvas}{bg=black!5}}
\newcommand{\tcw}{\textcolor{structure.bg}}
\newcommand{\mynoteonly}{| handout:0}
\newcommand{\nottheirhandout}{handout:0}

%For handout + mynotes
%%% Usage:
%% $ cat handout+mynotes.tex unitXX-foo.tex | pdflatex
%% $ pdfjam --nup 1x4 -o unitXX-foo-wmn{a/b/...}.pdf texput.pdf '1,{1stpage}-{lastpage}'
\documentclass[handout]{beamer}
\usepackage{pgfpages}
%%\pgfpagesuselayout{2 on 1}[letterpaper,border shrink=5mm] 
\setbeameroption{show notes on second screen=left}
\newcommand{\tcw}{\textcolor{black}}
\newcommand{\mynoteonly}{}
\newcommand{\nottheirhandout}{}

\newcommand{\igrphx}[2][width=\linewidth]{\includegraphics[#1]{images/#2}}

\renewcommand{\strut}{\rule{0pt}{3ex}}

% Frame note commands, with time budget updating
\newcounter{timeTotal}
\newcommand{\printUpdateTimeTotal}[1]{ \addtocounter{timeTotal}{#1}
  \textrm{(#1 min budgeted)} \hfill \textbf{Finish by
    \arabic{timeTotal}\ min from start}\\[1ex]}
\newcommand{\tnote}[3][10]{#2 \note{#2: #3 \\[.5ex]} \printUpdateTimeTotal{#1}}
\newcommand{\itnote}[2][10]{\note{ \begin{itemize} #2 \end{itemize}
\vfill \noindent  \printUpdateTimeTotal{#1} }}
\newcommand{\ennote}[2][10]{\note{ \begin{enumerate} #2 \end{enumerate}
\vfill \noindent  \printUpdateTimeTotal{#1} }}
\newcommand{\Note}[2][10]{\note{ #2 \mbox{ }\\ \vfill \noindent  \printUpdateTimeTotal{#1} }}

\newenvironment{Column}[1][.5\linewidth]{\begin{column}{#1}}{\end{column}}

\mode<handout>{\beamertemplatesolidbackgroundcolor{black!5}} 
\mode<article>{\usepackage{fullpage}}
\mode<presentation>
{
  \usetheme{Boadilla}
  % or ...

  \setbeamercovered{transparent}
  % or whatever (possibly just delete it)
}
\usepackage{url}
\usepackage{ulem}

\usepackage[english]{babel}
% or whatever

\usepackage[latin1]{inputenc}
% or whatever

\usepackage{times}
\usepackage[T1]{fontenc}
% Or whatever. Note that the encoding and the font should match. If T1
% does not look nice, try deleting the line with the fontenc.

\AtBeginSubsection[]
{
  \begin{frame}<beamer>
    \frametitle{Outline}
    \tableofcontents[subsectionstyle=show/shaded/hide]
  \end{frame}
}

\AtBeginSection[]
{
  \begin{frame}<beamer>
    \frametitle{Outline}
    \tableofcontents[hideothersubsections,sectionstyle=show/shaded]
  \end{frame}
}


% If you wish to uncover everything in a step-wise fashion, uncomment
% the following command: 

%\beamerdefaultoverlayspecification{<+->}

%USEFUL CODE TEMPLATES:
%\begin{itemize}[<+-| alert@+>]


% \begin{frame}[fragile]


% \begin{columns}
% \column{.4\textwidth}  
%   \begin{itemize}
%   \item<1-| alert@1> Why sample twice?
%   \item<2-| alert@2> Illustrative example.
%   \item<3-| alert@3> PSU's and SSU's.
%   \item<4-| alert@4> Stratification in combination with two-stage cluster sampling.
%   \item<5-| alert@5> Detailed example.
%   \end{itemize}
% \column{.6\textwidth}
% \only<2| handout:0>{
% \includegraphics[width=\textwidth]{nursingHomes}}
% \only<5| handout:1>{
% \includegraphics[width=\textwidth]{hosp_cover}}
% \end{columns}

% FOR INCLUDING R CODE. 
%\usepackage{listings}
%\lstset{language=R}
% \begin{frame}[fragile]
% \frametitle{Some code}
% \begin{lstlisting}
% > plot(myobj)
% > rm(myobj)
% \end{lstlisting}  
% \end{frame}

%\addtocounter{framenumber}{-1}

\author{Stats 617}
\date{Winter 2016}

\usepackage{amsmath,amsthm}
\usepackage{wasysym,pifont}
\usepackage{Sweave}
\usepackage{ulem}
\usepackage{textcomp}
\usepackage{versions}
%% amsthm-type theorem environment specifications -- 
%% see amsthdoc.pdf in amscls documentation
\theoremstyle{plain}
\newtheorem{prop}{Proposition}[section]
\newtheorem{lem}[prop]{Lemma}

\newtheorem*{thm}{Proposition}
\newtheorem*{cor}{Corollary}

\theoremstyle{definition}
\newtheorem{defn}{Definition}[section]

\newcommand{\Pdistsym}{P}
\newcommand{\Pdistsymn}{P_n}
\newcommand{\Qdistsym}{Q}
\newcommand{\Qdistsymn}{Q_n}
\newcommand{\Qdistsymni}{Q_{n_i}}
\newcommand{\Qdistsymt}{Q[t]}
\newcommand{\dQdP}{\ensuremath{\frac{dQ}{dP}}}
\newcommand{\dQdPn}{\ensuremath{\frac{dQ_{n}}{dP_{n}}}}
\newcommand{\EE}{\ensuremath{\mathbf{E}}}
\newcommand{\EEp}{\ensuremath{\mathbf{E}_{P}}}
\newcommand{\EEpn}{\ensuremath{\mathbf{E}_{P_{n}}}}
\newcommand{\EEq}{\ensuremath{\mathbf{E}_{Q}}}
\newcommand{\EEqn}{\ensuremath{\mathbf{E}_{Q_{n}}}}
\newcommand{\EEqni}{\ensuremath{\mathbf{E}_{Q_{n[i]}}}}
\newcommand{\EEqt}{\ensuremath{\mathbf{E}_{Q[t]}}}
\newcommand{\PP}{\ensuremath{\mathbf{Pr}}}
\newcommand{\PPp}{\ensuremath{\mathbf{Pr}_{P}}}
\newcommand{\PPpn}{\ensuremath{\mathbf{Pr}_{P_{n}}}}
\newcommand{\PPq}{\ensuremath{\mathbf{Pr}_{Q}}}
\newcommand{\PPqn}{\ensuremath{\mathbf{Pr}_{Q_{n}}}}
\newcommand{\PPqt}{\ensuremath{\mathbf{Pr}_{Q[t]}}}
\newcommand{\var}{\ensuremath{\mathbf{V}}}
\newcommand{\varp}{\ensuremath{\mathbf{V}_{P}}}
\newcommand{\varpn}{\ensuremath{\mathbf{V}_{P_{n}}}}
\newcommand{\varq}{\ensuremath{\mathbf{V}_{Q}}}
\newcommand{\cov}{\ensuremath{\mathbf{Cov}}}
\newcommand{\covp}{\ensuremath{\mathbf{Cov}_{P}}}
\newcommand{\covpn}{\ensuremath{\mathbf{Cov}_{P_{n}}}}
\newcommand{\covq}{\ensuremath{\mathbf{Cov}_{Q}}}

\newcommand{\hatvar}{\ensuremath{\widehat{\mathrm{Var}}}}
\newcommand{\hatcov}{\ensuremath{\widehat{\mathrm{Cov}}}}

\newcommand{\sehat}{\ensuremath{\widehat{\mathrm{se}}}}

\newcommand{\combdiff}[1]{\ensuremath{\Delta_{{z}}[#1]}}
\newcommand{\Combdiff}[1]{\ensuremath{\Delta_{{Z}}[#1]}}

\newcommand{\psvec}{\ensuremath{\varphi}}
\newcommand{\psvecgc}{\ensuremath{\tilde{\varphi}}}


\newcommand{\atob}[2]{\ensuremath{#1\!\! :\!\! #2}}
\newcommand{\stratA}{\ensuremath{\mathbf{S}}}
\newcommand{\stratAnumstrat}{\ensuremath{S}}
\newcommand{\sAsi}{\ensuremath{s}}

\newcommand{\permsd}{\ensuremath{\sigma_{\Pdistsym}}}
% \newcommand{\dz}[1]{\ensuremath{d_{z}[{#1}]}}
\newcommand{\dZ}[1]{\ensuremath{d_{Z}[{#1}]}}
\newcommand{\tz}[1]{\ensuremath{t_{{z}}[#1]}}
\newcommand{\tZ}[1]{\ensuremath{t_{{Z}}[#1]}}


\newlength{\tabcolsepadj}
\setlength{\tabcolsepadj}{1.3mm}

%%% NEWBLOCK UNDEFINED BUG
\def\newblock{\hskip .11em plus .33em minus .07em}

%%% tightlist undefined control sequence bug 
%%% http://tex.stackexchange.com/questions/257418/error-tightlist-converting-md-file-into-pdf-using-pandoc
\providecommand{\tightlist}{%
  \setlength{\itemsep}{0pt}\setlength{\parskip}{0pt}}
\newcommand{\igrphx}[2][width=\linewidth]{\includegraphics[#1]{images/#2}}

\renewcommand{\strut}{\rule{0pt}{3ex}}

\newenvironment{Column}[1][.5\linewidth]{\begin{column}{#1}}{\end{column}}

\usetikzlibrary{arrows} % also see \tikzstyle spec below after \begin{doc}

\newcommand{\mlpnode}[1]{\tikz[baseline=-.5ex] \coordinate (#1) {};}
%\newcommand{\mlpnode}[1]{\raisebox{.5ex}{\pnode{#1}}}


\usepackage{xspace}

\includeversion{pedantic} % \excludeversion

\tikzstyle{every picture}+=[remember picture]


\date{ICPSR Session 2 (July 18, 2022)}


\usepackage{tikz} 
\usetikzlibrary{arrows} % also see \tikzstyle spec below after \begin{doc}

\usepackage{xspace}
\usepackage[round]{natbib}

\usepackage{versions}
\includeversion{pedantic} % \excludeversion

\usepackage{./mytexdefs2}
\usepackage{./mytexdefs}
\newcounter{saveenum}

\title{Unit 5: Multivariate distance matching}
% \author, date moved to beamer-preamble-*-all.tex

\begin{document}

\newcounter{saveenumi}
\tikzstyle{every picture}+=[remember picture]

  \begin{frame}
    \frametitle{Outline \& Readings}

\tableofcontents[subsectionstyle=show/hide/hide]

 \alert{Readings for this Unit:} \textit{DOS}, ch.8.
\end{frame}

\section[Pair and multiple controls matching]{Pair matching and matching with multiple controls}

\newlength{\boywidth} 
\newlength{\girlwidth} 

\newcommand{\igrphxG}[1]{\includegraphics[width=\girlwidth]{../2014/lectures/images/#1}}
\newcommand{\igrphxB}[1]{\includegraphics[width=\boywidth]{../2014/lectures/images/#1}}


\begin{frame}
  \frametitle{Optimal (communitarian) vs greedy (individualistic) matching}

\settowidth{\boywidth}{Bachelors\ }
\setlength{\girlwidth}{.5\boywidth}


\begin{center}
    \begin{tabular}{l|cccc}
      & \multicolumn{4}{c}{Bachelorettes} \\
Bachelors & \igrphxG{amandamarsh}       & \igrphxG{TristaRehn}
& \igrphxG{emilymaynard2}       & \igrphxG{chantal2} \\ \hline
\igrphxB{alexmichel2} & 0 & 1 & 1 & 10 \\
\igrphxB{bradWomack} &10& 0 & 10 & 10 \\
\igrphxB{ChrisHarrison} &  1 & 1 & $\infty$ & $\infty$ \\ \hline
    \end{tabular}
  \end{center}

\end{frame}


\begin{frame}
  \frametitle{Costs of nuclear plants}
\framesubtitle<2->{When expanding, cheaper to build on the site of an existing plant?}
\begin{columns}
\column{.5\textwidth}
\begin{center}
  \igrphx[width=.5\textwidth]{coxSnell}
\end{center}

\column{.5\textwidth}
  \only<2-| handout:0>{\igrphx[width=\textwidth]{Nuclear_plant_at_Grafenrheinfeld}}
\end{columns}

\note{This is a \textit{prospective} design.  Explain.

(Presentation version has Nuclear plant image on 2nd overlay)
}
\end{frame}


\begin{frame}
\frametitle{New and refurbished nuclear plants}
\note{
  \begin{itemize}
  \item This is what many understand by matching.  [Greedy pair match follows.]
  \item Sometimes matching means  ``with-replacement'' matching. We'll permit that too, but in a structured way, and keeping track of how many replacements. 
  \item Why keep track of the replacements?  --B/c they affect standard errors of the matched comparison.  (``Effective sample size'' to be introduced a few slides later.)
  \end{itemize}
}
\begin{minipage}[t]{2in}
\begin{center}
Existing site\\
{\small
% latex table generated in R 3.0.1 by xtable 1.7-1 package
% Wed Aug 14 15:26:53 2013
\begin{tabular}{lr}
  \hline
 & capacity \\ 
  \hline
A & {660} {\mlpnode{NA}} \\ 
  B & {660} {\mlpnode{NB}} \\ 
  C & {420} {\mlpnode{NC}} \\ 
  D & {130} {\mlpnode{ND}} \\ 
  E & {650} {\mlpnode{NE}} \\ 
  F & {430} {\mlpnode{NF}} \\ 
  G & {420} {\mlpnode{NG}} \\ 
   \hline
\end{tabular}}
\end{center}
\bigskip
\bigskip
{\footnotesize  ``capacity'' is net capacity of the power plant, in MWe above
400.

\only<9->{\alert<@+| handout:0>{This is matching ``without replacement.''}}
\only<10->{\alert<@+| handout:0>{Contrast with matching ``with replacement.''}}
}
\end{minipage}
\begin{minipage}[t]{2in}
\begin{center}
New site\\
{\scriptsize
% latex table generated in R 3.0.1 by xtable 1.7-1 package
% Wed Aug 14 15:26:53 2013
\begin{tabular}{lr}
  \hline
 & capacity \\ 
  \hline
{\mlpnode{NH}\mbox{}} {H} & 290 \\ 
  {\mlpnode{NI}\mbox{}} {I} & 660 \\ 
  {\mlpnode{NJ}\mbox{}} {J} & 660 \\ 
  {\mlpnode{NK}\mbox{}} {K} & 110 \\ 
  {\mlpnode{NL}\mbox{}} {L} & 420 \\ 
  {\mlpnode{NM}\mbox{}} {M} &  60 \\ 
  {\mlpnode{NN}\mbox{}} {N} & 390 \\ 
  {\mlpnode{NO}\mbox{}} {O} & 160 \\ 
  {\mlpnode{NP}\mbox{}} {P} & 390 \\ 
  {\mlpnode{NQ}\mbox{}} {Q} & 130 \\ 
  {\mlpnode{NR}\mbox{}} {R} & 650 \\ 
  {\mlpnode{NS}\mbox{}} {S} & 450 \\ 
  {\mlpnode{NT}\mbox{}} {T} & 380 \\ 
  {\mlpnode{NU}\mbox{}} {U} & 440 \\ 
  {\mlpnode{NV}\mbox{}} {V} & 690 \\ 
  {\mlpnode{NW}\mbox{}} {W} & 510 \\ 
  {\mlpnode{NX}\mbox{}} {X} & 390 \\ 
  {\mlpnode{NY}\mbox{}} {Y} & 140 \\ 
  {\mlpnode{NZ}\mbox{}} {Z} & 730 \\ 
   \hline
\end{tabular}}
\end{center}
\end{minipage}


% \only<2->{ \ncline{NA}{NI} }
% \only<3->{ \ncline{NB}{NJ} }
% \only<4->{ \ncline{NC}{NL} }
% \only<5->{ \ncline{ND}{NQ} }
% \only<6->{ \ncline{NE}{NR} }
% \only<7->{ \ncline{NF}{NU} }
% \only<8->{ \ncline{NG}{NN} }
\begin{tikzpicture}[overlay]
  \path[draw,gray]<2-> (NA) edge (NI);
 \path[draw,gray]<3-> (NB) edge (NJ);
 \path[draw,gray]<4-> (NC) edge (NL);
 \path[draw,gray]<5-> (ND) edge (NQ);
 \path[draw,gray]<6-> (NE) edge (NR);
 \path[draw,gray]<7-> (NF) edge (NU);
 \path[draw,gray]<8-> (NG) edge (NN);
 \end{tikzpicture}\only<10-| handout:0>{
\begin{tikzpicture}[overlay]
  \path[draw,red] (NA) edge [out=0, in=-180] (NI);
 \path[draw,red] (NB) edge [out=0, in=-180] (NJ);
 \path[draw,red] (NC) edge [out=0, in=-180] (NL);
 \path[draw,red] (NF) edge [out=0, in=-180] (NL);
 \path[draw,red] (NG) edge [out=0, in=-180] (NL);
 \path[draw,red] (ND) edge [out=0, in=-180] (NQ);
 \path[draw,red] (NE) edge [out=0, in=-180] (NR);
 \end{tikzpicture}}
\end{frame}



\begin{frame}[fragile]
  \frametitle{A simple model for matched
    analysis\footnote{\textit{cf.} Abadie \& Imbens (2012, JASA).}}
\note{
This model is both an OLS regression w/ matched pair and treatment dummies; a two-way ANOVA; and a model of the matched pair differences as i.i.d.

Note carefully that matching and outcome analysis are \textit{separate} steps, even in this simple example.  In practice there'll be even more preliminaries to the analysis.

The model asks you to believe that matching on capacity is a
sufficient adjustment.  We'll be able to do better than that, but what
we do with those better matches will be much the same.

After showing them this, STOP and have them review and work thru
worksheet up to first bundle of exercises (create \texttt{pm},
\texttt{tm}; compare balance on \texttt{t2}, \texttt{date}).
%(At \url{http://www.stat.lsa.umich.edu/~bbh/fullmatch-vignette.pdf}.)

% \begin{pedantic}
This will be first explicit use of object orientation, assignments.  Explain.
%\end{pedantic}
}
\begin{Schunk}
\begin{Sinput}
> pm <- pairmatch( pr~cap, data=nuke.nopt )
> ( cost.adj <- lm(cost~pr+pm, data=nuke.nopt) )
\end{Sinput}
\end{Schunk}
% Note hard-coding of following:
\begin{Schunk}
\begin{Soutput}
Call:
lm(formula = cost ~ pr + pm, data = nuke.nopt)

Coefficients:
(Intercept)           pr        pmm.2  
     433.40        29.40       199.17  
<...>
\end{Soutput}
\end{Schunk}
\begin{Schunk}
\begin{Sinput}
> summary(cost.adj)
\end{Sinput}
\end{Schunk}
$\vdots$
\begin{Schunk}
\begin{Soutput}
            Estimate Std. Error t value Pr(>|t|)
(Intercept)   554.20         81  6.8788  0.00047
pr            -12.86         57 -0.2257  0.82894
pm1.2          -0.16        107 -0.0015  0.99885
\end{Soutput}
\end{Schunk}
$\vdots$
\end{frame}

\begin{frame}
\frametitle{Matching on several variables at once}
\note{ Note differences in scale of variables.  How to put differences in date and capacity onto a common scale? }
\begin{minipage}[t]{2in}
\begin{center}
Existing site\\
{\small
% latex table generated in R 3.0.1 by xtable 1.7-1 package
% Wed Aug 14 15:26:54 2013
\begin{tabular}{lrr}
  \hline
 & date & capacity \\ 
  \hline
A & 2.3 & {660} {\mlpnode{NA}} \\ 
  B & 3.0 & {660} {\mlpnode{NB}} \\ 
  C & 3.4 & {420} {\mlpnode{NC}} \\ 
  D & 3.4 & {130} {\mlpnode{ND}} \\ 
  E & 3.9 & {650} {\mlpnode{NE}} \\ 
  F & 5.9 & {430} {\mlpnode{NF}} \\ 
  G & 5.1 & {420} {\mlpnode{NG}} \\ 
   \hline
\end{tabular}}
\end{center}
\bigskip
\bigskip
{\footnotesize ``date'' is date of construction, in years after
1965; ``capacity'' is net capacity of the power plant, in MWe above
400.}
\end{minipage}
\begin{minipage}[t]{2in}
\begin{center}
New site\\
{\scriptsize
% latex table generated in R 3.0.1 by xtable 1.7-1 package
% Wed Aug 14 15:26:54 2013
\begin{tabular}{lrr}
  \hline
 & date & capacity \\ 
  \hline
{\mlpnode{NH}\mbox{}} {H} & 3.6 & 290 \\ 
  {\mlpnode{NI}\mbox{}} {I} & 2.3 & 660 \\ 
  {\mlpnode{NJ}\mbox{}} {J} & 3.0 & 660 \\ 
  {\mlpnode{NK}\mbox{}} {K} & 2.9 & 110 \\ 
  {\mlpnode{NL}\mbox{}} {L} & 3.2 & 420 \\ 
  {\mlpnode{NM}\mbox{}} {M} & 3.4 &  60 \\ 
  {\mlpnode{NN}\mbox{}} {N} & 3.3 & 390 \\ 
  {\mlpnode{NO}\mbox{}} {O} & 3.6 & 160 \\ 
  {\mlpnode{NP}\mbox{}} {P} & 3.8 & 390 \\ 
  {\mlpnode{NQ}\mbox{}} {Q} & 3.4 & 130 \\ 
  {\mlpnode{NR}\mbox{}} {R} & 3.9 & 650 \\ 
  {\mlpnode{NS}\mbox{}} {S} & 3.9 & 450 \\ 
  {\mlpnode{NT}\mbox{}} {T} & 3.4 & 380 \\ 
  {\mlpnode{NU}\mbox{}} {U} & 4.5 & 440 \\ 
  {\mlpnode{NV}\mbox{}} {V} & 4.2 & 690 \\ 
  {\mlpnode{NW}\mbox{}} {W} & 3.8 & 510 \\ 
  {\mlpnode{NX}\mbox{}} {X} & 4.7 & 390 \\ 
  {\mlpnode{NY}\mbox{}} {Y} & 5.4 & 140 \\ 
  {\mlpnode{NZ}\mbox{}} {Z} & 6.1 & 730 \\ 
   \hline
\end{tabular}}
\end{center}
\end{minipage}

\end{frame}

%\addtocounter{frame}{-1}
%
\begin{frame}
\frametitle{Matching on several variables at once} \framesubtitle{Finding a common scale}
\begin{minipage}[t]{2in}
\begin{center}
Existing site\\
{\small
% latex table generated in R 3.0.1 by xtable 1.7-1 package
% Wed Aug 14 15:26:54 2013
\begin{tabular}{lrr}
  \hline
 & z.date & z.cap \\ 
  \hline
A & -1.6 & {1.2} {\mlpnode{NA}} \\ 
  B & -0.9 & {1.2} {\mlpnode{NB}} \\ 
  C & -0.4 & {0} {\mlpnode{NC}} \\ 
  D & -0.4 & {-1.4} {\mlpnode{ND}} \\ 
  E & 0.1 & {1.1} {\mlpnode{NE}} \\ 
  F & 2.2 & {0} {\mlpnode{NF}} \\ 
  G & 1.3 & {0} {\mlpnode{NG}} \\ 
   \hline
\end{tabular}}
\end{center}
\bigskip
\bigskip
\bigskip
{\footnotesize Here, ``z.date'' is date of
construction in s.d.'s from the sample-average date of construction; ``z.cap'' is the capacity of the power plant (in s.d.'s from mean capacity).}
\end{minipage}
\begin{minipage}[t]{2in}
\begin{center}
New site\\
{\scriptsize
% latex table generated in R 3.0.1 by xtable 1.7-1 package
% Wed Aug 14 15:26:54 2013
\begin{tabular}{lrr}
  \hline
 & z.date & z.cap \\ 
  \hline
{\mlpnode{NH}\mbox{}} {H} & -0.3 & -0.7 \\ 
  {\mlpnode{NI}\mbox{}} {I} & -1.6 & 1.2 \\ 
  {\mlpnode{NJ}\mbox{}} {J} & -0.9 & 1.2 \\ 
  {\mlpnode{NK}\mbox{}} {K} & -0.9 & -1.5 \\ 
  {\mlpnode{NL}\mbox{}} {L} & -0.7 & -0.0 \\ 
  {\mlpnode{NM}\mbox{}} {M} & -0.4 & -1.8 \\ 
  {\mlpnode{NN}\mbox{}} {N} & -0.5 & -0.2 \\ 
  {\mlpnode{NO}\mbox{}} {O} & -0.3 & -1.3 \\ 
  {\mlpnode{NP}\mbox{}} {P} & -0.1 & -0.2 \\ 
  {\mlpnode{NQ}\mbox{}} {Q} & -0.4 & -1.4 \\ 
  {\mlpnode{NR}\mbox{}} {R} & 0.1 & 1.1 \\ 
  {\mlpnode{NS}\mbox{}} {S} & 0.1 & 0.1 \\ 
  {\mlpnode{NT}\mbox{}} {T} & -0.4 & -0.2 \\ 
  {\mlpnode{NU}\mbox{}} {U} & 0.7 & 0.1 \\ 
  {\mlpnode{NV}\mbox{}} {V} & 0.4 & 1.3 \\ 
  {\mlpnode{NW}\mbox{}} {W} & -0.1 & 0.4 \\ 
  {\mlpnode{NX}\mbox{}} {X} & 0.9 & -0.2 \\ 
  {\mlpnode{NY}\mbox{}} {Y} & 1.7 & -1.4 \\ 
  {\mlpnode{NZ}\mbox{}} {Z} & 2.3 & 1.5 \\ 
   \hline
\end{tabular}}
\end{center}
\end{minipage}
\end{frame}
\note{

\begin{itemize}
\item \textsc{Don't linger.}
  \item Conversion to standard units
\item Mahalanobis distance combines squared differences on
  standardized versions of the two variables, addressing correlations
  between variables by appropriately reducing the contributions of
  highly correlated variables.
\item For more info, look up Mahalanobis distance on your own.
\end{itemize}
}


\begin{frame}<1>[fragile,label=MahalDFr]
\frametitle{Combining discrepancies via Mahalanobis distance}
\addtolength{\tabcolsep}{-\tabcolsepadj} \vfill
%\begin{center}
{\footnotesize
\begin{tabular}{|r|rrrrrrrrrrrrrrrrrrr|}
\hline
\multicolumn{2}{|l}{Exist-} & \multicolumn{18}{c|}{New sites} \\
\multicolumn{1}{|l}{ing}& H & I & J & K & L & M & N & O & P & Q & R & S & T & U & V & W & X & Y & Z \\ 
  \hline
A & 2.2 & 0.0 & 0.7 & 2.7 & 1.4 & 3.1 & 1.7 & 2.7 & 2.0 & 2.8 & 1.6 & 1.9 & 1.8 & 2.4 & 1.9 & 1.6 & 2.7 & 4.0 & 3.8 \\ 
  B & 1.9 & 0.7 & 0.0 & 2.6 & 1.2 & 2.9 & 1.3 & 2.5 & 1.5 & 2.6 & 0.9 & 1.4 & 1.4 & 1.9 & 1.2 & 1.1 & 2.2 & 3.5 & 3.1 \\ 
  C & 0.7 & 1.6 & 1.2 & 1.6 & 0.3 & 1.7 & 0.2 & 1.3 & 0.4 & 1.4 & 1.2 & 0.5 & 0.2 & 1.1 & 1.5 & 0.5 & 1.3 & 2.4 & 3.1 \\ 
  D & 0.8 & 2.8 & 2.6 & 0.5 & 1.4 & 0.3 & 1.3 & 0.2 & 1.3 & 0.0 & 2.5 & 1.6 & 1.2 & 1.9 & 2.8 & 1.9 & 1.8 & 2.0 & 3.9 \\ 
  E & 1.8 & 1.6 & 0.9 & 2.7 & 1.3 & 2.9 & 1.4 & 2.4 & 1.3 & 2.5 & 0.0 & 1.0 & 1.4 & 1.1 & 0.3 & 0.7 & 1.5 & 2.9 & 2.2 \\ 
  F & 2.5 & 3.8 & 3.2 & 3.4 & 2.8 & 3.1 & 2.6 & 2.7 & 2.2 & 2.9 & 2.3 & 2.0 & 2.6 & 1.4 & 2.2 & 2.2 & 1.3 & 1.5 & 1.4 \\ 
  G & 1.7 & 3.0 & 2.4 & 2.6 & 1.9 & 2.4 & 1.8 & 2.0 & 1.4 & 2.2 & 1.6 & 1.2 & 1.7 & 0.6 & 1.6 & 1.4 & 0.4 & 1.4 & 1.8 \\ 
   \hline
\end{tabular}
}
\vfill \addtolength{\tabcolsep}{\tabcolsepadj}
%\end{center}

\only<1\mynoteonly>{
  \begin{itemize}
  \item Similar to sum of squared differences, along \texttt{z.cap}, \texttt{z.date}.
\item However, Mahalanobis corrects for collinearity of \texttt{z.cap}, \texttt{z.date}.
  \end{itemize}
}
\only<2>{
R code:
\begin{semiverbatim}
> match\_on(pr\textasciitilde cap+date, data=nuke.nopt)
\end{semiverbatim}
}
\only<3\mynoteonly>{
  \begin{itemize}
  \item Mahalanobis designed with Normal variables in mind.
  \item If you use it on a categorical variable, and there is a rare category, Mahalanobis tries extra-hard to match on it.
  \end{itemize}
}
\end{frame}

\note{
 \textsc{Don't linger.}
 
\texttt{optmatch} computes squared Mahalanobis distances by default.
  
  Example: (note corr betw. cap and date is just .0025)
{\footnotesize
  \begin{semiverbatim}
> nuke.nopt[1:3,c('cap', 'date')]

   cap date

H  687   69

I 1065   67

A 1065   67

\end{semiverbatim}

\begin{tabular}{l|rrrrrr}
& \multicolumn{2}{c}{diff}    &\multicolumn{2}{c}{$z$} &\multicolumn{2}{c}{$z^2$} \\
 &cap &date                 &cap &date               & cap & date               \\
$H-A$ &$-378$  &1.2         &$-1.8$ & 1.3            & 3.3  & 1.6        \\
$I-A$    &0  &0.0           &0.0 & 0.0               & 0.0  & 0.0               \\
\end{tabular}
}
  }


% \begin{frame}
% \frametitle{A formal definition of the Mahalanobis distance}

% The \emph{Mahalanobis distance} between units $a, b$ in terms of quantitative
% variables $(X_1, \ldots, X_k) \equiv \mathbf{X}$ is given by
% $$ d(a, b) \equiv (\mathbf{x}_a - \mathbf{x}_b)^t C^{-1} (\mathbf{x}_a - \mathbf{x}_b),$$
% where $C$ is either (i) the sample covariance matrix of $x_1, \ldots, x_k$
% in the control group or (ii) the covariance matrix of $x_1, \ldots,
% x_k$ formed by pooling control and treatment groups.

% \note{

% Mahalanobis distance was designed for multivariate Normal covariates,
% and it works best for combining variables whose distributions don't
% deviate sharply from Normal.  In the UM space assignment study, logs
% of grant funding were closer to Normal than were grant totals
% themselves; the Mahalanobis distance used combined logs of PI and
% co-PI grant funding totals.

% If you have a rare category, Mahalanobis will strain extra-hard to
% match on it, at the expense of matching more poorly on other
% variables.  }

% \end{frame}


\begin{frame}
\frametitle{Pair-matched new and refurbished plants}
\begin{minipage}[t]{2in}
\begin{center}
Existing site\\
{\small
% latex table generated in R 3.0.1 by xtable 1.7-1 package
% Wed Aug 14 15:26:54 2013
\begin{tabular}{lrr}
  \hline
 & z.date & z.cap \\ 
  \hline
A & -1.6 & {1.2} {\mlpnode{NA}} \\ 
  B & -0.9 & {1.2} {\mlpnode{NB}} \\ 
  C & -0.4 & {0} {\mlpnode{NC}} \\ 
  D & -0.4 & {-1.4} {\mlpnode{ND}} \\ 
  E & 0.1 & {1.1} {\mlpnode{NE}} \\ 
  F & 2.2 & {0} {\mlpnode{NF}} \\ 
  G & 1.3 & {0} {\mlpnode{NG}} \\ 
   \hline
\end{tabular}}
\end{center}
\bigskip
{R code:\\

\begin{semiverbatim}
dcdist = match\_on(pr \textasciitilde\ date+cap, data=nuke.nopt) ;
\end{semiverbatim}
}
\end{minipage}
\begin{minipage}[t]{2in}
\begin{center}
New site\\
{\scriptsize
% latex table generated in R 3.0.1 by xtable 1.7-1 package
% Wed Aug 14 15:26:54 2013
\begin{tabular}{lrr}
  \hline
 & z.date & z.cap \\ 
  \hline
{\mlpnode{NH}\mbox{}} {H} & -0.3 & -0.7 \\ 
  {\mlpnode{NI}\mbox{}} {I} & -1.6 & 1.2 \\ 
  {\mlpnode{NJ}\mbox{}} {J} & -0.9 & 1.2 \\ 
  {\mlpnode{NK}\mbox{}} {K} & -0.9 & -1.5 \\ 
  {\mlpnode{NL}\mbox{}} {L} & -0.7 & -0.0 \\ 
  {\mlpnode{NM}\mbox{}} {M} & -0.4 & -1.8 \\ 
  {\mlpnode{NN}\mbox{}} {N} & -0.5 & -0.2 \\ 
  {\mlpnode{NO}\mbox{}} {O} & -0.3 & -1.3 \\ 
  {\mlpnode{NP}\mbox{}} {P} & -0.1 & -0.2 \\ 
  {\mlpnode{NQ}\mbox{}} {Q} & -0.4 & -1.4 \\ 
  {\mlpnode{NR}\mbox{}} {R} & 0.1 & 1.1 \\ 
  {\mlpnode{NS}\mbox{}} {S} & 0.1 & 0.1 \\ 
  {\mlpnode{NT}\mbox{}} {T} & -0.4 & -0.2 \\ 
  {\mlpnode{NU}\mbox{}} {U} & 0.7 & 0.1 \\ 
  {\mlpnode{NV}\mbox{}} {V} & 0.4 & 1.3 \\ 
  {\mlpnode{NW}\mbox{}} {W} & -0.1 & 0.4 \\ 
  {\mlpnode{NX}\mbox{}} {X} & 0.9 & -0.2 \\ 
  {\mlpnode{NY}\mbox{}} {Y} & 1.7 & -1.4 \\ 
  {\mlpnode{NZ}\mbox{}} {Z} & 2.3 & 1.5 \\ 
   \hline
\end{tabular}}
\end{center}
\end{minipage}
\begin{tikzpicture}[overlay]
  \path[draw,gray] (NA) edge (NI);
 \path[draw,gray] (NB) edge (NJ);
 \path[draw,gray] (NC) edge (NN);
 \path[draw,gray] (ND) edge (NQ);
 \path[draw,gray] (NE) edge (NR);
 \path[draw,gray] (NF) edge (NX);
 \path[draw,gray] (NG) edge (NU);
 \end{tikzpicture}
\begin{semiverbatim}
pairmatch(dcdist, data=nuke.nopt ) \# \textrm{\textit{or} just}

pairmatch( pr \textasciitilde\ date+cap, data=nuke.nopt ) 
\end{semiverbatim}

\end{frame}
\note[itemize]{
\item This match contrasts in two ways with the match we generated a few slides ago: matching on 2 variables rather than one; this is an \textit{optimal} match.  
 \item Optimal means that the matching algorithm minimizes the global sum of matched differences.  Greedy may not give a match that's best-possible in that sense.
}

\begin{frame}
\frametitle{Matching with $k$ controls}
\begin{minipage}[t]{2in}
\begin{center}
Existing site\\
{\small
% latex table generated in R 3.0.1 by xtable 1.7-1 package
% Wed Aug 14 15:26:54 2013
\begin{tabular}{lrr}
  \hline
 & z.date & z.cap \\ 
  \hline
A & -1.6 & {1.2} {\mlpnode{NA}} \\ 
  B & -0.9 & {1.2} {\mlpnode{NB}} \\ 
  C & -0.4 & {0} {\mlpnode{NC}} \\ 
  D & -0.4 & {-1.4} {\mlpnode{ND}} \\ 
  E & 0.1 & {1.1} {\mlpnode{NE}} \\ 
  F & 2.2 & {0} {\mlpnode{NF}} \\ 
  G & 1.3 & {0} {\mlpnode{NG}} \\ 
   \hline
\end{tabular}}
\end{center}
\bigskip
\bigskip
\bigskip
\bigskip
{R code:
}
\end{minipage}
\begin{minipage}[t]{2in}
\begin{center}
New site\\
{\scriptsize
% latex table generated in R 3.0.1 by xtable 1.7-1 package
% Wed Aug 14 15:26:54 2013
\begin{tabular}{lrr}
  \hline
 & z.date & z.cap \\ 
  \hline
{\mlpnode{NH}\mbox{}} {H} & -0.3 & -0.7 \\ 
  {\mlpnode{NI}\mbox{}} {I} & -1.6 & 1.2 \\ 
  {\mlpnode{NJ}\mbox{}} {J} & -0.9 & 1.2 \\ 
  {\mlpnode{NK}\mbox{}} {K} & -0.9 & -1.5 \\ 
  {\mlpnode{NL}\mbox{}} {L} & -0.7 & -0.0 \\ 
  {\mlpnode{NM}\mbox{}} {M} & -0.4 & -1.8 \\ 
  {\mlpnode{NN}\mbox{}} {N} & -0.5 & -0.2 \\ 
  {\mlpnode{NO}\mbox{}} {O} & -0.3 & -1.3 \\ 
  {\mlpnode{NP}\mbox{}} {P} & -0.1 & -0.2 \\ 
  {\mlpnode{NQ}\mbox{}} {Q} & -0.4 & -1.4 \\ 
  {\mlpnode{NR}\mbox{}} {R} & 0.1 & 1.1 \\ 
  {\mlpnode{NS}\mbox{}} {S} & 0.1 & 0.1 \\ 
  {\mlpnode{NT}\mbox{}} {T} & -0.4 & -0.2 \\ 
  {\mlpnode{NU}\mbox{}} {U} & 0.7 & 0.1 \\ 
  {\mlpnode{NV}\mbox{}} {V} & 0.4 & 1.3 \\ 
  {\mlpnode{NW}\mbox{}} {W} & -0.1 & 0.4 \\ 
  {\mlpnode{NX}\mbox{}} {X} & 0.9 & -0.2 \\ 
  {\mlpnode{NY}\mbox{}} {Y} & 1.7 & -1.4 \\ 
  {\mlpnode{NZ}\mbox{}} {Z} & 2.3 & 1.5 \\ 
   \hline
\end{tabular}}
\end{center}
\end{minipage}
\begin{tikzpicture}[overlay]
  \path[draw,gray] (NA) edge (NI);
 \path[draw,gray] (NA) edge (NJ);
 \path[draw,gray] (NB) edge (NL);
 \path[draw,gray] (NB) edge (NW);
 \path[draw,gray] (NC) edge (NN);
 \path[draw,gray] (NC) edge (NT);
 \path[draw,gray] (ND) edge (NO);
 \path[draw,gray] (ND) edge (NQ);
 \path[draw,gray] (NE) edge (NR);
 \path[draw,gray] (NE) edge (NV);
 \path[draw,gray] (NF) edge (NY);
 \path[draw,gray] (NF) edge (NZ);
 \path[draw,gray] (NG) edge (NU);
 \path[draw,gray] (NG) edge (NX);
 \end{tikzpicture}\begin{semiverbatim}
pairmatch( pr \textasciitilde\ date+cap, controls=2, data=nuke.nopt )
\end{semiverbatim}
\end{frame}

\section[Greedy vs Optimal Matching]{Greedy vs Optimal Matching}


\begin{frame}[fragile]
\frametitle{Role of the matching algorithm: greedy matching}
\begin{minipage}[t]{2in}
\begin{center}
Existing site\\
{\small
% latex table generated in R 3.0.1 by xtable 1.7-1 package
% Wed Aug 14 15:26:57 2013
\begin{tabular}{lrr}
  \hline
 & z.date & z.cap \\ 
  \hline
A & -1.6 & {1.2} {\mlpnode{NA}} \\ 
  B & -0.9 & {1.2} {\mlpnode{NB}} \\ 
  C & -0.4 & {0} {\mlpnode{NC}} \\ 
  D & -0.4 & {-1.4} {\mlpnode{ND}} \\ 
  E & 0.1 & {1.1} {\mlpnode{NE}} \\ 
  F & 2.2 & {0} {\mlpnode{NF}} \\ 
  G & 1.3 & {0} {\mlpnode{NG}} \\ 
   \hline
\end{tabular}}
\end{center}
\bigskip
\bigskip
\bigskip
{\footnotesize Here, ``date'' is \emph{rank of} date of
construction, in years after 1965, and  ``capacity'' is
\emph{rank of} net capacity of the power plant, in MWe
above 400.}
\end{minipage}
\begin{minipage}[t]{2in}
\begin{center}
New site\\
{\scriptsize
% latex table generated in R 3.0.1 by xtable 1.7-1 package
% Wed Aug 14 15:26:57 2013
\begin{tabular}{lrr}
  \hline
 & z.date & z.cap \\ 
  \hline
{\mlpnode{NH}\mbox{}} {H} & -0.3 & -0.7 \\ 
  {\mlpnode{NI}\mbox{}} {I} & -1.6 & 1.2 \\ 
  {\mlpnode{NJ}\mbox{}} {J} & -0.9 & 1.2 \\ 
  {\mlpnode{NK}\mbox{}} {K} & -0.9 & -1.5 \\ 
  {\mlpnode{NL}\mbox{}} {L} & -0.7 & -0.0 \\ 
  {\mlpnode{NM}\mbox{}} {M} & -0.4 & -1.8 \\ 
  {\mlpnode{NN}\mbox{}} {N} & -0.5 & -0.2 \\ 
  {\mlpnode{NO}\mbox{}} {O} & -0.3 & -1.3 \\ 
  {\mlpnode{NP}\mbox{}} {P} & -0.1 & -0.2 \\ 
  {\mlpnode{NQ}\mbox{}} {Q} & -0.4 & -1.4 \\ 
  {\mlpnode{NR}\mbox{}} {R} & 0.1 & 1.1 \\ 
  {\mlpnode{NS}\mbox{}} {S} & 0.1 & 0.1 \\ 
  {\mlpnode{NT}\mbox{}} {T} & -0.4 & -0.2 \\ 
  {\mlpnode{NU}\mbox{}} {U} & 0.7 & 0.1 \\ 
  {\mlpnode{NV}\mbox{}} {V} & 0.4 & 1.3 \\ 
  {\mlpnode{NW}\mbox{}} {W} & -0.1 & 0.4 \\ 
  {\mlpnode{NX}\mbox{}} {X} & 0.9 & -0.2 \\ 
  {\mlpnode{NY}\mbox{}} {Y} & 1.7 & -1.4 \\ 
  {\mlpnode{NZ}\mbox{}} {Z} & 2.3 & 1.5 \\ 
   \hline
\end{tabular}}
\end{center}
\end{minipage}

\begin{tikzpicture}[overlay]
  \path[draw,red]<2-> (NA) edge (NI);
 \path[draw,red]<3-> (NA) edge (NJ);
 \path[draw,red]<4-> (NB) edge (NR);
 \path[draw,red]<5-> (NB) edge (NW);
 \path[draw,red]<6-> (NC) edge (NN);
 \path[draw,red]<7-> (NC) edge (NT);
 \path[draw,red]<8-> (ND) edge (NO);
 \path[draw,red]<9-> (ND) edge (NQ);
 \path[draw,red]<10-> (NE) edge (NS);
 \path[draw,red]<11-> (NE) edge (NV);
 \path[draw,red]<12-> (NF) edge (NU);
 \path[draw,red]<13-> (NF) edge (NX);
 \path[draw,red]<14-> (NG) edge (NM);
 \path[draw,red]<15-> (NG) edge (NZ);
 \end{tikzpicture}
\end{frame}

\begin{frame}[fragile]

\begin{minipage}[t]{2in}
\begin{center}
Existing site\\
{\small
% latex table generated in R 3.0.1 by xtable 1.7-1 package
% Wed Aug 14 15:26:57 2013
\begin{tabular}{lrr}
  \hline
 & z.date & z.cap \\ 
  \hline
A & -1.6 & {1.2} {\mlpnode{NA}} \\ 
  B & -0.9 & {1.2} {\mlpnode{NB}} \\ 
  C & -0.4 & {0} {\mlpnode{NC}} \\ 
  D & -0.4 & {-1.4} {\mlpnode{ND}} \\ 
  E & 0.1 & {1.1} {\mlpnode{NE}} \\ 
  F & 2.2 & {0} {\mlpnode{NF}} \\ 
  G & 1.3 & {0} {\mlpnode{NG}} \\ 
   \hline
\end{tabular}}
\end{center}
\bigskip

\textcolor{blue}{Optimal vs. Greedy matching}
\bigskip

{\footnotesize By evaluating potential matches all together rather than
  sequentially, optimal matching (\textcolor{blue}{blue lines}) reduces
  the mean matched distance from 1.17 to
 0.56.}

\end{minipage}
\begin{minipage}[t]{2in}
\begin{center}
New site\\
{\scriptsize
% latex table generated in R 3.0.1 by xtable 1.7-1 package
% Wed Aug 14 15:26:57 2013
\begin{tabular}{lrr}
  \hline
 & z.date & z.cap \\ 
  \hline
{\mlpnode{NH}\mbox{}} {H} & -0.3 & -0.7 \\ 
  {\mlpnode{NI}\mbox{}} {I} & -1.6 & 1.2 \\ 
  {\mlpnode{NJ}\mbox{}} {J} & -0.9 & 1.2 \\ 
  {\mlpnode{NK}\mbox{}} {K} & -0.9 & -1.5 \\ 
  {\mlpnode{NL}\mbox{}} {L} & -0.7 & -0.0 \\ 
  {\mlpnode{NM}\mbox{}} {M} & -0.4 & -1.8 \\ 
  {\mlpnode{NN}\mbox{}} {N} & -0.5 & -0.2 \\ 
  {\mlpnode{NO}\mbox{}} {O} & -0.3 & -1.3 \\ 
  {\mlpnode{NP}\mbox{}} {P} & -0.1 & -0.2 \\ 
  {\mlpnode{NQ}\mbox{}} {Q} & -0.4 & -1.4 \\ 
  {\mlpnode{NR}\mbox{}} {R} & 0.1 & 1.1 \\ 
  {\mlpnode{NS}\mbox{}} {S} & 0.1 & 0.1 \\ 
  {\mlpnode{NT}\mbox{}} {T} & -0.4 & -0.2 \\ 
  {\mlpnode{NU}\mbox{}} {U} & 0.7 & 0.1 \\ 
  {\mlpnode{NV}\mbox{}} {V} & 0.4 & 1.3 \\ 
  {\mlpnode{NW}\mbox{}} {W} & -0.1 & 0.4 \\ 
  {\mlpnode{NX}\mbox{}} {X} & 0.9 & -0.2 \\ 
  {\mlpnode{NY}\mbox{}} {Y} & 1.7 & -1.4 \\ 
  {\mlpnode{NZ}\mbox{}} {Z} & 2.3 & 1.5 \\ 
   \hline
\end{tabular}}
\end{center}
\end{minipage}

% \ncline[linecolor=gray]{NA}{NI}
% \ncline[linecolor=gray]{NA}{NJ}
% \ncline[linecolor=gray]{NB}{NL}
% \ncline[linecolor=gray]{NB}{NN}
% \ncline[linecolor=gray]{NC}{NP}
% \ncline[linecolor=gray]{NC}{NT}
% \ncline[linecolor=gray]{ND}{NM}
% \ncline[linecolor=gray]{ND}{NQ}
% \ncline[linecolor=gray]{NE}{NR}
% \ncline[linecolor=gray]{NE}{NS}
% \ncline[linecolor=gray]{NF}{NU}
% \ncline[linecolor=gray]{NF}{NX}
% \ncline[linecolor=gray]{NG}{NW}
% \ncline[linecolor=gray]{NG}{NY}

\begin{tikzpicture}[overlay]
  \path[draw,red] (NA) edge (NI);
 \path[draw,red] (NA) edge (NJ);
 \path[draw,red] (NB) edge (NR);
 \path[draw,red] (NB) edge (NW);
 \path[draw,red] (NC) edge (NN);
 \path[draw,red] (NC) edge (NT);
 \path[draw,red] (ND) edge (NO);
 \path[draw,red] (ND) edge (NQ);
 \path[draw,red] (NE) edge (NS);
 \path[draw,red] (NE) edge (NV);
 \path[draw,red] (NF) edge (NU);
 \path[draw,red] (NF) edge (NX);
 \path[draw,red] (NG) edge (NM);
 \path[draw,red] (NG) edge (NZ);
 \end{tikzpicture}\begin{tikzpicture}[overlay]
  \path[draw,blue] (NA) edge [out=0, in=-180] (NI);
 \path[draw,blue] (NA) edge [out=0, in=-180] (NJ);
 \path[draw,blue] (NB) edge [out=0, in=-180] (NL);
 \path[draw,blue] (NB) edge [out=0, in=-180] (NW);
 \path[draw,blue] (NC) edge [out=0, in=-180] (NN);
 \path[draw,blue] (NC) edge [out=0, in=-180] (NT);
 \path[draw,blue] (ND) edge [out=0, in=-180] (NO);
 \path[draw,blue] (ND) edge [out=0, in=-180] (NQ);
 \path[draw,blue] (NE) edge [out=0, in=-180] (NR);
 \path[draw,blue] (NE) edge [out=0, in=-180] (NV);
 \path[draw,blue] (NF) edge [out=0, in=-180] (NY);
 \path[draw,blue] (NF) edge [out=0, in=-180] (NZ);
 \path[draw,blue] (NG) edge [out=0, in=-180] (NU);
 \path[draw,blue] (NG) edge [out=0, in=-180] (NX);
 \end{tikzpicture}
% \ncarc[linecolor=blue]{NA}{NI}
% \ncarc[linecolor=blue]{NA}{NW}
% \ncarc[linecolor=blue]{NB}{NJ}
% \ncarc[linecolor=blue]{NB}{NL}
% \ncarc[linecolor=blue]{NC}{NN}
% \ncarc[linecolor=blue]{NC}{NT}
% \ncarc[linecolor=blue]{ND}{NM}
% \ncarc[linecolor=blue]{ND}{NQ}
% \ncarc[linecolor=blue]{NE}{NR}
% \ncarc[linecolor=blue]{NE}{NS}
% \ncarc[linecolor=blue]{NF}{NU}
% \ncarc[linecolor=blue]{NF}{NZ}
% \ncarc[linecolor=blue]{NG}{NX}
% \ncarc[linecolor=blue]{NG}{NY}

\end{frame}




\section{Matching structure and effective sample size}
\begin{frame}[fragile]
\frametitle{Evaluating matches with different structures}

Here are two optimal matches from the same sample.  How do they compare?

  \begin{columns}
    \column{.5\linewidth}%
{
  \begin{center}
{\usebeamercolor[fg]{titlelike}    Pair matching: }
  \end{center}
{\footnotesize
    \begin{semiverbatim}
> pm = pairmatch(my.ppty)
> summary(pm)
Structure of matched sets:
 1:1  0:1
 322 4103
Effective Sample Size:  322
(equivalent number of matched pairs).

sum(matched.distances)=16.9
(within 0.0464 of optimum).
Percentiles of matched distances:
    0%    50%    95%   100%
0.0000 0.0024 0.2710 0.6400
    \end{semiverbatim}
}
}
  \column{.5\linewidth}%
{
  \begin{center}
    {\usebeamercolor[fg]{titlelike} Matched triples:}
  \end{center}
{\footnotesize
\begin{semiverbatim}
> tm = pairmatch(my.ppty, 2)
> summary(tm)
Structure of matched sets:
 1:2  0:1
 322 3781
Effective Sample Size:  429
(equivalent number of matched pairs).

sum(matched.distances)=85.1
(within {0.458} of optimum).
Percentiles of matched distances:
   0%   50%   95%  100%
0.000 0.020 0.609 1.020
\end{semiverbatim}
}
}
\end{columns}

\note{
  \begin{itemize}%[<+-| alert@+>]
  \item With \atob{1}{2} matching rather than \atob{1}{1} matching, harder to get good matches.
  \item On the other hand, using more data may help with variance.
  \item Summary info helps to manage the bias-variance tradeoff.
  \item \textsc{Don't linger} on effective s.s. issue --- instead
    point to my PWH festschrift paper.  (If there's time/interest, there's more detail on it on the next slide.)
  \item covariate balance, not indicated here (yet), also speaks to bias --- more on this presently.
   \end{itemize}
}
\end{frame}

\begin{frame}
  \frametitle{Tracking effective sample size}

In 2-sample comparisons, total sample size can be a misleading as a measure of information content.  Example:
\begin{itemize}
\item say $Y$ has same variance, $\sigma^{2}$,in the Tx and the Ctl population.
\item Ben H. samples 10 Tx and 40 Ctls, and
\item Justin M. samples 25 Tx and 25 Ctls
\end{itemize}
--- so that total sample sizes are the same.  However,

\begin{eqnarray*}
  V_{BH}(\bar{y}_{t} - \bar{y}_{c}) &=& \frac{\sigma^{2}}{10} + \frac{\sigma^{2}}{40}=.125\sigma^{2}\mbox{;}\\
  V_{JM}(\bar{y}_{t} - \bar{y}_{c}) &=& \frac{\sigma^{2}}{25} + \frac{\sigma^{2}}{25}=.08\sigma^{2}.\\
\end{eqnarray*}

Similarly, a matched triple is roughly $[(\sigma^{2}/1 + \sigma^{2}/2)/(\sigma^{2}/1 + \sigma^{2}/1)]^{-1}= 1.33$ times as informative as a matched pair.

\note{Use pooled 2-sample t statistic SE formula to compare 1-1 vs 1-2 matched sets' contribution to variance:
$$
\begin{array}{c|c}
  \atob{1}{1} & \atob{1}{2} \\
M^{-2}\sum_{m=1}^{M} (\sigma^{2}/1 + \sigma^{2}/1) & M^{-2}\sum_{m=1}^{M} (\sigma^{2}/1 + \sigma^{2}/2) \\
\frac{2\sigma^{2}}{M} & \frac{1.5\sigma^{2}}{M} \\
\end{array}
$$
So 20 matched pairs is comparable to 15 matched triples.

(Correspondingly, h-mean of $n_{t},n_{c}$ for a pair is 1, while for a triple it's $[(1/1 + 1/2)/2]^{-1}=4/3$.)
}

\end{frame}

\note{
  {} [\textsc{Skip unless specifically requested}, instead referring
  interested readers to Hansen, 2011]\\
The variance of the \texttt{pr}-coeff in \texttt{v \textasciitilde pr + match} is
$$
 \frac{2 \sigma^{2}}{\sum_{s} h_{s}}, \hspace{3em} h_{s} = \left( \frac{n_{ts}^{-1} + n_{cs}^{-1} }{2}  \right)^{-1} ,
$$
assuming the OLS model and homoskedastic errors.  (This is b/c the anova formulation is equivalent to harmonic-mean weighting, under which $V(\sum_{s}w_{s}(\bar{v}_{ts} - \bar v_{cs})) = \sum_{s} w_{s}^{2}(n_{ts}^{-1} + n_{cs}^{-1}) \sigma^{2} = \sigma^{2} \sum_{s} w_{s}^{2} 2 h_{s}^{-1} = 2\sigma^{2} \sum_{s}w_{s}/\sum_{s}h_{s} = 2\sigma^{2}/\sum_{s} h_{s}$.)

For matched pairs, of course, $h_{s}=1$.  Harmonic mean of 1, 2 is $4/3$. Etc.   }


\begin{frame}
  \frametitle{Weighting matched sets by contributions to effective s.s.}


% \nocite{hansen:bowers:2008}
\enlargethispage*{1000pt}
\begin{minipage}[t]{2in}
\begin{center}
Existing site\\
{\scriptsize
% latex table generated in R 3.0.1 by xtable 1.7-1 package
% Wed Aug 14 15:26:54 2013
\begin{tabular}{lrr}
  \hline
 & date & capacity \\ 
  \hline
A & 2.3 & {660} {\mlpnode{NA}} \\ 
  B & 3.0 & {660} {\mlpnode{NB}} \\ 
  C & 3.4 & {420} {\mlpnode{NC}} \\ 
  D & 3.4 & {130} {\mlpnode{ND}} \\ 
  E & 3.9 & {650} {\mlpnode{NE}} \\ 
  F & 5.9 & {430} {\mlpnode{NF}} \\ 
  G & 5.1 & {420} {\mlpnode{NG}} \\ 
   \hline
\end{tabular}}
\end{center}
\bigskip
{\footnotesize
Differences in mean capacity\\
% latex table generated in R 3.0.1 by xtable 1.7-1 package
% Wed Aug 14 15:26:54 2013
\begin{tabular}{lrr}
  \hline
 & diff. & pair.equiv \\ 
  \hline
AI & 0 & 1.0 \\ 
  BJ & 0 & 1.0 \\ 
  CL & 0 & 1.0 \\ 
  DMQT & -58 & 1.5 \\ 
  ERSW & 110 & 1.5 \\ 
  FU & -17 & 1.0 \\ 
  GX & 35 & 1.0 \\ 
   \hline
\end{tabular}}
\end{minipage}
\begin{minipage}[t]{2in}
\begin{center}
New site\\
{\scriptsize
% latex table generated in R 3.0.1 by xtable 1.7-1 package
% Wed Aug 14 15:26:54 2013
\begin{tabular}{lrr}
  \hline
 & date & capacity \\ 
  \hline
{\mlpnode{NH}\mbox{}} {H} & 3.6 & 290 \\ 
  {\mlpnode{NI}\mbox{}} {I} & 2.3 & 660 \\ 
  {\mlpnode{NJ}\mbox{}} {J} & 3.0 & 660 \\ 
  {\mlpnode{NK}\mbox{}} {K} & 2.9 & 110 \\ 
  {\mlpnode{NL}\mbox{}} {L} & 3.2 & 420 \\ 
  {\mlpnode{NM}\mbox{}} {M} & 3.4 &  60 \\ 
  {\mlpnode{NN}\mbox{}} {N} & 3.3 & 390 \\ 
  {\mlpnode{NO}\mbox{}} {O} & 3.6 & 160 \\ 
  {\mlpnode{NP}\mbox{}} {P} & 3.8 & 390 \\ 
  {\mlpnode{NQ}\mbox{}} {Q} & 3.4 & 130 \\ 
  {\mlpnode{NR}\mbox{}} {R} & 3.9 & 650 \\ 
  {\mlpnode{NS}\mbox{}} {S} & 3.9 & 450 \\ 
  {\mlpnode{NT}\mbox{}} {T} & 3.4 & 380 \\ 
  {\mlpnode{NU}\mbox{}} {U} & 4.5 & 440 \\ 
  {\mlpnode{NV}\mbox{}} {V} & 4.2 & 690 \\ 
  {\mlpnode{NW}\mbox{}} {W} & 3.8 & 510 \\ 
  {\mlpnode{NX}\mbox{}} {X} & 4.7 & 390 \\ 
  {\mlpnode{NY}\mbox{}} {Y} & 5.4 & 140 \\ 
  {\mlpnode{NZ}\mbox{}} {Z} & 6.1 & 730 \\ 
   \hline
\end{tabular}}
\end{center}
%\hyperlink{computeNotesFr}{\beamergotobutton{Jump to ``Worksheet updates''}}
\end{minipage}
\begin{tikzpicture}[overlay]
  \path[draw,gray] (NA) edge (NI);
 \path[draw,gray] (NB) edge (NJ);
 \path[draw,gray] (NC) edge (NL);
 \path[draw,gray] (ND) edge (NM);
 \path[draw,gray] (ND) edge (NQ);
 \path[draw,gray] (ND) edge (NT);
 \path[draw,gray] (NE) edge (NR);
 \path[draw,gray] (NE) edge (NS);
 \path[draw,gray] (NE) edge (NW);
 \path[draw,gray] (NF) edge (NU);
 \path[draw,gray] (NG) edge (NX);
 \end{tikzpicture}
\end{frame}

\note[itemize]{
  \item {} \textsc{Don't linger.}  (Rather than detailed explanation
    below, default to a gesture back to previously presented material
    about h-weighting.)
  \item {}   The table at lower left indicates paired differences, or means of differences between a Tx and each control matched to it, along with weights attaching to matched sets.
  \item {}  Matching ``reduces'' the difference of group means from 80 MWe
($883-803$) to 14 MWe (sum of products of columns in prev table).
\item This is the same harmonic weighting, ``precision weighting,'' we spoke of earlier.
\item By contrast, using ETT weighting, it comes out to 10 MWe.
}

\section{Incorporating restrictions on who can be matched to whom}


\begin{frame}[label=incorpRestrictFr]
\frametitle{Incorporating restrictions on who can be matched to whom}


Placing restrictions on who can be matched to whom narrows the
range of possibilities and makes the algorithm run quicker\footnote{Computationally important in large problems,  $n_cn_t \geq 500,000$ or so.}.  Two
ways to do this:
\begin{itemize}[<+-| alert@+>]
\item Use categorical variables to divide the matching problem into two or more.
\item Use quantitative or ordinal variables to introduce calipers.
\end{itemize}

\end{frame}

% \begin{frame}
% \frametitle{Exact matching on categorical variables}

% \alert{Categorical variables:}  Matching within categories, or \alert{exact matching} is possible, but should be done with care.  The more stringent your criterion for an acceptable match, the smaller
% your effective sample size.  Consider {collapsing categories}, particularly rare ones.

% An alternative to exact matching on a variable is to match on a
% multivariate distance that includes it.
% \note{

% In contrast, by matching on a MV distance such as Mahalanobis, you permit some cross-category matches but ``incentivize'' matching within categories.

% Because Mahalanobis devotes extra attention to rare categories, when
% folding a categorical covariate into an MH distance one should still
% collapsing categories.  On the other hand, one can include in the
% propensity score, which doesn't generally suffer as acutely on this
% point.  }
% \end{frame}

% \begin{frame}
% \frametitle{Introducing restrictions on who can be matched to whom}
% \framesubtitle{Dividing the matching problem into two or more}

% In the UM space assignment study, once it was decided that matching
% was to be done within department, each department
% could be treated as a self-standing matching problem.
% Thus each of the eight departments listed below gives rise to its own
% matching problem.
% \begin{center}
%   \begin{tabular}{lrrrrrrrrr}
% \hline
%  & A & B & C & D & E & F & G & H & Overall \\
% \hline
% Women &   3 &   5 &   7 &   5 &   7 &   6 &   6 &   3 &  42 \\
% Men &  13 &  36 &  21 &  27 &  55 &  22 &  65 &  11 & 250 \\
% \hline
% \end{tabular}
% \end{center}

% \end{frame}

\begin{frame}[fragile]
  \frametitle{Matching within subclasses in R}

In \texttt{optmatch}, the decision to match within subclasses is made
prior to matching, so that distances need only be calculated within subclasses.

\begin{Schunk}
\begin{Sinput}
> em1 <- exactMatch(pr ~ pt, data=nuclearplants)
> pairmatch(pr ~ cap + date, data=nuclearplants, within=em1) 
> pairmatch(my.ppty, 
+          within=exactMatch(pr ~ pt, data=nuclearplants))
\end{Sinput}
\end{Schunk}
% \begin{semiverbatim}
% match\_on(pr \textasciitilde\ cap + date, data=nuclearplants, \\
% structure.fmla= \textasciitilde\ pt)

% match\_on(my.ppty.mod, data=nuclearplants, \\
% structure.fmla= \textasciitilde\ pt)
% \end{semiverbatim}

\begin{itemize}
\item (These generate Mahalanobis and propensity distances for all of the nuclear plants, not just the \texttt{pt==0} plants, after subclassifying the plants according to the value of \texttt{pt}.)
\item<+-> For a sense of how this affects the size of the matching
  problem, compare \texttt{with(nuclearplants, prod( table(pr) ) )}  to
  \texttt{num\_eligible\_matches(em1)}. 
\end{itemize}
\end{frame}
\note{
% NB: for the second \texttt{mdist()} call, ``pt'' has to be in the
% propensity score model.  (But that's a good idea anyway.)
}
%\mbox{}
\begin{frame}
\frametitle{Introducing restrictions on who can be matched to whom:
  calipers}
In the nuclear plants example, suppose we choose to insist upon a
  \alert{caliper} of 3 years in the date of construction.  This would
  forbid four potential matches, indicated below in \textcolor{red}{red}.



\addtolength{\tabcolsep}{-\tabcolsepadj} \vfill
\begin{center}
{\footnotesize
\begin{tabular}{|r|rrrrrrrrrrrrrrrrrrr|}
\hline
\multicolumn{2}{|l}{Exist-} & \multicolumn{18}{c|}{New sites} \\
\multicolumn{1}{|l|}{ing}& H & I & J & K & L & M & N & O & P & Q & R & S & T & U & V & W & X & Y & Z \\ \hline
A & 1.3 & 0.0 & 0.7 & 0.6 & 0.9 & 1.1 & 1.0 & 1.3 & 1.4 & 1.1 & 1.6 & 1.6 & 1.1 & 2.2 & 1.9 & 1.4 & 2.4 & \textcolor{red}{3.1} & \textcolor{red}{3.8} \\
  B & 0.6 & 0.7 & 0.0 & 0.1 & 0.2 & 0.4 & 0.3 & 0.6 & 0.8 & 0.4 & 0.9 & 0.9 & 0.4 & 1.5 & 1.2 & 0.8 & 1.7 & 2.5 & \textcolor{red}{3.1} \\
  C & 0.2 & 1.1 & 0.4 & 0.5 & 0.3 & 0.0 & 0.1 & 0.2 & 0.3 & 0.0 & 0.5 & 0.5 & 0.0 & 1.1 & 0.8 & 0.3 & 1.3 & 2.0 & 2.7 \\
  D & 0.2 & 1.1 & 0.4 & 0.5 & 0.3 & 0.0 & 0.1 & 0.2 & 0.3 & 0.0 & 0.5 & 0.5 & 0.0 & 1.1 & 0.8 & 0.3 & 1.3 & 2.0 & 2.7 \\
  E & 0.3 & 1.6 & 0.9 & 1.0 & 0.8 & 0.5 & 0.6 & 0.3 & 0.2 & 0.5 & 0.0 & 0.0 & 0.5 & 0.6 & 0.3 & 0.2 & 0.8 & 1.5 & 2.2 \\
  F & 2.4 & \textcolor{red}{3.7} & 3.0 & 3.0 & 2.8 & 2.5 & 2.6 & 2.4 & 2.2 & 2.5 & 2.0 & 2.0 & 2.5 & 1.4 & 1.8 & 2.2 & 1.3 & 0.5 & 0.2 \\
  G & 1.5 & 2.8 & 2.1 & 2.2 & 1.9 & 1.7 & 1.8 & 1.5 & 1.4 & 1.7 & 1.2 & 1.2 & 1.7 & 0.6 & 0.9 & 1.4 & 0.4 & 0.3 & 1.0 \\
\hline
\end{tabular}
}
\vfill \addtolength{\tabcolsep}{\tabcolsepadj}
\end{center}
\end{frame}


% \begin{frame}
% \frametitle{How closely should you match on the variables selected
% for adjustment?}

% \alert{Quantitative or ordinal variables and calipers.}  In the UM
% space assignment study, exploratory analysis suggested applying the
% log transformation ($x \mapsto \log (1 + x)$) to grant totals and
% imposing a \alert{caliper} of $1.0$ on the log of grant funding as PI.

% Quantitative variables can also be incorporated into a
% multivariate distance.

% \note{

% The space assignment study caliper requirement says \ldots.

%   Equivalently, no woman may be matched to a man whose PI grant
%   funding differs from hers either by being less than a tenth of it or
%   more than ten times it.


% Imposing a caliper differs from dividing the variable into categories
% and imposing exact matching on that category.
% }

% \end{frame}



%\mbox{}
\begin{frame}[fragile]
\frametitle{Coding a caliper when using \texttt{optmatch} and \texttt{R}}

To incorporate such a caliper, first create a distance measuring difference on year of construction\mbox{;} then use it to add a caliper:

\begin{Schunk}
\begin{Sinput}
> datecal <- match_on(pr~date, method="euclidean", 
+                     data=nuke.nopt)
> datecal <- caliper(datecal, width=3)
\end{Sinput}
\end{Schunk}
\enlargethispage*{550pt}
% {\small 
% (With optmatch version 0.7-3 or earlier, this requires the $1\times 1$ matrix
% \texttt{diag(1)} to have been assigned a ``dimnames'' attribute.) }
Taken together with \texttt{datecal}, the distance \texttt{match\_on(pr \textasciitilde
  cap + date, data=nuke.nopt)} becomes:

\addtolength{\tabcolsep}{-\tabcolsepadj} \vfill
\begin{center}
{\footnotesize
\begin{tabular}{|r|rrrrrrrrrrrrrrrrrrr|}
\hline
\multicolumn{2}{|l}{Exist-} & \multicolumn{18}{c|}{New sites} \\
\multicolumn{1}{|l}{ing}& H & I & J & K & L & M & N & O & P & Q & R & S & T & U & V & W & X & Y & Z \\ 
  \hline
A & 2.2 & 0.0 & 0.7 & 2.7 & 1.4 & 3.1 & 1.7 & 2.7 & 2.0 & 2.8 & 1.6 & 1.9 & 1.8 & 2.4 & 1.9 & 1.6 & 2.7 & \texttt{Inf} & \texttt{Inf} \\ 
  B & 1.9 & 0.7 & 0.0 & 2.6 & 1.2 & 2.9 & 1.3 & 2.5 & 1.5 & 2.6 & 0.9 & 1.4 & 1.4 & 1.9 & 1.2 & 1.1 & 2.2 & 3.5 & \texttt{Inf} \\ 
  C & 0.7 & 1.6 & 1.2 & 1.6 & 0.3 & 1.7 & 0.2 & 1.3 & 0.4 & 1.4 & 1.2 & 0.5 & 0.2 & 1.1 & 1.5 & 0.5 & 1.3 & 2.4 & 3.1 \\ 
  D & 0.8 & 2.8 & 2.6 & 0.5 & 1.4 & 0.3 & 1.3 & 0.2 & 1.3 & 0.0 & 2.5 & 1.6 & 1.2 & 1.9 & 2.8 & 1.9 & 1.8 & 2.0 & 3.9 \\ 
  E & 1.8 & 1.6 & 0.9 & 2.7 & 1.3 & 2.9 & 1.4 & 2.4 & 1.3 & 2.5 & 0.0 & 1.0 & 1.4 & 1.1 & 0.3 & 0.7 & 1.5 & 2.9 & 2.2 \\ 
  F & 2.5 & \texttt{Inf} & 3.2 & 3.4 & 2.8 & 3.1 & 2.6 & 2.7 & 2.2 & 2.9 & 2.3 & 2.0 & 2.6 & 1.4 & 2.2 & 2.2 & 1.3 & 1.5 & 1.4 \\ 
  G & 1.7 & 3.0 & 2.4 & 2.6 & 1.9 & 2.4 & 1.8 & 2.0 & 1.4 & 2.2 & 1.6 & 1.2 & 1.7 & 0.6 & 1.6 & 1.4 & 0.4 & 1.4 & 1.8 \\ 
   \hline
\end{tabular}
}
\vfill \addtolength{\tabcolsep}{\tabcolsepadj}
\end{center}

R code:
\texttt{pairmatch(pr \textasciitilde cap + date, within=datecal,  data=nuke.nopt)}

\end{frame}
\Note{
}


\end{document}

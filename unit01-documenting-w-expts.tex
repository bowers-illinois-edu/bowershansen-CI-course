%  For slides only
%%% Usage:
%% $ cat slidesonly.tex unitXX-foo.tex | pdflatex
%% $ pdfjam -o unitXX-slides{a/b/...}.pdf texput.pdf '1,{1stpage}-{lastpage}'

\documentclass{beamer} 
\newcommand{\tcw}{\textcolor{black}}
\newcommand{\mynoteonly}{}
\newcommand{\nottheirhandout}{handout:0}

% % For handout
%%% Usage:
%% $ cat handout.tex unitXX-foo.tex | pdflatex
%% $ pdfjam -o unitXX-handout{a/b/...}.pdf texput.pdf '{1stpage}-{lastpage}'
\documentclass[handout]{beamer}
\usepackage{pgfpages}
\pgfpagesuselayout{4 on 1}[letterpaper,border shrink=5mm]%\nofiles

\mode<handout>{\setbeamercolor{background canvas}{bg=black!5}}
\newcommand{\tcw}{\textcolor{structure.bg}}
\newcommand{\mynoteonly}{| handout:0}
\newcommand{\nottheirhandout}{handout:0}

%For handout + mynotes
%% Usage:
%% $ cat handout+mynotes.tex unitXX-foo.tex | pdflatex
%% $ pdfjam --nup 1x4 -o unitXX-foo-wmn{a/b/...}.pdf texput.pdf '1,{1stpage}-{lastpage}'
\documentclass[handout]{beamer}
\usepackage{pgfpages}
%%\pgfpagesuselayout{2 on 1}[letterpaper,border shrink=5mm] 
\setbeameroption{show notes on second screen=left}
\newcommand{\tcw}{\textcolor{black}}
\newcommand{\mynoteonly}{}
\newcommand{\nottheirhandout}{}

\newcommand{\igrphx}[2][width=\linewidth]{\includegraphics[#1]{images/#2}}

\renewcommand{\strut}{\rule{0pt}{3ex}}

% Frame note commands, with time budget updating
\newcounter{timeTotal}
\newcommand{\printUpdateTimeTotal}[1]{ \addtocounter{timeTotal}{#1}
  \textrm{(#1 min budgeted)} \hfill \textbf{Finish by
    \arabic{timeTotal}\ min from start}\\[1ex]}
\newcommand{\tnote}[3][10]{#2 \note{#2: #3 \\[.5ex]} \printUpdateTimeTotal{#1}}
\newcommand{\itnote}[2][10]{\note{ \begin{itemize} #2 \end{itemize}
\vfill \noindent  \printUpdateTimeTotal{#1} }}
\newcommand{\ennote}[2][10]{\note{ \begin{enumerate} #2 \end{enumerate}
\vfill \noindent  \printUpdateTimeTotal{#1} }}
\newcommand{\Note}[2][10]{\note{ #2 \mbox{ }\\ \vfill \noindent  \printUpdateTimeTotal{#1} }}

\newenvironment{Column}[1][.5\linewidth]{\begin{column}{#1}}{\end{column}}

\mode<handout>{\beamertemplatesolidbackgroundcolor{black!5}} 
\mode<article>{\usepackage{fullpage}}
\mode<presentation>
{
  \usetheme{Boadilla}
  % or ...

  \setbeamercovered{transparent}
  % or whatever (possibly just delete it)
}
\usepackage{url}
\usepackage{ulem}

\usepackage[english]{babel}
% or whatever

\usepackage[latin1]{inputenc}
% or whatever

\usepackage{times}
\usepackage[T1]{fontenc}
% Or whatever. Note that the encoding and the font should match. If T1
% does not look nice, try deleting the line with the fontenc.

\AtBeginSubsection[]
{
  \begin{frame}<beamer>
    \frametitle{Outline}
    \tableofcontents[subsectionstyle=show/shaded/hide]
  \end{frame}
}

\AtBeginSection[]
{
  \begin{frame}<beamer>
    \frametitle{Outline}
    \tableofcontents[hideothersubsections,sectionstyle=show/shaded]
  \end{frame}
}


% If you wish to uncover everything in a step-wise fashion, uncomment
% the following command: 

%\beamerdefaultoverlayspecification{<+->}

%USEFUL CODE TEMPLATES:
%\begin{itemize}[<+-| alert@+>]


% \begin{frame}[fragile]


% \begin{columns}
% \column{.4\textwidth}  
%   \begin{itemize}
%   \item<1-| alert@1> Why sample twice?
%   \item<2-| alert@2> Illustrative example.
%   \item<3-| alert@3> PSU's and SSU's.
%   \item<4-| alert@4> Stratification in combination with two-stage cluster sampling.
%   \item<5-| alert@5> Detailed example.
%   \end{itemize}
% \column{.6\textwidth}
% \only<2| handout:0>{
% \includegraphics[width=\textwidth]{nursingHomes}}
% \only<5| handout:1>{
% \includegraphics[width=\textwidth]{hosp_cover}}
% \end{columns}

% FOR INCLUDING R CODE. 
%\usepackage{listings}
%\lstset{language=R}
% \begin{frame}[fragile]
% \frametitle{Some code}
% \begin{lstlisting}
% > plot(myobj)
% > rm(myobj)
% \end{lstlisting}  
% \end{frame}

%\addtocounter{framenumber}{-1}


\usepackage{amsmath,amsthm}
\usepackage{wasysym,pifont}
\usepackage{Sweave}
\usepackage{ulem}
\usepackage{textcomp}
\usepackage{versions}
%% amsthm-type theorem environment specifications -- 
%% see amsthdoc.pdf in amscls documentation
\theoremstyle{plain}
\newtheorem{prop}{Proposition}[section]
\newtheorem{lem}[prop]{Lemma}

\newtheorem*{thm}{Proposition}
\newtheorem*{cor}{Corollary}

\theoremstyle{definition}
\newtheorem{defn}{Definition}[section]

\newcommand{\Pdistsym}{P}
\newcommand{\Pdistsymn}{P_n}
\newcommand{\Qdistsym}{Q}
\newcommand{\Qdistsymn}{Q_n}
\newcommand{\Qdistsymni}{Q_{n_i}}
\newcommand{\Qdistsymt}{Q[t]}
\newcommand{\dQdP}{\ensuremath{\frac{dQ}{dP}}}
\newcommand{\dQdPn}{\ensuremath{\frac{dQ_{n}}{dP_{n}}}}
\newcommand{\EE}{\ensuremath{\mathbf{E}}}
\newcommand{\EEp}{\ensuremath{\mathbf{E}_{P}}}
\newcommand{\EEpn}{\ensuremath{\mathbf{E}_{P_{n}}}}
\newcommand{\EEq}{\ensuremath{\mathbf{E}_{Q}}}
\newcommand{\EEqn}{\ensuremath{\mathbf{E}_{Q_{n}}}}
\newcommand{\EEqni}{\ensuremath{\mathbf{E}_{Q_{n[i]}}}}
\newcommand{\EEqt}{\ensuremath{\mathbf{E}_{Q[t]}}}
\newcommand{\PP}{\ensuremath{\mathbf{Pr}}}
\newcommand{\PPp}{\ensuremath{\mathbf{Pr}_{P}}}
\newcommand{\PPpn}{\ensuremath{\mathbf{Pr}_{P_{n}}}}
\newcommand{\PPq}{\ensuremath{\mathbf{Pr}_{Q}}}
\newcommand{\PPqn}{\ensuremath{\mathbf{Pr}_{Q_{n}}}}
\newcommand{\PPqt}{\ensuremath{\mathbf{Pr}_{Q[t]}}}
\newcommand{\var}{\ensuremath{\mathbf{V}}}
\newcommand{\varp}{\ensuremath{\mathbf{V}_{P}}}
\newcommand{\varpn}{\ensuremath{\mathbf{V}_{P_{n}}}}
\newcommand{\varq}{\ensuremath{\mathbf{V}_{Q}}}
\newcommand{\cov}{\ensuremath{\mathbf{Cov}}}
\newcommand{\covp}{\ensuremath{\mathbf{Cov}_{P}}}
\newcommand{\covpn}{\ensuremath{\mathbf{Cov}_{P_{n}}}}
\newcommand{\covq}{\ensuremath{\mathbf{Cov}_{Q}}}

\newcommand{\hatvar}{\ensuremath{\widehat{\mathrm{Var}}}}
\newcommand{\hatcov}{\ensuremath{\widehat{\mathrm{Cov}}}}

\newcommand{\sehat}{\ensuremath{\widehat{\mathrm{se}}}}

\newcommand{\combdiff}[1]{\ensuremath{\Delta_{{z}}[#1]}}
\newcommand{\Combdiff}[1]{\ensuremath{\Delta_{{Z}}[#1]}}

\newcommand{\psvec}{\ensuremath{\varphi}}
\newcommand{\psvecgc}{\ensuremath{\tilde{\varphi}}}


\newcommand{\atob}[2]{\ensuremath{#1\!\! :\!\! #2}}
\newcommand{\stratA}{\ensuremath{\mathbf{S}}}
\newcommand{\stratAnumstrat}{\ensuremath{S}}
\newcommand{\sAsi}{\ensuremath{s}}

\newcommand{\permsd}{\ensuremath{\sigma_{\Pdistsym}}}
% \newcommand{\dz}[1]{\ensuremath{d_{z}[{#1}]}}
\newcommand{\dZ}[1]{\ensuremath{d_{Z}[{#1}]}}
\newcommand{\tz}[1]{\ensuremath{t_{{z}}[#1]}}
\newcommand{\tZ}[1]{\ensuremath{t_{{Z}}[#1]}}


\newlength{\tabcolsepadj}
\setlength{\tabcolsepadj}{1.3mm}

%%% NEWBLOCK UNDEFINED BUG
\def\newblock{\hskip .11em plus .33em minus .07em}

%%% tightlist undefined control sequence bug 
%%% http://tex.stackexchange.com/questions/257418/error-tightlist-converting-md-file-into-pdf-using-pandoc
\providecommand{\tightlist}{%
  \setlength{\itemsep}{0pt}\setlength{\parskip}{0pt}}
\newcommand{\igrphx}[2][width=\linewidth]{\includegraphics[#1]{images/#2}}

\renewcommand{\strut}{\rule{0pt}{3ex}}

\newenvironment{Column}[1][.5\linewidth]{\begin{column}{#1}}{\end{column}}

\usepackage{tikz}
\usepackage{tikz-cd}
\usepackage{textpos}

\usetikzlibrary{arrows} % also see \tikzstyle spec below after \begin{doc}

\newcommand{\mlpnode}[1]{\tikz[baseline=-.5ex] \coordinate (#1) {};}
%\newcommand{\mlpnode}[1]{\raisebox{.5ex}{\pnode{#1}}}


\usepackage{xspace}

\includeversion{pedantic} % \excludeversion
\tikzstyle{every picture}+=[remember picture]


% copied from CSCAR svn repo, `dissdefs.tex`
%\usepackage{xspace}
\newcommand{\satm}{\mbox{\textsc{sat-m}}\xspace}
\newcommand{\satv}{\mbox{\textsc{sat-v}}\xspace}
\newcommand{\upm}{\mbox{\textsc{urm}}\xspace}
\newcommand{\asian}{\mbox{\textsc{asian}}\xspace}
\newcommand{\presatm}{\mbox{\textsc{pre-m}}\xspace}
\newcommand{\premath}{\mbox{\textsc{pre-m}}\xspace}
\newcommand{\presatv}{\mbox{\textsc{pre-v}}\xspace}
\newcommand{\preverb}{\mbox{\textsc{pre-v}}\xspace}
\newcommand{\parentsinc}{\mbox{\textsc{incm}}\xspace}
\newcommand{\gpa}{\mbox{\textsc{gpa}}\xspace}
\newcommand{\dadsed}{\mbox{\textsc{dadsed}}\xspace}
\newcommand{\momsed}{\mbox{\textsc{momsed}}\xspace}
\newcommand{\avgeng}{\mbox{\textsc{e-gpa}}\xspace}
\newcommand{\avgmath}{\mbox{\textsc{m-gpa}}\xspace}
\newcommand{\avgnatsci}{\mbox{\textsc{ns-gpa}}\xspace}
\newcommand{\avgssci}{\mbox{\textsc{ss-gpa}}\xspace}
\newcommand{\coach}{\mbox{\textsc{coach}}\xspace}
\newcommand{\dadcoll}{\mbox{\textsc{dadcoll}}\xspace}
\newcommand{\aavg}{\mbox{\textsc{a-avg}}\xspace}

\newcommand{\teeW}{\ensuremath{t_{W}}}
\newcommand{\pcor}{\ensuremath{\rho_{y \cdot w|z\mathbf{x}}}}





\date{ICPSR Session 2 (July 19, 2021)}


\title{Unit 1: Documenting causation with experiments}
% \autho moved to beamer-preamble-*-all.tex

\begin{document}


\begin{frame}[shrink]
    \frametitle{Outline \& Readings}

\tableofcontents[subsectionstyle=show/hide/hide]

{\usebeamercolor[fg]{titlelike} Announcements:}\\

\input{announcement-of-the-day}  % 
                                % announcement-of-the-day.tex not part of repo
\end{frame}

\section{Overview}

\begin{frame}[label=CIFr]{``Causal Inference''}

 \textit{Causal inference} is about design and analysis of experiments
 and observational studies so as to create fair comparisons that
 enable you to measure effects of interventions.


\end{frame}

% old (2015-) version of above frame:
% \begin{frame}[label=CIFr]{``Causal inference'' again}

% `Causal inference'' is about design and analysis of randomized and
% nonrandomized studies with goals:
% \begin{enumerate}
% \item Meaningful comparisons, ie comparisons that reveal causation (if present) simply and w/ minimal overhead;
% \item On scrutiny, EITHER indicated causation occurred OR \ldots
% \end{enumerate}


% (where a short, circumscribed list follows the ``OR'')

% \end{frame}


% \begin{frame}[label=whatWeWillCoverFr]{Methods we'll study in the course}
%   \begin{itemize}
%   \item Randomization-based analysis for experiments
%   \item Natural experiments, quasi-experiments,
%     difference-in-differences
%   \item Effect estimation \& inference in the presence of interference
%   \item Propensity scores, propensity score matching
% %  \item Regression discontinuity
%   \item Omitted variable sensitivity analysis
%   \item Instrumental variables \& principal stratification
%   \item \ldots
%   \end{itemize}
% \end{frame}

% \Note{Don't linger, I come back to this slide after Causal Inference
%   in Experiments section}

\begin{frame}[handout:0]{Three traditions in quantitative analysis}
\framesubtitle{A brief, tendentious review}

Modern causal inference in the 19th c., emerges over the 20th c, hits
it stride during this century. Dominant strains of quantitative
analysis in social science, by century:

\begin{itemize}
\item[19th] social physics
\item[20th] sampling of populations
\item[21st] causal inference (a prediction)
\end{itemize}

  
\end{frame}

\section{Causal inference in simple experiments}

\begin{frame}{Centrality of experiment to modern CI}
  \begin{itemize}
  \item Experiments are conceptually and practically central
  \item Not just any experiments, particular types, featuring
    \textit{control groups} and \textit{random assignment}\footnote{In
    this regard statisticians' concept of ``experiment'' is slightly
    idiosyncratic.  Compare Kinder \& Palfrey's (1993) definition to that of
    Rosenbaum (2010, ch 2.)}
\item Central characteristics of an ideal situation
  \item (Shares outward manifestations of Stats 101 stats ---
    randomness, p-values, ``bias,'' confidence levels etc --- but
    \textit{without} the ideas that our sample is sampled from some
    superpopulation, or a probabilistic data generating mechanism.
    At least, those ideas are optional.) 
  \item The ideal reveals its merits in a variety of disciplines over
    the 20th c.
  \end{itemize}
\end{frame}

\subsection{Testing hypotheses of no effect in randomized experiments}
\begin{frame}{Example 1a: Salk vaccine trial}
\framesubtitle{
   The version that is remembered \footnote{E.g.
      \href{http://www.cengage.com/resource_uploads/downloads/0534094929_46500.pdf}{Meier
        1972}, ``The biggest public health experiment ever\ldots''}, in statistics \& public health. }


\igrphx{SalkVtable-rctonly}

\end{frame}
\itnote{
\item Vaccines contentious
\item Geographic variation
\item year to year variation
\item randomized
\item double blinded
\item large, national sample
\item Exercise: Abstract a relevant 2x2 table
}

\begin{frame}{The hypothesis of absolutely no effect,  i}
\framesubtitle{Fisher's null \& a Fisherian test statistic for Salk trial}
  \begin{itemize}
  \item In Placebo group, 162 reported cases ($n_{0}=201K$).  Among
    Vaccinated, 82 of $n_{1}=201K$; rate = $4.1\times 10^{-4}$ 
   \item The hypothesis of absolutely no effect --- Fisher's (1935) $H_{0}$ --- says that however
     randomization had turned out, polio would have
     been reported for the same $162 +
     82 = 244$ children (and no others).
   \item Impossible?  No.  Improbable?  Let's see.  Under $H_{0}$:
     \begin{itemize}
     \item $\mathbf{y} = (0, \ldots, 0; 1, \ldots 1)$, w/ 244= 1's.
       Not random. 
     \item $\mathbf{Z} =$ random shuffle of $n_{0}$ 0s and $n_{1}$
      1s, where $(n_{0}; n_{1}) = (201,229 ; 200,745)$.  
     \item A \textit{random variable}  (r.v.) representing what the mean of
       the vaccine group might have been is 
       $n_{1}^{-1}\sum_{i=1}^{n} Z_{i}y_{i}$, $n = n_{0} + n_{1}$, or
       $n_{1}^{-1}\mathbf{Z}'\mathbf{y}$.
     \item Basic probability gives
       $\mathrm{E}_{0}(n_{1}^{-1}\mathbf{Z}'\mathbf{y}) = \bar y =
       244/402K \approx 6\times 10^{-4}$. 
     \end{itemize}
  \end{itemize}
  \end{frame}

  \begin{frame}{The hypothesis of absolutely no effect,  ii}
\framesubtitle{Fisherian ``s.e.'' \& large-sample $p$-value}

\begin{itemize}
\item \ldots we're in the midst of calculating aspects of the \textit{randomization
  distribution} of our \textit{test statistic}
$n_{1}^{-1}\mathbf{Z}'\mathbf{y}$, assuming $H_{0}$.  (For the
time being!) \pause We found
$\mathrm{E}_{0}(n_{1}^{-1}\mathbf{Z}'\mathbf{y}) \approx 6\times
10^{-4}$  Continuing, 
\begin{itemize}
     \item By statistical theory, 
       $\mathrm{Var}_{0}(n_{1}^{-1}\mathbf{Z}'\mathbf{y}) = n_{1}^{-1}
  \frac{n_{0}}{n} \frac{\sum_{i=1}^{n} (y_{i} - \bar
    y)^{2}}{n-1}$.
  \item (Counter to similar formulas from Stats 101, the sum and
    average are over all 400K participants, not just the vaccine
    group.) 
  \item This evaluates to
  $\left(4\times 10^{-5}\right)^{2}$.
     \item Stat. theory also suggests $n_{1}^{-1}\mathbf{Z}'\mathbf{y}$ is
       approximately Normal.
     \item With the \textit{realized} treatment assignment
       $\mathbf{z}$, we  observed $n_{1}^{-1}\mathbf{z}'\mathbf{y} =4 \times
       10^{-4}$.  That's 5 null s.e.'s less than
       $\mathrm{E}_{0}(n_{1}^{-1}\mathbf{Z}'\mathbf{y})$!
     \end{itemize}
   \item As \texttt{pnorm(-5)} is \texttt{2.87e-07}, we obtain an
     approximate p-value of $3 \times 10^{-7}$.
   \item (One can associate a p-value with these data + $H_{0}$ without the
     Normal approximation, as we'll see later.)
\end{itemize}
\end{frame}
% \itnote{
% \item[2016] at the
% blackboard I walked them through strict null for the Salk trial,
% tested it approximately using N-approx, $\mathrm{sd}_{0}(\hat p)
% \approx \sqrt{p_{0}q_{0}/n_{1}}$.  This seemed to work well.
% }

\begin{frame}{Potential outcomes}
  
  \begin{itemize}[<+->]
  \item The type of probability calculation we just saw originates
    with Fisher (1935). It entertains counter-to-fact
    \textit{assignments} of treatment conditions to experimental
    subjects, holding responses fixed at their observed values.
  \item Neyman (1923) and Rubin (1970s) also posited
    unobserved, counter-to-fact \textit{potential outcomes} for each
    $i$, bringing mathematical precision to ``treatment effect''.
   \item Rosenbaum's \textit{DOS} covers this material in detail
       (Ch.2, ``Causal inference in randomized experiments''). And we'll
       see more of it as we go.
   \item For now, let's ask Fisher's question --- What evidence does
     the experiment provide \textit{against} the notion that the
     treatment is entirely without effect --- in a few different contexts.
  \end{itemize}

\end{frame}
\note{Fisher tea-tasting now}

\begin{frame}{Exercises}

  \begin{columns}
    \begin{Column}[.35\linewidth]
%  \begin{itemize}
    %\item
      Data at right are daily emergency-room admissions, from a paired
    ``quasi-experiment.'' Cols = Control, Treatment. 
  %\item
% \end{itemize}      
    \end{Column}
    \begin{Column}[.65\linewidth]
      \igrphx{scalonetal93-tab3-small}
    \end{Column}
  \end{columns}  

    To analyze as if tosses of a fair coin determined which days
    were C and T, you can simulate  hypothetical random assignments
    (of ``unlucky Friday'' status).
  
  \begin{enumerate}
   \item Using a coin from your pocket, simulate such an assignment for
    each of the 7 months.  Calculate columnwise means; record the
    difference of means.
  \item Pool findings and repeat as needed, until you have 3-5 simulated
    differences of means. 
  \item Compare actual actual result to simulations: stand out, or fit
    right in?
  \end{enumerate}
  
\end{frame}
\end{document}
\subsection{Conceptual framework}
\begin{frame}{Goals of experimental design}

Experiment design aims to create a situation in which causation is
readily detected. If the putative causes
really do have their intended effects, you'll tend to see it; if they
don't, you'll tend not to.
\pause

\begin{itemize}[<+->]
\item \textrm{Random
    assignment} may not be necessary, but it sure helps.
  This week and next will theorize this point along lines
  of Fisher, Neyman and their current-day followers.
\item In comparative assessments of study designs, random assignment also benefits
  from its association with other design characteristics of
  high-quality studies:  distinctions among covariates, treatments and
  outcomes; comparability of treatment and control groups; \ldots;
  formal study planning (Rosenbaum, \textit{DOS}, ch. 1.)
\end{itemize}

\end{frame}

\begin{frame}{Example 2: overconfidence \& theories of intelligence}

\igrphx{ehrlingeretal16-titlebanner}

\end{frame}

\begin{frame}[handout:0]{Example 2: overconfidence \& theories of intelligence}

\igrphx{ehrlingeretal16sec2para1}

\end{frame}

\begin{frame}[handout:0]{Example 2: overconfidence \& theories of intelligence}

\begin{tabular}{c}
  \igrphx{ehrlingeretal16-4-1-2-0}\\
\igrphx{ehrlingeretal16-4-1-2-1}\\
\end{tabular}

\end{frame}

\subsection{The randomization distribution }

\begin{frame}{Fisher's tea-tasting experiment}
  \framesubtitle{The data}

The observed results of the experiment --- Fisher and the lady's, not
ours --- can be summarized as follows:
  
\begin{center}
    \begin{tabular}{l|l|ll|}
    Unit & $\mathbf{z}$ & $\mathbf{y_c}$ & $\mathbf{y_t}$ \\ \hline
    1    & 1          & ?                & 1 \\
    2    & 1          & ?                & 1  \\
    3    & 1          & ?                & 1  \\
    4    & 1          & ?                & 1  \\
    5    & 0          & 0                & ?  \\
    6    & 0          & 0                & ?  \\
    7    & 0          & 0                & ?  \\
    8    & 0          & 0                & ?  \\
    \end{tabular}
\end{center}  

(This summary involves potential outcomes, which were Neyman's idea
not Fisher's.)
\end{frame}


\begin{frame}{Fisher's tea-tasting experiment}

  \framesubtitle{An abstract representation of the states of the world
    it aims to distinguish}

\begin{columns}
    \begin{Column}
      \usebeamerfont{title}\usebeamercolor{title}{\color{fg} No effect}

      \begin{tabular}{l|l|l}
    Unit & $\mathbf{y}_c$ & $\mathbf{y}_t$ \\ \hline
    1    & 1                & 1 \\
    2    & 1                & 1  \\
    3    & 1                & 1  \\
    4    & 1                & 1  \\
    5    & 0                & 0  \\
    6    & 0                & 0  \\
    7    & 0                & 0  \\
    8    & 0                & 0  \\
    \end{tabular}      
    \end{Column}
    \begin{Column}
\usebeamerfont{title}\usebeamercolor{title}{\color{fg} Perfect
  discrimination}

\begin{tabular}{l|l|l}
    Unit & $\mathbf{y}_c$ & $\mathbf{y}_t$ \\ \hline
    1    & 0                & 1 \\
    2    & 0                & 1  \\
    3    & 0                & 1  \\
    4    & 0                & 1  \\
    5    & 0                & 1  \\
    6    & 0                & 1  \\
    7    & 0                & 1  \\
    8    & 0                & 1  \\
    \end{tabular}
    \end{Column}
\end{columns}
\end{frame}

\begin{frame}{Fisher's tea-tasting experiment}
  \framesubtitle{Probabilities of assignment}
  \begin{itemize}
  \item In the experiment, $\mathbf{Z} = (Z_{1}, Z_{2}, \ldots,
    Z_{8})$ is selected at random from ${8 \choose 4} = 70$ candidates
    (see \texttt{day2-fishercombs.pdf}).
  \item \textit{Temporarily assuming} the null hypothesis of no effect to
    be true, only one of these has $\mathbf{Z}'\mathbf{y}$ as large as
    4. So, $\PP_{0}(\mathbf{Z}'\mathbf{y} \geq 4) = 1/70 = .014$.
  \item Under the alternative hypothesis of perfect discrimination,  $\PP_{0}(\mathbf{Z}'\mathbf{y} \geq 4) = 1$.
  \item Fisher's conceptualization at its best w/ a single sharply
    distinct alternative hypothesis. (Despite ``alternative hypothesis'' being Neyman's notion, not
    Fisher's.)  To adapt it to questions of degree, we'll blend in
    other concepts of Neyman.  
  \end{itemize}
\end{frame}

\section{Observational studies}


\begin{frame}<1-3>[label=quasivstrueFr]{Nonrandomized studies as
    ``experiments''?}
  \begin{itemize}[<+->]
  \item Controlled experiments vs found ``experiments''
  \item Found randomization as a hypothesis
  \item Theoretical vs empirical scrutiny of hypothesis
  \item Informal empirical scrutiny
  \item Formal hypothesis testing (instructor agenda alert!)
  \end{itemize}
\end{frame}



\begin{frame}{A paired observational study}
  
  \begin{columns}
    \begin{Column}
  \begin{itemize}
  \item Example: Vehicular accidents on Friday the
      13ths\footnotemark
  \item An \textit{observational} paired comparison.  (Rosenbaum's
    \textit{DOS} starts w/
    a randomized paired study.)
  \item (Some refer to this study design as ``fixed effects.'')
  \end{itemize}      
    \end{Column}
    \begin{Column}
      \igrphx{scalonetal93-tab3}
    \end{Column}
  \end{columns}
  \pause
  
  \begin{itemize}
  \item
  Difficulty of ``randomizing'' Fri. the 13th makes analogy to an
  experiment slightly odd. That feeling may go away.  If not, it may
  points to a genuine weakness of the study (cf. \textit{DOS} ch. 1)\ldots    
  \end{itemize}
  \footnotetext{T.J. Scanlon et al. (1993), ``Is Friday the 13th
        Bad For Your Health?'', BMJ 307 1584--86.}
\end{frame}

\itnote{
\item Numbers of emergency admissions to a UK hospital due to
  vehicular accidents, on consecutive fridays the second of which is a
  Friday the 13th. 
\item (\S~2.1--2.2.1 of \textit{DOS} creates a small paired expt from
  the NSW data.)
\item It's a weakness of the study that the treatment is not sharply
  defined and potentially applicable to any day.  Accordingly,
  e.g. high reports of anxiety on Thurs 12ths would be difficult to
  categorize as confounding variables or treatment effects.  
\item Sharper conceptualization of the study can sometimes help.
  Orienting it on the timeline helps as well.  
}

\begin{frame}<1>[label=timelineFr]{Timeline, with entry points for biases}
  % commands I'll want to renew, to get slides right in different versions
  \newcommand{\selpt}{ }%{{\fbox{\strut \hspace{2em} \strut}}}
  \newcommand{\biasrow}{}%{ Biases: &   & & & & \\ }

  \begin{center}
    \begin{tabular}{lp{.15\linewidth}p{.15\linewidth}p{.15\linewidth}p{.15\linewidth}p{.15\linewidth}}
      & \multicolumn{2}{c}{\underline{$T< 0$}} & \underline{$T=0$} & \multicolumn{2}{c|}{\underline{$T>
        0$}} \\
& subjects identified & subjects recruited & treatment assigned &
treatment delivered & outcomes  measured\\ \hline
& \fbox{Healthy}   &            &          & & \\
 & \fbox{Sick}        &  \selpt      &     & & \\
 &                           &            &   \fbox{T}  & &  \\
 & \fbox{Urban}       &            &               & \fbox{T} & \\
 & \fbox{Rural}  &  \selpt      &           &  & \\
 &                            &            &                 & \fbox{C} & \\
 & \fbox{Rich}         &            &    \fbox{C} &  & \\
 & \fbox{Poor}        &  \selpt      &             &  & \\
 &   (etc)                        &            &                &  \\
\biasrow% T unlike C at beginning of study,
                        %sample not rep of pop,
                        % differential uptake, placebo effects,
                        % differential attrition
   \end{tabular}
  \end{center}
\end{frame}
\itnote{
% \item[W2016]  Having assigned the coffee/mortality study, assess it
% against this timeline.
\item How sharply defined are the 3 epochs?
\item what about comparability of groups?
\item Potential for unmeasured variables, in light of the Lawlor et al
findings as summarized by Rosenbaum, \textit{DOS} ch1?}


\begin{frame}{Aside: Potential outcomes for in the Fri the 13th study}

  \begin{center}
    \begin{tabular}{ll|l|r|l|l|r|l|l}
\hline
$i$ &year & month & Fri6th & $y_{i1C}$ & $y_{i1T}$ & Fri13th & $y_{i2C}$ & $y_{i2T}$\\
\hline
1 & '89 & Oct. & 9 & 9  & ? & 13 & ? & 13  \\
\hline                                  
2 & '90 & Jul. & 6 & 6  & ? & 12 & ? & 12  \\
\hline                                  
3 & '91 & Sep. & 11 & 11 & ? & 14 & ? & 14  \\
\hline                                  
4 & '91 & Dec. & 11 & 11 & ? & 10 & ? & 10  \\
\hline                                  
5 & '92 & Mar. & 3 & 3  & ? & 4 &  ? &4\\
\hline                                  
6 & '92 & Nov. & 5 & 5  & ? & 12 & ? & 12  \\
\hline
\end{tabular}

\end{center}

\begin{itemize}
\item Potential outcomes similarly defined in other experiments and
  observational studies with paired structure.  
\item (We're following Rosenbaum's (2010, ch2) subscript notation.)
\item  The structure of the ``experiment'' precludes observing $y_{i1T}$
 and $y_{i2T}$ simultaneously, but we assume both to exist and permit
 them to be distinct.
\end{itemize}  
\end{frame}


\begin{frame}{Example 1b: Salk vaccine trial (the other one)}

\igrphx{SalkVtable-full}
\end{frame}
\itnote{
\item Explain differences
\item both versions are experiments in the everyday sense
\item both control for age and physical locale
\item large size, presence of these controls lends legitimacy to both
\item In this case I maintain that markers of good design are an
  illusion, it's an overtly unfair comparison.  Why isn't this
  apples-to-apples?
\item Exercises: abstract $2\times 2$ table. \textrm{Think-pair-share}
\item Demonstration of advantage of random assigment per se
\item We'll see how to get the ``right answer'', despite shortcomings.  The move is
  primarily conceptual, not technical, having to do w/ IVs.
}




\againframe{timelineFr}
\itnote{
\item Compare "Ohio State" vaccine study, Fri 13th study against this
  timeline
\item Time permitting start on unit01-Rex.pdf }
\end{document}
\section*{Closing remarks}

%\againframe{CIFr}
 
\begin{frame}{Course prerequisites}
  \begin{enumerate}
\item experience using a command-based computer program (as opposed to a menu-driven one)
\item willingness to try on attitudes suitable to
  non-commercial software: expect to err; trust but validate.
\item willingness to learn R  (if it's new to you, consider taking the course)
\item willingness to learn markdown (you can do this on your own)
\item Conditional probability and conditional expectation: prior
  exposure, or openness to self-study
  \item mathematical proofs, enough to distinguish correct from incorrect


  \end{enumerate}
\end{frame}



\end{document}

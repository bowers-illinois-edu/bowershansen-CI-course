%  For slides only
%%% Usage:
%% $ cat slidesonly.tex unitXX-foo.tex | pdflatex
%% $ pdfjam -o unitXX-slides{a/b/...}.pdf texput.pdf '1,{1stpage}-{lastpage}'

\documentclass{beamer} 
\newcommand{\tcw}{\textcolor{black}}
\newcommand{\mynoteonly}{}
\newcommand{\nottheirhandout}{handout:0}

% % For handout
%% Usage:
%% $ cat handout.tex unitXX-foo.tex | pdflatex
%% $ pdfjam -o unitXX-handout{a/b/...}.pdf texput.pdf '{1stpage}-{lastpage}'
\documentclass[handout]{beamer}
\usepackage{pgfpages}
\pgfpagesuselayout{4 on 1}[letterpaper,border shrink=5mm]%\nofiles

\mode<handout>{\setbeamercolor{background canvas}{bg=black!5}}
\newcommand{\tcw}{\textcolor{structure.bg}}
\newcommand{\mynoteonly}{| handout:0}
\newcommand{\nottheirhandout}{handout:0}

%For handout + mynotes
%%% Usage:
%% $ cat handout+mynotes.tex unitXX-foo.tex | pdflatex
%% $ pdfjam --nup 1x4 -o unitXX-foo-wmn{a/b/...}.pdf texput.pdf '1,{1stpage}-{lastpage}'
\documentclass[handout]{beamer}
\usepackage{pgfpages}
%%\pgfpagesuselayout{2 on 1}[letterpaper,border shrink=5mm] 
\setbeameroption{show notes on second screen=left}
\newcommand{\tcw}{\textcolor{black}}
\newcommand{\mynoteonly}{}
\newcommand{\nottheirhandout}{}

\newcommand{\igrphx}[2][width=\linewidth]{\includegraphics[#1]{images/#2}}

\renewcommand{\strut}{\rule{0pt}{3ex}}

% Frame note commands, with time budget updating
\newcounter{timeTotal}
\newcommand{\printUpdateTimeTotal}[1]{ \addtocounter{timeTotal}{#1}
  \textrm{(#1 min budgeted)} \hfill \textbf{Finish by
    \arabic{timeTotal}\ min from start}\\[1ex]}
\newcommand{\tnote}[3][10]{#2 \note{#2: #3 \\[.5ex]} \printUpdateTimeTotal{#1}}
\newcommand{\itnote}[2][10]{\note{ \begin{itemize} #2 \end{itemize}
\vfill \noindent  \printUpdateTimeTotal{#1} }}
\newcommand{\ennote}[2][10]{\note{ \begin{enumerate} #2 \end{enumerate}
\vfill \noindent  \printUpdateTimeTotal{#1} }}
\newcommand{\Note}[2][10]{\note{ #2 \mbox{ }\\ \vfill \noindent  \printUpdateTimeTotal{#1} }}

\newenvironment{Column}[1][.5\linewidth]{\begin{column}{#1}}{\end{column}}

\mode<handout>{\beamertemplatesolidbackgroundcolor{black!5}} 
\mode<article>{\usepackage{fullpage}}
\mode<presentation>
{
  \usetheme{Boadilla}
  % or ...

  \setbeamercovered{transparent}
  % or whatever (possibly just delete it)
}
\usepackage{url}
\usepackage{ulem}

\usepackage[english]{babel}
% or whatever

\usepackage[latin1]{inputenc}
% or whatever

\usepackage{times}
\usepackage[T1]{fontenc}
% Or whatever. Note that the encoding and the font should match. If T1
% does not look nice, try deleting the line with the fontenc.

\AtBeginSubsection[]
{
  \begin{frame}<beamer>
    \frametitle{Outline}
    \tableofcontents[subsectionstyle=show/shaded/hide]
  \end{frame}
}

\AtBeginSection[]
{
  \begin{frame}<beamer>
    \frametitle{Outline}
    \tableofcontents[hideothersubsections,sectionstyle=show/shaded]
  \end{frame}
}


% If you wish to uncover everything in a step-wise fashion, uncomment
% the following command: 

%\beamerdefaultoverlayspecification{<+->}

%USEFUL CODE TEMPLATES:
%\begin{itemize}[<+-| alert@+>]


% \begin{frame}[fragile]


% \begin{columns}
% \column{.4\textwidth}  
%   \begin{itemize}
%   \item<1-| alert@1> Why sample twice?
%   \item<2-| alert@2> Illustrative example.
%   \item<3-| alert@3> PSU's and SSU's.
%   \item<4-| alert@4> Stratification in combination with two-stage cluster sampling.
%   \item<5-| alert@5> Detailed example.
%   \end{itemize}
% \column{.6\textwidth}
% \only<2| handout:0>{
% \includegraphics[width=\textwidth]{nursingHomes}}
% \only<5| handout:1>{
% \includegraphics[width=\textwidth]{hosp_cover}}
% \end{columns}

% FOR INCLUDING R CODE. 
%\usepackage{listings}
%\lstset{language=R}
% \begin{frame}[fragile]
% \frametitle{Some code}
% \begin{lstlisting}
% > plot(myobj)
% > rm(myobj)
% \end{lstlisting}  
% \end{frame}

%\addtocounter{framenumber}{-1}


\usepackage{amsmath,amsthm}
\usepackage{wasysym,pifont}
\usepackage{Sweave}
\usepackage{ulem}
\usepackage{textcomp}
\usepackage{versions}
%% amsthm-type theorem environment specifications -- 
%% see amsthdoc.pdf in amscls documentation
\theoremstyle{plain}
\newtheorem{prop}{Proposition}[section]
\newtheorem{lem}[prop]{Lemma}

\newtheorem*{thm}{Proposition}
\newtheorem*{cor}{Corollary}

\theoremstyle{definition}
\newtheorem{defn}{Definition}[section]

\newcommand{\Pdistsym}{P}
\newcommand{\Pdistsymn}{P_n}
\newcommand{\Qdistsym}{Q}
\newcommand{\Qdistsymn}{Q_n}
\newcommand{\Qdistsymni}{Q_{n_i}}
\newcommand{\Qdistsymt}{Q[t]}
\newcommand{\dQdP}{\ensuremath{\frac{dQ}{dP}}}
\newcommand{\dQdPn}{\ensuremath{\frac{dQ_{n}}{dP_{n}}}}
\newcommand{\EE}{\ensuremath{\mathbf{E}}}
\newcommand{\EEp}{\ensuremath{\mathbf{E}_{P}}}
\newcommand{\EEpn}{\ensuremath{\mathbf{E}_{P_{n}}}}
\newcommand{\EEq}{\ensuremath{\mathbf{E}_{Q}}}
\newcommand{\EEqn}{\ensuremath{\mathbf{E}_{Q_{n}}}}
\newcommand{\EEqni}{\ensuremath{\mathbf{E}_{Q_{n[i]}}}}
\newcommand{\EEqt}{\ensuremath{\mathbf{E}_{Q[t]}}}
\newcommand{\PP}{\ensuremath{\mathbf{Pr}}}
\newcommand{\PPp}{\ensuremath{\mathbf{Pr}_{P}}}
\newcommand{\PPpn}{\ensuremath{\mathbf{Pr}_{P_{n}}}}
\newcommand{\PPq}{\ensuremath{\mathbf{Pr}_{Q}}}
\newcommand{\PPqn}{\ensuremath{\mathbf{Pr}_{Q_{n}}}}
\newcommand{\PPqt}{\ensuremath{\mathbf{Pr}_{Q[t]}}}
\newcommand{\var}{\ensuremath{\mathbf{V}}}
\newcommand{\varp}{\ensuremath{\mathbf{V}_{P}}}
\newcommand{\varpn}{\ensuremath{\mathbf{V}_{P_{n}}}}
\newcommand{\varq}{\ensuremath{\mathbf{V}_{Q}}}
\newcommand{\cov}{\ensuremath{\mathbf{Cov}}}
\newcommand{\covp}{\ensuremath{\mathbf{Cov}_{P}}}
\newcommand{\covpn}{\ensuremath{\mathbf{Cov}_{P_{n}}}}
\newcommand{\covq}{\ensuremath{\mathbf{Cov}_{Q}}}

\newcommand{\hatvar}{\ensuremath{\widehat{\mathrm{Var}}}}
\newcommand{\hatcov}{\ensuremath{\widehat{\mathrm{Cov}}}}

\newcommand{\sehat}{\ensuremath{\widehat{\mathrm{se}}}}

\newcommand{\combdiff}[1]{\ensuremath{\Delta_{{z}}[#1]}}
\newcommand{\Combdiff}[1]{\ensuremath{\Delta_{{Z}}[#1]}}

\newcommand{\psvec}{\ensuremath{\varphi}}
\newcommand{\psvecgc}{\ensuremath{\tilde{\varphi}}}


\newcommand{\atob}[2]{\ensuremath{#1\!\! :\!\! #2}}
\newcommand{\stratA}{\ensuremath{\mathbf{S}}}
\newcommand{\stratAnumstrat}{\ensuremath{S}}
\newcommand{\sAsi}{\ensuremath{s}}

\newcommand{\permsd}{\ensuremath{\sigma_{\Pdistsym}}}
% \newcommand{\dz}[1]{\ensuremath{d_{z}[{#1}]}}
\newcommand{\dZ}[1]{\ensuremath{d_{Z}[{#1}]}}
\newcommand{\tz}[1]{\ensuremath{t_{{z}}[#1]}}
\newcommand{\tZ}[1]{\ensuremath{t_{{Z}}[#1]}}


\newlength{\tabcolsepadj}
\setlength{\tabcolsepadj}{1.3mm}

%%% NEWBLOCK UNDEFINED BUG
\def\newblock{\hskip .11em plus .33em minus .07em}

%%% tightlist undefined control sequence bug 
%%% http://tex.stackexchange.com/questions/257418/error-tightlist-converting-md-file-into-pdf-using-pandoc
\providecommand{\tightlist}{%
  \setlength{\itemsep}{0pt}\setlength{\parskip}{0pt}}
\newcommand{\igrphx}[2][width=\linewidth]{\includegraphics[#1]{images/#2}}

\renewcommand{\strut}{\rule{0pt}{3ex}}

\newenvironment{Column}[1][.5\linewidth]{\begin{column}{#1}}{\end{column}}

\usepackage{tikz}
\usepackage{tikz-cd}
\usepackage{textpos}

\usetikzlibrary{arrows} % also see \tikzstyle spec below after \begin{doc}

\newcommand{\mlpnode}[1]{\tikz[baseline=-.5ex] \coordinate (#1) {};}
%\newcommand{\mlpnode}[1]{\raisebox{.5ex}{\pnode{#1}}}


\usepackage{xspace}

\includeversion{pedantic} % \excludeversion
\tikzstyle{every picture}+=[remember picture]


% copied from CSCAR svn repo, `dissdefs.tex`
%\usepackage{xspace}
\newcommand{\satm}{\mbox{\textsc{sat-m}}\xspace}
\newcommand{\satv}{\mbox{\textsc{sat-v}}\xspace}
\newcommand{\upm}{\mbox{\textsc{urm}}\xspace}
\newcommand{\asian}{\mbox{\textsc{asian}}\xspace}
\newcommand{\presatm}{\mbox{\textsc{pre-m}}\xspace}
\newcommand{\premath}{\mbox{\textsc{pre-m}}\xspace}
\newcommand{\presatv}{\mbox{\textsc{pre-v}}\xspace}
\newcommand{\preverb}{\mbox{\textsc{pre-v}}\xspace}
\newcommand{\parentsinc}{\mbox{\textsc{incm}}\xspace}
\newcommand{\gpa}{\mbox{\textsc{gpa}}\xspace}
\newcommand{\dadsed}{\mbox{\textsc{dadsed}}\xspace}
\newcommand{\momsed}{\mbox{\textsc{momsed}}\xspace}
\newcommand{\avgeng}{\mbox{\textsc{e-gpa}}\xspace}
\newcommand{\avgmath}{\mbox{\textsc{m-gpa}}\xspace}
\newcommand{\avgnatsci}{\mbox{\textsc{ns-gpa}}\xspace}
\newcommand{\avgssci}{\mbox{\textsc{ss-gpa}}\xspace}
\newcommand{\coach}{\mbox{\textsc{coach}}\xspace}
\newcommand{\dadcoll}{\mbox{\textsc{dadcoll}}\xspace}
\newcommand{\aavg}{\mbox{\textsc{a-avg}}\xspace}

\newcommand{\teeW}{\ensuremath{t_{W}}}
\newcommand{\pcor}{\ensuremath{\rho_{y \cdot w|z\mathbf{x}}}}





\date{ICPSR Session 2 (July 19, 2021)}


\title{Unit 1: Documenting causation with experiments}
% \autho moved to beamer-preamble-*-all.tex

\begin{document}


\begin{frame}[shrink]
    \frametitle{Outline \& Readings}

\tableofcontents[subsectionstyle=show/hide/hide]

{\usebeamercolor[fg]{titlelike} Announcements:}\\

  \input{announcement-of-the-day}  

\end{frame}
\itnote{
\item Day 2: Starting in part 2, I'll demo use of R to do some Fisher test calcs for the Salk polio trial. 
\item Then transition into them trying out some of these calculations in R
\item Then we'll look at observational studies
}

\section{Overview}

\begin{frame}[label=CIFr]{``Causal Inference''}

 \textit{Causal inference} is about design and analysis of experiments
 and observational studies so as to create fair comparisons that
 enable you to measure effects of interventions.


\end{frame}

% old (2015-) version of above frame:
% \begin{frame}[label=CIFr]{``Causal inference'' again}

% `Causal inference'' is about design and analysis of randomized and
% nonrandomized studies with goals:
% \begin{enumerate}
% \item Meaningful comparisons, ie comparisons that reveal causation (if present) simply and w/ minimal overhead;
% \item On scrutiny, EITHER indicated causation occurred OR \ldots
% \end{enumerate}


% (where a short, circumscribed list follows the ``OR'')

% \end{frame}


\begin{frame}[label=whatWeWillCoverFr]{Methods we'll study in the course}
  \begin{itemize}
  \item Randomization-based analysis for experiments
  \item Natural experiments, quasi-experiments,
    difference-in-differences
  \item Effect estimation \& inference in the presence of interference
  \item Propensity scores, propensity score matching
%  \item Regression discontinuity
  \item Omitted variable sensitivity analysis
  \item Instrumental variables \& principal stratification
  \item \ldots
  \end{itemize}
\end{frame}

\Note{Don't linger, I come back to this slide after Causal Inference
  in Experiments section}

\begin{frame}{Three traditions in quantitative analysis}
\framesubtitle{A brief, tendentious review}

Modern causal inference in the 19th c., emerges over the 20th c, hits
it stride during this century. Dominant strains of quantitative
analysis in social science, by century:

\begin{itemize}
\item[19th] social physics
\item[20th] sampling of populations
\item[21st] causal inference (a prediction)
\end{itemize}

  
\end{frame}

\section{Causal inference in simple experiments}

\begin{frame}{Centrality of experiment to modern CI}
  \begin{itemize}
  \item Experiments are conceptually and practically central
  \item Not just any experiments, particular types, featuring
    \textit{control groups} and \textit{random assignment}\footnote{In
    this regard statisticians' concept of ``experiment'' is slightly
    idiosyncratic.  Compare Kinder \& Palfrey's (1993) definition to that of
    Rosenbaum (2010, ch 2.)}
  \item Central characteristics of an ideal situation
  \item The ideal reveals its merits in a variety of disciplines over
    the 20th c.
  \end{itemize}
\end{frame}

\subsection{Testing hypotheses of no effect in randomized experiments}
\begin{frame}{Example 1a: Salk vaccine trial}
\framesubtitle{
   The version that is remembered \footnote{E.g.
      \href{http://www.cengage.com/resource_uploads/downloads/0534094929_46500.pdf}{Meier
        1972}, ``The biggest public health experiment ever\ldots''}, in statistics \& public health. }


\igrphx{SalkVtable-rctonly}

\end{frame}
\itnote{
\item Vaccines contentious
\item Geographic variation
\item year to year variation
\item randomized
\item double blinded
\item large, national sample
\item Exercise: Abstract a relevant 2x2 table
}

\begin{frame}{The hypothesis of absolutely no effect}

    \vfill
  \end{frame}

\itnote{

\item Then break to discuss testing of hypotheses of no effect, i.e. start worksheet.
\medskip

\item[2016] at the
blackboard I walked them through strict null for the Salk trial,
tested it approximately using N-approx, $\mathrm{sd}_{0}(\hat p)
\approx \sqrt{p_{0}q_{0}/n_{1}}$.  This seemed to work well.
}

\begin{frame}{Potential outcomes}
  
  \begin{itemize}[<+->]
  \item Distributions of response under treatment and control: $\mu_{t},
    \mu_{c}$; $\mu_{t}(x), \mu_{c}(x)$.
  \item Counterfactual conditional.
  \item Unit-level potential outcomes (Rubin, 1970s): $y_{t}, y_{c}$.
  \end{itemize}

\uncover<+->{Rosenbaum's \textit{DOS} covers this material in detail (Ch.2, ``Causal inference in randomized experiments'').}
\end{frame}
\itnote{
\item[2017] (Skipped dispensible material about
  $x$/$\mathbf{x}$/$\mu_{\cdot}(\mathbf{x})$.) 
}
\subsection{Conceptual framework}
\begin{frame}{Goals of experimental design}

Experiment design aims to create a situation in which causation is
readily detected. If the putative causes
really do have their intended effects, you'll tend to see it; if they
don't, you'll tend not to.
\pause

\begin{itemize}[<+->]
\item \textrm{Random
    assignment}  sure helps. Units 1-3 theorize this point along lines
  of Fisher, Neyman and their current-day followers.
\item In comparative assessments of study designs, random assignment also benefits
  from its association with other design characteristics of
  high-quality studies:  distinctions among covariates, treatments and
  outcomes; comparability of treatment and control groups; \ldots;
  formal study planning (Rosenbaum, \textit{DOS}, ch. 1.)
\end{itemize}

\end{frame}

\begin{frame}{Example 2: overconfidence \& theories of intelligence}

\igrphx{ehrlingeretal16-titlebanner}

\end{frame}

\begin{frame}[handout:0]{Example 2: overconfidence \& theories of intelligence}

\igrphx{ehrlingeretal16sec2para1}

\end{frame}

\begin{frame}[handout:0]{Example 2: overconfidence \& theories of intelligence}

\begin{tabular}{c}
\igrphx{ehrlingeretal16-4-1-2-0}\\
\igrphx{ehrlingeretal16-4-1-2-1}\\
\end{tabular}

\end{frame}

\begin{frame}<1>[label=timelineFr]{Timeline, with entry points for biases}
  % commands I'll want to renew, to get slides right in different versions
  \newcommand{\selpt}{ }%{{\fbox{\strut \hspace{2em} \strut}}}
  \newcommand{\biasrow}{}%{ Biases: &   & & & & \\ }

  \begin{center}
    \begin{tabular}{lp{.15\linewidth}p{.15\linewidth}p{.15\linewidth}p{.15\linewidth}p{.15\linewidth}}
      & \multicolumn{2}{c}{\underline{$T< 0$}} & \underline{$T=0$} & \multicolumn{2}{c|}{\underline{$T>
        0$}} \\
& subjects identified & subjects recruited & treatment assigned &
treatment delivered & outcomes  measured\\ \hline
& \fbox{Healthy}   &            &          & & \\
 & \fbox{Sick}        &  \selpt      &     & & \\
 &                           &            &   \fbox{T}  & &  \\
 & \fbox{Urban}       &            &               & \fbox{T} & \\
 & \fbox{Rural}  &  \selpt      &           &  & \\
 &                            &            &                 & \fbox{C} & \\
 & \fbox{Rich}         &            &    \fbox{C} &  & \\
 & \fbox{Poor}        &  \selpt      &             &  & \\
 &   (etc)                        &            &                &  \\
\biasrow% T unlike C at beginning of study,
                        %sample not rep of pop,
                        % differential uptake, placebo effects,
                        % differential attrition
   \end{tabular}
  \end{center}
\end{frame}
\itnote{
% \item[W2016]  Having assigned the coffee/mortality study, assess it
% against this timeline.
\item How sharply defined are the 3 epochs?
\item what about comparability of groups?
\item Potential for unmeasured variables, in light of the Lawlor et al
findings as summarized by Rosenbaum, \textit{DOS} ch1?}


\section{Observational studies}


\begin{frame}{Example 1b: Salk vaccine trial (the other one)}

\igrphx{SalkVtable-full}
\end{frame}
\itnote{
\item Explain differences
\item both versions are experiments in the everyday sense
\item both control for age and physical locale
\item large size, presence of these controls lends legitimacy to both
\item In this case I maintain that markers of good design are an
  illusion, it's an overtly unfair comparison.  Why isn't this
  apples-to-apples?
\item Exercises: abstract $2\times 2$ table. \textrm{Think-pair-share}
\item Demonstration of advantage of random assigment per se
\item We'll see how to get the ``right answer'', despite shortcomings.  The move is
  primarily conceptual, not technical, having to do w/ IVs.
}



\begin{frame}<1-3>[label=quasivstrueFr]{Quasi-experiments as
    experiments?}
  \begin{itemize}[<+->]
  \item Controlled experiments vs found ``experiments''
  \item Found randomization as a hypothesis
  \item Theoretical vs empirical scrutiny of hypothesis
  \item Informal empirical scrutiny
  \item Formal hypothesis testing (instructor agenda alert!)
  \end{itemize}
\end{frame}



\begin{frame}{Paired comparisons}
  
  \begin{columns}
    \begin{Column}
  \begin{itemize}
  \item Example: Vehicular accidents on Friday the
      13ths\footnotemark
  \item An \textit{observational} paired comparison.  (Rosenbaum's
    \texttt{Design of Observational Studies} w/
    a randomized paired study.)
  \item This study design is sometimes called ``fixed effects.''
  \end{itemize}      
    \end{Column}
    \begin{Column}
      \igrphx{scalonetal93-tab3}
    \end{Column}
  \end{columns}
\footnotetext{T.J. Scanlon et al. (1993), ``Is Friday the 13th
        Bad For Your Health?'', BMJ 307 1584--86.}
\end{frame}

\itnote{
\item Numbers of emergency admissions to a UK hospital due to
  vehicular accidents, on consecutive fridays the second of which is a
  Friday the 13th. 
\item (\S~2.1--2.2.1 of \textit{DOS} creates a small paired expt from
  the NSW data.)
}

\againframe{timelineFr}

\begin{frame}{The hypothesis of no effect (again)}

  \begin{center}
    \begin{tabular}{l|l|r|l|l|r|l|l}
\hline
year & month & Fri6th & y\_11C & y\_11T & Fri13th & y\_12C & y\_12T\\
\hline
'89 & Oct. & 9 &  &  & 13 &  & \\
\hline
'90 & Jul. & 6 &  &  & 12 &  & \\
\hline
'91 & Sep. & 11 &  &  & 14 &  & \\
\hline
'91 & Dec. & 11 &  &  & 10 &  & \\
\hline
'92 & Mar. & 3 &  &  & 4 &  & \\
\hline
'92 & Nov. & 5 &  &  & 12 &  & \\
\hline
\end{tabular}

  \end{center}

  
\end{frame}
\section*{Closing remarks}

%\againframe{CIFr}
 
\begin{frame}{Course prerequisites}
  \begin{enumerate}
\item experience using a command-based computer program (as opposed to a menu-driven one)
\item willingness to try on attitudes suitable to
  non-commercial software: expect to err; trust but validate.
\item willingness to learn R  (if it's new to you, consider taking the course)
\item willingness to learn markdown (you can do this on your own)
\item Conditional probability and conditional expectation: prior
  exposure, or openness to self-study
  \item mathematical proofs, enough to distinguish correct from incorrect


  \end{enumerate}
\end{frame}



\end{document}

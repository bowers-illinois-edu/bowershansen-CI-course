%%%%%%%%%%%%%%%%%%%%%%%%%%%%%%%%%%%%%%%%%
% Beamer Presentation
% LaTeX Template
% Version 1.0 (10/11/12)
%
% This template has been downloaded from:
% http://www.LaTeXTemplates.com
%
% License:
% CC BY-NC-SA 3.0 (http://creativecommons.org/licenses/by-nc-sa/3.0/)
%
%%%%%%%%%%%%%%%%%%%%%%%%%%%%%%%%%%%%%%%%%

%------------------------------------------------------------------------------------------------
%	PACKAGES AND THEMES
%------------------------------------------------------------------------------------------------

\documentclass[table, xcolor = {dvipsnames}, 9pt]{beamer}
\usepackage{tikz}
\usetikzlibrary{calc}
\usetikzlibrary{positioning}
\usetikzlibrary{arrows.meta}
\usetikzlibrary{external}
\mode<presentation> {

% The Beamer class comes with a number of default slide themes
% which change the colors and layouts of slides. Below this is a list
% of all the themes, uncomment each in turn to see what they look like.

\usetheme{default}
%\usetheme{AnnArbor}
%\usetheme{Antibes}
%\usetheme{Bergen}
%\usetheme{Berkeley}
%\usetheme{Berlin}
%\usetheme{Boadilla}
%\usetheme{CambridgeUS}
%\usetheme{Copenhagen}
%\usetheme{Darmstadt}
%\usetheme{Dresden}
%\usetheme{Frankfurt}
%\usetheme{Goettingen}
%\usetheme{Hannover}
%\usetheme{Ilmenau}
%\usetheme{JuanLesPins}
%\usetheme{Luebeck}
%\usetheme{Madrid}
\usetheme{metropolis}
%\usetheme{Malmoe}
%\usetheme{Marburg}
%\usetheme{Montpellier}
%\usetheme{PaloAlto}
%\usetheme{Pittsburgh}
%\usetheme{Rochester}
%\usetheme{Singapore}
%\usetheme{Szeged}
%\usetheme{Warsaw}

% As well as themes, the Beamer class has a number of color themes
% for any slide theme. Uncomment each of these in turn to see how it
% changes the colors of your current slide theme.

%\usecolortheme{albatross}
%\usecolortheme{beaver}
%\usecolortheme{beetle}
%\usecolortheme{crane}
%\usecolortheme{dolphin}
%\usecolortheme{dove}
%\usecolortheme{fly}
%\usecolortheme{lily}
%\usecolortheme{orchid}
%\usecolortheme{rose}
\usecolortheme{seagull}
%\usecolortheme{seahorse}
%\usecolortheme{whale}
%\usecolortheme{wolverine}
\usefonttheme{professionalfonts}
%\setbeamertemplate{footline} % To remove the footer line in all slides uncomment this line
%\setbeamertemplate{footline}[page number] % To replace the footer line in all slides with a simple slide count uncomment this line

%\setbeamertemplate{navigation symbols}{} % To remove the navigation symbols from the bottom of all slides uncomment this line
}

\usepackage{graphicx} % Allows including images
\usepackage{booktabs} % Allows the use of \toprule, \midrule and \bottomrule in tables
\usepackage{tikz}
\usepackage{multirow}
\usepackage{natbib}
\usepackage{hyperref}
\usepackage{diagbox}
\usepackage{makecell}
\usepackage{xparse}
\usepackage{subfig}
\usepackage{amsmath}
\usepackage{amsfonts,amsthm,amsmath,amssymb}    
\usepackage{bbm}
\usepackage{bm}
\usepackage{empheq}
\usepackage{pgfplots}
\usepackage{animate}
\usepgfplotslibrary{colorbrewer}

\newcommand\mybox[2][]{\tikz[overlay]\node[fill=lightgray,inner sep=2pt, anchor=text, rectangle, rounded corners=1mm,#1] {#2};\phantom{#2}}
\hypersetup{unicode=true,
            bookmarksnumbered=true,
            bookmarksopen=true,
            bookmarksopenlevel=2,
            breaklinks=false,
            pdfborder={0 0 1},
            hypertexnames=false,
            pdfstartview={XYZ null null 1}}
\usepackage{xcolor}
\newcommand\myheading[1]{%
  \par\bigskip
  {\Large\bfseries#1}\par\smallskip}
\newcommand\given[1][]{\:#1\vert\:}
\theoremstyle{plain}
\newtheorem{thm}{Theorem}
\newtheorem{prop}{Proposition\thisthmnumber}
\newtheorem{lem}{Lemma\thisthmnumber}
\newtheorem{cor}{Corollary}
\newtheorem{defin}{Definition}
\newtheorem{algo}{Algorithm}
\newcommand*\diff{\mathop{}\!\mathrm{d}}
\newcommand*\Diff[1]{\mathop{}\!\mathrm{d^#1}}
\newcommand{\thisthmnumber}{}
\newcommand{\tikzmark}[1]{\tikz[baseline,remember picture] \coordinate (#1) {};}
\newcommand*{\QEDA}{\hfill\ensuremath{\blacksquare}}%
\newcommand*{\QEDB}{\hfill\ensuremath{\square}}%
\DeclareMathOperator{\E}{\rm{E}}
\DeclareMathOperator{\R}{\mathbb{R}}
\DeclareMathOperator{\N}{\mathbb{N}}
\DeclareMathOperator{\Var}{\rm{Var}}
\DeclareMathOperator{\Cov}{\rm{Cov}}
\DeclareMathOperator{\Supp}{\rm{Supp}}
\DeclareMathOperator{\e}{\rm{e}}
\DeclareMathOperator{\F}{\mathcal{F}}
\DeclareMathOperator{\Z}{\mathcal{Z}}
\DeclareMathOperator{\logit}{\rm{logit}}
\DeclareMathOperator{\indep}{{\perp\!\!\!\perp}}
\DeclareMathOperator{\rank}{rank}
\DeclareMathOperator*{\argmin}{arg\,min}
\DeclareMathOperator*{\argmax}{arg\,max}
%\DeclareMathOperator{\Pr}{\rm{Pr}}
%------------------------------------------------------------------------
%	TITLE PAGE
%-----------------------------------------------------------------------
\pagestyle{empty}
\title[]{Sensitivity Analysis for Weak Null} % The short title appears at the bottom of every slide, the full title is only on the title page

\author{Thomas Leavitt} % Your name
\institute[Columbia University] % Your institution as it will appear on the bottom of every slide, may be shorthand to save space
{
Columbia University \\ % Your institution for the title page
\medskip
\textit{tl2624@columbia.edu} % Your email address
}
\date{\today} % Date, can be changed to a custom date

\NewDocumentEnvironment{statement}{mo}
 {%
  \IfValueT{#2}{\renewcommand{\thisthmnumber}{ #2}}\begin{#1}%
 }
 {\end{#1}}

\begin{document}

\begin{frame}
\titlepage % Print the title page as the first slide
\end{frame}

%\begin{frame}
%\frametitle{Overview} % Table of contents slide, comment this block out to remove it
%\tableofcontents % Throughout your presentation, if you choose to use \section{} and \subsection{} commands, these will automatically be printed on this slide as an overview of your presentation
%\end{frame}

%------------------------------------------------------------------------
%	PRESENTATION SLIDES
%----------------------------------------------------------------
\section{Review of Sensitivity Analysis}
\begin{frame}[t]
\frametitle{Review of Sensitivity Analysis} 
\vfill
\begin{itemize}
\item When we impute missing potential outcomes according to a sharp null hypothesis, we can then find the vector $\bm{u}$ that maximizes the p-value.
\item If we are not directly imputing missing potential outcomes, as we do in a sharp null, then we cannot compute the $\bm{u}$ that maximizes the p-value under that sharp null
\item Once we have imputed potential outcomes, we can directly compute $\bm{u}$. So imagine all possible sharp causal hypotheses consistent with a given average effect. Then imagine the $\bm{u}$ that maximizes the expected value of the test stat under every configuration of potential outcomes. We take the maximum of all of these expected values. 
\end{itemize}  
\vfill
\end{frame}
%----------------------------------------------------------------
\section{Review of Neymanian Hypothesis Testing}
\begin{frame}[t]
\frametitle{Hypothesis Tests for Weak Nulls} 
\vfill
\begin{itemize}
\item $\dfrac{\hat{\tau} - \E\left[\hat{\tau}\right]}{\sqrt{\Var\left[\hat{\tau}\right]}} \overset{d}{\to} \mathcal{N}\left(0, 1\right)$, where $\Var\left[\hat{\tau}\right]$ is unknown
\item Our conservative variance estimator $\widehat{\Var}\left[\hat{\tau}\right]$ converges in probability to a constant greater than or equal to true limiting variance
\begin{itemize}
\item Define true limiting variance as $\nu = \lim \limits_{N \to \infty} \Var\left[\hat{\tau}\right]$
\item Then note that $\widehat{\Var}\left[\hat{\tau}\right] \overset{p}{\to} \eta \geq \nu$
\end{itemize}  
\item This implies asymptotic validity: As $N \to \infty$, Probability Type I error $\leq \alpha$
\end{itemize}
\vfill
\end{frame}
%----------------------------------------------------------------
\begin{frame}[t]
\frametitle{Proof} 
\vfill
\begin{itemize}
\item Our test statistic is $T_N = \dfrac{\hat{\tau} - \tau_0}{\sqrt{\widehat{\Var}\left[\hat{\tau}\right]}}$, which is  equivalent to $Z_N\dfrac{\sqrt{\Var\left[\hat{\tau}\right]}}{\sqrt{\widehat{\Var}\left[\hat{\tau}\right]}}$, where $Z_N = \dfrac{\hat{\tau} - \tau_0}{\sqrt{\Var\left[\hat{\tau}\right]}}$
\item If $\tau_0 = \tau$, then $Z_N \overset{d}{\to} Z$
\item $\sqrt{\widehat{\Var}\left[\hat{\tau}\right]} \bigg / \sqrt{\Var\left[\hat{\tau}\right]} \overset{p}{\to} c \geq 1$; so, by Slutsky's theorem and continous mapping theorem, $T_N \overset{d}{\to} Z / \sqrt{c}$
\item Therefore, as $N \to \infty$, the Type I error probability tends to the limit
\begin{align*}
\Pr\left(\dfrac{\left\lvert Z \right\rvert}{\sqrt{c}} > \Phi^{-1}(1 - \alpha / 2)\right) \\ 
\Pr\left(\left\lvert Z \right\rvert > \sqrt{c}\Phi^{-1}(1 - \alpha / 2)\right) \leq \Phi(1 - \alpha/2)
\end{align*}
\item Hence, as $N \to \infty$, our type I error probability will be less than or equal to $\alpha$
\end{itemize}
\vfill
\end{frame}
%----------------------------------------------------------------
\begin{frame}[t]
\frametitle{Proof} 
\vfill
\begin{itemize}
\item The proof above assumes uniform random assignment, but how know whether asymptotic validity holds under violations of random assignment? We need two things:
\item If we want asymptotic validity at a given $\Gamma$ over all $\bm{u} \in \mathcal{U} = [0, 1]^N$ and for all configurations of potential outcomes consistent with weak null, then
\begin{enumerate}
\item we need to know some $\eta$ such that $\eta \geq \max_{\bm{u} \in \mathcal{U}} \E\left[\hat{\tau}\right]$ and
\item an asymptotically conservative variance estimator $\hat{\nu}$ in which $\hat{\nu} / \nu \overset{p}{\to} c \geq 1$ as $N \to \infty$
\end{enumerate}
\begin{align*}
\Pr\left(\dfrac{\left\lvert Z \right\rvert - d}{\sqrt{c}} > \Phi^{-1}(1 - \alpha / 2)\right) \\ 
\Pr\left(\left\lvert Z \right\rvert > d + \sqrt{c}\Phi^{-1}(1 - \alpha / 2)\right) \leq \Phi(1 - \alpha/2)
\end{align*}
\end{itemize}
\vfill
\end{frame}
%----------------------------------------------------------------


\section{\normalsize Introduction}
\begin{frame}[t]
\frametitle{Sensitivity analysis for weak nulls} 
\vfill
\begin{itemize}
\item Denote the sample average treatment effect in block $b$ by $\hat{\tau}_b$
\item An unbiased estimator of the SATE is $\hat{\tau} = \sum \limits_{b = 1}^B (n_b/N)\hat{\tau}_b$
\item Use the test statistic $\hat{\tau} - \tau_0$

\end{itemize}  
\vfill
\end{frame}
%----------------------------------------------------------------
\begin{frame}[t]
\frametitle{Sensitivity analysis for weak nulls} 
\vfill
\begin{itemize}
\item We want to control the type I error probability? 
\item Suppose that Rosenbaum's sensitivity analysis model holds at a value of $\Gamma$ and that the null hypothesis is true, $\tau_0 = \bar{\tau}$
\item The expected value of the test statistic still depends on the unknown vector $\bm{u}$
\item We use the standard Normal distribution to conduct hypothesis testing
\item The larger is $\E\left[\hat{\tau}\right]$, the smaller is the standardized test statistic and the higher is our p-value
\item So we find $\max \limits_{\bm{u} \in \mathcal{U}} \E\left[\hat{\tau}\right]$
\item Explicitly calculating $\max \limits_{\bm{u} \in \mathcal{U}} \E\left[\hat{\tau}\right]$ requires knowledge of potential outcomes, so we can't actually do this
\end{itemize}  
\vfill
\end{frame}
%----------------------------------------------------------------
\begin{frame}[t]
\frametitle{Sensitivity analysis for weak nulls} 
\vfill
\begin{itemize}
\item So what can we do?
\item \citet{fogarty2020} shows that we can compute a tight upper bound for $\max \limits_{\bm{u} \in \mathcal{U}} \E\left[\hat{\tau}\right]$
\item What does this mean? It means that this value is the smallest possible value greater than the maximum expected value of the test statistic over all possible configurations of $\bm{u}$ and all configurations of potential outcomes consistent with $\tau_0$
\item By taking the largest possible value of $\E\left[\hat{\tau}\right]$, we ensure that our test is conservative, yielding a p-value greater than the p-value of the true expected value:
\begin{equation}
\dfrac{\hat{\tau} - \E_{\bm{u}}\left[\hat{\tau}\right]}{\widehat{\Var}_{\bm{u}}\left[\hat{\tau}\right]} \geq \dfrac{\hat{\tau} - \max_{\bm{u} \in \mathcal{U}} \E_{\bm{u}}\left[\hat{\tau}\right]}{\widehat{\Var}_{\bm{u}}\left[\hat{\tau}\right]} 
\end{equation}
\end{itemize}  
\vfill
\end{frame}
%----------------------------------------------------------------
\begin{frame}[t]
\frametitle{Sensitivity analysis for weak nulls} 
\vfill
\begin{itemize}
\item But what about the variance? It still depends on the unknowable $\bm{u}$ . . . 
\item \citet{fogarty2018} provides a conservative variance estimator that converges in probability to a value greater than or equal to the true variance
\item We don't need to know the missing potential outcomes or the unknown confounded $\bm{u}$ to implement this estimator
\end{itemize}  
\vfill
\end{frame}
%----------------------------------------------------------------
\section{Details}
\begin{frame}[t]
\frametitle{} 
\vfill
\begin{itemize}
\item The bounds on the individual treatment assignment probabilities without conditioning on $\sum_{i = 1}^N Z_i = n_T$ also imply bounds on $\Pr\left(\bm{Z} = \bm{z} \given n_T\right)$. These bounds are given by
\begin{equation}
1 \bigg / \sum \limits_{k = 0}^{n_T} \binom{n_T}{n_T - k} \binom{n_C}{n_C - k} \Gamma^k \leq \Pr\left(\bm{Z} = \bm{z} \given n_T\right) \leq \Gamma^{n_T} \bigg/ \sum \limits_{k = 0}^{n_T} \binom{n_T}{k} \binom{n_C}{n_T - k} \Gamma^k
\end{equation}
\item Optimal full matching: one treated to many controls or one control to many treated
\item $\Gamma \implies \dfrac{1}{\Gamma(n_b - 1) + 1} \leq \Pr(\bm{Z} = \bm{z}) \leq \dfrac{\Gamma}{(n_b - 1) + \Gamma}$
\item \citet[][Theorem 2]{fogarty2020}: Test statistic has known upper bound when sensitivity model holds at $\Gamma$ and average treatment effect is $\bar{\tau} = \tau_0$
\item Worst-case IPW estimator:
\begin{equation}
\tilde{W}_b = \dfrac{1}{n_b}\left(\dfrac{\hat{\tau}_b}{\Gamma / \left(n_b - 1 + \Gamma\right) \mathbbm{1}\left\{\hat{\tau}_b \geq 0\right\} + 1 / \left(\Gamma(n_b - 1) + 1\right)\mathbbm{1}\left\{\tau_b < 0\right\}}\right)
\end{equation}
\item The IPW with the worst-case expectation is when we weight by the largest possible probability when $\hat{\tau}_b \geq 0$ and the least possible probability when $\hat{\tau}_b < 0$. This maximizes the expectation of our test statistic, which can be no larger than $\bar{\tau}_b$
\end{itemize}  
\vfill
\end{frame}
%----------------------------------------------------------------

\begin{frame}[allowframebreaks]
\frametitle{References} 
\scriptsize
\bibliographystyle{chicago}
\bibliography{bibliography}    % name your BibTeX data base
\end{frame}
%----------------------------------------------------------------
%----------------------------------------------------------------
\end{document}
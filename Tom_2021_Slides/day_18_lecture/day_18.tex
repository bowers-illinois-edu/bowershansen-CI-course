%%%%%%%%%%%%%%%%%%%%%%%%%%%%%%%%%%%%%%%%%
% Beamer Presentation
% LaTeX Template
% Version 1.0 (10/11/12)
%
% This template has been downloaded from:
% http://www.LaTeXTemplates.com
%
% License:
% CC BY-NC-SA 3.0 (http://creativecommons.org/licenses/by-nc-sa/3.0/)
%
%%%%%%%%%%%%%%%%%%%%%%%%%%%%%%%%%%%%%%%%%

%------------------------------------------------------------------------------------------------
% PACKAGES AND THEMES
%------------------------------------------------------------------------------------------------

\documentclass[table, xcolor = {dvipsnames}, 9pt]{beamer}
\usepackage{tikz}
\usetikzlibrary{calc}
\usetikzlibrary{positioning}
\usetikzlibrary{arrows.meta}
\usetikzlibrary{external}
\mode<presentation> {

% The Beamer class comes with a number of default slide themes
% which change the colors and layouts of slides. Below this is a list
% of all the themes, uncomment each in turn to see what they look like.

\usetheme{default}
%\usetheme{AnnArbor}
%\usetheme{Antibes}
%\usetheme{Bergen}
%\usetheme{Berkeley}
%\usetheme{Berlin}
%\usetheme{Boadilla}
%\usetheme{CambridgeUS}
%\usetheme{Copenhagen}
%\usetheme{Darmstadt}
%\usetheme{Dresden}
%\usetheme{Frankfurt}
%\usetheme{Goettingen}
%\usetheme{Hannover}
%\usetheme{Ilmenau}
%\usetheme{JuanLesPins}
%\usetheme{Luebeck}
%\usetheme{Madrid}
\usetheme{metropolis}
%\usetheme{Malmoe}
%\usetheme{Marburg}
%\usetheme{Montpellier}
%\usetheme{PaloAlto}
%\usetheme{Pittsburgh}
%\usetheme{Rochester}
%\usetheme{Singapore}
%\usetheme{Szeged}
%\usetheme{Warsaw}

% As well as themes, the Beamer class has a number of color themes
% for any slide theme. Uncomment each of these in turn to see how it
% changes the colors of your current slide theme.

%\usecolortheme{albatross}
%\usecolortheme{beaver}
%\usecolortheme{beetle}
%\usecolortheme{crane}
%\usecolortheme{dolphin}
%\usecolortheme{dove}
%\usecolortheme{fly}
%\usecolortheme{lily}
%\usecolortheme{orchid}
%\usecolortheme{rose}
\usecolortheme{seagull}
%\usecolortheme{seahorse}
%\usecolortheme{whale}
%\usecolortheme{wolverine}
\usefonttheme{professionalfonts}
%\setbeamertemplate{footline} % To remove the footer line in all slides uncomment this line
%\setbeamertemplate{footline}[page number] % To replace the footer line in all slides with a simple slide count uncomment this line

%\setbeamertemplate{navigation symbols}{} % To remove the navigation symbols from the bottom of all slides uncomment this line
}

\usepackage{graphicx} % Allows including images
\usepackage{booktabs} % Allows the use of \toprule, \midrule and \bottomrule in tables
\usepackage{tikz}
\usepackage{multirow}
\usepackage{natbib}
\usepackage{hyperref}
\usepackage{diagbox}
\usepackage{makecell}
\usepackage{xparse}
\usepackage{subfig}
\usepackage{amsmath}
\usepackage{amsfonts,amsthm,amsmath,amssymb}    
\usepackage{bbm}
\usepackage{bm}
\usepackage{empheq}
\usepackage{pgfplots}
\usepackage{animate}
\usepgfplotslibrary{colorbrewer}

\newcommand\mybox[2][]{\tikz[overlay]\node[fill=lightgray,inner sep=2pt, anchor=text, rectangle, rounded corners=1mm,#1] {#2};\phantom{#2}}
\hypersetup{unicode=true,
            bookmarksnumbered=true,
            bookmarksopen=true,
            bookmarksopenlevel=2,
            breaklinks=false,
            pdfborder={0 0 1},
            hypertexnames=false,
            pdfstartview={XYZ null null 1}}
\usepackage{xcolor}
\newcommand\myheading[1]{%
  \par\bigskip
  {\Large\bfseries#1}\par\smallskip}
\newcommand\given[1][]{\:#1\vert\:}
\theoremstyle{plain}
\newtheorem{thm}{Theorem}
\newtheorem{prop}{Proposition\thisthmnumber}
\newtheorem{lem}{Lemma\thisthmnumber}
\newtheorem{cor}{Corollary}
\newtheorem{defin}{Definition}
\newtheorem{algo}{Algorithm}
\newcommand*\diff{\mathop{}\!\mathrm{d}}
\newcommand*\Diff[1]{\mathop{}\!\mathrm{d^#1}}
\newcommand{\thisthmnumber}{}
\newcommand{\tikzmark}[1]{\tikz[baseline,remember picture] \coordinate (#1) {};}
\newcommand*{\QEDA}{\hfill\ensuremath{\blacksquare}}%
\newcommand*{\QEDB}{\hfill\ensuremath{\square}}%
\DeclareMathOperator{\E}{\rm{E}}
\DeclareMathOperator{\R}{\mathbb{R}}
\DeclareMathOperator{\Var}{\rm{Var}}
\DeclareMathOperator{\Cov}{\rm{Cov}}
\DeclareMathOperator{\e}{\rm{e}}
\DeclareMathOperator{\logit}{\rm{logit}}
\DeclareMathOperator{\indep}{{\perp\!\!\!\perp}}
\DeclareMathOperator{\rank}{rank}
\DeclareMathOperator*{\argmin}{arg\,min}
\DeclareMathOperator*{\argmax}{arg\,max}
%\DeclareMathOperator{\Pr}{\rm{Pr}}
%------------------------------------------------------------------------
% TITLE PAGE
%-----------------------------------------------------------------------
\pagestyle{empty}
\title[]{Difference-in-Differences Designs} % The short title appears at the bottom of every slide, the full title is only on the title page

\author{Thomas Leavitt, Jake Bowers and Ben Hansen} % Your name
\institute[Columbia University] % Your institution as it will appear on the bottom of every slide, may be shorthand to save space
{
%Columbia University \\ % Your institution for the title page
\medskip
%\textit{tl2624@columbia.edu} % Your email address
}
\date{\today} % Date, can be changed to a custom date

\NewDocumentEnvironment{statement}{mo}
 {%
  \IfValueT{#2}{\renewcommand{\thisthmnumber}{ #2}}\begin{#1}%
 }
 {\end{#1}}

\begin{document}

\begin{frame}
\titlepage % Print the title page as the first slide
\end{frame}

%\begin{frame}
%\frametitle{Overview} % Table of contents slide, comment this block out to remove it
%\tableofcontents % Throughout your presentation, if you choose to use \section{} and \subsection{} commands, these will automatically be printed on this slide as an overview of your presentation
%\end{frame}

%------------------------------------------------------------------------
% PRESENTATION SLIDES
%------------------------------------------------------------------------
\section{Introduction}
\begin{frame}[t]
\frametitle{Causal inference and DID}
\vfill
\begin{itemize}
\item Scholars typically want to infer causal effects \vfill
\begin{itemize}
\item \pause Neyman--Rubin causal model \citep{neyman1923,rubin1974}  \vfill
\end{itemize} 
\item \pause Randomized experiments are ``gold standard'' for inferring causal effects \vfill
\item \pause But experiments often \textit{not} feasible or ethical \vfill
\item \pause Usually observational designs aim to justify \textit{as-if-randomized} assumption \vfill
\item \pause Other observational designs justify assumptions about how treated units' outcomes would have evolved in absence of treatment \vfill
\begin{itemize}
\item \pause Difference-in-Differences (DID) is archetypal example of such designs \vfill
\end{itemize} 
\end{itemize}
\vfill
\end{frame}
%-----------------------------------------------------------------------
\begin{frame}[t]
\frametitle{Canonical DID design}
\vfill
\begin{itemize}
\item Canonical setup: \vfill
\begin{itemize}
\item \pause Two populations and two time periods \vfill
\item \pause One population treated between two time periods and the other is not \vfill
\end{itemize}
\item \pause Key assumption: Parallel trends \pause \vfill
\end{itemize}
\vskip-2ex
\begin{figure}
\includegraphics[width = 0.9\linewidth]{figures/diff_in_diffs_plot.pdf}
\vskip1ex
\end{figure}
\vfill
\end{frame}
%-----------------------------------------------------------------------
\begin{frame}[t]
\frametitle{Canonical DID design}
\vfill
\begin{itemize}
\item DID estimator: \pause 
\begin{align*}
\text{treated sample's after-minus-before mean } - \\ \text{control sample's after-minus-before mean} 
\end{align*} \vfill
\item \pause Parallel trends $\implies$ expectation of DID estimator $=$ ATT \\ 
\pause (Average Treatment effect in Treated population) \vfill
\end{itemize}
\vfill
\end{frame}
%-----------------------------------------------------------------------
\begin{frame}[t]
\frametitle{DID Estimation}
\vfill
\begin{itemize}
\item DID estimator for population ($\mathcal{P}$) estimand: \pause \vfill
\normalsize
{\large
\begin{equation*}
\begin{split}
 \tikzmark{est}\widehat{\text{DID}}_{\mathcal{P}} & = \tikzmark{ntreat}\left(\frac{1}{n_1}\right)\tikzmark{treatsum}\sum_{i = n_0 + 1}^{n} \left(\tikzmark{treatedpotout}Y_{iT}(1) - \tikzmark{treatedout}Y_{iT-1}\right) - \tikzmark{ncontrol}\left(\frac{1}{n_0}\right)\tikzmark{controlsum}\sum_{i = 1}^{n_0} \left(\tikzmark{postcontrolout}Y_{iT} - \tikzmark{precontrolout}Y_{iT-1}\right)
\end{split} 
\end{equation*}
\begin{tikzpicture}[overlay, remember picture]
  
  %\node (betaj) [below of = ntreat, node distance = 9.25 em, anchor = west, xshift = 0cm] {\footnotesize \textsf{Number of treated units}};
    %\draw[<-, in = 500, out = -90] (ntreat.south)++(1.5em,-3.5ex) to (betaj.west);

  %\node (betaj) [below of = treatsum, node distance = 8.25 em, anchor = west, xshift = 0cm] {\footnotesize \textsf{Sum over treated units}};
    %\draw[<-, in = 500, out = -90] (treatsum.south)++(1.5em,-3.5ex) to (betaj.west);

    %\node (betaj) [below of = treatedpotout, node distance = 7.25 em, anchor = west, xshift = 0cm] {\footnotesize \textsf{Treated unit's treated potential outcome in post-treatment period}};
    %\draw[<-, in = 500, out = -90] (treatedpotout.south)++(1.5em,-1ex) to (betaj.west);

    %\node (betaj) [below of = treatedout, node distance = 6.25 em, anchor = west, xshift = 0cm] {\footnotesize \textsf{Treated unit's pre-treatment outcome}};
    %\draw[<-, in = 500, out = -90] (treatedout.south)++(1.5em,-1ex) to (betaj.west);

    %\node (betaj) [below of = ncontrol, node distance = 5.25 em, anchor = west, xshift = 0cm] {\footnotesize \textsf{Number of control units}};
    %\draw[<-, in = 500, out = -90] (ncontrol.south)++(1.5em,-3.5ex) to (betaj.west);

    %\node (betaj) [below of = controlsum, node distance = 4.25 em, anchor = west, xshift = 0cm] {\footnotesize \textsf{Sum over control units}};
    %\draw[<-, in = 500, out = -90] (controlsum.south)++(1em,-3.5ex) to (betaj.west);

    %\node (betatyrt) [above right of = postcontrolout, node distance = 4.25 em, anchor = east, xshift = -1cm] {\footnotesize \textsf{Control unit's post-treatment outcome}};
    %\draw[<-, in = 400, out = 90] (postcontrolout.north)++(1em, 2ex) to (betatyrt.east);

    %\node (betatyrt) [above right of = precontrolout, node distance = 6.25 em, anchor = east, xshift = -2cm] {\footnotesize \textsf{Control unit's pre-treatment outcome}};
    %\draw[<-, in = 400, out = 90] (precontrolout.north)++(1em, 2ex) to (betatyrt.east);

\end{tikzpicture}
}
\end{itemize}
\vfill
\end{frame}
%-----------------------------------------------------------------------
\begin{frame}[t]
\frametitle{DID Estimation}
\vfill
\begin{itemize}
\item DID estimator for population ($\mathcal{P}$) estimand: \vfill
\normalsize
{\large
\begin{equation*}
\begin{split}
 \tikzmark{est}\widehat{\text{DID}}_{\mathcal{P}} & = \tikzmark{ntreat}\left(\frac{1}{n_1}\right)\tikzmark{treatsum}\sum_{i = n_0 + 1}^{n} \left(\tikzmark{treatedpotout}Y_{iT}(1) - \tikzmark{treatedout}Y_{iT-1}\right) - \tikzmark{ncontrol}\left(\frac{1}{n_0}\right)\tikzmark{controlsum}\sum_{i = 1}^{n_0} \left(\tikzmark{postcontrolout}Y_{iT} - \tikzmark{precontrolout}Y_{iT-1}\right)
\end{split} 
\end{equation*}
\begin{tikzpicture}[overlay, remember picture]
  
  %\node (betaj) [below of = ntreat, node distance = 9.25 em, anchor = west, xshift = 0cm] {\footnotesize \textsf{Number of treated units}};
    %\draw[<-, in = 500, out = -90] (ntreat.south)++(1.5em,-3.5ex) to (betaj.west);

  %\node (betaj) [below of = treatsum, node distance = 8.25 em, anchor = west, xshift = 0cm] {\footnotesize \textsf{Sum over treated units}};
    %\draw[<-, in = 500, out = -90] (treatsum.south)++(1.5em,-3.5ex) to (betaj.west);

    \node (betaj) [below of = treatedpotout, node distance = 7.25 em, anchor = west, xshift = 0cm] {\footnotesize \textsf{Treated unit's treated potential outcome in post-treatment period}};
    \draw[<-, in = 500, out = -90] (treatedpotout.south)++(1.5em,-1ex) to (betaj.west);

    %\node (betaj) [below of = treatedout, node distance = 6.25 em, anchor = west, xshift = 0cm] {\footnotesize \textsf{Treated unit's pre-treatment outcome}};
    %\draw[<-, in = 500, out = -90] (treatedout.south)++(1.5em,-1ex) to (betaj.west);

    %\node (betaj) [below of = ncontrol, node distance = 5.25 em, anchor = west, xshift = 0cm] {\footnotesize \textsf{Number of control units}};
    %\draw[<-, in = 500, out = -90] (ncontrol.south)++(1.5em,-3.5ex) to (betaj.west);

    %\node (betaj) [below of = controlsum, node distance = 4.25 em, anchor = west, xshift = 0cm] {\footnotesize \textsf{Sum over control units}};
    %\draw[<-, in = 500, out = -90] (controlsum.south)++(1em,-3.5ex) to (betaj.west);

    %\node (betatyrt) [above right of = postcontrolout, node distance = 4.25 em, anchor = east, xshift = -1cm] {\footnotesize \textsf{Control unit's post-treatment outcome}};
    %\draw[<-, in = 400, out = 90] (postcontrolout.north)++(1em, 2ex) to (betatyrt.east);

    %\node (betatyrt) [above right of = precontrolout, node distance = 6.25 em, anchor = east, xshift = -2cm] {\footnotesize \textsf{Control unit's pre-treatment outcome}};
    %\draw[<-, in = 400, out = 90] (precontrolout.north)++(1em, 2ex) to (betatyrt.east);

\end{tikzpicture}
}
\end{itemize}
\vfill
\end{frame}
%-----------------------------------------------------------------------
\begin{frame}[t]
\frametitle{DID Estimation}
\vfill
\begin{itemize}
\item DID estimator for population ($\mathcal{P}$) estimand: \vfill
\normalsize
{\large
\begin{equation*}
\begin{split}
 \tikzmark{est}\widehat{\text{DID}}_{\mathcal{P}} & = \tikzmark{ntreat}\left(\frac{1}{n_1}\right)\tikzmark{treatsum}\sum_{i = n_0 + 1}^{n} \left(\tikzmark{treatedpotout}Y_{iT}(1) - \tikzmark{treatedout}Y_{iT-1}\right) - \tikzmark{ncontrol}\left(\frac{1}{n_0}\right)\tikzmark{controlsum}\sum_{i = 1}^{n_0} \left(\tikzmark{postcontrolout}Y_{iT} - \tikzmark{precontrolout}Y_{iT-1}\right)
\end{split} 
\end{equation*}
\begin{tikzpicture}[overlay, remember picture]
  
  %\node (betaj) [below of = ntreat, node distance = 9.25 em, anchor = west, xshift = 0cm] {\footnotesize \textsf{Number of treated units}};
    %\draw[<-, in = 500, out = -90] (ntreat.south)++(1.5em,-3.5ex) to (betaj.west);

  %\node (betaj) [below of = treatsum, node distance = 8.25 em, anchor = west, xshift = 0cm] {\footnotesize \textsf{Sum over treated units}};
    %\draw[<-, in = 500, out = -90] (treatsum.south)++(1.5em,-3.5ex) to (betaj.west);

    \node (betaj) [below of = treatedpotout, node distance = 7.25 em, anchor = west, xshift = 0cm] {\footnotesize \textsf{Treated unit's treated potential outcome in post-treatment period}};
    \draw[<-, in = 500, out = -90] (treatedpotout.south)++(1.5em,-1ex) to (betaj.west);

    \node (betaj) [below of = treatedout, node distance = 6.25 em, anchor = west, xshift = 0cm] {\footnotesize \textsf{Treated unit's pre-treatment outcome}};
    \draw[<-, in = 500, out = -90] (treatedout.south)++(1.5em,-1ex) to (betaj.west);

    %\node (betaj) [below of = ncontrol, node distance = 5.25 em, anchor = west, xshift = 0cm] {\footnotesize \textsf{Number of control units}};
    %\draw[<-, in = 500, out = -90] (ncontrol.south)++(1.5em,-3.5ex) to (betaj.west);

    %\node (betaj) [below of = controlsum, node distance = 4.25 em, anchor = west, xshift = 0cm] {\footnotesize \textsf{Sum over control units}};
    %\draw[<-, in = 500, out = -90] (controlsum.south)++(1em,-3.5ex) to (betaj.west);

    %\node (betatyrt) [above right of = postcontrolout, node distance = 4.25 em, anchor = east, xshift = -1cm] {\footnotesize \textsf{Control unit's post-treatment outcome}};
    %\draw[<-, in = 400, out = 90] (postcontrolout.north)++(1em, 2ex) to (betatyrt.east);

    %\node (betatyrt) [above right of = precontrolout, node distance = 6.25 em, anchor = east, xshift = -2cm] {\footnotesize \textsf{Control unit's pre-treatment outcome}};
    %\draw[<-, in = 400, out = 90] (precontrolout.north)++(1em, 2ex) to (betatyrt.east);

\end{tikzpicture}
}
\end{itemize}
\vfill
\end{frame}
%------------------------------------------------------------------------
\begin{frame}[t]
\frametitle{DID Estimation}
\vfill
\begin{itemize}
\item DID estimator for population ($\mathcal{P}$) estimand: \vfill 
\normalsize
{\large
\begin{equation*}
\begin{split}
 \tikzmark{est}\widehat{\text{DID}}_{\mathcal{P}} & = \tikzmark{ntreat}\left(\frac{1}{n_1}\right)\tikzmark{treatsum}\sum_{i = n_0 + 1}^{n} \left(\tikzmark{treatedpotout}Y_{iT}(1) - \tikzmark{treatedout}Y_{iT-1}\right) - \tikzmark{ncontrol}\left(\frac{1}{n_0}\right)\tikzmark{controlsum}\sum_{i = 1}^{n_0} \left(\tikzmark{postcontrolout}Y_{iT} - \tikzmark{precontrolout}Y_{iT-1}\right)
\end{split} 
\end{equation*}
\begin{tikzpicture}[overlay, remember picture]
  
  %\node (betaj) [below of = ntreat, node distance = 9.25 em, anchor = west, xshift = 0cm] {\footnotesize \textsf{Number of treated units}};
    %\draw[<-, in = 500, out = -90] (ntreat.south)++(1.5em,-3.5ex) to (betaj.west);

  \node (betaj) [below of = treatsum, node distance = 8.25 em, anchor = west, xshift = 0cm] {\footnotesize \textsf{Sum over treated units}};
    \draw[<-, in = 500, out = -90] (treatsum.south)++(1.5em,-3.5ex) to (betaj.west);

    \node (betaj) [below of = treatedpotout, node distance = 7.25 em, anchor = west, xshift = 0cm] {\footnotesize \textsf{Treated unit's treated potential outcome in post-treatment period}};
    \draw[<-, in = 500, out = -90] (treatedpotout.south)++(1.5em,-1ex) to (betaj.west);

    \node (betaj) [below of = treatedout, node distance = 6.25 em, anchor = west, xshift = 0cm] {\footnotesize \textsf{Treated unit's pre-treatment outcome}};
    \draw[<-, in = 500, out = -90] (treatedout.south)++(1.5em,-1ex) to (betaj.west);

    %\node (betaj) [below of = ncontrol, node distance = 5.25 em, anchor = west, xshift = 0cm] {\footnotesize \textsf{Number of control units}};
    %\draw[<-, in = 500, out = -90] (ncontrol.south)++(1.5em,-3.5ex) to (betaj.west);

    %\node (betaj) [below of = controlsum, node distance = 4.25 em, anchor = west, xshift = 0cm] {\footnotesize \textsf{Sum over control units}};
    %\draw[<-, in = 500, out = -90] (controlsum.south)++(1em,-3.5ex) to (betaj.west);

    %\node (betatyrt) [above right of = postcontrolout, node distance = 4.25 em, anchor = east, xshift = -1cm] {\footnotesize \textsf{Control unit's post-treatment outcome}};
    %\draw[<-, in = 400, out = 90] (postcontrolout.north)++(1em, 2ex) to (betatyrt.east);

    %\node (betatyrt) [above right of = precontrolout, node distance = 6.25 em, anchor = east, xshift = -2cm] {\footnotesize \textsf{Control unit's pre-treatment outcome}};
    %\draw[<-, in = 400, out = 90] (precontrolout.north)++(1em, 2ex) to (betatyrt.east);

\end{tikzpicture}
}
\end{itemize}
\vfill
\end{frame}
%------------------------------------------------------------------------
\begin{frame}[t]
\frametitle{DID Estimation}
\vfill
\begin{itemize}
\item DID estimator for population ($\mathcal{P}$) estimand: \vfill
\normalsize
{\large
\begin{equation*}
\begin{split}
 \tikzmark{est}\widehat{\text{DID}}_{\mathcal{P}} & = \tikzmark{ntreat}\left(\frac{1}{n_1}\right)\tikzmark{treatsum}\sum_{i = n_0 + 1}^{n} \left(\tikzmark{treatedpotout}Y_{iT}(1) - \tikzmark{treatedout}Y_{iT-1}\right) - \tikzmark{ncontrol}\left(\frac{1}{n_0}\right)\tikzmark{controlsum}\sum_{i = 1}^{n_0} \left(\tikzmark{postcontrolout}Y_{iT} - \tikzmark{precontrolout}Y_{iT-1}\right)
\end{split} 
\end{equation*}
\begin{tikzpicture}[overlay, remember picture]
  
  \node (betaj) [below of = ntreat, node distance = 9.25 em, anchor = west, xshift = 0cm] {\footnotesize \textsf{Number of treated units}};
    \draw[<-, in = 500, out = -90] (ntreat.south)++(1.5em,-3.5ex) to (betaj.west);

  \node (betaj) [below of = treatsum, node distance = 8.25 em, anchor = west, xshift = 0cm] {\footnotesize \textsf{Sum over treated units}};
    \draw[<-, in = 500, out = -90] (treatsum.south)++(1.5em,-3.5ex) to (betaj.west);

    \node (betaj) [below of = treatedpotout, node distance = 7.25 em, anchor = west, xshift = 0cm] {\footnotesize \textsf{Treated unit's treated potential outcome in post-treatment period}};
    \draw[<-, in = 500, out = -90] (treatedpotout.south)++(1.5em,-1ex) to (betaj.west);

    \node (betaj) [below of = treatedout, node distance = 6.25 em, anchor = west, xshift = 0cm] {\footnotesize \textsf{Treated unit's pre-treatment outcome}};
    \draw[<-, in = 500, out = -90] (treatedout.south)++(1.5em,-1ex) to (betaj.west);

    %\node (betaj) [below of = ncontrol, node distance = 5.25 em, anchor = west, xshift = 0cm] {\footnotesize \textsf{Number of control units}};
    %\draw[<-, in = 500, out = -90] (ncontrol.south)++(1.5em,-3.5ex) to (betaj.west);

    %\node (betaj) [below of = controlsum, node distance = 4.25 em, anchor = west, xshift = 0cm] {\footnotesize \textsf{Sum over control units}};
    %\draw[<-, in = 500, out = -90] (controlsum.south)++(1em,-3.5ex) to (betaj.west);

    %\node (betatyrt) [above right of = postcontrolout, node distance = 4.25 em, anchor = east, xshift = -1cm] {\footnotesize \textsf{Control unit's post-treatment outcome}};
    %\draw[<-, in = 400, out = 90] (postcontrolout.north)++(1em, 2ex) to (betatyrt.east);

    %\node (betatyrt) [above right of = precontrolout, node distance = 6.25 em, anchor = east, xshift = -2cm] {\footnotesize \textsf{Control unit's pre-treatment outcome}};
    %\draw[<-, in = 400, out = 90] (precontrolout.north)++(1em, 2ex) to (betatyrt.east);

\end{tikzpicture}
}
\end{itemize}
\vfill
\end{frame}
%-----------------------------------------------------------------------
\begin{frame}[t]
\frametitle{DID Estimation}
\vfill
\begin{itemize}
\item DID estimator for population ($\mathcal{P}$) estimand: \vfill
\normalsize
{\large
\begin{equation*}
\begin{split}
 \tikzmark{est}\widehat{\text{DID}}_{\mathcal{P}} & = \tikzmark{ntreat}\left(\frac{1}{n_1}\right)\tikzmark{treatsum}\sum_{i = n_0 + 1}^{n} \left(\tikzmark{treatedpotout}Y_{iT}(1) - \tikzmark{treatedout}Y_{iT-1}\right) - \tikzmark{ncontrol}\left(\frac{1}{n_0}\right)\tikzmark{controlsum}\sum_{i = 1}^{n_0} \left(\tikzmark{postcontrolout}Y_{iT} - \tikzmark{precontrolout}Y_{iT-1}\right)
\end{split} 
\end{equation*}
\begin{tikzpicture}[overlay, remember picture]
  
  \node (betaj) [below of = ntreat, node distance = 9.25 em, anchor = west, xshift = 0cm] {\footnotesize \textsf{Number of treated units}};
    \draw[<-, in = 500, out = -90] (ntreat.south)++(1.5em,-3.5ex) to (betaj.west);

  \node (betaj) [below of = treatsum, node distance = 8.25 em, anchor = west, xshift = 0cm] {\footnotesize \textsf{Sum over treated units}};
    \draw[<-, in = 500, out = -90] (treatsum.south)++(1.5em,-3.5ex) to (betaj.west);

    \node (betaj) [below of = treatedpotout, node distance = 7.25 em, anchor = west, xshift = 0cm] {\footnotesize \textsf{Treated unit's treated potential outcome in post-treatment period}};
    \draw[<-, in = 500, out = -90] (treatedpotout.south)++(1.5em,-1ex) to (betaj.west);

    \node (betaj) [below of = treatedout, node distance = 6.25 em, anchor = west, xshift = 0cm] {\footnotesize \textsf{Treated unit's pre-treatment outcome}};
    \draw[<-, in = 500, out = -90] (treatedout.south)++(1.5em,-1ex) to (betaj.west);

    %\node (betaj) [below of = ncontrol, node distance = 5.25 em, anchor = west, xshift = 0cm] {\footnotesize \textsf{Number of control units}};
    %\draw[<-, in = 500, out = -90] (ncontrol.south)++(1.5em,-3.5ex) to (betaj.west);

    %\node (betaj) [below of = controlsum, node distance = 4.25 em, anchor = west, xshift = 0cm] {\footnotesize \textsf{Sum over control units}};
    %\draw[<-, in = 500, out = -90] (controlsum.south)++(1em,-3.5ex) to (betaj.west);

    \node (betatyrt) [above right of = postcontrolout, node distance = 4.25 em, anchor = east, xshift = -1cm] {\footnotesize \textsf{Control unit's post-treatment outcome}};
    \draw[<-, in = 400, out = 90] (postcontrolout.north)++(1em, 2ex) to (betatyrt.east);

    %\node (betatyrt) [above right of = precontrolout, node distance = 6.25 em, anchor = east, xshift = -2cm] {\footnotesize \textsf{Control unit's pre-treatment outcome}};
    %\draw[<-, in = 400, out = 90] (precontrolout.north)++(1em, 2ex) to (betatyrt.east);

\end{tikzpicture}
}
\end{itemize}
\vfill
\end{frame}
%------------------------------------------------------------------------
\begin{frame}[t]
\frametitle{DID Estimation}
\vfill
\begin{itemize}
\item DID estimator for population ($\mathcal{P}$) estimand: \vfill 
\normalsize
{\large
\begin{equation*}
\begin{split}
 \tikzmark{est}\widehat{\text{DID}}_{\mathcal{P}} & = \tikzmark{ntreat}\left(\frac{1}{n_1}\right)\tikzmark{treatsum}\sum_{i = n_0 + 1}^{n} \left(\tikzmark{treatedpotout}Y_{iT}(1) - \tikzmark{treatedout}Y_{iT-1}\right) - \tikzmark{ncontrol}\left(\frac{1}{n_0}\right)\tikzmark{controlsum}\sum_{i = 1}^{n_0} \left(\tikzmark{postcontrolout}Y_{iT} - \tikzmark{precontrolout}Y_{iT-1}\right)
\end{split} 
\end{equation*}
\begin{tikzpicture}[overlay, remember picture]
  
  \node (betaj) [below of = ntreat, node distance = 9.25 em, anchor = west, xshift = 0cm] {\footnotesize \textsf{Number of treated units}};
    \draw[<-, in = 500, out = -90] (ntreat.south)++(1.5em,-3.5ex) to (betaj.west);

  \node (betaj) [below of = treatsum, node distance = 8.25 em, anchor = west, xshift = 0cm] {\footnotesize \textsf{Sum over treated units}};
    \draw[<-, in = 500, out = -90] (treatsum.south)++(1.5em,-3.5ex) to (betaj.west);

    \node (betaj) [below of = treatedpotout, node distance = 7.25 em, anchor = west, xshift = 0cm] {\footnotesize \textsf{Treated unit's treated potential outcome in post-treatment period}};
    \draw[<-, in = 500, out = -90] (treatedpotout.south)++(1.5em,-1ex) to (betaj.west);

    \node (betaj) [below of = treatedout, node distance = 6.25 em, anchor = west, xshift = 0cm] {\footnotesize \textsf{Treated unit's pre-treatment outcome}};
    \draw[<-, in = 500, out = -90] (treatedout.south)++(1.5em,-1ex) to (betaj.west);

    %\node (betaj) [below of = ncontrol, node distance = 5.25 em, anchor = west, xshift = 0cm] {\footnotesize \textsf{Number of control units}};
    %\draw[<-, in = 500, out = -90] (ncontrol.south)++(1.5em,-3.5ex) to (betaj.west);

    %\node (betaj) [below of = controlsum, node distance = 4.25 em, anchor = west, xshift = 0cm] {\footnotesize \textsf{Sum over control units}};
    %\draw[<-, in = 500, out = -90] (controlsum.south)++(1em,-3.5ex) to (betaj.west);

    \node (betatyrt) [above right of = postcontrolout, node distance = 4.25 em, anchor = east, xshift = -1cm] {\footnotesize \textsf{Control unit's post-treatment outcome}};
    \draw[<-, in = 400, out = 90] (postcontrolout.north)++(1em, 2ex) to (betatyrt.east);

    \node (betatyrt) [above right of = precontrolout, node distance = 6.25 em, anchor = east, xshift = -2cm] {\footnotesize \textsf{Control unit's pre-treatment outcome}};
    \draw[<-, in = 400, out = 90] (precontrolout.north)++(1em, 2ex) to (betatyrt.east);

\end{tikzpicture}
}
\end{itemize}
\vfill
\end{frame}
%------------------------------------------------------------------------
\begin{frame}[t]
\frametitle{DID Estimation}
\vfill
\begin{itemize}
\item DID estimator for population ($\mathcal{P}$) estimand: \vfill
\normalsize
{\large
\begin{equation*}
\begin{split}
 \tikzmark{est}\widehat{\text{DID}}_{\mathcal{P}} & = \tikzmark{ntreat}\left(\frac{1}{n_1}\right)\tikzmark{treatsum}\sum_{i = n_0 + 1}^{n} \left(\tikzmark{treatedpotout}Y_{iT}(1) - \tikzmark{treatedout}Y_{iT-1}\right) - \tikzmark{ncontrol}\left(\frac{1}{n_0}\right)\tikzmark{controlsum}\sum_{i = 1}^{n_0} \left(\tikzmark{postcontrolout}Y_{iT} - \tikzmark{precontrolout}Y_{iT-1}\right)
\end{split} 
\end{equation*}
\begin{tikzpicture}[overlay, remember picture]
  
  \node (betaj) [below of = ntreat, node distance = 9.25 em, anchor = west, xshift = 0cm] {\footnotesize \textsf{Number of treated units}};
    \draw[<-, in = 500, out = -90] (ntreat.south)++(1.5em,-3.5ex) to (betaj.west);

  \node (betaj) [below of = treatsum, node distance = 8.25 em, anchor = west, xshift = 0cm] {\footnotesize \textsf{Sum over treated units}};
    \draw[<-, in = 500, out = -90] (treatsum.south)++(1.5em,-3.5ex) to (betaj.west);

    \node (betaj) [below of = treatedpotout, node distance = 7.25 em, anchor = west, xshift = 0cm] {\footnotesize \textsf{Treated unit's treated potential outcome in post-treatment period}};
    \draw[<-, in = 500, out = -90] (treatedpotout.south)++(1.5em,-1ex) to (betaj.west);

    \node (betaj) [below of = treatedout, node distance = 6.25 em, anchor = west, xshift = 0cm] {\footnotesize \textsf{Treated unit's pre-treatment outcome}};
    \draw[<-, in = 500, out = -90] (treatedout.south)++(1.5em,-1ex) to (betaj.west);

    %\node (betaj) [below of = ncontrol, node distance = 5.25 em, anchor = west, xshift = 0cm] {\footnotesize \textsf{Number of control units}};
    %\draw[<-, in = 500, out = -90] (ncontrol.south)++(1.5em,-3.5ex) to (betaj.west);

    \node (betaj) [below of = controlsum, node distance = 4.25 em, anchor = west, xshift = 0cm] {\footnotesize \textsf{Sum over control units}};
    \draw[<-, in = 500, out = -90] (controlsum.south)++(1em,-3.5ex) to (betaj.west);

    \node (betatyrt) [above right of = postcontrolout, node distance = 4.25 em, anchor = east, xshift = -1cm] {\footnotesize \textsf{Control unit's post-treatment outcome}};
    \draw[<-, in = 400, out = 90] (postcontrolout.north)++(1em, 2ex) to (betatyrt.east);

    \node (betatyrt) [above right of = precontrolout, node distance = 6.25 em, anchor = east, xshift = -2cm] {\footnotesize \textsf{Control unit's pre-treatment outcome}};
    \draw[<-, in = 400, out = 90] (precontrolout.north)++(1em, 2ex) to (betatyrt.east);

\end{tikzpicture}
}
\end{itemize}
\vfill
\end{frame}
%-----------------------------------------------------------------------
\begin{frame}[t]
\frametitle{DID Estimation}
\vfill
\begin{itemize} 
\item DID estimator for population ($\mathcal{P}$) estimand: \vfill
\normalsize
{\large
\begin{equation*}
\begin{split}
 \tikzmark{est}\widehat{\text{DID}}_{\mathcal{P}} & = \tikzmark{ntreat}\left(\frac{1}{n_1}\right)\tikzmark{treatsum}\sum_{i = n_0 + 1}^{n} \left(\tikzmark{treatedpotout}Y_{iT}(1) - \tikzmark{treatedout}Y_{iT-1}\right) - \tikzmark{ncontrol}\left(\frac{1}{n_0}\right)\tikzmark{controlsum}\sum_{i = 1}^{n_0} \left(\tikzmark{postcontrolout}Y_{iT} - \tikzmark{precontrolout}Y_{iT-1}\right)
\end{split} 
\end{equation*}
\begin{tikzpicture}[overlay, remember picture]
  
  \node (betaj) [below of = ntreat, node distance = 9.25 em, anchor = west, xshift = 0cm] {\footnotesize \textsf{Number of treated units}};
    \draw[<-, in = 500, out = -90] (ntreat.south)++(1.5em,-3.5ex) to (betaj.west);

  \node (betaj) [below of = treatsum, node distance = 8.25 em, anchor = west, xshift = 0cm] {\footnotesize \textsf{Sum over treated units}};
    \draw[<-, in = 500, out = -90] (treatsum.south)++(1.5em,-3.5ex) to (betaj.west);

    \node (betaj) [below of = treatedpotout, node distance = 7.25 em, anchor = west, xshift = 0cm] {\footnotesize \textsf{Treated unit's treated potential outcome in post-treatment period}};
    \draw[<-, in = 500, out = -90] (treatedpotout.south)++(1.5em,-1ex) to (betaj.west);

    \node (betaj) [below of = treatedout, node distance = 6.25 em, anchor = west, xshift = 0cm] {\footnotesize \textsf{Treated unit's pre-treatment outcome}};
    \draw[<-, in = 500, out = -90] (treatedout.south)++(1.5em,-1ex) to (betaj.west);

    \node (betaj) [below of = ncontrol, node distance = 5.25 em, anchor = west, xshift = 0cm] {\footnotesize \textsf{Number of control units}};
    \draw[<-, in = 500, out = -90] (ncontrol.south)++(1.5em,-3.5ex) to (betaj.west);

    \node (betaj) [below of = controlsum, node distance = 4.25 em, anchor = west, xshift = 0cm] {\footnotesize \textsf{Sum over control units}};
    \draw[<-, in = 500, out = -90] (controlsum.south)++(1em,-3.5ex) to (betaj.west);

    \node (betatyrt) [above right of = postcontrolout, node distance = 4.25 em, anchor = east, xshift = -1cm] {\footnotesize \textsf{Control unit's post-treatment outcome}};
    \draw[<-, in = 400, out = 90] (postcontrolout.north)++(1em, 2ex) to (betatyrt.east);

    \node (betatyrt) [above right of = precontrolout, node distance = 6.25 em, anchor = east, xshift = -2cm] {\footnotesize \textsf{Control unit's pre-treatment outcome}};
    \draw[<-, in = 400, out = 90] (precontrolout.north)++(1em, 2ex) to (betatyrt.east);

\end{tikzpicture}
}
\end{itemize}
\vfill
\end{frame}
%-----------------------------------------------------------------------
\begin{frame}[t]
\frametitle{DID Estimation}
\begin{itemize}
\item Under independent and identically distributed (IID) sampling,  \pause 
\begin{equation*}
\begin{split}
\E\left[\widehat{\text{DID}}_{\mathcal{P}}\right] & = \underbrace{\E\left[Y_{iT}(1) - Y_{iT-1} \given z_{iT} = 1\right] - \E\left[Y_{iT} - Y_{iT-1} \given z_{iT} = 0\right]}_{\text{Descriptive difference between treated and control populations}}
\end{split} 
\end{equation*}
%\begin{align*}
%\E\left[\widehat{\text{DID}}_{\mathcal{P}}\right] & = \underbrace{\E\left[Y_{iT}(1) - Y_{iT-1} \given z_{iT} = 1\right] - \E\left[Y_{iT} - Y_{iT-1} \given z_{iT} = 0\right]}_{\text{Descriptive difference between treated and control populations}}
%\end{align*} \large
\item \pause Parallel trends: \pause 
\vskip3ex 
\small
{\large
\begin{equation*}
\begin{split}
\E\left[Y_{iT}(0) \tikzmark{treatedafterbeforemeannotreat}- Y_{iT-1} \given z_{iT} = 1\right] & = \E\left[Y_{iT} - Y_{iT-1} \given z_{iT} = 0\right]
\end{split} 
\end{equation*}
\begin{tikzpicture}[overlay, remember picture]
  
    \node (treated) [above right of = treatedafterbeforemeannotreat, node distance = 4 em, anchor = west, xshift = -2cm] {\footnotesize \textsf{Treated population's counterfactual after-minus-before mean}};
    \draw[<-, in = -500, out = 90] (treatedafterbeforemeannotreat.west)++(.5em, 1.75ex) to (treated.west);

\end{tikzpicture}
} \normalsize
\vspace{-1em}
\item Rearranging parallel trends assumption yields
\small
\vspace{0.75em}
\begin{align*}
\underbrace{\E\left[Y_{iT}(0) \given z_{iT} = 1\right]}_{\text{Counterfactual}} - \E\left[Y_{T-1} \given z_{iT} = 1 \right]+ \E\left[Y_{iT-1} \given z_{iT} = 0\right] & = \E\left[Y_{iT} \given z_{iT} = 0\right]
\end{align*}
\normalsize
\vspace{0.75em}
\item \pause Substitute for $\E\left[Y_{iT} \given z_{iT} = 0\right]$ in DID estimator \pause
\small\begin{align*}
\underbrace{\E\left[Y_{iT}(1) - Y_{iT-1} \given z_{iT} = 1\right] - \E\left[Y_{iT} - Y_{iT-1} \given z_{iT} = 0\right]}_{\text{Descriptive difference between treated and control populations}} \\ = \underbrace{\E\left[Y_{iT}(1) - Y_{iT}(0) \given z_{iT} = 1\right]}_{\substack{\text{Average treatment effect in} \\ \text{treated population } (\text{ATT}_{\mathcal{P}})}}
\end{align*} 
\normalsize
\end{itemize}
\end{frame}
%------------------------------------------------------------------------
\section{Problems with standard DID}
\begin{frame}{Problems with standard DID}
\begin{enumerate}
\item Estimation: Parallel trends assumption is restrictive
\item \pause Inference: Accounts only for sampling uncertainty
\end{enumerate}
\end{frame}
%------------------------------------------------------------------------
\section{Estimation}
\begin{frame}[t]
\frametitle{Estimation}
\vfill
\begin{itemize}
\item Nonparametric identification is possible when parallel trends is plausible: \\~\\ \pause
\item[] \vfill
\begin{figure}
\includegraphics[width = \linewidth]{figures/diff_in_diffs_parallel_trend_plot.pdf}
\vskip1ex
\end{figure} 
\end{itemize}
\vfill
\end{frame}
%-----------------------------------------------------------------------
\begin{frame}[t]
\frametitle{Estimation}
\vfill
\begin{itemize}
\item Common belief holds that: \\ \pause If parallel trends is implausible, then nonparametric identification is \textit{not} possible \vfill
\item \pause But why should we abandon DID if another trend assumption is plausible? \\~\\
\pause 
\begin{figure}
\includegraphics[width = 0.9\linewidth]{figures/diff_in_diffs_nonparallel_trend_plot.pdf}
\vskip1ex
\end{figure}
\end{itemize}
\vfill
\end{frame}
%-----------------------------------------------------------------------
\section{Inference}
\begin{frame}[t]
\frametitle{Inference}
\vfill 
\begin{itemize}
\item Two primary types of uncertainty in causal inference \citep{abadieetal2020}: \vfill
\begin{enumerate}
\item \pause \textbf{Sampling}: \pause Inability to observe all units in a target population \vfill
\item \pause \textbf{Causal}: \pause Inability to observe counterfactual potential outcomes \textit{among whichever units one samples}  \vfill
\end{enumerate}
\item \pause E.g., randomized experiment on random sample from population: \vfill
\begin{itemize}
\item \pause Uncertainty in estimator of population average causal effect reflects variation in \vfill
\begin{enumerate}
\item \pause average causal effect across possible \textit{samples} \vfill
\item \pause causal estimates across possible \textit{assignments} conditional on each possible sample \vfill
\end{enumerate} 
\end{itemize}
\item \pause Quantification of uncertainty (standard errors, $p$-values, confidence intervals) in DID reflects only sampling uncertainty \vfill
\item[] \pause $\implies$ We are too likely to detect an effect when none exists \vfill
\item \pause Quantification of causal uncertainty is difficult because DID based on parallel trends, \textit{not} as-if-randomized assumption \\ 
\citep{manskipepper2018} \vfill
\end{itemize}
\vfill
\end{frame}
%-----------------------------------------------------------------------
\section{Decomposition of ATT and generalization of DID}
\begin{frame}[t]
\frametitle{Decomposition of ATT and generalization of DID}
\begin{itemize}
\item To decompose DID, define $\Delta_{\mathcal{P}}$ as difference in counterfactual trends: \pause \vskip4ex 
\normalsize
{\large
\begin{equation*}
\begin{split}
\Delta_{\mathcal{P}} = \underbrace{\E\left[\tikzmark{controlpotouttreated}Y_{iT}(0) \given z_{iT} = 1\right]}_{\text{Inestimable}} - \underbrace{\E\left[Y_{iT-1} \given z_{iT} = 1\right]}_{\text{Estimable}} - \\ \left[\underbrace{\E\left[Y_{iT} \given z_{iT} = 0\right]}_{\text{Estimable}} - \underbrace{\E\left[Y_{iT-1} \given z_{iT} = 0\right]}_{\text{Estimable}}\right]
\end{split} 
\end{equation*}
\begin{tikzpicture}[overlay, remember picture]
  
    \node (betatyrt) [above right of = controlpotouttreated, node distance = 4 em, anchor = east, xshift = 1cm] {\footnotesize \textsf{Treated unit's control potential outcome}};
    \draw[<-, in = -400, out = 90] (controlpotouttreated.east)++(1.25em, 1.75ex) to (betatyrt.east);

\end{tikzpicture}
}
\item \pause $\E\left[\widehat{\text{DID}}_{\mathcal{P}}\right] - \Delta_{\mathcal{P}} = \text{ATT}_{\mathcal{P}}$
\item \pause Parallel trends is special case in which \normalsize $\Delta_{\mathcal{P}} = 0$ \large
\item \pause Decompose $\text{ATT}_{\mathcal{P}}$  as $\text{ATT}_{\mathcal{P}} = \underbrace{\E\left[\widehat{\text{DID}}_{\mathcal{P}}\right]}_{\substack{\text{Descriptive} \\ \text{parameter}}} - \underbrace{\Delta_{\mathcal{P}}}_{\substack{\text{Counterfactual} \\ \text{parameter}}}$
\end{itemize}
\end{frame}
%-----------------------------------------------------------------------
\begin{frame}[t]
\frametitle{Decomposition of ATT and generalization of DID}
\vfill
\begin{itemize}
\item The decomposition of $\text{ATT}_{\mathcal{P}} = \E\left[\widehat{\text{DID}}_{\mathcal{P}}\right] - \Delta_{\mathcal{P}}$ illustrates problem parallel trends poses for inference: \vfill
\begin{itemize}
\item \pause Parallel trends $\implies$ \vfill
\item[] \pause Descriptive difference between treated and control populations equal to average causal effect in treated population \vfill
\item \pause Imagine one samples entire population $\mathcal{P} = \mathcal{S}$ $\implies$ \vfill
\item[] \pause Descriptive difference between treated and control populations is known \vfill
\item \pause Hence, parallel trends assumption $\implies$ \vfill 
\item[] \pause $\text{ATT}_{\mathcal{P}}$ is known with certainty when $\mathcal{P} = \mathcal{S}$ \vfill
\end{itemize}
\item \pause Instead, let inference of $\text{ATT}_{\mathcal{P}}$ reflect sampling \textit{and} causal uncertainty: \pause \vfill
\begin{align*}
\text{ATT}_{\mathcal{P}} = \underbrace{\E\left[\widehat{\text{DID}}_{\mathcal{P}}\right]}_{\substack{\text{Sampling} \\ \text{uncertainty}}} - \underbrace{\Delta_{\mathcal{P}}}_{\substack{\text{Causal} \\ \text{uncertainty}}}
\end{align*} \vfill
\end{itemize}
\vfill
\end{frame}
%------------------------------------------------------------------------
\section{\large Finite sample estimation and inference}
%\begin{frame}
%\frametitle{Causal uncertainty}
%\begin{itemize}
%\item Assume target is ATT in finite sample such that  
%\begin{align*}
%\text{ATT}_{\mathcal{S}} & = \left(\frac{1}{n_1}\right) \sum \limits_{i = n_0 + 1}^n \left(y_{iT}(1) - y_{iT}(0)\right) \\ 
%\Delta_{\mathcal{S}} & = \left(\frac{1}{n_1}\right)\sum_{i = n_0 + 1}^{n} y_{iT}(0) - \left(\frac{1}{n_1}\right)\sum_{i = n_0 + 1}^{n} y_{iT-1} \\ & - \left[\left(\frac{1}{n_0}\right)\sum_{i = 1}^{n_0} y_{iT} - \left(\frac{1}{n_0}\right)\sum_{i = 1}^{n_0} y_{iT-1}\right]
%\end{align*} 
%\item How can we eliminate the $\Delta_{\mathcal{S}}$ nuisance parameter?
%\item Under as-if-randomized assumption, we usually 
%\begin{enumerate}
%\item condition on a sufficient statistic (matching) or 
%\item[] use plug-in estimation (inverse propensity weighted estimator)
%\item assess sensitivity over range of values of nuisance parameter
%\end{enumerate}
%\item These approaches will not work for DID
%\item Instead, marginalize over nuisance parameter with respect to its prior \\ \citep{bergeretal1999,liseo2005}
%\item But let this prior distribution be
%\begin{enumerate}
%\item \textit{conservative} (entropy maximizing) and 
%\item \textit{data-driven} (uses information from pre-treatment data)
%\end{enumerate} 
%\end{itemize}
%\end{frame}
%-----------------------------------------------------------------------
\begin{frame}[t, label = Finite sample uncertainty]
\frametitle{Finite sample estimation and inference}
\begin{itemize}
\item To focus on \textit{causal} estimation and inference, consider finite sample ($\mathcal{S}$) \pause
\item Only one term in $\Delta_{\mathcal{S}}$ is unobserved: \pause \vskip4ex
\normalsize
{\large
\begin{equation*}
\begin{split}
\Delta_{\mathcal{S}} = \underbrace{\left(\frac{1}{n_1}\right)\sum_{i = n_0 + 1}^{n} \tikzmark{controlpotouttreated} y_{iT}(0)}_{\text{unobserved}} - \underbrace{\left(\frac{1}{n_1}\right)\sum_{i = n_0 + 1}^{n} y_{iT-1}}_{\text{observed}} - \\ \left[\underbrace{\left(\frac{1}{n_0}\right)\sum_{i = 1}^{n_0} y_{iT}}_{\text{observed}} - \underbrace{\left(\frac{1}{n_0}\right)\sum_{i = 1}^{n_0} y_{iT-1}}_{\text{observed}}\right]
\end{split} 
\end{equation*}
\begin{tikzpicture}[overlay, remember picture]
  
    \node (betatyrt) [above right of = controlpotouttreated, node distance = 4 em, anchor = east, xshift = 1cm] {\footnotesize \textsf{Treated unit's counterfactual potential outcome}};
    \draw[<-, in = -400, out = 90] (controlpotouttreated.east)++(1.25em, 1.75ex) to (betatyrt.east);

\end{tikzpicture}
}

%\item \pause $\Delta_{\mathcal{S}}$ is a nuisance parameter
%\item \pause Eliminate it by marginalizing over nuisance parameter with respect to its prior \citep{bergeretal1999,liseo2005}
%\item \pause But let this prior distribution be
%\begin{enumerate}
%\item \pause \textit{conservative} (entropy maximizing) and 
%\item \pause \textit{data-driven} (uses information from pre-treatment data)
%\end{enumerate} 
\item \pause Predictive distribution on $\left(\frac{1}{n_1}\right)\sum \limits_{i = n_0 + 1}^{n} y_{iT}(0) \implies$ distribution on $\Delta_{\mathcal{S}}$
\end{itemize}
\end{frame}
%-----------------------------------------------------------------------
\section{Algorithm for estimation and inference}
\begin{frame}[t]
\frametitle{Algorithm for estimation and inference}
\vfill
\begin{itemize}
\item[] \textbf{Estimation} \vfill
\item \pause Separately fit machine-learning models to pre-treatment data in treated and control groups \vfill
\item \pause Extrapolate to post-treatment period and impute unobserved term in $\Delta_{\mathcal{S}}$ \vfill
\end{itemize}
\vfill
\end{frame}
%-----------------------------------------------------------------------
\begin{frame}[t]
\frametitle{Algorithm for estimation and inference}
\vfill
%\begin{itemize}
%\item FE linear time trend models:
%\begin{align*} 
%y_{it} & = \alpha_i + \beta_1 t + \epsilon_{it} \text{ for } i \in \left\{1, \dots , n_0\right\} \text{ and } t \in \left\{0, \dots , T - 1\right\} \\
%y_{it} & = \alpha_i + \beta_1 t + \epsilon_{it} \text{ for } i \in \left\{n_0 + 1, \dots , n\right\} \text{ and } t \in \left\{0, \dots , T - 1\right\}
%\end{align*}
%\item Data are fixed and parameters are random
%\item Distribution on $\epsilon_{it}$ derived from random distribution on parameters that is calibrated to pre-treatment data.
%\end{itemize}
%\vskip-2ex
\begin{figure}
\includegraphics[width = 0.95\linewidth]{figures/time_trend_plot.pdf}
\caption{DID design to estimate effect of Madrid train bombings (March 11, 2004) on PP vote share \citep{montalvo2011}. Nonresident voters (control) voted by mail March 2 -- 7, 2004 and resident voters (treated) voted in person March 14, 2004}
\end{figure}
\vfill
\end{frame}
%-----------------------------------------------------------------------
\begin{frame}[t]
\frametitle{Algorithm for estimation and inference}
\vfill
\begin{itemize}
\item[] \textbf{Estimation} \vfill
\item \pause Impute unobserved term in $\Delta_{\mathcal{S}}$ as either \vfill
\begin{itemize}
\item \pause \textbf{Accurate treated prediction (ATP) imputation}: \vfill
\pause
\begin{align*}
\bar{Y}^{\text{ imp}}_{T}(0) & \equiv \underbrace{\left(\frac{1}{n_1}\right) \sum \limits_{i = n_0 + 1}^{n} \hat{Y}_{iT}(0)}_{\text{Average of predictions in treated group}} \text{ or }
\end{align*}
\vfill
\item \pause \textbf{Equal deviation from predictions (EDP) imputation}: \vfill
\pause
\begin{align*}
\bar{Y}^{\text{ imp}}_{T}(0) & \equiv \underbrace{\left(\frac{1}{n_0}\right)\sum \limits_{i = 1}^{n_0} \left(y_{iT} - \hat{y}_{iT}\right)}_{\text{Average of prediction errors in control group}} + \underbrace{\left(\frac{1}{n_1}\right) \sum \limits_{i = n_0 + 1}^{n} \hat{Y}_{iT}(0)}_{\text{Average of predictions in treated group}}
\end{align*} \vfill
\end{itemize} \vfill
\item \pause Then estimate $\text{ATT}_{\mathcal{S}}$ as $\widehat{\text{DID}}_{\mathcal{S}} - \Delta^{\text{imp}}_{\mathcal{S}}$ \vfill
%\item Uncertainty represented via ``informal Bayesian approach'' \\ \citep[][]{gelmanhill2006,miratrix2019}
%\begin{itemize}
%\item Set hyperparameters of $\mathcal{N}_K\left(\bm{\mu}, \bm{\Sigma}\right)$ to estimated regression coefficients and estimated variance-covariance matrix
%\item Repeatedly draw coefficient vector from $\mathcal{N}_K\left(\bm{\mu}, \bm{\Sigma}\right)$ and generate predictions and imputations, $\Delta^{\text{imp}}_{\mathcal{S}}$
%\end{itemize}
%\item Point estimate $\text{ATT}_{\mathcal{S}}$ is $\E\left[\widehat{\text{DID}}_{\mathcal{S}} - \Delta^{\text{imp}}_{\mathcal{S}}\right]$
%\item Variance surrounding estimate of $\text{ATT}_{\mathcal{S}}$ is $\Var\left[\widehat{\text{DID}}_{\mathcal{S}} - \Delta^{\text{imp}}_{\mathcal{S}}\right]$
\end{itemize}
\vfill
\end{frame}
%-----------------------------------------------------------------------
\begin{frame}[t]
\frametitle{Algorithm for estimation and inference}
\vfill 
\begin{itemize}
\item[] \textbf{Inference} \vfill
\item \pause Intuitively, predictive distribution of plausible projected counterfactuals \vfill
\item \pause Formally, Normal data-driven prior distribution on nuisance parameter $\Delta_{\mathcal{S}}$ \vfill
\item \pause Normal prior maximizes entropy among distributions with finite variance \\ \citep{coverthomas1991} \vfill
\item \pause Hyperparameters of Normal prior calibrated to pre-treatment trends \vfill
\item \pause Uncertainty in counterfactual predictions decreasing in \vfill
\begin{enumerate}
\item \pause Stability of pre-treatment trends \vfill
\item \pause Amount of pre-treatment data \vfill
\end{enumerate} 
\item \pause Hence, prior distribution is \vfill
\begin{enumerate}
\item \pause Conservative (entropy maximizing) \vfill
\item \pause Data-driven (uses information from pre-treatment data) \vfill
\item \pause Intuitive (fits how we already reason about whether parallel trends holds) \vfill
\end{enumerate} 
\end{itemize}
\vfill
\end{frame}
%-----------------------------------------------------------------------
\begin{frame}[allowframebreaks]
\frametitle{References} 
\scriptsize
\bibliographystyle{chicago}
\bibliography{master_bibliography}   % name your BibTeX data base
\end{frame}
%------------------------------------------------------------------------
\end{document}